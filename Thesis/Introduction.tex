\section{Introduction}

The Discrete Fourier Transform (DFT) is a staple operation within Computer Science, Physics, and other fields with many applications.
Fast Fourier Transforms are implementations of the DFT with improved performance characteristics.
Most current implementations, such as WFFT\cite{Frigo2005}, take the form of large libraries written in low-level languages. 
A key component of these libraries is the use of multiple implementations of the same algorithm, with each implementation (or kernel) containing optimisations suited towards specific input sizes and hardware profiles. 
When the user wants to compute the result of a Fourier Transform, the library chooses the optimal kernel based on the input size and the user's hardware.

The large number of kernels makes it very challenging to verify that a given FFT library provides the same result as the naïve DFT.
This is because to do so would involve analysing the low-level implementation of each kernel, individually, and proving that it gives the same result as the naïve DFT for all possible inputs.
An alternate approach is as follows. % This sentence doesn't add much, but I'm not sure that it flows too well without it
Instead of analysing existing code to confirm its correctness, we can create a single specification of the FFT such that it can be instantiated to any kernel, giving us a usable kernel and formal proof that said kernel computes the expected values.

Agda is a dependently typed functional language which allows for formal properties of programs written in it to be proven.\cite{Norell2007} 
This paper discusses the use of Agda to create a general case implementation of the FFT which can be proven to always compute the same value as the naïve DFT.
This general case definition can then be used to generate the kernels for any size in a low-level, efficient language.
This allows the proof of correctness to be propagated down to any kernels generated from it allowing for a library of formally correct kernels to be generated with associated guarantees of its correctness.

