\documentclass[12pt, twoside]{article}
\usepackage{graphicx} % Required for inserting images
\usepackage{fancyhdr} % Required to put information in a right and left justified header
\usepackage{enumitem} % Required to make itemize left margin not massive 
\usepackage{hyperref} % Allows linking to the bib in citations
\usepackage{amsmath} % Used to make maths good
\usepackage{amssymb} % USed for maths symbols
\usepackage{minted}
\usepackage{pdflscape}
\usepackage{todonotes}
% Required to set the margin sizes to avoid \vspace{-9em}
\usepackage{geometry}
\geometry{
    a4paper,
    left=35mm,
    right=24mm,
    top=24mm,
    bottom=24mm,
}
\usepackage{quiver}
\usepackage{tikz}
\setlength{\headheight}{18.0pt}
\renewcommand{\familydefault}{\rmdefault}
\setlength{\parindent}{0pt}  % Disable paragraph indentation
\pagenumbering{roman}

% Used to stop me submitting the wrong version
%\usepackage{draftwatermark}
%\SetWatermarkText{Draft}
%\SetWatermarkScale{4}
%\SetWatermarkAngle{30}

% https://tex.stackexchange.com/questions/318142/typeseeting-a-multiset-with-double-curly-braces
\newcommand*{\xlbrace}{[\mskip-5mu[}
\newcommand*{\xrbrace}{]\mskip-5mu]}

\usepackage{bm}
\usepackage{agda}
\usepackage{newunicodechar}


\newunicodechar{₁}{\ensuremath{_1}}
\newunicodechar{₂}{\ensuremath{_2}}
\newunicodechar{₃}{\ensuremath{_3}}
\newunicodechar{₄}{\ensuremath{_4}}
\newunicodechar{₅}{\ensuremath{_5}}
\newunicodechar{ι}{\ensuremath{\iota}}
\newunicodechar{ₗ}{\ensuremath{_l}}
\newunicodechar{ₐ}{\ensuremath{_a}}
\newunicodechar{ₖ}{\ensuremath{_k}}
\newunicodechar{ᵢ}{\ensuremath{_i}}
\newunicodechar{ⱼ}{\ensuremath{_j}}
\newunicodechar{ᵥ}{\ensuremath{_v}}
\newunicodechar{ₕ}{\ensuremath{_h}}

\newunicodechar{ᵣ}{\ensuremath{_r}}
\newunicodechar{ʳ}{\ensuremath{^r}}
\newunicodechar{ˡ}{\ensuremath{^l}}
\newunicodechar{ℕ}{\ensuremath{\mathbb{N}}}
\newunicodechar{∀}{\ensuremath{\forall}}
\newunicodechar{≡}{\ensuremath{\equiv}}
\newunicodechar{≈}{\ensuremath{\approx}}
\newunicodechar{∎}{\ensuremath{\blacksquare}}
\newunicodechar{⊛}{\ensuremath{\circledast}}
\newunicodechar{⊗}{\ensuremath{\otimes}}
\newunicodechar{⊕}{\ensuremath{\oplus}}
\newunicodechar{φ}{\ensuremath{\phi}}
\newunicodechar{ψ}{\ensuremath{\psi}}
\newunicodechar{ε}{\ensuremath{\epsilon}}
\newunicodechar{λ}{\ensuremath{\lambda}}
\newunicodechar{σ}{\ensuremath{\sigma}}
\newunicodechar{′}{{'}}
\newunicodechar{∷}{\ensuremath{\dblcolon}}
\newunicodechar{↔}{\ensuremath{\leftrightarrow}}
\newunicodechar{↦}{\ensuremath{\mapsto}}
\newunicodechar{∘}{\ensuremath{\circ}}
\newunicodechar{⁻}{\ensuremath{^{-}}}
\newunicodechar{∙}{\ensuremath{\boldsymbol{\cdot}}}
\newunicodechar{▹}{\ensuremath{\triangleright}}
\newunicodechar{Σ}{\ensuremath{\Sigma}}
\newunicodechar{∈}{\ensuremath{\in}}
\newunicodechar{⟦}{\ensuremath{\llbracket}}
\newunicodechar{⟧}{\ensuremath{\rrbracket}}
\newunicodechar{⇒}{\ensuremath{\Rightarrow}}
\newunicodechar{⟪}{\ensuremath{\xlbrace}}
\newunicodechar{⟫}{\ensuremath{\xrbrace}}
\newunicodechar{⊠}{\ensuremath{\boxtimes}}
\newunicodechar{⊞}{\ensuremath{\boxplus}}
\newunicodechar{∋}{\ensuremath{\ni}}
\newunicodechar{∶}{\ensuremath{\bm{:}}}
\newunicodechar{↦}{\ensuremath{\mapsto}}
\newunicodechar{ℂ}{\ensuremath{\mathbb{C}}} 

\newunicodechar{ʹ}{\ensuremath{\prime}}
\newunicodechar{≢}{\ensuremath{\not\equiv}}
\newunicodechar{⦃}{\textlbrackdbl}
\newunicodechar{⦄}{\textrbrackdbl}
\newunicodechar{ω}{\ensuremath{\omega}}
\newunicodechar{ₙ}{\ensuremath{_n}}
\newunicodechar{₀}{\ensuremath{_0}}
\newunicodechar{⋆}{\ensuremath{\cdot}}
\newunicodechar{♯}{\ensuremath{\sharp}}
\newunicodechar{♭}{\ensuremath{\flat}}
\newunicodechar{ₛ}{\ensuremath{_s}}
\newunicodechar{≅}{\ensuremath{\cong}}
\newunicodechar{ᵗ}{\ensuremath{^t}}
\newunicodechar{⊡}{\ensuremath{\boxdot}}
% Some shortcut commands for agda symbols
\newcommand{\AD}[1]{\AgdaDatatype{#1}}
\newcommand{\AC}[1]{\AgdaInductiveConstructor{#1}}
\newcommand{\AF}[1]{\AgdaFunction{#1}}
\newcommand{\AB}[1]{\AgdaBound{#1}}
\newcommand{\AK}[1]{\AgdaKeyword{#1}}
\newcommand{\AR}[1]{\AgdaField{#1}}
\newcommand{\AM}[1]{\AgdaModule{#1}}
\newcommand{\AN}[1]{\AgdaNumber{#1}}
\newcommand{\AS}[1]{\AgdaString{#1}}

\renewcommand{\AgdaCommentFontStyle}[1]{\textrm{#1}}
\renewcommand{\AgdaFontStyle}[1]{\textrm{#1}}
\renewcommand{\AgdaKeywordFontStyle}[1]{\textrm{#1}}

\usepackage[T1]{fontenc}
\usepackage{microtype}
\DisableLigatures[-]{encoding=T1}



%TC:ignore
\pagestyle{fancy}
\lhead{
    \large
    % Electronics and Computer Science \\
    % Faculty of Engineering and Physical Sciences \\
    % University of Southampton 
    Final Report
    }
\rhead{
    \large 15th August 2025
}

\begin{document}
\thispagestyle{fancy}
\begin{center}
    \vspace*{20mm}
    \Large
    Electronics and Computer Science \\
    Faculty of Engineering and Physical Sciences \\
    University of Southampton 

    
    \vspace*{20mm}
    \Large Callum Gilchrist (\verb|cg3g22@soton.ac.uk|)
    
    \large 15th August 2025
    
    \vspace*{20mm}
    \Huge Formal Verification of Fast Fourier Transforms
    \vspace*{60mm}
    
    % \large Project Brief
    \large \textbf{Supervisor:} Artjoms Šinkarovs (\verb|a.sinkarovs@soton.ac.uk|)
    \large \textbf{Second Examiner:} Vahid Yazdanpanah (\verb|v.yazdanpanah@soton.ac.uk|)

    \vspace*{10mm}
    \normalsize A Project Report submitted for the award of \\
    \large \textbf{BSc Computer Science}
    
    \vspace*{10mm}
    
\end{center}
\clearpage
\addcontentsline{toc}{section}{Abstract}
\section*{Abstract}
Discrete Fourier Transforms (DFTs) are key operations within Digital Signal Processing and other fields, Fast Fourier Transforms (FFTs) allow for the time complexity of computing the DFT to be significantly reduced. 
Implementations of the FFT often comprise large, low-level libraries written with efficiency in mind, making verification of their correctness challenging.
Agda is a dependently typed functional language which implements Martin-Löf type theory allowing proofs to be embedded within code.
As a result of including these proofs, programs written in Agda can contain formal guarantees of their correctness, for the FFT this requires proving that the DFT is equal to the FFT for all cases.


In this project, I have thus far created an Agda definition of the DFT and FFT.
I now plan to implement proof that the two definitions are equal for all possible inputs.
Given such a proof I will then use the FFT definition to generate a low-level library of my own which can perform Fourier Transforms with strict guarantees on its correctness.

\clearpage
%TC:endignore
\include{Thesis/StatementOfOriginality}
\clearpage
% \addcontentsline{toc}{section}{Nomenclature}
% \section*{Nomenclature}
% \begin{table}[h]
%     \centering
%     \begin{tabular}{|c|c|}
%         \hline
%          DFT & Discrete Fourier Transform  \\
%         \hline
%          FFT & Fast Fourier Transform  \\
%         \hline
%     \end{tabular}
%     \caption{Nomenclature}
%     \label{tab:my_label}
% \end{table}
% \clearpage
\addcontentsline{toc}{section}{Contents}
\tableofcontents
\clearpage
% Add space between paragraphs, but only after the TOC
\setlength{\parskip}{0.5em}
\pagenumbering{arabic}
\clearpage
\section{Introduction}

The Discrete Fourier Transform (DFT) is a staple operation within Computer 
Science, Physics, and other fields with many applications.
Fast Fourier Transforms are implementations of the DFT with improved 
performance characteristics.
One such use of the FFT is polynomial multiplication, the time 
complexity of which can be reduced from $\mathcal{O}\left(n^2\right)$ to
$\mathcal{O}\left(n\log n\right)$.

Most current implementations, such as WFFT\cite{Frigo2005}, take the form of 
large libraries written in low-level languages. 
A key component of these libraries is the use of multiple implementations of 
the same algorithm, with each implementation (or kernel) containing 
optimisations suited towards specific input sizes and hardware profiles. 
When the user wants to compute the result of a Fourier Transform, the library 
chooses the optimal kernel based on the input size and the user's hardware.

The large number of kernels makes it very challenging to verify that a given 
FFT library provides the same result as the naïve DFT.
This is because to do so would involve analysing the low-level implementation 
of each kernel, individually, and proving that it gives the same result as the 
naïve DFT for all possible inputs.
An alternate approach is as follows. % This sentence doesn't add much, 
                                     % but I'm not sure that it flows too well 
                                     % without it
Instead of analysing existing code to confirm its correctness, we can create a 
single specification of the FFT such that it can be instantiated to any kernel, 
giving us a usable kernel and formal proof that said kernel computes the 
expected values.

Agda is a dependently typed functional language which allows for formal 
properties of programs written in it to be proven.\cite{Norell2007} 
This paper discusses the use of Agda to create a general case implementation 
of the FFT which is then proven to always compute the same value as the naïve 
DFT.

Such an implementation would allow for future research in Agda to make use of
the FFT in definitions, before substituting it with the DFT when it comes to
generating proofs.
This would be useful for research into algorithms which utilise the FFT, such as
efficient polynomial multiplication.
Such an implementation would also allow for future generation of low-level, 
efficient kernels, with a formally verified basis.

\section{Background}
\subsection{Fourier Transforms}
Described as ``perhaps the most ubiquitous algorithm in use today''\cite{Top10Algos},
Fourier Transforms are mathematical operations which transform functions between 
the time domain and the frequency domain.
Fourier Transforms, and derivatives of, receive their name from the French 
mathematician and physicist Jean-Baptiste-Joseph Fourier who proposed in his 
$1822$\cite{Fourier1822} treatise that any given function can be represented as 
a harmonic series.\cite{Saribulut2013} 
% https://mathoverflow.net/questions/417034/who-introduced-the-discrete-fourier-transform
While bearing Fourier's name, some early forms of the Discrete Fourier 
Transform (DFT), a Fourier Transform which works on evenly spaced samples of a 
function, can be found before Fourier's time.
As discussed by Heideman and Johnson in ``Gauss and the History of the Fast 
Fourier Transform''\cite{Heideman1985}, the earliest known example of this can 
be found in work published by Alexis-Claude Clairaut in $1754$\cite{Clairaut1754}.
Clairaut defined a variation of the DFT which exclusively used what we now refer 
to as the cosine component, thus restricting the input domain to the set of 
even functions\footnote{The term ``even function'' refers to the set of 
functions $f(x)$ such that $f(-x)=f(x)$, that is to say, the set of functions 
which are symmetric over the y-axis.  \cite{Gelfand1990}\cite{Tolstov1962}}.\cite{Heideman1985}
Carl Friedrich Gauss extended Clairaut's definition to make use of both cosine 
and sine components, removing the need for the input domain to be restricted 
to the set of even functions and allowing for the analysis of any periodic 
function.\cite{Gauss1866}\cite{Heideman1985}
This definition was published posthumously in $1866$, however, it is believed 
that it was originally written in $1805$.\cite{Heideman1985} 

We can use the historical definitions discussed above to create our modern 
definition for the DFT as follows.
Given an input sequence $x = 
        (x_0,x_1,\dots,x_{n-1})$,
            where $x_i\in\mathbb{C}
            $,
our transformed sequence $X =
        (X_0,X_1,\dots,X_{n-1})$,
            where $X_i\in\mathbb{C}
            $,
is given as follows.

\begin{align}
    X_j &= \sum_{k=0}^{N-1}x_k\omega_N^{kj}\label{eq:DFT_Definition}
\end{align}
\begin{equation}
    \text{where}~\omega_N~=e^{-\frac{2\pi i}{N}}
    = \cos{\left(\frac{2\pi}{N}\right)}-i\sin{\left(\frac{2\pi}{N}\right)}\label{eq:ComplexRootsOfUnity}
\end{equation}
% Original
%\begin{align}
%    X_j &= \sum_{k=0}^{N-1}\omega_N^{jk}x_k\label{eq:DFT_Definition} \\
%    \text{where}~\omega_N~&=e^{-\frac{2\pi i}{N}}\\
%    &= \cos{\left(\frac{2\pi}{N}\right)}-i\sin{\left(\frac{2\pi}{N}\right)}
%\end{align}

The DFT Eq. \ref{eq:DFT_Definition} has applications in a variety of fields, 
such as digital signal processing\cite{Bellanger2024}.
When implemented naïvely, however, it has poor performance scaling, requiring  
``$\mathcal{O}\left(n^2\right)$ complex operations'' \cite{VanLoan1992}.
Methods to reduce the number of complex operations required when computing the DFT were first investigated by Gauss in his $1805$ treatise such that the ``tediousness of mechanical calculations''\cite{Gauss1866} could be reduced.\cite{Heideman1985}
In part due to his lack of research into the complexity scaling factor of his method, Gauss's research into how computation complexity could be reduced was not widely recognised until $1977$ when H. H. Goldstine highlighted Gauss's research in an article for the Journal of Applied Mathematics and Mechanics.\cite{Heideman1985}\cite{Heinrich1980}
While the DFT continued to be of great use to mathematicians through the 20th century, and with Gauss's work on complexity remaining hidden, some attempts (such as those by Danielson and Lanczos \cite{Danielson1942} and by Good \cite{Good1958}) were made to create Fast Fourier Transform algorithms (FFT algorithms) which could reduce the complexity of computation to $\mathcal{O}\left(n\log n\right)$.
These algorithms, however, where only applicable to a subset of the domain\cite{Good1958}, succeeded only in reducing the constant on $\mathcal{O}\left(n^2\right)$, or did not directly perform the computational complexity\cite{Danielson1942}.

In $1965$ James William Cooley and John Tukey succeeded in discovering an FFT algorithm through the inadvertent reinvention of Gauss's algorithm for fast computation of the DFT; This would henceforth be known as the Cooley-Tukey FFT Algorithm.\cite{Cooley1965}\cite{Heideman1985}
This FFT Algorithm allows for a given DFT to be computed with $\mathcal{O}\left(n\log n\right)$ complex operations through recursive splitting of the input.\cite{Cooley1965}
Although other FFT Algorithms were discovered before and after the Cooley-Tukey FFT, it is commonly considered to be ``the most important FFT''\cite{Frigo2005}.
This is because this improvement in time complexity allowed algorithms with 
time complexity previously bounded by use of the DFT, to reduce this complexity 
to, at the lowest, $\mathcal{O}\left(n^2\right)$.
In the example of polynomial multiplication, this allowed for the computation
to be moved into the frequency domain, reducing the time complexity from 
$\Theta\left(n^2\right)$ to $\Theta\left(n\log n\right)$.\cite{IntroToAlgos}


The Cooley-Tukey FFT can be derived from the DFT Eq. \ref{eq:DFT_Definition} by splitting any non-prime input $n$ into the composite $n=r_1r_2$ and expressing the indices $k$ and $j$ as follows.
\begin{align}\label{eq:IndexManipulation}
    \begin{aligned}
        j&=j_1r_1+j_0 \\
        \text{where }~
        j_0&=(0,1,\dots,r_1-1) \\
        j_1&=(0,1,\dots,r_2-1) 
    \end{aligned}
    \begin{aligned}
        &~&~&~
    \end{aligned}
    \begin{aligned}
        k&=k_1r_2+k_0 \\
        \text{where }~k_0&=(0,1,\dots,r_2-1) \\
        k_1&=(0,1,\dots,r_1-1)
    \end{aligned}
\end{align}
Eq. \ref{eq:DFT_Definition} can then be arranged to take the following form.
\begin{align}
    X_{j_1r_1+j_0}&=\sum^{r_2-1}_{k_0=0}\left[\left(\sum^{r_1-1}_{k_1=0}x_{k_1r_2+k_0}\omega_{r_1}^{k_1j_0}\right)\omega_{r_1r_2}^{k_0j_1}\right]\omega_{r_2}^{k_0j_1}
    \label{eq:FFTDefinitionFromDFT}
\end{align}
When written in this form our recursive step, and thus the core idea of the Cooley-Tukey FFT, be easily observed by noting that the inner sum takes the form of a DFT of length $r_1$.
% \begin{align}
%     j&=j_1r_1+j_0, &k&=k_1r_2+k_0 \\
%     \text{where }~
%      j_0&=(0,1,\dots,r_1-1) 
%     &k_0&=(0,1,\dots,r_2-1) \\
%      j_1&=(0,1,\dots,r_2-1) 
%     &k_1&=(0,1,\dots,r_1-1)
% \end{align}


%\begin{align}
%    n&=n_1
%\end{align}\cite{Cooley1965AnSeries}
%
%We can then represent

%Methods to reduce the number of complex operations required when computing DFTs were first investigated by Gauss in his $1805$ treatise such that the ``tediousness of mechanical calculations''\cite{Gauss1868Nachlass:Tractata} could be reduced.\cite{Heideman1985GaussTransform}
%However, for a variety of reasons, Gauss's research into how computation complexity could be reduced was not widely recognised. \cite{Heideman1985GaussTransform}\cite{The one which worked out Gauss is a cool kid}
%In 1965 James William Cooley and John Tukey inadvertently reinvented Gauss's algorithm for fast computation of the DFT, in what would henceforth be known as the Cooley-Tukey Fast Fourier Transform (FFT) Algorithm.\cite{Cooley1965AnSeries}\cite{Heideman1985GaussTransform}
%This FFT Algorithm allows for a given DFT to be computed with $\mathcal{O}\left(n\log n\right)$ complex operations through recursive splitting on the input vector.\cite{Cooley1965AnSeries}


%never published investigations into the resultant reduction in computational complexity meaning that his methods, which reduced the complexity to $\mathcal{O}\left( N\log N \right)$ would go unnoticed until *AFTER C-T but I ain't mentioned C-T yet...*.



\subsection{Agda}
Agda\footnote{Reference to ``Agda'' throughout this report will always refer to 
version 2 unless explicitly stated otherwise} is a functional programming 
language which implements Martin-Löf Type Theory.\cite{Norell2007}\cite{Martin-Lf1984}
Martin-Löf type theory provides the definition of, and Agda allows for the 
construction of, dependent types.\cite{Norell2007}
These types allow for the definition of invariant properties which are checked
at compile time.\cite{Norell2007}
As well as making a variety of common errors, such as out-of-bound indexing,
unreachable, invariance properties can be used to guarantee functional
properties.\cite{Norell2007}
When evidence that properties hold is non trivial, proofs that the properties
hold must be provided.
This allows for strong guarantees to be formed on any program defined in Agda.
These proofs allow systems to be built which are provably correct allowing for a high confidence in their reliability.\cite{Norell2007}

Agda is not the only such proof assistant, and others exist with Agda's main 
contender being Coq.
Coq considers programs and proofs - or as it refers to them, tactics - 
separately.
This means that ``every concept has to be learned twice''\cite{PLFA}, for its 
program component and tactic component. \cite{Barras1999}
In Agda, however, proofs and programs are considered in the same light, removing
the need for this additional syntax and simplifying the programs within it.

\subsection{Related work}
FFTW\cite{Frigo2005} is a \verb|C| code library which is generally accepted within academia and industry as the fastest method with which the FFT can be correctly computed.\cite{Frigo1999} 
It achieves this title by implementing its own ``special-purpose compiler''\cite{Frigo1999}, \verb|genfft|, this compiler accepts the size of the transform as input and outputs a kernel - a \verb|c| code implementations of some known algorithm (i.e. the Cooley-Tukey FFT \cite{Cooley1965} Eq .\ref{eq:FFTDefinitionFromDFT}) optimised for that sized transform and the current hardware.\cite{Frigo1999} 
Although it is known through rigorous testing and real-world use that FFTW is correct, there is no formal verification of its correctness. %to provide a formal verification of the library would also be a very challenging undertaking requiring analysis of the low-level code which makes up the library.
% An alternate approach to formally verifying its correctens

As FFTW does not come with such formal guarantees 
separate definitions of various FFTs have been created before in proof assistance such as Coq\cite{Barras1999} and Hol\cite{Gordon1993} with various methods and goals. In the paper ``Certifying the Fast Fourier Transform with Coq''\cite{Capretta2001}, Capretta makes use of binary trees to create a definition of the Cooley-Tukey FFT\cite{Cooley1965} for the radix-2 case (when $r_1=2$).
This definition is then proven to be extensionally equal to that of the DFT.
This provides a good definition for the radix-2 case of the FFT, allowing for it to be built on to create future proofs should they require the FFT, however, it does not cover the generalisation on the radix restricting the proof to specific splitting strategies.
% Theres probably allot more waffle I can put in here

In another paper, ``A Methodology for the Formal Verification of FFT Algorithms in 
HOL'',\cite{Akbarpour2004} Akbarpour and Tahar create two definitions of the 
Cooley-Tukey FFT\cite{Cooley1965} in Hol for the radix-2 and radix-4 cases.
With a primary focus on the radix-2 case Akbarpour and Tahar go on to show 
equivalence to the DFT across various levels of abstraction.\cite{Akbarpour2004}
At one stage of this abstraction, Akbarpour and Tahar introduce floating and 
fixed point arithmetic, showing an analysis of the resultant errors.\cite{Akbarpour2004}

Much like Capretta\cite{Capretta2001}, this paper also does not make use of 
a general radix, however, it does highlight how its methodology can be used to 
analyse general radix FFT implementations.
Currently, all previous work to formally verify the Cooley Tukey FFT has used 
fixed radices, while most common implementations utilise mixed radices which 
``are adapted to the hardware''\cite{Frigo2005}.
This show a gap in the existing research, as no verification on these mixed
radix cases is present.

%TC:ignore
\begin{code}[hide]%
\>[0]\<%
\\
\>[0]\AgdaKeyword{module}\AgdaSpace{}%
\AgdaModule{Model}\AgdaSpace{}%
\AgdaKeyword{where}%
\>[2I]\AgdaComment{--\ This\ allows\ me\ to\ use\ arbitrary\ module\ names\ from\ here\ }\<%
\\
\>[.][@{}l@{}]\<[2I]%
\>[19]\AgdaComment{--\ onwards}\<%
\\
%
\\[\AgdaEmptyExtraSkip]%
\>[0]\AgdaKeyword{open}\AgdaSpace{}%
\AgdaKeyword{import}\AgdaSpace{}%
\AgdaModule{Real}\AgdaSpace{}%
\AgdaKeyword{using}\AgdaSpace{}%
\AgdaSymbol{(}\AgdaRecord{Real}\AgdaSymbol{)}\<%
\\
%
\\[\AgdaEmptyExtraSkip]%
\>[0]\AgdaKeyword{open}\AgdaSpace{}%
\AgdaKeyword{import}\AgdaSpace{}%
\AgdaModule{Data.Nat.Base}\AgdaSpace{}%
\AgdaKeyword{using}\AgdaSpace{}%
\AgdaSymbol{(}\AgdaDatatype{ℕ}\AgdaSymbol{;}\AgdaSpace{}%
\AgdaRecord{NonZero}\AgdaSymbol{;}\AgdaSpace{}%
\AgdaInductiveConstructor{suc}\AgdaSymbol{)}\AgdaSpace{}%
\AgdaKeyword{renaming}\AgdaSpace{}%
\AgdaSymbol{(}\AgdaOperator{\AgdaPrimitive{\AgdaUnderscore{}*\AgdaUnderscore{}}}\AgdaSpace{}%
\AgdaSymbol{to}\AgdaSpace{}%
\AgdaOperator{\AgdaPrimitive{\AgdaUnderscore{}*ₙ\AgdaUnderscore{}}}\AgdaSymbol{;}\AgdaSpace{}%
\AgdaOperator{\AgdaPrimitive{\AgdaUnderscore{}+\AgdaUnderscore{}}}\AgdaSpace{}%
\AgdaSymbol{to}\AgdaSpace{}%
\AgdaOperator{\AgdaPrimitive{\AgdaUnderscore{}+ₙ\AgdaUnderscore{}}}\AgdaSymbol{)}\<%
\\
\>[0]\AgdaKeyword{open}\AgdaSpace{}%
\AgdaKeyword{import}\AgdaSpace{}%
\AgdaModule{Data.Nat.Properties}\AgdaSpace{}%
\AgdaKeyword{using}\AgdaSpace{}%
\AgdaSymbol{(}\AgdaFunction{m*n≢0}\AgdaSymbol{)}\<%
\\
%
\\[\AgdaEmptyExtraSkip]%
\>[0]\AgdaKeyword{import}\AgdaSpace{}%
\AgdaModule{Relation.Binary.PropositionalEquality}\AgdaSpace{}%
\AgdaSymbol{as}\AgdaSpace{}%
\AgdaModule{Eq}\<%
\\
\>[0]\AgdaKeyword{open}\AgdaSpace{}%
\AgdaModule{Eq}\AgdaSpace{}%
\AgdaKeyword{using}\AgdaSpace{}%
\AgdaSymbol{(}\AgdaOperator{\AgdaDatatype{\AgdaUnderscore{}≡\AgdaUnderscore{}}}\AgdaSymbol{;}\AgdaSpace{}%
\AgdaInductiveConstructor{refl}\AgdaSymbol{;}\AgdaSpace{}%
\AgdaFunction{cong}\AgdaSymbol{;}\AgdaSpace{}%
\AgdaFunction{sym}\AgdaSymbol{)}\<%
\\
\>[0]\AgdaKeyword{open}\AgdaSpace{}%
\AgdaModule{Eq.≡-Reasoning}\<%
\\
%
\\[\AgdaEmptyExtraSkip]%
\>[0]\AgdaKeyword{import}\AgdaSpace{}%
\AgdaModule{Algebra.Structures}\AgdaSpace{}%
\AgdaSymbol{as}\AgdaSpace{}%
\AgdaModule{AlgebraStructures}\<%
\\
\>[0]\AgdaKeyword{open}\AgdaSpace{}%
\AgdaModule{AlgebraStructures}%
\>[24]\AgdaKeyword{using}\AgdaSpace{}%
\AgdaSymbol{(}\AgdaRecord{IsCommutativeMonoid}\AgdaSymbol{)}\<%
\\
\>[0]\AgdaKeyword{import}\AgdaSpace{}%
\AgdaModule{Algebra.Definitions}\AgdaSpace{}%
\AgdaSymbol{as}\AgdaSpace{}%
\AgdaModule{AlgebraDefinitions}\<%
\\
\>[0]\AgdaKeyword{open}\AgdaSpace{}%
\AgdaKeyword{import}\AgdaSpace{}%
\AgdaModule{Algebra.Core}\<%
\\
%
\\[\AgdaEmptyExtraSkip]%
\>[0]\AgdaKeyword{open}\AgdaSpace{}%
\AgdaKeyword{import}\AgdaSpace{}%
\AgdaModule{Function.Base}\AgdaSpace{}%
\AgdaKeyword{using}\AgdaSpace{}%
\AgdaSymbol{(}\AgdaOperator{\AgdaFunction{\AgdaUnderscore{}\$\AgdaUnderscore{}}}\AgdaSymbol{;}\AgdaSpace{}%
\AgdaOperator{\AgdaFunction{\AgdaUnderscore{}∘\AgdaUnderscore{}}}\AgdaSymbol{)}\<%
\\
%
\\[\AgdaEmptyExtraSkip]%
\>[0]\AgdaKeyword{open}\AgdaSpace{}%
\AgdaKeyword{import}\AgdaSpace{}%
\AgdaModule{Data.Product.Base}\AgdaSpace{}%
\AgdaKeyword{using}\AgdaSpace{}%
\AgdaSymbol{(}\AgdaOperator{\AgdaFunction{\AgdaUnderscore{}×\AgdaUnderscore{}}}\AgdaSymbol{;}\AgdaSpace{}%
\AgdaField{proj₁}\AgdaSymbol{;}\AgdaSpace{}%
\AgdaField{proj₂}\AgdaSymbol{)}\AgdaSpace{}%
\AgdaKeyword{renaming}\AgdaSpace{}%
\AgdaSymbol{(}\AgdaOperator{\AgdaInductiveConstructor{\AgdaUnderscore{},\AgdaUnderscore{}}}\AgdaSpace{}%
\AgdaSymbol{to}\AgdaSpace{}%
\AgdaOperator{\AgdaInductiveConstructor{⟨\AgdaUnderscore{},\AgdaUnderscore{}⟩}}\AgdaSymbol{)}\<%
\end{code}
%TC:endignore
% From Equation \ref{eq:DFT_Definition} we have a definition of the DFT which we 
% know to be correct, but this is not yet in a useable form. 
% In order to prove against this definition we must define our DFT in Agda given 
% some definition of Complex numbers, and some definition of Vectors. 
% 
% As we did above, we must then perform a likewise conversion for our FFT 
% definition, Equation \ref{eq:FFT_Definition}.
% Although it would be possible to implement this definition with respect to
% an input Vector, using instead an input Matrix of arbitrary shape will make both
% the definition, and the proofs applicable to any radix or mix of radices.
% 
% Given that Vectors can be considered one dimenti


\section{Implementation}

Before the DFT and FFT can be reasoned on, it is important to invest into the
definition of some data structures which can accurately encode all required data
and operations upon that data.
% These data structures hold useful properties about the data within them, these 
% properties can then be utilised by my implementation.
Well defined data structures allow us to abstract useful properties on the data
held within them.
This allows for reasoning to be performed on a high level where:
\begin{itemize}
\item Variations of the Fourier Transform, 
        such as the Inverse Fourier Transform, can be 
        instantiated without a need to modify the underlying code.
\item Definition can be made highly compact.
\item The possibility for out-of-bound indexing errors can be eliminated by 
        construction.
\item The FFT can be defined for an input tensor of arbitrary shape, allowing 
        multiple kernels to be defined using just one definition.
\item The FFT can be described over a number system of arbitrary structure,
        allowing it to be instantiated for any number system with the correct
        properties
\end{itemize}


\subsection{Complex Numbers}
\label{sec:complex_numbers}

% The Agda Standard library does not provide definitions for Complex numbers, it
% is therefore necessary for us to design and decide upon an encoding.
It is well known \cite{TheDFT}
that the DFT and FFT can be implemented on an arbitrary field-\textit{like}
\footnote{This structure is only field-\textit{like} because it does not require multiplicative inverses}
structure with roots of unity.
Agda allows this this idea to be captured precisely though the creation of a
structure, \AF{Cplx}, which axiomatizes this field and its properties which the FFT
and correctness proof rely on.
This is similar to Java interfaces, defining the carrier and operations, but also
allows for the properties (such as the associativity of addition) of this field to 
be defined.

This isolation allows the definition of the DFT, FFT and proofs to be instantiated
for any implementation of \AF{Cplx}.
This generality allows the use of any modular field of 
sufficient size which holds the required properties, allowing operations such as 
fast multiplication to be performed upon these fields.
% As Agda provides a builtin wrapper around IEEE754 floats\cite{IEEE754}
% the examples shown in this paper, use a simple implementation of \AF{Cplx} built 
% from pair of floating point numbers.

With the required operations and properties in mind, a structure can
be formed to encapsulate complex.
This structure first defines the carrier set, \AF{ℂ}, and the of basic operations
any implementation of complex must contain, each defined in a similar way to \AF{\_+\_} below.

\begin{AgdaMultiCode}
%TC:ignore
\begin{code}[hide]%
\>[0]\AgdaKeyword{module}\AgdaSpace{}%
\AgdaModule{ComplexMini}\AgdaSpace{}%
\AgdaSymbol{(}\AgdaBound{real}\AgdaSpace{}%
\AgdaSymbol{:}\AgdaSpace{}%
\AgdaRecord{Real}\AgdaSymbol{)}\AgdaSpace{}%
\AgdaKeyword{where}\<%
\\
\>[0][@{}l@{\AgdaIndent{0}}]%
\>[2]\AgdaKeyword{open}\AgdaSpace{}%
\AgdaModule{Real.Real}\AgdaSpace{}%
\AgdaBound{real}\AgdaSpace{}%
\AgdaKeyword{using}\AgdaSpace{}%
\AgdaSymbol{(}\AgdaField{ℝ}\AgdaSymbol{;}\AgdaSpace{}%
\AgdaFunction{0ℝ}\AgdaSymbol{;}\AgdaSpace{}%
\AgdaFunction{1ℝ}\AgdaSymbol{;}\AgdaSpace{}%
\AgdaFunction{-1ℝ}\AgdaSymbol{)}\<%
\\
%
\\[\AgdaEmptyExtraSkip]%
%
\>[2]\AgdaKeyword{import}\AgdaSpace{}%
\AgdaModule{Algebra.Structures}\AgdaSpace{}%
\AgdaSymbol{as}\AgdaSpace{}%
\AgdaModule{AlgebraStructures}\<%
\\
%
\>[2]\AgdaKeyword{import}\AgdaSpace{}%
\AgdaModule{Algebra.Definitions}\AgdaSpace{}%
\AgdaSymbol{as}\AgdaSpace{}%
\AgdaModule{AlgebraDefinitions}\<%
\\
%
\\[\AgdaEmptyExtraSkip]%
%
\>[2]\AgdaKeyword{private}\<%
\\
\>[2][@{}l@{\AgdaIndent{0}}]%
\>[4]\AgdaKeyword{variable}\<%
\\
\>[4][@{}l@{\AgdaIndent{0}}]%
\>[6]\AgdaGeneralizable{N}\AgdaSpace{}%
\AgdaGeneralizable{m}\AgdaSpace{}%
\AgdaGeneralizable{r₁}\AgdaSpace{}%
\AgdaGeneralizable{x}\AgdaSpace{}%
\AgdaGeneralizable{y}\AgdaSpace{}%
\AgdaGeneralizable{k₀}\AgdaSpace{}%
\AgdaGeneralizable{k₁}\AgdaSpace{}%
\AgdaSymbol{:}\AgdaSpace{}%
\AgdaDatatype{ℕ}\<%
\\
%
\>[4]\AgdaKeyword{postulate}%
\>[15]\AgdaComment{--\ I\ realise\ how\ horrendusly\ cursed\ postulating\ an\ instance\ is...}\<%
\\
\>[4][@{}l@{\AgdaIndent{0}}]%
\>[6]\AgdaKeyword{instance}\AgdaSpace{}%
\AgdaComment{--\ but\ it\ works...}\<%
\\
\>[6][@{}l@{\AgdaIndent{0}}]%
\>[8]\AgdaPostulate{nonZero-n}\AgdaSpace{}%
\AgdaSymbol{:}\AgdaSpace{}%
\AgdaRecord{NonZero}\AgdaSpace{}%
\AgdaGeneralizable{N}\<%
\\
%
\>[8]\AgdaPostulate{nonZero-r₁}\AgdaSpace{}%
\AgdaSymbol{:}\AgdaSpace{}%
\AgdaRecord{NonZero}\AgdaSpace{}%
\AgdaGeneralizable{r₁}\<%
\\
%
\>[8]\AgdaPostulate{nonZero-x}\AgdaSpace{}%
\AgdaSymbol{:}\AgdaSpace{}%
\AgdaRecord{NonZero}\AgdaSpace{}%
\AgdaGeneralizable{x}\<%
\\
\>[0]\<%
\end{code}
%TC:endignore
\begin{code}%
\>[0][@{}l@{\AgdaIndent{1}}]%
\>[2]\AgdaKeyword{record}\AgdaSpace{}%
\AgdaRecord{Cplx}\AgdaSpace{}%
\AgdaSymbol{:}\AgdaSpace{}%
\AgdaPrimitive{Set₁}\AgdaSpace{}%
\AgdaKeyword{where}\<%
\end{code}
%TC:ignore
\begin{code}[hide]%
\>[2][@{}l@{\AgdaIndent{1}}]%
\>[4]\AgdaKeyword{infix}%
\>[11]\AgdaNumber{8}\AgdaSpace{}%
\AgdaOperator{\AgdaField{-\AgdaUnderscore{}}}\<%
\\
%
\>[4]\AgdaKeyword{infixl}\AgdaSpace{}%
\AgdaNumber{7}\AgdaSpace{}%
\AgdaOperator{\AgdaField{\AgdaUnderscore{}*\AgdaUnderscore{}}}\<%
\\
%
\>[4]\AgdaKeyword{infixl}\AgdaSpace{}%
\AgdaNumber{6}\AgdaSpace{}%
\AgdaOperator{\AgdaField{\AgdaUnderscore{}+\AgdaUnderscore{}}}\ \AgdaUnderscore{}-\AgdaUnderscore{}\<%
\end{code}
%TC:endignore
\begin{code}%
%
\>[4]\AgdaKeyword{field}\<%
\\
\>[4][@{}l@{\AgdaIndent{0}}]%
\>[6]\AgdaField{ℂ}\AgdaSpace{}%
\AgdaSymbol{:}\AgdaSpace{}%
\AgdaPrimitive{Set}\<%
\\
%
\>[6]\AgdaOperator{\AgdaField{\AgdaUnderscore{}+\AgdaUnderscore{}}}\AgdaSpace{}%
\AgdaSymbol{:}\AgdaSpace{}%
\AgdaField{ℂ}\AgdaSpace{}%
\AgdaSymbol{→}\AgdaSpace{}%
\AgdaField{ℂ}\AgdaSpace{}%
\AgdaSymbol{→}\AgdaSpace{}%
\AgdaField{ℂ}\<%
\\
%
\>[6]\AgdaComment{--\ ...}\<%
\end{code}
%TC:ignore
\begin{code}[hide]%
%
\>[6]\AgdaOperator{\AgdaField{\AgdaUnderscore{}*\AgdaUnderscore{}}}\AgdaSpace{}%
\AgdaSymbol{:}\AgdaSpace{}%
\AgdaField{ℂ}\AgdaSpace{}%
\AgdaSymbol{→}\AgdaSpace{}%
\AgdaField{ℂ}\AgdaSpace{}%
\AgdaSymbol{→}\AgdaSpace{}%
\AgdaField{ℂ}\<%
\\
%
\>[6]\AgdaOperator{\AgdaField{\AgdaUnderscore{}-\AgdaUnderscore{}}}\AgdaSpace{}%
\AgdaSymbol{:}\AgdaSpace{}%
\AgdaField{ℂ}\AgdaSpace{}%
\AgdaSymbol{→}\AgdaSpace{}%
\AgdaField{ℂ}\AgdaSpace{}%
\AgdaSymbol{→}\AgdaSpace{}%
\AgdaField{ℂ}\<%
\\
%
\>[6]\AgdaOperator{\AgdaField{-\AgdaUnderscore{}}}%
\>[10]\AgdaSymbol{:}\AgdaSpace{}%
\AgdaField{ℂ}\AgdaSpace{}%
\AgdaSymbol{→}\AgdaSpace{}%
\AgdaField{ℂ}\<%
\\
%
\\[\AgdaEmptyExtraSkip]%
%
\>[6]\AgdaField{fromℝ}\AgdaSpace{}%
\AgdaSymbol{:}\AgdaSpace{}%
\AgdaField{ℝ}\AgdaSpace{}%
\AgdaSymbol{→}\AgdaSpace{}%
\AgdaField{ℂ}\<%
\\
%
\\[\AgdaEmptyExtraSkip]%
%
\>[6]\AgdaOperator{\AgdaField{e\textasciicircum{}i\AgdaUnderscore{}}}\AgdaSpace{}%
\AgdaSymbol{:}\AgdaSpace{}%
\AgdaField{ℝ}\AgdaSpace{}%
\AgdaSymbol{→}\AgdaSpace{}%
\AgdaField{ℂ}\<%
\\
%
\>[6]\AgdaField{ℂ-conjugate}\AgdaSpace{}%
\AgdaSymbol{:}\AgdaSpace{}%
\AgdaField{ℂ}\AgdaSpace{}%
\AgdaSymbol{→}\AgdaSpace{}%
\AgdaField{ℂ}\<%
\\
%
\\[\AgdaEmptyExtraSkip]%
%
\>[6]\AgdaComment{--+ω\ :\ ∀\ (N\ :\ ℕ)\ (k\ :\ ℕ)\ →\ ℂ}\<%
\\
%
\>[6]\AgdaComment{--\ Instance\ arguments\ seem\ pretty\ good\ https://agda.readthedocs.io/en/v2.5.4/language/instance-arguments.html}\<%
\\
%
\>[6]\AgdaComment{--\ ω\ really\ goes\ here}\<%
\\
%
\\[\AgdaEmptyExtraSkip]%
%
\>[4]\AgdaFunction{0ℂ}%
\>[8]\AgdaSymbol{:}\AgdaSpace{}%
\AgdaField{ℂ}\<%
\\
%
\>[4]\AgdaFunction{0ℂ}%
\>[8]\AgdaSymbol{=}\AgdaSpace{}%
\AgdaField{fromℝ}\AgdaSpace{}%
\AgdaSymbol{(}\AgdaFunction{0ℝ}\AgdaSymbol{)}\<%
\\
%
\>[4]\AgdaFunction{-1ℂ}\AgdaSpace{}%
\AgdaSymbol{:}\AgdaSpace{}%
\AgdaField{ℂ}\<%
\\
%
\>[4]\AgdaFunction{-1ℂ}\AgdaSpace{}%
\AgdaSymbol{=}\AgdaSpace{}%
\AgdaField{fromℝ}\AgdaSpace{}%
\AgdaSymbol{(}\AgdaFunction{-1ℝ}\AgdaSymbol{)}\<%
\\
%
\>[4]\AgdaFunction{1ℂ}%
\>[8]\AgdaSymbol{:}\AgdaSpace{}%
\AgdaField{ℂ}\<%
\\
%
\>[4]\AgdaFunction{1ℂ}%
\>[8]\AgdaSymbol{=}\AgdaSpace{}%
\AgdaField{fromℝ}\AgdaSpace{}%
\AgdaSymbol{(}\AgdaFunction{1ℝ}\AgdaSymbol{)}\<%
\\
%
\\[\AgdaEmptyExtraSkip]%
%
\>[4]\AgdaKeyword{open}\AgdaSpace{}%
\AgdaModule{AlgebraStructures}%
\>[28]\AgdaSymbol{\{}\AgdaArgument{A}\AgdaSpace{}%
\AgdaSymbol{=}\AgdaSpace{}%
\AgdaField{ℂ}\AgdaSymbol{\}}\AgdaSpace{}%
\AgdaOperator{\AgdaDatatype{\AgdaUnderscore{}≡\AgdaUnderscore{}}}\<%
\\
%
\>[4]\AgdaKeyword{open}\AgdaSpace{}%
\AgdaModule{AlgebraDefinitions}\AgdaSpace{}%
\AgdaSymbol{\{}\AgdaArgument{A}\AgdaSpace{}%
\AgdaSymbol{=}\AgdaSpace{}%
\AgdaField{ℂ}\AgdaSymbol{\}}\AgdaSpace{}%
\AgdaOperator{\AgdaDatatype{\AgdaUnderscore{}≡\AgdaUnderscore{}}}\<%
\\
\>[0]\<%
\\
%
\>[4]\AgdaKeyword{field}\<%
\end{code}
%TC:endignore
Addition, multiplication and negation must be proven to form a commutative ring,
meaning that a set of properties, such as multiplication distributes over addition
must hold. \cite{CommRingTheory}
\begin{code}%
\>[4][@{}l@{\AgdaIndent{1}}]%
\>[6]\AgdaComment{--\ ...}\<%
\\
%
\>[6]\AgdaField{+-*-isCommutativeRing}\AgdaSpace{}%
\AgdaSymbol{:}\AgdaSpace{}%
\AgdaRecord{IsCommutativeRing}\AgdaSpace{}%
\AgdaOperator{\AgdaField{\AgdaUnderscore{}+\AgdaUnderscore{}}}\AgdaSpace{}%
\AgdaOperator{\AgdaField{\AgdaUnderscore{}*\AgdaUnderscore{}}}\AgdaSpace{}%
\AgdaOperator{\AgdaField{-\AgdaUnderscore{}}}\AgdaSpace{}%
\AgdaFunction{0ℂ}\AgdaSpace{}%
\AgdaFunction{1ℂ}\<%
\\
%
\>[6]\AgdaComment{--\ ...}\<%
\end{code}
\paragraph{Roots of unity}\label{para:roots_of_unity} as described for Complex numbers in Equation 
\ref{eq:ComplexRootsOfUnity}, must be defined for some non-zero divisor $N$ 
and some power $K$, along with some properties on them.
To ensure that the divisor $N$ is never zero, a \AF{NonZero} proof argument is 
required on $N$, guaranteeing division by zero to be impossible.
This \AF{NonZero} property is an instance argument, allowing an instance 
resolution algorithm\cite{InstanceArgs}
to perform automatic resolution on it, simplifying further proofs.
\begin{code}%
%
\>[6]\AgdaComment{--\ ...}\<%
\\
%
\>[6]\AgdaField{-ω}\AgdaSpace{}%
\AgdaSymbol{:}\AgdaSpace{}%
\AgdaSymbol{(}\AgdaBound{N}\AgdaSpace{}%
\AgdaSymbol{:}\AgdaSpace{}%
\AgdaDatatype{ℕ}\AgdaSymbol{)}\AgdaSpace{}%
\AgdaSymbol{→}\AgdaSpace{}%
\AgdaSymbol{.\{\{}\AgdaSpace{}%
\AgdaBound{nonZero-n}\AgdaSpace{}%
\AgdaSymbol{:}\AgdaSpace{}%
\AgdaRecord{NonZero}\AgdaSpace{}%
\AgdaBound{N}\AgdaSpace{}%
\AgdaSymbol{\}\}}\AgdaSpace{}%
\AgdaSymbol{→}\AgdaSpace{}%
\AgdaSymbol{(}\AgdaBound{k}\AgdaSpace{}%
\AgdaSymbol{:}\AgdaSpace{}%
\AgdaDatatype{ℕ}\AgdaSymbol{)}\AgdaSpace{}%
\AgdaSymbol{→}\AgdaSpace{}%
\AgdaField{ℂ}\<%
\\
%
\>[6]\AgdaField{ω-N-0}%
\>[17]\AgdaSymbol{:}\AgdaSpace{}%
\AgdaField{-ω}\AgdaSpace{}%
\AgdaGeneralizable{N}\AgdaSpace{}%
\AgdaNumber{0}%
\>[43]\AgdaOperator{\AgdaDatatype{≡}}\AgdaSpace{}%
\AgdaFunction{1ℂ}\<%
\\
%
\>[6]\AgdaField{ω-N-mN}%
\>[17]\AgdaSymbol{:}\AgdaSpace{}%
\AgdaField{-ω}\AgdaSpace{}%
\AgdaGeneralizable{N}\AgdaSpace{}%
\AgdaSymbol{(}\AgdaGeneralizable{N}\AgdaSpace{}%
\AgdaOperator{\AgdaPrimitive{*ₙ}}\AgdaSpace{}%
\AgdaGeneralizable{m}\AgdaSymbol{)}%
\>[43]\AgdaOperator{\AgdaDatatype{≡}}\AgdaSpace{}%
\AgdaFunction{1ℂ}\<%
\\
%
\>[6]\AgdaField{ω-r₁x-r₁y}%
\>[17]\AgdaSymbol{:}\AgdaSpace{}%
\AgdaField{-ω}\AgdaSpace{}%
\AgdaSymbol{(}\AgdaGeneralizable{r₁}\AgdaSpace{}%
\AgdaOperator{\AgdaPrimitive{*ₙ}}\AgdaSpace{}%
\AgdaGeneralizable{x}\AgdaSymbol{)}\AgdaSpace{}%
\AgdaSymbol{(}\AgdaGeneralizable{r₁}\AgdaSpace{}%
\AgdaOperator{\AgdaPrimitive{*ₙ}}\AgdaSpace{}%
\AgdaGeneralizable{y}\AgdaSymbol{)}%
\>[43]\AgdaOperator{\AgdaDatatype{≡}}\AgdaSpace{}%
\AgdaField{-ω}\AgdaSpace{}%
\AgdaGeneralizable{x}\AgdaSpace{}%
\AgdaGeneralizable{y}\<%
\\
%
\>[6]\AgdaField{ω-N-k₀+k₁}%
\>[17]\AgdaSymbol{:}\AgdaSpace{}%
\AgdaField{-ω}\AgdaSpace{}%
\AgdaGeneralizable{N}\AgdaSpace{}%
\AgdaSymbol{(}\AgdaGeneralizable{k₀}\AgdaSpace{}%
\AgdaOperator{\AgdaPrimitive{+ₙ}}\AgdaSpace{}%
\AgdaGeneralizable{k₁}\AgdaSymbol{)}%
\>[43]\AgdaOperator{\AgdaDatatype{≡}}\AgdaSpace{}%
\AgdaSymbol{(}\AgdaField{-ω}\AgdaSpace{}%
\AgdaGeneralizable{N}\AgdaSpace{}%
\AgdaGeneralizable{k₀}\AgdaSymbol{)}\AgdaSpace{}%
\AgdaOperator{\AgdaField{*}}\AgdaSpace{}%
\AgdaSymbol{(}\AgdaField{-ω}\AgdaSpace{}%
\AgdaGeneralizable{N}\AgdaSpace{}%
\AgdaGeneralizable{k₁}\AgdaSymbol{)}\<%
\end{code}
\end{AgdaMultiCode}


\subsection{Tensors}
In Equations \ref{eq:DFT_Definition} and \ref{eq:FFTDefinitionFromDFT}, the DFT 
and FFT are both defined for any input vector $x$ of length $N$ and length 
$r_1\times r_2$ respectively. 
This implies that it would be possible to represent the input structure for both 
the DFT and the FFT in vector form, possibly using the Agda standard libraries functional
vector definition, \verb|Data.Vec.Functionals|.

Although this structure is ideal for the DFT, the FFTs relies on index splitting,
as described in Equation \ref{eq:IndexManipulation}, to decompose the input vector
into $r₁$ parts.
For vectors this would require low level index manipulation, for a single layer 
of splitting, this is not unreasonable, but can still complicate any definitions.
For multiple layers however, where the input is split into $n$ factors, this quickly
becomes complex as the multipliers and split position for each factor must be carried
through.
This would make an kind of reasoning on the FFT, as well as generalisation,
where the FFT is called iteratively, difficult as both would be
pulled down to require the same low level of index manipulation.

The need for this low level manipulation can be removed, by creating some
definition for shaped tensors, and allowing the FFT to 
accept these tensors as inputs.
These shaped tensors can also be considered as Multi-dimensional arrays.
As well as removing the need these low level manipulations, using this definition 
will also abstract the splitting of the input vector out of the FFT making any    % This may be better discussed in the FFT section...
definition radix independent.

\begin{AgdaAlign}
%TC:ignore
\begin{code}[hide]%
\>[0]\AgdaKeyword{module}\AgdaSpace{}%
\AgdaModule{Matrix}\AgdaSpace{}%
\AgdaKeyword{where}\<%
\\
\>[0][@{}l@{\AgdaIndent{0}}]%
\>[2]\AgdaKeyword{open}\AgdaSpace{}%
\AgdaKeyword{import}\AgdaSpace{}%
\AgdaModule{Data.Nat}\AgdaSpace{}%
\AgdaKeyword{using}\AgdaSpace{}%
\AgdaSymbol{(}\AgdaDatatype{ℕ}\AgdaSymbol{;}\AgdaSpace{}%
\AgdaInductiveConstructor{suc}\AgdaSymbol{;}\AgdaSpace{}%
\AgdaInductiveConstructor{zero}\AgdaSymbol{;}\AgdaSpace{}%
\AgdaRecord{NonZero}\AgdaSymbol{;}\AgdaSpace{}%
\AgdaOperator{\AgdaPrimitive{\AgdaUnderscore{}+\AgdaUnderscore{}}}\AgdaSymbol{;}\AgdaSpace{}%
\AgdaOperator{\AgdaPrimitive{\AgdaUnderscore{}*\AgdaUnderscore{}}}\AgdaSymbol{)}\<%
\\
%
\>[2]\AgdaKeyword{open}\AgdaSpace{}%
\AgdaKeyword{import}\AgdaSpace{}%
\AgdaModule{Data.Fin}\AgdaSpace{}%
\AgdaSymbol{as}\AgdaSpace{}%
\AgdaModule{F}\AgdaSpace{}%
\AgdaKeyword{using}\AgdaSpace{}%
\AgdaSymbol{(}\AgdaDatatype{Fin}\AgdaSymbol{;}\AgdaSpace{}%
\AgdaFunction{join}\AgdaSymbol{)}\AgdaSpace{}%
\AgdaKeyword{renaming}\AgdaSpace{}%
\AgdaSymbol{(}\AgdaInductiveConstructor{zero}\AgdaSpace{}%
\AgdaSymbol{to}\AgdaSpace{}%
\AgdaInductiveConstructor{fzero}\AgdaSymbol{;}\AgdaSpace{}%
\AgdaInductiveConstructor{suc}\AgdaSpace{}%
\AgdaSymbol{to}\AgdaSpace{}%
\AgdaInductiveConstructor{fsuc}\AgdaSymbol{)}\<%
\\
%
\>[2]\AgdaKeyword{open}\AgdaSpace{}%
\AgdaKeyword{import}\AgdaSpace{}%
\AgdaModule{Data.Product.Base}\AgdaSpace{}%
\AgdaKeyword{using}\AgdaSpace{}%
\AgdaSymbol{(}\AgdaOperator{\AgdaFunction{\AgdaUnderscore{}×\AgdaUnderscore{}}}\AgdaSymbol{)}\AgdaSpace{}%
\AgdaKeyword{renaming}\AgdaSpace{}%
\AgdaSymbol{(}\AgdaSpace{}%
\AgdaOperator{\AgdaInductiveConstructor{\AgdaUnderscore{},\AgdaUnderscore{}}}\AgdaSpace{}%
\AgdaSymbol{to}\AgdaSpace{}%
\AgdaOperator{\AgdaInductiveConstructor{⟨\AgdaUnderscore{},\AgdaUnderscore{}⟩}}\AgdaSymbol{)}\<%
\\
%
\>[2]\AgdaKeyword{open}\AgdaSpace{}%
\AgdaKeyword{import}\AgdaSpace{}%
\AgdaModule{Data.Sum.Base}\AgdaSpace{}%
\AgdaKeyword{using}\AgdaSpace{}%
\AgdaSymbol{(}\AgdaInductiveConstructor{inj₁}\AgdaSymbol{;}\AgdaSpace{}%
\AgdaInductiveConstructor{inj₂}\AgdaSymbol{)}\<%
\\
%
\\[\AgdaEmptyExtraSkip]%
%
\>[2]\AgdaKeyword{private}\<%
\\
\>[2][@{}l@{\AgdaIndent{0}}]%
\>[4]\AgdaKeyword{variable}\<%
\\
\>[4][@{}l@{\AgdaIndent{0}}]%
\>[6]\AgdaGeneralizable{n}\AgdaSpace{}%
\AgdaGeneralizable{m}\AgdaSpace{}%
\AgdaSymbol{:}\AgdaSpace{}%
\AgdaDatatype{ℕ}\<%
\\
%
\>[6]\AgdaGeneralizable{X}\AgdaSpace{}%
\AgdaGeneralizable{Y}\AgdaSpace{}%
\AgdaGeneralizable{Z}\AgdaSpace{}%
\AgdaSymbol{:}\AgdaSpace{}%
\AgdaPrimitive{Set}\<%
\end{code}
%TC:endignore
The shape of any given tensor can be described as a full binary tree of natural 
numbers.
Each leaf, \AF{ι n}, is one such natural number, one leaf 
can be considered to add one dimension to the overall shape. 
Each parent note, \AF{s ⊗ p}, joins two subtrees.
A given shape tree encodes the split of $N$ into $m$ many multipliers.

\begin{code}%
%
\>[2]\AgdaKeyword{data}\AgdaSpace{}%
\AgdaDatatype{Shape}\AgdaSpace{}%
\AgdaSymbol{:}\AgdaSpace{}%
\AgdaPrimitive{Set}\AgdaSpace{}%
\AgdaKeyword{where}\<%
\\
\>[2][@{}l@{\AgdaIndent{0}}]%
\>[4]\AgdaInductiveConstructor{ι}%
\>[8]\AgdaSymbol{:}\AgdaSpace{}%
\AgdaDatatype{ℕ}\AgdaSpace{}%
\AgdaSymbol{→}\AgdaSpace{}%
\AgdaDatatype{Shape}\<%
\\
%
\>[4]\AgdaOperator{\AgdaInductiveConstructor{\AgdaUnderscore{}⊗\AgdaUnderscore{}}}\AgdaSpace{}%
\AgdaSymbol{:}\AgdaSpace{}%
\AgdaDatatype{Shape}\AgdaSpace{}%
\AgdaSymbol{→}\AgdaSpace{}%
\AgdaDatatype{Shape}\AgdaSpace{}%
\AgdaSymbol{→}\AgdaSpace{}%
\AgdaDatatype{Shape}\<%
\end{code}

Defining shapes as trees in place of lists allows for more information to be 
encoded about the structure of the shape. 
This data loss can be identified by converting the below tensor shapes into their
list forms, both of which are \AF{s :: p :: r :: q :: []}.
% For the FFT, this additional information should allow for the structure of parallelism 
% to be defined by the shape of the input tensor for a parallelised implementation.

%TC:ignore
\begin{code}[hide]%
%
\>[2]\AgdaKeyword{private}\<%
\\
\>[2][@{}l@{\AgdaIndent{0}}]%
\>[4]\AgdaKeyword{variable}\<%
\\
\>[4][@{}l@{\AgdaIndent{0}}]%
\>[6]\AgdaGeneralizable{s}\AgdaSpace{}%
\AgdaGeneralizable{p}\AgdaSpace{}%
\AgdaSymbol{:}\AgdaSpace{}%
\AgdaDatatype{Shape}\<%
\\
%
\>[2]\AgdaFunction{tmp}\AgdaSpace{}%
\AgdaSymbol{:}\AgdaSpace{}%
\AgdaDatatype{Shape}\AgdaSpace{}%
\AgdaSymbol{→}\AgdaSpace{}%
\AgdaDatatype{Shape}\AgdaSpace{}%
\AgdaSymbol{→}\AgdaSpace{}%
\AgdaDatatype{Shape}\AgdaSpace{}%
\AgdaSymbol{→}\AgdaSpace{}%
\AgdaDatatype{Shape}\AgdaSpace{}%
\AgdaSymbol{→}\AgdaSpace{}%
\AgdaDatatype{Shape}\<%
\\
%
\>[2]\AgdaFunction{tmp}\AgdaSpace{}%
\AgdaBound{s}\AgdaSpace{}%
\AgdaBound{p}\AgdaSpace{}%
\AgdaBound{r}\AgdaSpace{}%
\AgdaBound{q}\AgdaSpace{}%
\AgdaSymbol{=}\AgdaSpace{}%
\AgdaKeyword{let}\<%
\end{code}
%TC:endignore
\begin{code}%
\>[2][@{}l@{\AgdaIndent{1}}]%
\>[4]\AgdaBound{s₁}\AgdaSpace{}%
\AgdaSymbol{=}\AgdaSpace{}%
\AgdaSymbol{(}%
\>[12]\AgdaBound{s}%
\>[15]\AgdaOperator{\AgdaInductiveConstructor{⊗}}%
\>[18]\AgdaBound{p}\AgdaSymbol{)}\AgdaSpace{}%
\AgdaOperator{\AgdaInductiveConstructor{⊗}}\AgdaSpace{}%
\AgdaSymbol{(}\AgdaBound{r}\AgdaSpace{}%
\AgdaOperator{\AgdaInductiveConstructor{⊗}}\AgdaSpace{}%
\AgdaBound{q}\AgdaSymbol{)}\<%
\\
%
\>[4]\AgdaBound{s₂}\AgdaSpace{}%
\AgdaSymbol{=}%
\>[12]\AgdaBound{s}%
\>[15]\AgdaOperator{\AgdaInductiveConstructor{⊗}}\AgdaSpace{}%
\AgdaSymbol{(}\AgdaBound{p}%
\>[21]\AgdaOperator{\AgdaInductiveConstructor{⊗}}\AgdaSpace{}%
\AgdaSymbol{(}\AgdaBound{r}\AgdaSpace{}%
\AgdaOperator{\AgdaInductiveConstructor{⊗}}\AgdaSpace{}%
\AgdaBound{q}\AgdaSymbol{))}\<%
\end{code}
%TC:ignore
\begin{code}[hide]%
%
\>[4]\AgdaKeyword{in}\AgdaSpace{}%
\AgdaInductiveConstructor{ι}\AgdaSpace{}%
\AgdaNumber{4}\<%
\end{code}
%TC:endignore

Array indices can then be inductively defined as a dependant type on Shapes.
This definition takes the same form as that of shapes and defines the position 
of a non-leaf nodes as being constructed by the positions of its two children 
nodes, while leaf nodes are bound by the length of that leaf.
This binding on the length of the leaf, allows the type checker to require
evidence that a positions index is not greater than the length, removing the possibility
for runtime out of bounds errors.


\begin{code}%
%
\>[2]\AgdaKeyword{data}\AgdaSpace{}%
\AgdaDatatype{Position}\AgdaSpace{}%
\AgdaSymbol{:}\AgdaSpace{}%
\AgdaDatatype{Shape}\AgdaSpace{}%
\AgdaSymbol{→}\AgdaSpace{}%
\AgdaPrimitive{Set}\AgdaSpace{}%
\AgdaKeyword{where}\<%
\\
\>[2][@{}l@{\AgdaIndent{0}}]%
\>[4]\AgdaInductiveConstructor{ι}%
\>[8]\AgdaSymbol{:}\AgdaSpace{}%
\AgdaDatatype{Fin}\AgdaSpace{}%
\AgdaGeneralizable{n}\AgdaSpace{}%
\AgdaSymbol{→}\AgdaSpace{}%
\AgdaDatatype{Position}\AgdaSpace{}%
\AgdaSymbol{(}\AgdaInductiveConstructor{ι}\AgdaSpace{}%
\AgdaGeneralizable{n}\AgdaSymbol{)}\<%
\\
%
\>[4]\AgdaOperator{\AgdaInductiveConstructor{\AgdaUnderscore{}⊗\AgdaUnderscore{}}}\AgdaSpace{}%
\AgdaSymbol{:}\AgdaSpace{}%
\AgdaDatatype{Position}\AgdaSpace{}%
\AgdaGeneralizable{s}\AgdaSpace{}%
\AgdaSymbol{→}\AgdaSpace{}%
\AgdaDatatype{Position}\AgdaSpace{}%
\AgdaGeneralizable{p}\AgdaSpace{}%
\AgdaSymbol{→}\AgdaSpace{}%
\AgdaDatatype{Position}\AgdaSpace{}%
\AgdaSymbol{(}\AgdaGeneralizable{s}\AgdaSpace{}%
\AgdaOperator{\AgdaInductiveConstructor{⊗}}\AgdaSpace{}%
\AgdaGeneralizable{p}\AgdaSymbol{)}\<%
\end{code}

\AF{Position} can then be used to define the tensor data encoding, such that
tensors form indexed types
accepting a position and returning the value at that position.

\begin{code}%
%
\>[2]\AgdaFunction{Ar}\AgdaSpace{}%
\AgdaSymbol{:}\AgdaSpace{}%
\AgdaDatatype{Shape}\AgdaSpace{}%
\AgdaSymbol{→}\AgdaSpace{}%
\AgdaPrimitive{Set}\AgdaSpace{}%
\AgdaSymbol{→}\AgdaSpace{}%
\AgdaPrimitive{Set}\<%
\\
%
\>[2]\AgdaFunction{Ar}\AgdaSpace{}%
\AgdaBound{s}\AgdaSpace{}%
\AgdaBound{X}\AgdaSpace{}%
\AgdaSymbol{=}\AgdaSpace{}%
\AgdaDatatype{Position}\AgdaSpace{}%
\AgdaBound{s}\AgdaSpace{}%
\AgdaSymbol{→}\AgdaSpace{}%
\AgdaBound{X}\<%
\end{code}
This means any given tensor of \AF{Shape} \AF{s} and type \AF{X} accepts a
\AF{Position} of shape \AF{s} and returns a value of type \AF{X}.
This is a similar definition to that used in \cite{BlockedSinkarovs}, and
provides a basis on which tensors can be discussed
\end{AgdaAlign}
\subsubsection{Tensor length}

Given the shape of an array, we can compute the number of elements it contains
by multiplying all components of the shape tree

\begin{code}%
%
\>[2]\AgdaFunction{\#}\AgdaSpace{}%
\AgdaSymbol{:}\AgdaSpace{}%
\AgdaDatatype{Shape}\AgdaSpace{}%
\AgdaSymbol{→}\AgdaSpace{}%
\AgdaDatatype{ℕ}\<%
\\
%
\>[2]\AgdaFunction{\#}\AgdaSpace{}%
\AgdaSymbol{(}\AgdaInductiveConstructor{ι}\AgdaSpace{}%
\AgdaBound{x}\AgdaSymbol{)}\AgdaSpace{}%
\AgdaSymbol{=}\AgdaSpace{}%
\AgdaBound{x}\<%
\\
%
\>[2]\AgdaFunction{\#}\AgdaSpace{}%
\AgdaSymbol{(}\AgdaBound{s}\AgdaSpace{}%
\AgdaOperator{\AgdaInductiveConstructor{⊗}}\AgdaSpace{}%
\AgdaBound{s₁}\AgdaSymbol{)}\AgdaSpace{}%
\AgdaSymbol{=}\AgdaSpace{}%
\AgdaFunction{\#}\AgdaSpace{}%
\AgdaBound{s}\AgdaSpace{}%
\AgdaOperator{\AgdaPrimitive{*}}\AgdaSpace{}%
\AgdaFunction{\#}\AgdaSpace{}%
\AgdaBound{s₁}\<%
\end{code}

When computing the DFT and FFT, the number of elements in a given tensor is used 
to determine the base, or principal, root of unity.
This base, however, cannot be zero as to avoid division be zero.
Therefore, \AF{ω} requires that a non zero proof argument to be provided.\ref{para:roots_of_unity}.
This can be easily achieved by restricting the DFT and FFT to operate only on
tensors the number of elements is greater than zero,
This means that any implementation of the DFT and FFT must be provided, or generate,
a proof argument that no leaf is of zero length.
For the simplicity of this paper we use the notation $Ar∔$ % Judge me how you will for using this symbol here and then changing what it really is in newunicodechar... its cursed, its horrible... but it works
to indicate that a tensor is provided such a proof argument.
This notation cannot be used in the final implementation where the non zero property
must be provided explicitly, however, this obfuscates the key points and so this improved
notation is used here.


\subsubsection{One-dimensional tensors}
Given the definition of tensors, we can begin by defining some basic operations 
which can be used upon them.
The first operations we shall define will operate exclusively on single dimensional
tensors, which are often referred to for succinctness as vectors.

\paragraph{Head and Tail} Head and tail operations can be defined to 
allow for the deconstruction of any tensor of shape \AF{ι (suc n)}. 
\AF{head₁} returns the first element of the tensor, while
\AF{tail₁} returns all following elements in a tensor of shape \AF{ι n}.
These operations allow for recursion over vectors to be defined.

\begin{code}%
%
\>[2]\AgdaFunction{head₁}\AgdaSpace{}%
\AgdaSymbol{:}\AgdaSpace{}%
\AgdaFunction{Ar}\AgdaSpace{}%
\AgdaSymbol{(}\AgdaInductiveConstructor{ι}\AgdaSpace{}%
\AgdaSymbol{(}\AgdaInductiveConstructor{suc}\AgdaSpace{}%
\AgdaGeneralizable{n}\AgdaSymbol{))}\AgdaSpace{}%
\AgdaGeneralizable{X}\AgdaSpace{}%
\AgdaSymbol{→}\AgdaSpace{}%
\AgdaGeneralizable{X}\<%
\\
%
\>[2]\AgdaFunction{head₁}\AgdaSpace{}%
\AgdaBound{ar}\AgdaSpace{}%
\AgdaSymbol{=}\AgdaSpace{}%
\AgdaBound{ar}\AgdaSpace{}%
\AgdaSymbol{(}\AgdaInductiveConstructor{ι}\AgdaSpace{}%
\AgdaInductiveConstructor{fzero}\AgdaSymbol{)}\<%
\\
%
\\[\AgdaEmptyExtraSkip]%
%
\>[2]\AgdaFunction{tail₁}\AgdaSpace{}%
\AgdaSymbol{:}\AgdaSpace{}%
\AgdaFunction{Ar}\AgdaSpace{}%
\AgdaSymbol{(}\AgdaInductiveConstructor{ι}\AgdaSpace{}%
\AgdaSymbol{(}\AgdaInductiveConstructor{suc}\AgdaSpace{}%
\AgdaGeneralizable{n}\AgdaSymbol{))}\AgdaSpace{}%
\AgdaGeneralizable{X}\AgdaSpace{}%
\AgdaSymbol{→}\AgdaSpace{}%
\AgdaFunction{Ar}\AgdaSpace{}%
\AgdaSymbol{(}\AgdaInductiveConstructor{ι}\AgdaSpace{}%
\AgdaGeneralizable{n}\AgdaSymbol{)}\AgdaSpace{}%
\AgdaGeneralizable{X}\<%
\\
%
\>[2]\AgdaFunction{tail₁}\AgdaSpace{}%
\AgdaBound{ar}\AgdaSpace{}%
\AgdaSymbol{(}\AgdaInductiveConstructor{ι}\AgdaSpace{}%
\AgdaBound{x}\AgdaSymbol{)}\AgdaSpace{}%
\AgdaSymbol{=}\AgdaSpace{}%
\AgdaBound{ar}\AgdaSpace{}%
\AgdaSymbol{(}\AgdaInductiveConstructor{ι}\AgdaSpace{}%
\AgdaSymbol{(}\AgdaInductiveConstructor{fsuc}\AgdaSpace{}%
\AgdaBound{x}\AgdaSymbol{))}\<%
\end{code}

% This wouldn't be good for a paper, but I feel like its useful to observe when
% describing for the thesis
One feature of Agda used regularly is seen here, pattern matching.
This is a feature found in most functional languages that use algebraic types,
such as Haskell, and allows for the breaking down of some types of input 
fields to the types they are built on. 
In the above example \AF{ι x} is of type \AF{Position (suc n)}, 
which is deconstructed to expose \AF{x} of type \AF{Fin (suc n)}.

\paragraph{Sum} From Equation \ref{eq:DFT_Definition}, it can be seen that
an operation to sum all elements in a given array is required.
By defining this operation generally, over any binary operation and neutral 
element, we are able to represent any fold-like operations including the sum
operation we require.
This definition is can be instantiated for any commutative monoid 
\AF{(X, \_⋆\_, ε)} where 
\begin{itemize}
  \item \AF{X} is a set
  \item \AF{\_⋆\_} is some operation \AF{X → X → X}, such that 
  \begin{itemize}
      \item \AF{x ⋆ y ≡ y ⋆ x}
      \item \AF{(x ⋆ y) ⋆ z ≡ x ⋆ (y ⋆ z)}
  \end{itemize}
  \item \AF{ε} is an identity element in \AF{X} such that \AF{ε ⋆ x ≡ x}
\end{itemize}
With the above definition, sum can be defined as below.
\begin{AgdaMultiCode}
\begin{code}%
\>[0]\AgdaKeyword{module}\AgdaSpace{}%
\AgdaModule{Sum}\<%
\\
\>[0][@{}l@{\AgdaIndent{0}}]%
\>[4]\AgdaSymbol{\{}\AgdaBound{A}\AgdaSpace{}%
\AgdaSymbol{:}\AgdaSpace{}%
\AgdaPrimitive{Set}\AgdaSymbol{\}}\<%
\\
%
\>[4]\AgdaSymbol{(}\AgdaOperator{\AgdaBound{\AgdaUnderscore{}⋆\AgdaUnderscore{}}}\AgdaSpace{}%
\AgdaSymbol{:}\AgdaSpace{}%
\AgdaFunction{Op₂}\AgdaSpace{}%
\AgdaBound{A}\AgdaSymbol{)}\<%
\\
%
\>[4]\AgdaSymbol{(}\AgdaBound{ε}\AgdaSpace{}%
\AgdaSymbol{:}\AgdaSpace{}%
\AgdaBound{A}\AgdaSymbol{)}\<%
\\
%
\>[4]\AgdaSymbol{(}\AgdaBound{isCommutativeMonoid}\AgdaSpace{}%
\AgdaSymbol{:}\AgdaSpace{}%
\AgdaRecord{IsCommutativeMonoid}\AgdaSpace{}%
\AgdaSymbol{\{}\AgdaArgument{A}\AgdaSpace{}%
\AgdaSymbol{=}\AgdaSpace{}%
\AgdaBound{A}\AgdaSymbol{\}}\AgdaSpace{}%
\AgdaOperator{\AgdaDatatype{\AgdaUnderscore{}≡\AgdaUnderscore{}}}\AgdaSpace{}%
\AgdaOperator{\AgdaBound{\AgdaUnderscore{}⋆\AgdaUnderscore{}}}\AgdaSpace{}%
\AgdaBound{ε}\AgdaSymbol{)}\<%
\\
\>[0][@{}l@{\AgdaIndent{0}}]%
\>[2]\AgdaKeyword{where}\<%
\end{code}
%TC:ignore
\begin{code}[hide]%
%
\>[2]\AgdaKeyword{open}\AgdaSpace{}%
\AgdaKeyword{import}\AgdaSpace{}%
\AgdaModule{Data.Product.Base}\AgdaSpace{}%
\AgdaKeyword{using}\AgdaSpace{}%
\AgdaSymbol{(}\AgdaField{proj₁}\AgdaSymbol{;}\AgdaSpace{}%
\AgdaField{proj₂}\AgdaSymbol{)}\<%
\\
%
\\[\AgdaEmptyExtraSkip]%
%
\>[2]\AgdaKeyword{open}\AgdaSpace{}%
\AgdaKeyword{import}\AgdaSpace{}%
\AgdaModule{Data.Nat.Base}\AgdaSpace{}%
\AgdaKeyword{using}\AgdaSpace{}%
\AgdaSymbol{(}\AgdaDatatype{ℕ}\AgdaSymbol{;}\AgdaSpace{}%
\AgdaInductiveConstructor{zero}\AgdaSymbol{;}\AgdaSpace{}%
\AgdaInductiveConstructor{suc}\AgdaSymbol{;}\AgdaSpace{}%
\AgdaOperator{\AgdaPrimitive{\AgdaUnderscore{}+\AgdaUnderscore{}}}\AgdaSymbol{;}\AgdaSpace{}%
\AgdaOperator{\AgdaPrimitive{\AgdaUnderscore{}*\AgdaUnderscore{}}}\AgdaSymbol{)}\<%
\\
%
\>[2]\AgdaKeyword{open}\AgdaSpace{}%
\AgdaKeyword{import}\AgdaSpace{}%
\AgdaModule{Data.Nat.Properties}\AgdaSpace{}%
\AgdaKeyword{using}\AgdaSpace{}%
\AgdaSymbol{(}\AgdaFunction{*-zeroʳ}\AgdaSymbol{)}\<%
\\
%
\>[2]\AgdaKeyword{open}\AgdaSpace{}%
\AgdaKeyword{import}\AgdaSpace{}%
\AgdaModule{Data.Fin.Base}\AgdaSpace{}%
\AgdaKeyword{using}\AgdaSpace{}%
\AgdaSymbol{()}\AgdaSpace{}%
\AgdaKeyword{renaming}\AgdaSpace{}%
\AgdaSymbol{(}\AgdaInductiveConstructor{zero}\AgdaSpace{}%
\AgdaSymbol{to}\AgdaSpace{}%
\AgdaInductiveConstructor{fzero}\AgdaSymbol{;}\AgdaSpace{}%
\AgdaInductiveConstructor{suc}\AgdaSpace{}%
\AgdaSymbol{to}\AgdaSpace{}%
\AgdaInductiveConstructor{fsuc}\AgdaSymbol{)}\<%
\\
%
\\[\AgdaEmptyExtraSkip]%
%
\>[2]\AgdaKeyword{open}\AgdaSpace{}%
\AgdaKeyword{import}\AgdaSpace{}%
\AgdaModule{Matrix}\AgdaSpace{}%
\AgdaKeyword{using}\AgdaSpace{}%
\AgdaSymbol{(}\AgdaFunction{Ar}\AgdaSymbol{;}\AgdaSpace{}%
\AgdaDatatype{Position}\AgdaSymbol{;}\AgdaSpace{}%
\AgdaInductiveConstructor{ι}\AgdaSymbol{;}\AgdaSpace{}%
\AgdaOperator{\AgdaInductiveConstructor{\AgdaUnderscore{}⊗\AgdaUnderscore{}}}\AgdaSymbol{;}\AgdaSpace{}%
\AgdaFunction{head₁}\AgdaSymbol{;}\AgdaSpace{}%
\AgdaFunction{tail₁}\AgdaSymbol{;}\AgdaSpace{}%
\AgdaFunction{splitArₗ}\AgdaSymbol{;}\AgdaSpace{}%
\AgdaFunction{splitArᵣ}\AgdaSymbol{)}\<%
\\
%
\>[2]\AgdaKeyword{open}\AgdaSpace{}%
\AgdaKeyword{import}\AgdaSpace{}%
\AgdaModule{Matrix.Equality}\AgdaSpace{}%
\AgdaKeyword{using}\AgdaSpace{}%
\AgdaSymbol{(}\AgdaOperator{\AgdaFunction{\AgdaUnderscore{}≅\AgdaUnderscore{}}}\AgdaSymbol{;}\AgdaSpace{}%
\AgdaFunction{reduce-≅}\AgdaSymbol{)}\<%
\\
%
\>[2]\AgdaKeyword{open}\AgdaSpace{}%
\AgdaKeyword{import}\AgdaSpace{}%
\AgdaModule{Matrix.Properties}\AgdaSpace{}%
\AgdaKeyword{using}\AgdaSpace{}%
\AgdaSymbol{(}\AgdaFunction{tail₁-const}\AgdaSymbol{)}\<%
\\
%
\\[\AgdaEmptyExtraSkip]%
%
\>[2]\AgdaKeyword{open}\AgdaSpace{}%
\AgdaKeyword{import}\AgdaSpace{}%
\AgdaModule{Matrix.Reshape}\AgdaSpace{}%
\AgdaKeyword{using}\AgdaSpace{}%
\AgdaSymbol{(}\AgdaFunction{reshape}\AgdaSymbol{;}\AgdaSpace{}%
\AgdaFunction{reindex}\AgdaSymbol{;}\AgdaSpace{}%
\AgdaFunction{|s|≡|sᵗ|}\AgdaSymbol{;}\AgdaSpace{}%
\AgdaOperator{\AgdaFunction{\AgdaUnderscore{}⟨\AgdaUnderscore{}⟩}}\AgdaSymbol{;}\AgdaSpace{}%
\AgdaInductiveConstructor{split}\AgdaSymbol{;}\AgdaSpace{}%
\AgdaOperator{\AgdaInductiveConstructor{\AgdaUnderscore{}∙\AgdaUnderscore{}}}\AgdaSymbol{;}\AgdaSpace{}%
\AgdaInductiveConstructor{eq}\AgdaSymbol{)}\<%
\\
%
\\[\AgdaEmptyExtraSkip]%
%
\\[\AgdaEmptyExtraSkip]%
%
\>[2]\AgdaComment{-----------------------------------------}\<%
\\
%
\>[2]\AgdaComment{---\ Pull\ out\ properties\ of\ the\ monoid\ ---}\<%
\\
%
\>[2]\AgdaComment{-----------------------------------------}\<%
\\
%
\\[\AgdaEmptyExtraSkip]%
%
\>[2]\AgdaKeyword{open}\AgdaSpace{}%
\AgdaModule{AlgebraDefinitions}\AgdaSpace{}%
\AgdaSymbol{\{}\AgdaArgument{A}\AgdaSpace{}%
\AgdaSymbol{=}\AgdaSpace{}%
\AgdaBound{A}\AgdaSymbol{\}}\AgdaSpace{}%
\AgdaOperator{\AgdaDatatype{\AgdaUnderscore{}≡\AgdaUnderscore{}}}\<%
\\
%
\\[\AgdaEmptyExtraSkip]%
%
\>[2]\AgdaKeyword{open}\AgdaSpace{}%
\AgdaModule{IsCommutativeMonoid}\AgdaSpace{}%
\AgdaBound{isCommutativeMonoid}\AgdaSpace{}%
\AgdaKeyword{using}\AgdaSpace{}%
\AgdaSymbol{(}\AgdaFunction{identity}\AgdaSymbol{;}\AgdaSpace{}%
\AgdaFunction{assoc}\AgdaSymbol{;}\AgdaSpace{}%
\AgdaField{comm}\AgdaSymbol{)}\<%
\\
%
\\[\AgdaEmptyExtraSkip]%
%
\\[\AgdaEmptyExtraSkip]%
%
\>[2]\AgdaKeyword{private}\<%
\\
\>[2][@{}l@{\AgdaIndent{0}}]%
\>[4]\AgdaFunction{identityˡ}\AgdaSpace{}%
\AgdaSymbol{:}\AgdaSpace{}%
\AgdaFunction{LeftIdentity}\AgdaSpace{}%
\AgdaBound{ε}\AgdaSpace{}%
\AgdaOperator{\AgdaBound{\AgdaUnderscore{}⋆\AgdaUnderscore{}}}\<%
\\
%
\>[4]\AgdaFunction{identityˡ}\AgdaSpace{}%
\AgdaSymbol{=}\AgdaSpace{}%
\AgdaField{proj₁}\AgdaSpace{}%
\AgdaFunction{identity}\<%
\\
%
\\[\AgdaEmptyExtraSkip]%
%
\>[4]\AgdaFunction{identityʳ}\AgdaSpace{}%
\AgdaSymbol{:}\AgdaSpace{}%
\AgdaFunction{RightIdentity}\AgdaSpace{}%
\AgdaBound{ε}\AgdaSpace{}%
\AgdaOperator{\AgdaBound{\AgdaUnderscore{}⋆\AgdaUnderscore{}}}\<%
\\
%
\>[4]\AgdaFunction{identityʳ}\AgdaSpace{}%
\AgdaSymbol{=}\AgdaSpace{}%
\AgdaField{proj₂}\AgdaSpace{}%
\AgdaFunction{identity}\<%
\\
%
\\[\AgdaEmptyExtraSkip]%
%
\>[4]\AgdaKeyword{variable}\<%
\\
\>[4][@{}l@{\AgdaIndent{0}}]%
\>[6]\AgdaGeneralizable{n}\AgdaSpace{}%
\AgdaGeneralizable{m}\AgdaSpace{}%
\AgdaSymbol{:}\AgdaSpace{}%
\AgdaDatatype{ℕ}\<%
\\
%
\\[\AgdaEmptyExtraSkip]%
%
\>[2]\AgdaComment{----------------------}\<%
\\
%
\>[2]\AgdaComment{---\ Sum\ Definition\ ---}\<%
\\
%
\>[2]\AgdaComment{----------------------}\<%
\\
\>[0]\<%
\end{code}
%TC:endignore
\begin{code}%
\>[0][@{}l@{\AgdaIndent{1}}]%
\>[2]\AgdaFunction{sum}\AgdaSpace{}%
\AgdaSymbol{:}\AgdaSpace{}%
\AgdaSymbol{(}\AgdaBound{xs}\AgdaSpace{}%
\AgdaSymbol{:}\AgdaSpace{}%
\AgdaFunction{Ar}\AgdaSpace{}%
\AgdaSymbol{(}\AgdaInductiveConstructor{ι}\AgdaSpace{}%
\AgdaGeneralizable{n}\AgdaSymbol{)}\AgdaSpace{}%
\AgdaBound{A}\AgdaSymbol{)}\AgdaSpace{}%
\AgdaSymbol{→}\AgdaSpace{}%
\AgdaBound{A}\<%
\\
%
\>[2]\AgdaFunction{sum}\AgdaSpace{}%
\AgdaSymbol{\{}\AgdaInductiveConstructor{zero}\AgdaSymbol{\}}%
\>[15]\AgdaBound{xs}\AgdaSpace{}%
\AgdaSymbol{=}\AgdaSpace{}%
\AgdaBound{ε}\<%
\\
%
\>[2]\AgdaFunction{sum}\AgdaSpace{}%
\AgdaSymbol{\{}\AgdaInductiveConstructor{suc}\AgdaSpace{}%
\AgdaBound{n}\AgdaSymbol{\}}%
\>[15]\AgdaBound{xs}\AgdaSpace{}%
\AgdaSymbol{=}\AgdaSpace{}%
\AgdaSymbol{(}\AgdaFunction{head₁}\AgdaSpace{}%
\AgdaBound{xs}\AgdaSymbol{)}\AgdaSpace{}%
\AgdaOperator{\AgdaBound{⋆}}\AgdaSpace{}%
\AgdaSymbol{(}\AgdaFunction{sum}\AgdaSpace{}%
\AgdaOperator{\AgdaFunction{∘}}\AgdaSpace{}%
\AgdaFunction{tail₁}\AgdaSymbol{)}\AgdaSpace{}%
\AgdaBound{xs}\<%
\end{code}
\end{AgdaMultiCode}
For the DFT and FFT in this paper, this is instantiated over complex addition,
described as the monoid \AF{(ℂ, \_+\_, 0ℂ)}.
However, this definition allows for any fold-like operation to be defined for 
any instance of \AF{(X, \_⋆\_, ε)} meaning operations such as $\Pi$ can be 
instantiated with the same definition and general rules.
This is similar to how the DFT and FFT can be instantiated for any definition of
\AF{Cplx}.


%TC:ignore
\begin{code}[hide]%
\>[0]\AgdaKeyword{open}\AgdaSpace{}%
\AgdaKeyword{import}\AgdaSpace{}%
\AgdaModule{Complex}\AgdaSpace{}%
\AgdaKeyword{using}\AgdaSpace{}%
\AgdaSymbol{(}\AgdaRecord{Cplx}\AgdaSymbol{)}\<%
\\
\>[0]\AgdaKeyword{module}\AgdaSpace{}%
\AgdaModule{FFT}\AgdaSpace{}%
\AgdaSymbol{(}\AgdaBound{real}\AgdaSpace{}%
\AgdaSymbol{:}\AgdaSpace{}%
\AgdaRecord{Real}\AgdaSymbol{)}\AgdaSpace{}%
\AgdaSymbol{(}\AgdaBound{cplx}\AgdaSpace{}%
\AgdaSymbol{:}\AgdaSpace{}%
\AgdaRecord{Cplx}\AgdaSpace{}%
\AgdaBound{real}\AgdaSymbol{)}\AgdaSpace{}%
\AgdaKeyword{where}\<%
\\
\>[0][@{}l@{\AgdaIndent{0}}]%
\>[2]\AgdaKeyword{open}\AgdaSpace{}%
\AgdaModule{Cplx}\AgdaSpace{}%
\AgdaBound{cplx}\AgdaSpace{}%
\AgdaKeyword{using}\AgdaSpace{}%
\AgdaSymbol{(}\AgdaField{ℂ}\AgdaSymbol{;}\AgdaSpace{}%
\AgdaOperator{\AgdaField{\AgdaUnderscore{}*\AgdaUnderscore{}}}\AgdaSymbol{;}\AgdaSpace{}%
\AgdaField{-ω}\AgdaSymbol{;}\AgdaSpace{}%
\AgdaOperator{\AgdaField{e\textasciicircum{}i\AgdaUnderscore{}}}\AgdaSymbol{;}\AgdaSpace{}%
\AgdaOperator{\AgdaField{\AgdaUnderscore{}+\AgdaUnderscore{}}}\AgdaSymbol{;}\AgdaSpace{}%
\AgdaFunction{0ℂ}\AgdaSymbol{;}\AgdaSpace{}%
\AgdaField{+-*-isCommutativeRing}\AgdaSymbol{)}\<%
\\
%
\\[\AgdaEmptyExtraSkip]%
%
\>[2]\AgdaKeyword{open}\AgdaSpace{}%
\AgdaModule{AlgebraStructures}%
\>[26]\AgdaSymbol{\{}\AgdaArgument{A}\AgdaSpace{}%
\AgdaSymbol{=}\AgdaSpace{}%
\AgdaField{ℂ}\AgdaSymbol{\}}\AgdaSpace{}%
\AgdaOperator{\AgdaDatatype{\AgdaUnderscore{}≡\AgdaUnderscore{}}}\<%
\\
%
\>[2]\AgdaKeyword{open}\AgdaSpace{}%
\AgdaModule{IsCommutativeRing}\AgdaSpace{}%
\AgdaField{+-*-isCommutativeRing}\AgdaSpace{}%
\AgdaKeyword{using}\AgdaSpace{}%
\AgdaSymbol{(}\AgdaFunction{+-isCommutativeMonoid}\AgdaSymbol{)}\<%
\\
%
\\[\AgdaEmptyExtraSkip]%
%
\>[2]\AgdaKeyword{open}\AgdaSpace{}%
\AgdaKeyword{import}\AgdaSpace{}%
\AgdaModule{Data.Fin.Base}\AgdaSpace{}%
\AgdaKeyword{using}\AgdaSpace{}%
\AgdaSymbol{(}\AgdaDatatype{Fin}\AgdaSymbol{;}\AgdaSpace{}%
\AgdaFunction{toℕ}\AgdaSymbol{)}\AgdaSpace{}%
\AgdaKeyword{renaming}\AgdaSpace{}%
\AgdaSymbol{(}\AgdaInductiveConstructor{zero}\AgdaSpace{}%
\AgdaSymbol{to}\AgdaSpace{}%
\AgdaInductiveConstructor{fzero}\AgdaSymbol{;}\AgdaSpace{}%
\AgdaInductiveConstructor{suc}\AgdaSpace{}%
\AgdaSymbol{to}\AgdaSpace{}%
\AgdaInductiveConstructor{fsuc}\AgdaSymbol{)}\<%
\\
%
\>[2]\AgdaKeyword{open}\AgdaSpace{}%
\AgdaKeyword{import}\AgdaSpace{}%
\AgdaModule{Data.Nat.Base}\AgdaSpace{}%
\AgdaKeyword{using}\AgdaSpace{}%
\AgdaSymbol{(}\AgdaDatatype{ℕ}\AgdaSymbol{;}\AgdaSpace{}%
\AgdaInductiveConstructor{suc}\AgdaSymbol{;}\AgdaSpace{}%
\AgdaRecord{NonZero}\AgdaSymbol{)}\AgdaSpace{}%
\AgdaKeyword{renaming}\AgdaSpace{}%
\AgdaSymbol{(}\AgdaOperator{\AgdaPrimitive{\AgdaUnderscore{}+\AgdaUnderscore{}}}\AgdaSpace{}%
\AgdaSymbol{to}\AgdaSpace{}%
\AgdaOperator{\AgdaPrimitive{\AgdaUnderscore{}+ₙ\AgdaUnderscore{}}}\AgdaSymbol{;}\AgdaSpace{}%
\AgdaOperator{\AgdaPrimitive{\AgdaUnderscore{}*\AgdaUnderscore{}}}\AgdaSpace{}%
\AgdaSymbol{to}\AgdaSpace{}%
\AgdaOperator{\AgdaPrimitive{\AgdaUnderscore{}*ₙ\AgdaUnderscore{}}}\AgdaSymbol{)}\<%
\\
%
\>[2]\AgdaKeyword{open}\AgdaSpace{}%
\AgdaKeyword{import}\AgdaSpace{}%
\AgdaModule{Data.Nat.Properties}\AgdaSpace{}%
\AgdaKeyword{using}\AgdaSpace{}%
\AgdaSymbol{(}\AgdaFunction{nonZero?}\AgdaSymbol{)}\<%
\\
%
\>[2]\AgdaKeyword{open}\AgdaSpace{}%
\AgdaKeyword{import}\AgdaSpace{}%
\AgdaModule{Relation.Nullary}\<%
\\
%
\\[\AgdaEmptyExtraSkip]%
%
\>[2]\AgdaKeyword{open}\AgdaSpace{}%
\AgdaKeyword{import}\AgdaSpace{}%
\AgdaModule{Matrix}\AgdaSpace{}%
\AgdaKeyword{using}\AgdaSpace{}%
\AgdaSymbol{(}\AgdaFunction{Ar}\AgdaSymbol{;}\AgdaSpace{}%
\AgdaDatatype{Shape}\AgdaSymbol{;}\AgdaSpace{}%
\AgdaDatatype{Position}\AgdaSymbol{;}\AgdaSpace{}%
\AgdaInductiveConstructor{ι}\AgdaSymbol{;}\AgdaSpace{}%
\AgdaOperator{\AgdaInductiveConstructor{\AgdaUnderscore{}⊗\AgdaUnderscore{}}}\AgdaSymbol{;}\AgdaSpace{}%
\AgdaFunction{zipWith}\AgdaSymbol{;}\AgdaSpace{}%
\AgdaFunction{mapLeft}\AgdaSymbol{;}\AgdaSpace{}%
\AgdaFunction{length}\AgdaSymbol{)}\<%
\\
%
\>[2]\AgdaKeyword{open}\AgdaSpace{}%
\AgdaKeyword{import}\AgdaSpace{}%
\AgdaModule{Matrix.Sum}\AgdaSpace{}%
\AgdaOperator{\AgdaField{\AgdaUnderscore{}+\AgdaUnderscore{}}}\AgdaSpace{}%
\AgdaFunction{0ℂ}\AgdaSpace{}%
\AgdaFunction{+-isCommutativeMonoid}\AgdaSpace{}%
\AgdaKeyword{using}\AgdaSpace{}%
\AgdaSymbol{(}\AgdaFunction{sum}\AgdaSymbol{)}\<%
\\
%
\>[2]\AgdaKeyword{open}\AgdaSpace{}%
\AgdaKeyword{import}\AgdaSpace{}%
\AgdaModule{Matrix.Reshape}\AgdaSpace{}%
\AgdaKeyword{using}\AgdaSpace{}%
\AgdaSymbol{(}\AgdaFunction{recursive-transpose}\AgdaSymbol{;}\AgdaSpace{}%
\AgdaFunction{reshape}\AgdaSymbol{;}\AgdaSpace{}%
\AgdaInductiveConstructor{swap}\AgdaSymbol{;}\AgdaSpace{}%
\AgdaOperator{\AgdaFunction{\AgdaUnderscore{}⟨\AgdaUnderscore{}⟩}}\AgdaSymbol{;}\AgdaSpace{}%
\AgdaFunction{♯}\AgdaSymbol{;}\AgdaSpace{}%
\AgdaFunction{recursive-transposeᵣ}\AgdaSymbol{)}\<%
\\
%
\>[2]\AgdaKeyword{open}\AgdaSpace{}%
\AgdaKeyword{import}\AgdaSpace{}%
\AgdaModule{Matrix.NonZero}\AgdaSpace{}%
\AgdaKeyword{using}\AgdaSpace{}%
\AgdaSymbol{(}\AgdaDatatype{NonZeroₛ}\AgdaSymbol{;}\AgdaSpace{}%
\AgdaInductiveConstructor{ι}\AgdaSymbol{;}\AgdaSpace{}%
\AgdaOperator{\AgdaInductiveConstructor{\AgdaUnderscore{}⊗\AgdaUnderscore{}}}\AgdaSymbol{;}\AgdaSpace{}%
\AgdaFunction{nonZeroₛ-s⇒nonZero-s}\AgdaSymbol{;}\AgdaSpace{}%
\AgdaFunction{nonZeroDec}\AgdaSymbol{;}\AgdaSpace{}%
\AgdaFunction{nonZeroₛ-s⇒nonZeroₛ-sᵗ}\AgdaSymbol{)}\<%
\\
%
\\[\AgdaEmptyExtraSkip]%
%
\>[2]\AgdaKeyword{private}\<%
\\
\>[2][@{}l@{\AgdaIndent{0}}]%
\>[4]\AgdaKeyword{variable}\<%
\\
\>[4][@{}l@{\AgdaIndent{0}}]%
\>[6]\AgdaGeneralizable{N}\AgdaSpace{}%
\AgdaSymbol{:}\AgdaSpace{}%
\AgdaDatatype{ℕ}\<%
\\
%
\>[6]\AgdaGeneralizable{s}\AgdaSpace{}%
\AgdaGeneralizable{p}\AgdaSpace{}%
\AgdaSymbol{:}\AgdaSpace{}%
\AgdaDatatype{Shape}\<%
\\
%
\\[\AgdaEmptyExtraSkip]%
%
\>[2]\AgdaFunction{Ar∔}\AgdaSpace{}%
\AgdaSymbol{:}\AgdaSpace{}%
\AgdaDatatype{Shape}\AgdaSpace{}%
\AgdaSymbol{→}\AgdaSpace{}%
\AgdaPrimitive{Set}\AgdaSpace{}%
\AgdaSymbol{→}\AgdaSpace{}%
\AgdaPrimitive{Set}\<%
\\
%
\>[2]\AgdaFunction{Ar∔}\AgdaSpace{}%
\AgdaSymbol{=}\AgdaSpace{}%
\AgdaFunction{Ar}\<%
\\
%
\\[\AgdaEmptyExtraSkip]%
%
\>[2]\AgdaComment{------------------------------------}\<%
\\
%
\>[2]\AgdaComment{---\ DFT\ and\ FFT\ helper\ functions\ ---}\<%
\\
%
\>[2]\AgdaComment{------------------------------------}\<%
\end{code}
%TC:endignore

\paragraph{Index's in a single dimension}. As defined above, \AF{Position} encodes 
the bounds on a given index, as well as the index itself. 
When calculating the DFT some arithmetic on this index is required,
this arithmetic would be overly complex if performed while the index is 
wrapped in a position, and so
helper functions are required to convert a given position to its index value.
This helper function for the single dimensional case is shown below.

\begin{code}%
%
\>[2]\AgdaFunction{iota}\AgdaSpace{}%
\AgdaSymbol{:}\AgdaSpace{}%
\AgdaFunction{Ar}\AgdaSpace{}%
\AgdaSymbol{(}\AgdaInductiveConstructor{ι}\AgdaSpace{}%
\AgdaGeneralizable{N}\AgdaSymbol{)}\AgdaSpace{}%
\AgdaDatatype{ℕ}\<%
\\
%
\>[2]\AgdaFunction{iota}\AgdaSpace{}%
\AgdaSymbol{(}\AgdaInductiveConstructor{ι}\AgdaSpace{}%
\AgdaBound{i}\AgdaSymbol{)}\AgdaSpace{}%
\AgdaSymbol{=}\AgdaSpace{}%
\AgdaFunction{toℕ}\AgdaSpace{}%
\AgdaBound{i}\<%
\end{code}

\subsection{DFT}
Given the above definition of the complex numbers, tensors, and methods on one 
dimensional tensors, the formation of the DFT is now trivial.
This is of the same shape as Equation \ref{eq:DFT_Definition}, requiring through 
use of \AF{Ar∔} that the length of any input vector is non zero, as to satisfy 
this same condition on the divisor of \AF{-ω} as defined in \ref{para:roots_of_unity}.

\begin{code}%
%
\>[2]\AgdaFunction{DFT}\AgdaSpace{}%
\AgdaSymbol{:}\AgdaSpace{}%
\AgdaFunction{Ar∔}\AgdaSpace{}%
\AgdaSymbol{(}\AgdaInductiveConstructor{ι}\AgdaSpace{}%
\AgdaGeneralizable{N}\AgdaSymbol{)}\AgdaSpace{}%
\AgdaField{ℂ}\AgdaSpace{}%
\AgdaSymbol{→}\AgdaSpace{}%
\AgdaFunction{Ar∔}\AgdaSpace{}%
\AgdaSymbol{(}\AgdaInductiveConstructor{ι}\AgdaSpace{}%
\AgdaGeneralizable{N}\AgdaSymbol{)}\AgdaSpace{}%
\AgdaField{ℂ}\<%
\\
%
\>[2]\AgdaFunction{DFT}\AgdaSpace{}%
\AgdaSymbol{\{}\AgdaBound{N}\AgdaSymbol{\}}\AgdaSpace{}%
\AgdaBound{xs}\AgdaSpace{}%
\AgdaBound{j}\AgdaSpace{}%
\AgdaSymbol{=}\AgdaSpace{}%
\AgdaFunction{sum}\AgdaSpace{}%
\AgdaSymbol{λ}\AgdaSpace{}%
\AgdaBound{k}\AgdaSpace{}%
\AgdaSymbol{→}\AgdaSpace{}%
\AgdaBound{xs}\AgdaSpace{}%
\AgdaBound{k}\AgdaSpace{}%
\AgdaOperator{\AgdaField{*}}\AgdaSpace{}%
\AgdaField{-ω}\AgdaSpace{}%
\AgdaBound{N}\AgdaSpace{}%
\AgdaSymbol{(}\AgdaFunction{iota}\AgdaSpace{}%
\AgdaBound{k}\AgdaSpace{}%
\AgdaOperator{\AgdaPrimitive{*ₙ}}\AgdaSpace{}%
\AgdaFunction{iota}\AgdaSpace{}%
\AgdaBound{j}\AgdaSymbol{)}\<%
\end{code}
%TC:ignore
\begin{code}[hide]%
\>[2][@{}l@{\AgdaIndent{1}}]%
\>[4]\AgdaKeyword{where}\<%
\\
\>[4][@{}l@{\AgdaIndent{0}}]%
\>[6]\AgdaKeyword{postulate}\<%
\\
\>[6][@{}l@{\AgdaIndent{0}}]%
\>[8]\AgdaKeyword{instance}\<%
\\
\>[8][@{}l@{\AgdaIndent{0}}]%
\>[10]\AgdaPostulate{\AgdaUnderscore{}}\AgdaSpace{}%
\AgdaSymbol{:}\AgdaSpace{}%
\AgdaRecord{NonZero}\AgdaSpace{}%
\AgdaBound{N}\<%
\end{code}
%TC:endignore


% It is then trivial to form the \AF{DFT} without this restriction, by 
% checking if a given array is of length zero, and returning that same array of
% length zero when this is the case.



\subsection{Reshape}
%TC:ignore
\begin{code}[hide]%
\>[0]\AgdaKeyword{module}\AgdaSpace{}%
\AgdaModule{Reshape}\AgdaSpace{}%
\AgdaKeyword{where}\<%
\\
%
\\[\AgdaEmptyExtraSkip]%
\>[0][@{}l@{\AgdaIndent{0}}]%
\>[2]\AgdaKeyword{open}\AgdaSpace{}%
\AgdaKeyword{import}\AgdaSpace{}%
\AgdaModule{Data.Nat}\AgdaSpace{}%
\AgdaKeyword{using}\AgdaSpace{}%
\AgdaSymbol{(}\AgdaDatatype{ℕ}\AgdaSymbol{;}\AgdaSpace{}%
\AgdaOperator{\AgdaPrimitive{\AgdaUnderscore{}*\AgdaUnderscore{}}}\AgdaSymbol{;}\AgdaSpace{}%
\AgdaRecord{NonZero}\AgdaSymbol{)}\<%
\\
%
\>[2]\AgdaKeyword{open}\AgdaSpace{}%
\AgdaKeyword{import}\AgdaSpace{}%
\AgdaModule{Data.Nat.Properties}\AgdaSpace{}%
\AgdaKeyword{using}\AgdaSpace{}%
\AgdaSymbol{(}\AgdaFunction{*-comm}\AgdaSymbol{)}\<%
\\
%
\>[2]\AgdaKeyword{open}\AgdaSpace{}%
\AgdaKeyword{import}\AgdaSpace{}%
\AgdaModule{Data.Fin}\AgdaSpace{}%
\AgdaSymbol{as}\AgdaSpace{}%
\AgdaModule{F}\AgdaSpace{}%
\AgdaKeyword{using}\AgdaSpace{}%
\AgdaSymbol{(}\AgdaDatatype{Fin}\AgdaSymbol{;}\AgdaSpace{}%
\AgdaFunction{combine}\AgdaSymbol{;}\AgdaSpace{}%
\AgdaFunction{remQuot}\AgdaSymbol{;}\AgdaSpace{}%
\AgdaFunction{quotRem}\AgdaSymbol{;}\AgdaSpace{}%
\AgdaFunction{toℕ}\AgdaSymbol{;}\AgdaSpace{}%
\AgdaFunction{cast}\AgdaSymbol{)}\<%
\\
%
\>[2]\AgdaKeyword{open}\AgdaSpace{}%
\AgdaKeyword{import}\AgdaSpace{}%
\AgdaModule{Data.Fin.Properties}\AgdaSpace{}%
\AgdaKeyword{using}\AgdaSpace{}%
\AgdaSymbol{(}\AgdaFunction{remQuot-combine}\AgdaSymbol{;}\AgdaSpace{}%
\AgdaFunction{combine-remQuot}\AgdaSymbol{;}\AgdaSpace{}%
\AgdaFunction{cast-is-id}\AgdaSymbol{;}\AgdaSpace{}%
\AgdaFunction{cast-trans}\AgdaSymbol{)}\<%
\\
%
\\[\AgdaEmptyExtraSkip]%
%
\>[2]\AgdaKeyword{open}\AgdaSpace{}%
\AgdaKeyword{import}\AgdaSpace{}%
\AgdaModule{Data.Product}\AgdaSpace{}%
\AgdaKeyword{using}\AgdaSpace{}%
\AgdaSymbol{(}\AgdaOperator{\AgdaInductiveConstructor{\AgdaUnderscore{},\AgdaUnderscore{}}}\AgdaSymbol{;}\AgdaSpace{}%
\AgdaField{proj₁}\AgdaSymbol{;}\AgdaSpace{}%
\AgdaField{proj₂}\AgdaSymbol{)}\<%
\\
%
\>[2]\AgdaKeyword{open}\AgdaSpace{}%
\AgdaKeyword{import}\AgdaSpace{}%
\AgdaModule{Matrix}\AgdaSpace{}%
\AgdaKeyword{using}\AgdaSpace{}%
\AgdaSymbol{(}\AgdaDatatype{Shape}\AgdaSymbol{;}\AgdaSpace{}%
\AgdaDatatype{Position}\AgdaSymbol{;}\AgdaSpace{}%
\AgdaFunction{Ar}\AgdaSymbol{;}\AgdaSpace{}%
\AgdaInductiveConstructor{ι}\AgdaSymbol{;}\AgdaSpace{}%
\AgdaOperator{\AgdaInductiveConstructor{\AgdaUnderscore{}⊗\AgdaUnderscore{}}}\AgdaSymbol{;}\AgdaSpace{}%
\AgdaFunction{length}\AgdaSymbol{)}\<%
\\
%
\\[\AgdaEmptyExtraSkip]%
%
\>[2]\AgdaKeyword{import}\AgdaSpace{}%
\AgdaModule{Relation.Binary.PropositionalEquality}\AgdaSpace{}%
\AgdaSymbol{as}\AgdaSpace{}%
\AgdaModule{Eq}\<%
\\
%
\>[2]\AgdaKeyword{open}\AgdaSpace{}%
\AgdaModule{Eq}\AgdaSpace{}%
\AgdaKeyword{using}\AgdaSpace{}%
\AgdaSymbol{(}\AgdaOperator{\AgdaDatatype{\AgdaUnderscore{}≡\AgdaUnderscore{}}}\AgdaSymbol{;}\AgdaSpace{}%
\AgdaInductiveConstructor{refl}\AgdaSymbol{;}\AgdaSpace{}%
\AgdaFunction{cong}\AgdaSymbol{;}\AgdaSpace{}%
\AgdaFunction{trans}\AgdaSymbol{;}\AgdaSpace{}%
\AgdaFunction{subst}\AgdaSymbol{;}\AgdaSpace{}%
\AgdaFunction{sym}\AgdaSymbol{)}\<%
\\
%
\>[2]\AgdaKeyword{open}\AgdaSpace{}%
\AgdaModule{Eq.≡-Reasoning}\<%
\\
%
\\[\AgdaEmptyExtraSkip]%
%
\>[2]\AgdaKeyword{variable}\<%
\\
\>[2][@{}l@{\AgdaIndent{0}}]%
\>[4]\AgdaGeneralizable{m}\AgdaSpace{}%
\AgdaGeneralizable{n}\AgdaSpace{}%
\AgdaGeneralizable{k}\AgdaSpace{}%
\AgdaSymbol{:}\AgdaSpace{}%
\AgdaDatatype{ℕ}\<%
\\
%
\>[4]\AgdaGeneralizable{s}\AgdaSpace{}%
\AgdaGeneralizable{p}\AgdaSpace{}%
\AgdaGeneralizable{q}\AgdaSpace{}%
\AgdaGeneralizable{r}\AgdaSpace{}%
\AgdaSymbol{:}\AgdaSpace{}%
\AgdaDatatype{Shape}\<%
\\
%
\>[4]\AgdaGeneralizable{X}\AgdaSpace{}%
\AgdaGeneralizable{Y}\AgdaSpace{}%
\AgdaGeneralizable{Z}\AgdaSpace{}%
\AgdaSymbol{:}\AgdaSpace{}%
\AgdaPrimitive{Set}\<%
\\
%
\\[\AgdaEmptyExtraSkip]%
%
\>[2]\AgdaKeyword{infixr}\AgdaSpace{}%
\AgdaNumber{5}\AgdaSpace{}%
\AgdaOperator{\AgdaInductiveConstructor{\AgdaUnderscore{}∙\AgdaUnderscore{}}}\<%
\\
%
\>[2]\AgdaKeyword{infixl}\AgdaSpace{}%
\AgdaNumber{10}\AgdaSpace{}%
\AgdaOperator{\AgdaInductiveConstructor{\AgdaUnderscore{}⊕\AgdaUnderscore{}}}\<%
\end{code}
%TC:endignore

When working with tensors, it is often necessary for elements to be rearranged, 
through operations such as transpose or flatten, without any additions or removals.
The naïve approach to this, would be to define each rearrange as a function of type \AF{Ar s X → Ar p X}.
This approach however, would operate on too large a space, meaning reasoning upon
such functions would be difficult and could not be generalised.
An alternate approach is to define a small language of reshapes.
This language captures a small set of rearrangements, as well as methods to allow
for their composition.
Generalised properties, such as how each reshape is applied to a position, can 
then be defined on this language or reshapes.

For this language, each reshape operations can be considered as a bijective function
from shape \AF{s} to shape \AF{p}. 
As this ensures that no tensor can loose or gain data, creating a strict reshape 
language will strengthen any reasoning in future proofs.
This also means that any reshape operation is reversible which will allow for the
formation of rules which are general to all operations in the reshape language.

The reshape language is defined as a set of operations from shape to shape as follows.
\begin{code}%
%
\>[2]\AgdaKeyword{data}\AgdaSpace{}%
\AgdaDatatype{Reshape}\AgdaSpace{}%
\AgdaSymbol{:}\AgdaSpace{}%
\AgdaDatatype{Shape}\AgdaSpace{}%
\AgdaSymbol{→}\AgdaSpace{}%
\AgdaDatatype{Shape}\AgdaSpace{}%
\AgdaSymbol{→}\AgdaSpace{}%
\AgdaPrimitive{Set}\AgdaSpace{}%
\AgdaKeyword{where}\<%
\\
\>[2][@{}l@{\AgdaIndent{0}}]%
\>[4]\AgdaInductiveConstructor{eq}%
\>[11]\AgdaSymbol{:}\AgdaSpace{}%
\AgdaDatatype{Reshape}\AgdaSpace{}%
\AgdaGeneralizable{s}\AgdaSpace{}%
\AgdaGeneralizable{s}%
\>[46]\AgdaComment{--\ Identity}\<%
\\
%
\>[4]\AgdaOperator{\AgdaInductiveConstructor{\AgdaUnderscore{}∙\AgdaUnderscore{}}}%
\>[11]\AgdaSymbol{:}\AgdaSpace{}%
\AgdaDatatype{Reshape}\AgdaSpace{}%
\AgdaGeneralizable{p}\AgdaSpace{}%
\AgdaGeneralizable{q}%
\>[46]\AgdaComment{--\ Composition\ of\ Reshapes}\<%
\\
%
\>[11]\AgdaSymbol{→}\AgdaSpace{}%
\AgdaDatatype{Reshape}\AgdaSpace{}%
\AgdaGeneralizable{s}\AgdaSpace{}%
\AgdaGeneralizable{p}\<%
\\
%
\>[11]\AgdaSymbol{→}\AgdaSpace{}%
\AgdaDatatype{Reshape}\AgdaSpace{}%
\AgdaGeneralizable{s}\AgdaSpace{}%
\AgdaGeneralizable{q}\<%
\\
%
\>[4]\AgdaOperator{\AgdaInductiveConstructor{\AgdaUnderscore{}⊕\AgdaUnderscore{}}}%
\>[11]\AgdaSymbol{:}\AgdaSpace{}%
\AgdaDatatype{Reshape}\AgdaSpace{}%
\AgdaGeneralizable{s}\AgdaSpace{}%
\AgdaGeneralizable{p}%
\>[46]\AgdaComment{--\ Left/\ Right\ application}\<%
\\
%
\>[11]\AgdaSymbol{→}\AgdaSpace{}%
\AgdaDatatype{Reshape}\AgdaSpace{}%
\AgdaGeneralizable{q}\AgdaSpace{}%
\AgdaGeneralizable{r}\<%
\\
%
\>[11]\AgdaSymbol{→}\AgdaSpace{}%
\AgdaDatatype{Reshape}\AgdaSpace{}%
\AgdaSymbol{(}\AgdaGeneralizable{s}\AgdaSpace{}%
\AgdaOperator{\AgdaInductiveConstructor{⊗}}\AgdaSpace{}%
\AgdaGeneralizable{q}\AgdaSymbol{)}\AgdaSpace{}%
\AgdaSymbol{(}\AgdaGeneralizable{p}\AgdaSpace{}%
\AgdaOperator{\AgdaInductiveConstructor{⊗}}\AgdaSpace{}%
\AgdaGeneralizable{r}\AgdaSymbol{)}\<%
\\
%
\>[4]\AgdaInductiveConstructor{split}%
\>[11]\AgdaSymbol{:}\AgdaSpace{}%
\AgdaDatatype{Reshape}\AgdaSpace{}%
\AgdaSymbol{(}\AgdaInductiveConstructor{ι}\AgdaSpace{}%
\AgdaSymbol{(}\AgdaGeneralizable{m}\AgdaSpace{}%
\AgdaOperator{\AgdaPrimitive{*}}\AgdaSpace{}%
\AgdaGeneralizable{n}\AgdaSymbol{))}\AgdaSpace{}%
\AgdaSymbol{(}\AgdaInductiveConstructor{ι}\AgdaSpace{}%
\AgdaGeneralizable{m}\AgdaSpace{}%
\AgdaOperator{\AgdaInductiveConstructor{⊗}}\AgdaSpace{}%
\AgdaInductiveConstructor{ι}\AgdaSpace{}%
\AgdaGeneralizable{n}\AgdaSymbol{)}%
\>[46]\AgdaComment{--\ "Vector"\ →\ 2D\ Tensor}\<%
\\
%
\>[4]\AgdaInductiveConstructor{flat}%
\>[11]\AgdaSymbol{:}\AgdaSpace{}%
\AgdaDatatype{Reshape}\AgdaSpace{}%
\AgdaSymbol{(}\AgdaInductiveConstructor{ι}\AgdaSpace{}%
\AgdaGeneralizable{m}\AgdaSpace{}%
\AgdaOperator{\AgdaInductiveConstructor{⊗}}\AgdaSpace{}%
\AgdaInductiveConstructor{ι}\AgdaSpace{}%
\AgdaGeneralizable{n}\AgdaSymbol{)}\AgdaSpace{}%
\AgdaSymbol{(}\AgdaInductiveConstructor{ι}\AgdaSpace{}%
\AgdaSymbol{(}\AgdaGeneralizable{m}\AgdaSpace{}%
\AgdaOperator{\AgdaPrimitive{*}}\AgdaSpace{}%
\AgdaGeneralizable{n}\AgdaSymbol{))}%
\>[46]\AgdaComment{--\ 2D\ Tensor\ →\ "Vector"}\<%
\\
%
\>[4]\AgdaInductiveConstructor{swap}%
\>[11]\AgdaSymbol{:}\AgdaSpace{}%
\AgdaDatatype{Reshape}\AgdaSpace{}%
\AgdaSymbol{(}\AgdaGeneralizable{s}\AgdaSpace{}%
\AgdaOperator{\AgdaInductiveConstructor{⊗}}\AgdaSpace{}%
\AgdaGeneralizable{p}\AgdaSymbol{)}\AgdaSpace{}%
\AgdaSymbol{(}\AgdaGeneralizable{p}\AgdaSpace{}%
\AgdaOperator{\AgdaInductiveConstructor{⊗}}\AgdaSpace{}%
\AgdaGeneralizable{s}\AgdaSymbol{)}%
\>[46]\AgdaComment{--\ Transposition}\<%
\end{code}

Using this definition of reshape and some standard library methods on Fin, 
it is then possible do define the application of reshape to positions and tensors.
\begin{code}%
%
\>[2]\AgdaOperator{\AgdaFunction{\AgdaUnderscore{}⟨\AgdaUnderscore{}⟩}}\AgdaSpace{}%
\AgdaSymbol{:}\AgdaSpace{}%
\AgdaDatatype{Position}\AgdaSpace{}%
\AgdaGeneralizable{p}\AgdaSpace{}%
\AgdaSymbol{→}\AgdaSpace{}%
\AgdaDatatype{Reshape}\AgdaSpace{}%
\AgdaGeneralizable{s}\AgdaSpace{}%
\AgdaGeneralizable{p}\AgdaSpace{}%
\AgdaSymbol{→}\AgdaSpace{}%
\AgdaDatatype{Position}\AgdaSpace{}%
\AgdaGeneralizable{s}\<%
\\
%
\>[2]\AgdaBound{i}%
\>[15]\AgdaOperator{\AgdaFunction{⟨}}\AgdaSpace{}%
\AgdaInductiveConstructor{eq}%
\>[25]\AgdaOperator{\AgdaFunction{⟩}}\AgdaSpace{}%
\AgdaSymbol{=}\AgdaSpace{}%
\AgdaBound{i}\<%
\\
%
\>[2]\AgdaBound{i}%
\>[15]\AgdaOperator{\AgdaFunction{⟨}}\AgdaSpace{}%
\AgdaBound{r}\AgdaSpace{}%
\AgdaOperator{\AgdaInductiveConstructor{∙}}\AgdaSpace{}%
\AgdaBound{r₁}%
\>[25]\AgdaOperator{\AgdaFunction{⟩}}\AgdaSpace{}%
\AgdaSymbol{=}\AgdaSpace{}%
\AgdaBound{i}\AgdaSpace{}%
\AgdaOperator{\AgdaFunction{⟨}}\AgdaSpace{}%
\AgdaBound{r}\AgdaSpace{}%
\AgdaOperator{\AgdaFunction{⟩}}\AgdaSpace{}%
\AgdaOperator{\AgdaFunction{⟨}}\AgdaSpace{}%
\AgdaBound{r₁}\AgdaSpace{}%
\AgdaOperator{\AgdaFunction{⟩}}\<%
\\
%
\>[2]\AgdaSymbol{(}\AgdaBound{i}\AgdaSpace{}%
\AgdaOperator{\AgdaInductiveConstructor{⊗}}\AgdaSpace{}%
\AgdaBound{j}\AgdaSymbol{)}%
\>[15]\AgdaOperator{\AgdaFunction{⟨}}\AgdaSpace{}%
\AgdaBound{r}\AgdaSpace{}%
\AgdaOperator{\AgdaInductiveConstructor{⊕}}\AgdaSpace{}%
\AgdaBound{r₁}%
\>[25]\AgdaOperator{\AgdaFunction{⟩}}\AgdaSpace{}%
\AgdaSymbol{=}\AgdaSpace{}%
\AgdaSymbol{(}\AgdaBound{i}\AgdaSpace{}%
\AgdaOperator{\AgdaFunction{⟨}}\AgdaSpace{}%
\AgdaBound{r}\AgdaSpace{}%
\AgdaOperator{\AgdaFunction{⟩}}\AgdaSymbol{)}\AgdaSpace{}%
\AgdaOperator{\AgdaInductiveConstructor{⊗}}\AgdaSpace{}%
\AgdaSymbol{(}\AgdaBound{j}\AgdaSpace{}%
\AgdaOperator{\AgdaFunction{⟨}}\AgdaSpace{}%
\AgdaBound{r₁}\AgdaSpace{}%
\AgdaOperator{\AgdaFunction{⟩}}\AgdaSymbol{)}\<%
\\
%
\>[2]\AgdaSymbol{(}\AgdaInductiveConstructor{ι}\AgdaSpace{}%
\AgdaBound{i}\AgdaSpace{}%
\AgdaOperator{\AgdaInductiveConstructor{⊗}}\AgdaSpace{}%
\AgdaInductiveConstructor{ι}\AgdaSpace{}%
\AgdaBound{j}\AgdaSymbol{)}%
\>[15]\AgdaOperator{\AgdaFunction{⟨}}\AgdaSpace{}%
\AgdaInductiveConstructor{split}%
\>[25]\AgdaOperator{\AgdaFunction{⟩}}\AgdaSpace{}%
\AgdaSymbol{=}\AgdaSpace{}%
\AgdaInductiveConstructor{ι}\AgdaSpace{}%
\AgdaSymbol{(}\AgdaFunction{combine}\AgdaSpace{}%
\AgdaBound{i}\AgdaSpace{}%
\AgdaBound{j}\AgdaSymbol{)}\<%
\\
%
\>[2]\AgdaInductiveConstructor{ι}\AgdaSpace{}%
\AgdaBound{i}%
\>[15]\AgdaOperator{\AgdaFunction{⟨}}\AgdaSpace{}%
\AgdaInductiveConstructor{flat}%
\>[25]\AgdaOperator{\AgdaFunction{⟩}}\AgdaSpace{}%
\AgdaSymbol{=}\AgdaSpace{}%
\AgdaKeyword{let}\AgdaSpace{}%
\AgdaBound{a}\AgdaSpace{}%
\AgdaOperator{\AgdaInductiveConstructor{,}}\AgdaSpace{}%
\AgdaBound{b}\AgdaSpace{}%
\AgdaSymbol{=}\AgdaSpace{}%
\AgdaFunction{remQuot}\AgdaSpace{}%
\AgdaSymbol{\AgdaUnderscore{}}\AgdaSpace{}%
\AgdaBound{i}\AgdaSpace{}%
\AgdaKeyword{in}\AgdaSpace{}%
\AgdaInductiveConstructor{ι}\AgdaSpace{}%
\AgdaBound{a}\AgdaSpace{}%
\AgdaOperator{\AgdaInductiveConstructor{⊗}}\AgdaSpace{}%
\AgdaInductiveConstructor{ι}\AgdaSpace{}%
\AgdaBound{b}\<%
\\
%
\>[2]\AgdaSymbol{(}\AgdaBound{i}\AgdaSpace{}%
\AgdaOperator{\AgdaInductiveConstructor{⊗}}\AgdaSpace{}%
\AgdaBound{j}\AgdaSymbol{)}%
\>[15]\AgdaOperator{\AgdaFunction{⟨}}\AgdaSpace{}%
\AgdaInductiveConstructor{swap}%
\>[25]\AgdaOperator{\AgdaFunction{⟩}}\AgdaSpace{}%
\AgdaSymbol{=}\AgdaSpace{}%
\AgdaBound{j}\AgdaSpace{}%
\AgdaOperator{\AgdaInductiveConstructor{⊗}}\AgdaSpace{}%
\AgdaBound{i}\<%
\\
%
\\[\AgdaEmptyExtraSkip]%
%
\>[2]\AgdaFunction{reshape}\AgdaSpace{}%
\AgdaSymbol{:}\AgdaSpace{}%
\AgdaDatatype{Reshape}\AgdaSpace{}%
\AgdaGeneralizable{s}\AgdaSpace{}%
\AgdaGeneralizable{p}\AgdaSpace{}%
\AgdaSymbol{→}\AgdaSpace{}%
\AgdaFunction{Ar}\AgdaSpace{}%
\AgdaGeneralizable{s}\AgdaSpace{}%
\AgdaGeneralizable{X}\AgdaSpace{}%
\AgdaSymbol{→}\AgdaSpace{}%
\AgdaFunction{Ar}\AgdaSpace{}%
\AgdaGeneralizable{p}\AgdaSpace{}%
\AgdaGeneralizable{X}\<%
\\
%
\>[2]\AgdaFunction{reshape}\AgdaSpace{}%
\AgdaBound{r}\AgdaSpace{}%
\AgdaBound{a}\AgdaSpace{}%
\AgdaBound{ix}\AgdaSpace{}%
\AgdaSymbol{=}\AgdaSpace{}%
\AgdaBound{a}\AgdaSpace{}%
\AgdaSymbol{(}\AgdaBound{ix}\AgdaSpace{}%
\AgdaOperator{\AgdaFunction{⟨}}\AgdaSpace{}%
\AgdaBound{r}\AgdaSpace{}%
\AgdaOperator{\AgdaFunction{⟩}}\AgdaSpace{}%
\AgdaSymbol{)}\<%
\end{code}
\subsubsection{Reverse}
As each reshape operation is a bijective function, it is trivial to define a reverse
method.
\begin{code}%
%
\>[2]\AgdaFunction{rev}\AgdaSpace{}%
\AgdaSymbol{:}\AgdaSpace{}%
\AgdaDatatype{Reshape}\AgdaSpace{}%
\AgdaGeneralizable{s}\AgdaSpace{}%
\AgdaGeneralizable{p}\AgdaSpace{}%
\AgdaSymbol{→}\AgdaSpace{}%
\AgdaDatatype{Reshape}\AgdaSpace{}%
\AgdaGeneralizable{p}\AgdaSpace{}%
\AgdaGeneralizable{s}\<%
\\
%
\>[2]\AgdaFunction{rev}\AgdaSpace{}%
\AgdaInductiveConstructor{eq}\AgdaSpace{}%
\AgdaSymbol{=}\AgdaSpace{}%
\AgdaInductiveConstructor{eq}\<%
\\
%
\>[2]\AgdaFunction{rev}\AgdaSpace{}%
\AgdaSymbol{(}\AgdaBound{r}\AgdaSpace{}%
\AgdaOperator{\AgdaInductiveConstructor{⊕}}\AgdaSpace{}%
\AgdaBound{r₁}\AgdaSymbol{)}\AgdaSpace{}%
\AgdaSymbol{=}\AgdaSpace{}%
\AgdaFunction{rev}\AgdaSpace{}%
\AgdaBound{r}\AgdaSpace{}%
\AgdaOperator{\AgdaInductiveConstructor{⊕}}\AgdaSpace{}%
\AgdaFunction{rev}\AgdaSpace{}%
\AgdaBound{r₁}\<%
\\
%
\>[2]\AgdaFunction{rev}\AgdaSpace{}%
\AgdaSymbol{(}\AgdaBound{r}\AgdaSpace{}%
\AgdaOperator{\AgdaInductiveConstructor{∙}}\AgdaSpace{}%
\AgdaBound{r₁}\AgdaSymbol{)}\AgdaSpace{}%
\AgdaSymbol{=}\AgdaSpace{}%
\AgdaFunction{rev}\AgdaSpace{}%
\AgdaBound{r₁}\AgdaSpace{}%
\AgdaOperator{\AgdaInductiveConstructor{∙}}\AgdaSpace{}%
\AgdaFunction{rev}\AgdaSpace{}%
\AgdaBound{r}\<%
\\
%
\>[2]\AgdaFunction{rev}\AgdaSpace{}%
\AgdaInductiveConstructor{split}\AgdaSpace{}%
\AgdaSymbol{=}\AgdaSpace{}%
\AgdaInductiveConstructor{flat}\<%
\\
%
\>[2]\AgdaFunction{rev}\AgdaSpace{}%
\AgdaInductiveConstructor{flat}\AgdaSpace{}%
\AgdaSymbol{=}\AgdaSpace{}%
\AgdaInductiveConstructor{split}\<%
\\
%
\>[2]\AgdaFunction{rev}\AgdaSpace{}%
\AgdaInductiveConstructor{swap}\AgdaSpace{}%
\AgdaSymbol{=}\AgdaSpace{}%
\AgdaInductiveConstructor{swap}\<%
\end{code}
From this operation, rules on reshape can be defined, allow for formation of
relations between reshape operations.
This allows for the reduction of the reshape language when operations such as 
\AF{split ∙ flat} occur.
%TC:ignore
\begin{code}[hide]%
%
\>[2]\AgdaKeyword{infixl}\AgdaSpace{}%
\AgdaNumber{11}\AgdaSpace{}%
\AgdaOperator{\AgdaFunction{\AgdaUnderscore{}ᵗ}}\<%
\\
%
\>[2]\AgdaKeyword{postulate}\<%
\end{code}
%TC:endignore
\begin{code}%
\>[2][@{}l@{\AgdaIndent{1}}]%
\>[4]\AgdaPostulate{rev-eq}\AgdaSpace{}%
\AgdaSymbol{:}\<%
\\
\>[4][@{}l@{\AgdaIndent{0}}]%
\>[6]\AgdaSymbol{∀}%
\>[937I]\AgdaSymbol{(}\AgdaBound{r}\AgdaSpace{}%
\AgdaSymbol{:}\AgdaSpace{}%
\AgdaDatatype{Reshape}\AgdaSpace{}%
\AgdaGeneralizable{s}\AgdaSpace{}%
\AgdaGeneralizable{p}\AgdaSymbol{)}\<%
\\
\>[.][@{}l@{}]\<[937I]%
\>[8]\AgdaSymbol{(}\AgdaBound{i}\AgdaSpace{}%
\AgdaSymbol{:}\AgdaSpace{}%
\AgdaDatatype{Position}\AgdaSpace{}%
\AgdaGeneralizable{p}\AgdaSpace{}%
\AgdaSymbol{)}\<%
\\
%
\>[6]\AgdaComment{---------------------}\<%
\\
%
\>[6]\AgdaSymbol{→}\AgdaSpace{}%
\AgdaBound{i}\AgdaSpace{}%
\AgdaOperator{\AgdaFunction{⟨}}\AgdaSpace{}%
\AgdaBound{r}\AgdaSpace{}%
\AgdaOperator{\AgdaInductiveConstructor{∙}}\AgdaSpace{}%
\AgdaFunction{rev}\AgdaSpace{}%
\AgdaBound{r}\AgdaSpace{}%
\AgdaOperator{\AgdaFunction{⟩}}\AgdaSpace{}%
\AgdaOperator{\AgdaDatatype{≡}}\AgdaSpace{}%
\AgdaBound{i}\<%
\\
%
\\[\AgdaEmptyExtraSkip]%
%
\>[4]\AgdaPostulate{rev-rev}\AgdaSpace{}%
\AgdaSymbol{:}\<%
\\
\>[4][@{}l@{\AgdaIndent{0}}]%
\>[6]\AgdaSymbol{∀}%
\>[956I]\AgdaSymbol{(}\AgdaBound{r}\AgdaSpace{}%
\AgdaSymbol{:}\AgdaSpace{}%
\AgdaDatatype{Reshape}\AgdaSpace{}%
\AgdaGeneralizable{s}\AgdaSpace{}%
\AgdaGeneralizable{p}\AgdaSymbol{)}\<%
\\
\>[.][@{}l@{}]\<[956I]%
\>[8]\AgdaSymbol{(}\AgdaBound{i}\AgdaSpace{}%
\AgdaSymbol{:}\AgdaSpace{}%
\AgdaDatatype{Position}\AgdaSpace{}%
\AgdaGeneralizable{p}\AgdaSpace{}%
\AgdaSymbol{)}\<%
\\
%
\>[6]\AgdaComment{-----------------------------}\<%
\\
%
\>[6]\AgdaSymbol{→}\AgdaSpace{}%
\AgdaBound{i}\AgdaSpace{}%
\AgdaOperator{\AgdaFunction{⟨}}\AgdaSpace{}%
\AgdaFunction{rev}\AgdaSpace{}%
\AgdaSymbol{(}\AgdaFunction{rev}\AgdaSpace{}%
\AgdaBound{r}\AgdaSymbol{)}\AgdaSpace{}%
\AgdaOperator{\AgdaFunction{⟩}}\AgdaSpace{}%
\AgdaOperator{\AgdaDatatype{≡}}\AgdaSpace{}%
\AgdaBound{i}\AgdaSpace{}%
\AgdaOperator{\AgdaFunction{⟨}}\AgdaSpace{}%
\AgdaBound{r}\AgdaSpace{}%
\AgdaOperator{\AgdaFunction{⟩}}\<%
\end{code}

\subsubsection{Recursive Reshaping}
While the above operations of reshape can be applied to tensors of a fixed shape
this language of reshapes can be improved with the creation of recursive reshape
operations.

\paragraph{Flatten and Unflatten} enable the recursive application of flat and 
split respectively.
This allows for an $N$-dimensional tensor to be flattened, and for any single dimensional
tensor of size \AF{length s} to be unflattened into a tensor of shape s.
\begin{code}%
%
\>[2]\AgdaFunction{♭}\AgdaSpace{}%
\AgdaSymbol{:}\AgdaSpace{}%
\AgdaDatatype{Reshape}\AgdaSpace{}%
\AgdaGeneralizable{s}\AgdaSpace{}%
\AgdaSymbol{(}\AgdaInductiveConstructor{ι}\AgdaSpace{}%
\AgdaSymbol{(}\AgdaFunction{length}\AgdaSpace{}%
\AgdaGeneralizable{s}\AgdaSymbol{))}\<%
\\
%
\>[2]\AgdaFunction{♭}\AgdaSpace{}%
\AgdaSymbol{\{}\AgdaInductiveConstructor{ι}\AgdaSpace{}%
\AgdaBound{x}%
\>[11]\AgdaSymbol{\}}\AgdaSpace{}%
\AgdaSymbol{=}\AgdaSpace{}%
\AgdaInductiveConstructor{eq}\<%
\\
%
\>[2]\AgdaFunction{♭}\AgdaSpace{}%
\AgdaSymbol{\{}\AgdaBound{s}\AgdaSpace{}%
\AgdaOperator{\AgdaInductiveConstructor{⊗}}\AgdaSpace{}%
\AgdaBound{s₁}\AgdaSymbol{\}}\AgdaSpace{}%
\AgdaSymbol{=}\AgdaSpace{}%
\AgdaInductiveConstructor{flat}\AgdaSpace{}%
\AgdaOperator{\AgdaInductiveConstructor{∙}}\AgdaSpace{}%
\AgdaFunction{♭}\AgdaSpace{}%
\AgdaOperator{\AgdaInductiveConstructor{⊕}}\AgdaSpace{}%
\AgdaFunction{♭}\<%
\\
%
\\[\AgdaEmptyExtraSkip]%
%
\>[2]\AgdaComment{--\ Unflatten\ is\ free\ from\ flatten}\<%
\\
%
\>[2]\AgdaFunction{♯}\AgdaSpace{}%
\AgdaSymbol{:}\AgdaSpace{}%
\AgdaDatatype{Reshape}\AgdaSpace{}%
\AgdaSymbol{(}\AgdaInductiveConstructor{ι}\AgdaSpace{}%
\AgdaSymbol{(}\AgdaFunction{length}\AgdaSpace{}%
\AgdaGeneralizable{s}\AgdaSymbol{))}\AgdaSpace{}%
\AgdaGeneralizable{s}\<%
\\
%
\>[2]\AgdaFunction{♯}\AgdaSpace{}%
\AgdaSymbol{=}\AgdaSpace{}%
\AgdaFunction{rev}\AgdaSpace{}%
\AgdaFunction{♭}\<%
\end{code}

\paragraph{Transpose} flips a tensor over its diagonal by swapping the left and
right sub-shape at each level.
Transpose applies swap to any non leaf nodes, allowing for any given 
function designed to operate on multi dimensional tenors, such as the FFT, to
do the same swap at each level.
It can be seen below that transpose is defined through use of the postfix operator, 
meaning the input shape goes before \AF{ᵗ}
\begin{code}%
%
\>[2]\AgdaOperator{\AgdaFunction{\AgdaUnderscore{}ᵗ}}\AgdaSpace{}%
\AgdaSymbol{:}\AgdaSpace{}%
\AgdaDatatype{Shape}\AgdaSpace{}%
\AgdaSymbol{→}\AgdaSpace{}%
\AgdaDatatype{Shape}\<%
\\
%
\>[2]\AgdaOperator{\AgdaFunction{\AgdaUnderscore{}ᵗ}}\AgdaSpace{}%
\AgdaSymbol{(}\AgdaInductiveConstructor{ι}\AgdaSpace{}%
\AgdaBound{x}%
\>[12]\AgdaSymbol{)}\AgdaSpace{}%
\AgdaSymbol{=}\AgdaSpace{}%
\AgdaInductiveConstructor{ι}\AgdaSpace{}%
\AgdaBound{x}\<%
\\
%
\>[2]\AgdaOperator{\AgdaFunction{\AgdaUnderscore{}ᵗ}}\AgdaSpace{}%
\AgdaSymbol{(}\AgdaBound{s}\AgdaSpace{}%
\AgdaOperator{\AgdaInductiveConstructor{⊗}}\AgdaSpace{}%
\AgdaBound{s₁}\AgdaSymbol{)}\AgdaSpace{}%
\AgdaSymbol{=}\AgdaSpace{}%
\AgdaSymbol{(}\AgdaBound{s₁}\AgdaSpace{}%
\AgdaOperator{\AgdaFunction{ᵗ}}\AgdaSymbol{)}\AgdaSpace{}%
\AgdaOperator{\AgdaInductiveConstructor{⊗}}\AgdaSpace{}%
\AgdaSymbol{(}\AgdaBound{s}\AgdaSpace{}%
\AgdaOperator{\AgdaFunction{ᵗ}}\AgdaSymbol{)}\<%
\end{code}


\subsection{Operations on tensors of Arbitrary rank}
In addition to the above reshape operations, some methods which can operate 
directly on multi dimensional tensors are required.
\paragraph{Zip With}\label{para:zipWith} performs point-wise application of a 
given function \AF{f} over two tensors of the same shape. 
%TC:ignore
\begin{code}[hide]%
\>[0]\AgdaKeyword{module}\AgdaSpace{}%
\AgdaModule{Matrix2}\AgdaSpace{}%
\AgdaKeyword{where}\<%
\\
\>[0][@{}l@{\AgdaIndent{0}}]%
\>[2]\AgdaKeyword{open}\AgdaSpace{}%
\AgdaKeyword{import}\AgdaSpace{}%
\AgdaModule{Matrix}\AgdaSpace{}%
\AgdaKeyword{using}\AgdaSpace{}%
\AgdaSymbol{(}\AgdaFunction{Ar}\AgdaSymbol{;}\AgdaSpace{}%
\AgdaDatatype{Shape}\AgdaSymbol{;}\AgdaSpace{}%
\AgdaDatatype{Position}\AgdaSymbol{;}\AgdaSpace{}%
\AgdaOperator{\AgdaInductiveConstructor{\AgdaUnderscore{}⊗\AgdaUnderscore{}}}\AgdaSymbol{)}\<%
\\
%
\>[2]\AgdaKeyword{private}\<%
\\
\>[2][@{}l@{\AgdaIndent{0}}]%
\>[4]\AgdaKeyword{variable}\<%
\\
\>[4][@{}l@{\AgdaIndent{0}}]%
\>[6]\AgdaGeneralizable{X}\AgdaSpace{}%
\AgdaGeneralizable{Y}\AgdaSpace{}%
\AgdaGeneralizable{Z}\AgdaSpace{}%
\AgdaSymbol{:}\AgdaSpace{}%
\AgdaPrimitive{Set}\<%
\\
%
\>[6]\AgdaGeneralizable{s}\AgdaSpace{}%
\AgdaGeneralizable{p}\AgdaSpace{}%
\AgdaSymbol{:}\AgdaSpace{}%
\AgdaDatatype{Shape}\<%
\end{code}
%TC:endignore
\begin{code}%
%
\>[2]\AgdaFunction{zipWith}\AgdaSpace{}%
\AgdaSymbol{:}\AgdaSpace{}%
\AgdaSymbol{(}\AgdaGeneralizable{X}\AgdaSpace{}%
\AgdaSymbol{→}\AgdaSpace{}%
\AgdaGeneralizable{Y}\AgdaSpace{}%
\AgdaSymbol{→}\AgdaSpace{}%
\AgdaGeneralizable{Z}\AgdaSymbol{)}\AgdaSpace{}%
\AgdaSymbol{→}\AgdaSpace{}%
\AgdaFunction{Ar}\AgdaSpace{}%
\AgdaGeneralizable{s}\AgdaSpace{}%
\AgdaGeneralizable{X}\AgdaSpace{}%
\AgdaSymbol{→}\AgdaSpace{}%
\AgdaFunction{Ar}\AgdaSpace{}%
\AgdaGeneralizable{s}\AgdaSpace{}%
\AgdaGeneralizable{Y}\AgdaSpace{}%
\AgdaSymbol{→}\AgdaSpace{}%
\AgdaFunction{Ar}\AgdaSpace{}%
\AgdaGeneralizable{s}\AgdaSpace{}%
\AgdaGeneralizable{Z}\<%
\\
%
\>[2]\AgdaFunction{zipWith}\AgdaSpace{}%
\AgdaBound{f}\AgdaSpace{}%
\AgdaBound{arr₁}\AgdaSpace{}%
\AgdaBound{arr₂}\AgdaSpace{}%
\AgdaBound{pos}\AgdaSpace{}%
\AgdaSymbol{=}\AgdaSpace{}%
\AgdaBound{f}\AgdaSpace{}%
\AgdaSymbol{(}\AgdaBound{arr₁}\AgdaSpace{}%
\AgdaBound{pos}\AgdaSymbol{)}\AgdaSpace{}%
\AgdaSymbol{(}\AgdaBound{arr₂}\AgdaSpace{}%
\AgdaBound{pos}\AgdaSymbol{)}\<%
\end{code}
This has many uses, below is shown one example where zipWith is used
over tensors $x$ and $y$, of shape \AF{(ι n ⊗ ι m)},
to add the values at each position.
This two dimensional shape is defined arbitrarily for ease of readability, 
however, \AF{zipWith} is not restricted on the shape meaning a tensor of any shape can
be used.

\begin{displaymath}
  \text{zipWith  \_+\_}
  \begin{bmatrix}
    x_{1,1} & \dots  & x_{1,n} \\
    \vdots  & \ddots & \vdots \\
    x_{m,1} & \dots  & x_{m,n}
  \end{bmatrix}
  \begin{bmatrix}
    y_{1,1} & \dots  & y_{1,n} \\
    \vdots  & \ddots & \vdots \\
    y_{m,1} & \dots  & y_{m,n}
  \end{bmatrix}
  \equiv 
  \begin{bmatrix}
    x_{1,1} + y_{1,1} & \dots  & x_{1,n} + y_{1,n} \\
    \vdots                  & \ddots & \vdots \\
    x_{m,1} + y_{m,1} & \dots  & x_{m,n} + y_{m,n}
  \end{bmatrix}
\end{displaymath}


\paragraph{Map} is similar to \AF{zipWith}, but operates over only one tensor, 
applying a function \AF{f} to each element.
\begin{code}%
%
\>[2]\AgdaFunction{map}\AgdaSpace{}%
\AgdaSymbol{:}\AgdaSpace{}%
\AgdaSymbol{(}\AgdaBound{f}\AgdaSpace{}%
\AgdaSymbol{:}\AgdaSpace{}%
\AgdaGeneralizable{X}\AgdaSpace{}%
\AgdaSymbol{→}\AgdaSpace{}%
\AgdaGeneralizable{Y}\AgdaSymbol{)}\AgdaSpace{}%
\AgdaSymbol{→}\AgdaSpace{}%
\AgdaFunction{Ar}\AgdaSpace{}%
\AgdaGeneralizable{s}\AgdaSpace{}%
\AgdaGeneralizable{X}\AgdaSpace{}%
\AgdaSymbol{→}\AgdaSpace{}%
\AgdaFunction{Ar}\AgdaSpace{}%
\AgdaGeneralizable{s}\AgdaSpace{}%
\AgdaGeneralizable{Y}\<%
\\
%
\>[2]\AgdaFunction{map}\AgdaSpace{}%
\AgdaBound{f}\AgdaSpace{}%
\AgdaBound{arr}\AgdaSpace{}%
\AgdaBound{i}\AgdaSpace{}%
\AgdaSymbol{=}\AgdaSpace{}%
\AgdaBound{f}\AgdaSpace{}%
\AgdaSymbol{(}\AgdaBound{arr}\AgdaSpace{}%
\AgdaBound{i}\AgdaSymbol{)}\<%
\end{code}
The functions \AF{nest} and \AF{unnest} can then be defined to create an 
isomorphism between tensors of the form \AF{Ar (s ⊗ p) X} and nested tensors 
of the form \AF{A s (Ar p X)}.
This allows for the definition of a new function \AF{mapLeft} which can apply a
given function to each \AF{p} shaped sub tensor.

\begin{code}%
%
\>[2]\AgdaFunction{nest}\AgdaSpace{}%
\AgdaSymbol{:}\AgdaSpace{}%
\AgdaFunction{Ar}\AgdaSpace{}%
\AgdaSymbol{(}\AgdaGeneralizable{s}\AgdaSpace{}%
\AgdaOperator{\AgdaInductiveConstructor{⊗}}\AgdaSpace{}%
\AgdaGeneralizable{p}\AgdaSymbol{)}\AgdaSpace{}%
\AgdaGeneralizable{X}\AgdaSpace{}%
\AgdaSymbol{→}\AgdaSpace{}%
\AgdaFunction{Ar}\AgdaSpace{}%
\AgdaGeneralizable{s}\AgdaSpace{}%
\AgdaSymbol{(}\AgdaFunction{Ar}\AgdaSpace{}%
\AgdaGeneralizable{p}\AgdaSpace{}%
\AgdaGeneralizable{X}\AgdaSymbol{)}\<%
\\
%
\>[2]\AgdaFunction{nest}\AgdaSpace{}%
\AgdaBound{arr}\AgdaSpace{}%
\AgdaBound{i}\AgdaSpace{}%
\AgdaBound{j}\AgdaSpace{}%
\AgdaSymbol{=}\AgdaSpace{}%
\AgdaBound{arr}\AgdaSpace{}%
\AgdaSymbol{(}\AgdaBound{i}\AgdaSpace{}%
\AgdaOperator{\AgdaInductiveConstructor{⊗}}\AgdaSpace{}%
\AgdaBound{j}\AgdaSymbol{)}\<%
\\
%
\\[\AgdaEmptyExtraSkip]%
%
\>[2]\AgdaFunction{unnest}\AgdaSpace{}%
\AgdaSymbol{:}\AgdaSpace{}%
\AgdaFunction{Ar}\AgdaSpace{}%
\AgdaGeneralizable{s}\AgdaSpace{}%
\AgdaSymbol{(}\AgdaFunction{Ar}\AgdaSpace{}%
\AgdaGeneralizable{p}\AgdaSpace{}%
\AgdaGeneralizable{X}\AgdaSymbol{)}\AgdaSpace{}%
\AgdaSymbol{→}\AgdaSpace{}%
\AgdaFunction{Ar}\AgdaSpace{}%
\AgdaSymbol{(}\AgdaGeneralizable{s}\AgdaSpace{}%
\AgdaOperator{\AgdaInductiveConstructor{⊗}}\AgdaSpace{}%
\AgdaGeneralizable{p}\AgdaSymbol{)}\AgdaSpace{}%
\AgdaGeneralizable{X}\<%
\\
%
\>[2]\AgdaFunction{unnest}\AgdaSpace{}%
\AgdaBound{arr}\AgdaSpace{}%
\AgdaSymbol{(}\AgdaBound{i}\AgdaSpace{}%
\AgdaOperator{\AgdaInductiveConstructor{⊗}}\AgdaSpace{}%
\AgdaBound{j}\AgdaSymbol{)}\AgdaSpace{}%
\AgdaSymbol{=}\AgdaSpace{}%
\AgdaBound{arr}\AgdaSpace{}%
\AgdaBound{i}\AgdaSpace{}%
\AgdaBound{j}\<%
\\
%
\\[\AgdaEmptyExtraSkip]%
%
\>[2]\AgdaFunction{mapLeft}\AgdaSpace{}%
\AgdaSymbol{:}\AgdaSpace{}%
\AgdaSymbol{∀}\AgdaSpace{}%
\AgdaSymbol{\{}\AgdaBound{s}\AgdaSpace{}%
\AgdaBound{p}\AgdaSpace{}%
\AgdaBound{t}\AgdaSpace{}%
\AgdaSymbol{:}\AgdaSpace{}%
\AgdaDatatype{Shape}\AgdaSymbol{\}}\AgdaSpace{}%
\AgdaSymbol{→}\AgdaSpace{}%
\AgdaSymbol{(}\AgdaFunction{Ar}\AgdaSpace{}%
\AgdaBound{p}\AgdaSpace{}%
\AgdaGeneralizable{X}\AgdaSpace{}%
\AgdaSymbol{→}\AgdaSpace{}%
\AgdaFunction{Ar}\AgdaSpace{}%
\AgdaBound{t}\AgdaSpace{}%
\AgdaGeneralizable{Y}\AgdaSymbol{)}\AgdaSpace{}%
\AgdaSymbol{→}\AgdaSpace{}%
\AgdaFunction{Ar}\AgdaSpace{}%
\AgdaSymbol{(}\AgdaBound{s}\AgdaSpace{}%
\AgdaOperator{\AgdaInductiveConstructor{⊗}}\AgdaSpace{}%
\AgdaBound{p}\AgdaSymbol{)}\AgdaSpace{}%
\AgdaGeneralizable{X}\AgdaSpace{}%
\AgdaSymbol{→}\AgdaSpace{}%
\AgdaFunction{Ar}\AgdaSpace{}%
\AgdaSymbol{(}\AgdaBound{s}\AgdaSpace{}%
\AgdaOperator{\AgdaInductiveConstructor{⊗}}\AgdaSpace{}%
\AgdaBound{t}\AgdaSymbol{)}\AgdaSpace{}%
\AgdaGeneralizable{Y}\<%
\\
%
\>[2]\AgdaFunction{mapLeft}\AgdaSpace{}%
\AgdaBound{f}\AgdaSpace{}%
\AgdaBound{arr}\AgdaSpace{}%
\AgdaSymbol{=}\AgdaSpace{}%
\AgdaFunction{unnest}\AgdaSpace{}%
\AgdaSymbol{(}\AgdaFunction{map}\AgdaSpace{}%
\AgdaBound{f}\AgdaSpace{}%
\AgdaSymbol{(}\AgdaFunction{nest}\AgdaSpace{}%
\AgdaBound{arr}\AgdaSymbol{))}\<%
\end{code}


\clearpage
\subsection{FFT}
Given the above operations, it is now possible to begin forming a definition for
the FFT.

Looking at the basic derivation of the Cooley Tukey FFT over an input vector
defined in Equation \ref{eq:FFTDefinitionFromDFT}, three distinct sections can
be observed.
\begin{align}
    X_{j_1r_1+j_0}
      &=\underbrace{\sum^{r_2-1}_{k_0=0}{
        \left[
          \underbrace{
            \left(
              \underbrace{
                \sum^{r_1-1}_{k_1=0}x_{k_1r_2+k_0}\omega_{r_1}^{k_1j_0}
              }_{Section A} \right
            ) \omega_{r_1r_2}^{k_0j_1}
          }_{Section B}
        \right]
        \omega_{r_2}^{k_0j_1}
      }}_{Section C}
    \label{eq:FFTDefinitionLabeled}
\end{align}
Section A takes the form of a DFT of length $r_1$.
In vector form, the first element of the input for this DFT is located at index $k₀$, 
each subsequent input is then found taken by making a step of $r_2$, $r_1$ times.
In vector form this is a relatively complex input to reason upon, when we can 
instead consider our input in tensor form, initially, as a tensor of shape \AF{ι r₁ ⊗ ι r₂}.
In this form, Section A can be considered to apply the DFT to each column of the
input tensor.
Similar to Section A, Section C then takes the form of a DFT of length $r_2$.
In our \AF{ι r₁ ⊗ ι r₂} tensor form, this is equivalent to the application of 
the DFT over the rows of the result of section B.

Section B differs to section A and C, and applies what are generally referred to
as, the twiddle factors.
In tensor form this section is equivalent to a point wise multiplication 
over each element from Section A.
This step can be represented in Agda as \AF{zipWith \_*\_}, on a tensor containing 
these "twiddle factors".

%TC:ignore
\begin{code}[hide]%
\>[0]\AgdaKeyword{module}\AgdaSpace{}%
\AgdaModule{FFT2}\AgdaSpace{}%
\AgdaSymbol{(}\AgdaBound{real}\AgdaSpace{}%
\AgdaSymbol{:}\AgdaSpace{}%
\AgdaRecord{Real}\AgdaSymbol{)}\AgdaSpace{}%
\AgdaSymbol{(}\AgdaBound{cplx}\AgdaSpace{}%
\AgdaSymbol{:}\AgdaSpace{}%
\AgdaRecord{Cplx}\AgdaSpace{}%
\AgdaBound{real}\AgdaSymbol{)}\AgdaSpace{}%
\AgdaKeyword{where}\<%
\\
\>[0][@{}l@{\AgdaIndent{0}}]%
\>[2]\AgdaKeyword{open}\AgdaSpace{}%
\AgdaKeyword{import}\AgdaSpace{}%
\AgdaModule{Matrix}\AgdaSpace{}%
\AgdaKeyword{using}\AgdaSpace{}%
\AgdaSymbol{(}\AgdaFunction{Ar}\AgdaSymbol{;}\AgdaSpace{}%
\AgdaDatatype{Shape}\AgdaSymbol{;}\AgdaSpace{}%
\AgdaDatatype{Position}\AgdaSymbol{;}\AgdaSpace{}%
\AgdaOperator{\AgdaInductiveConstructor{\AgdaUnderscore{}⊗\AgdaUnderscore{}}}\AgdaSymbol{;}\AgdaSpace{}%
\AgdaInductiveConstructor{ι}\AgdaSymbol{;}\AgdaSpace{}%
\AgdaFunction{length}\AgdaSymbol{;}\AgdaSpace{}%
\AgdaFunction{mapLeft}\AgdaSymbol{;}\AgdaSpace{}%
\AgdaFunction{zipWith}\AgdaSymbol{)}\<%
\\
%
\>[2]\AgdaKeyword{open}\AgdaSpace{}%
\AgdaKeyword{import}\AgdaSpace{}%
\AgdaModule{Matrix.NonZero}\<%
\\
%
\>[2]\AgdaKeyword{open}\AgdaSpace{}%
\AgdaKeyword{import}\AgdaSpace{}%
\AgdaModule{Matrix.Reshape}\AgdaSpace{}%
\AgdaKeyword{using}\AgdaSpace{}%
\AgdaSymbol{(}\AgdaOperator{\AgdaFunction{\AgdaUnderscore{}⟨\AgdaUnderscore{}⟩}}\AgdaSymbol{;}\AgdaSpace{}%
\AgdaFunction{♯}\AgdaSymbol{;}\AgdaSpace{}%
\AgdaFunction{reshape}\AgdaSymbol{;}\AgdaSpace{}%
\AgdaInductiveConstructor{swap}\AgdaSymbol{)}\AgdaSpace{}%
\AgdaKeyword{renaming}\AgdaSpace{}%
\AgdaSymbol{(}\AgdaFunction{recursive-transpose}\AgdaSpace{}%
\AgdaSymbol{to}\AgdaSpace{}%
\AgdaFunction{\AgdaUnderscore{}ᵗ}\AgdaSymbol{)}\<%
\\
%
\>[2]\AgdaKeyword{open}\AgdaSpace{}%
\AgdaKeyword{import}\AgdaSpace{}%
\AgdaModule{FFT}\AgdaSpace{}%
\AgdaBound{real}\AgdaSpace{}%
\AgdaBound{cplx}\AgdaSpace{}%
\AgdaKeyword{using}\AgdaSpace{}%
\AgdaSymbol{(}\AgdaFunction{iota}\AgdaSymbol{)}\AgdaSpace{}%
\AgdaKeyword{renaming}\AgdaSpace{}%
\AgdaSymbol{(}\AgdaFunction{DFT′}\AgdaSpace{}%
\AgdaSymbol{to}\AgdaSpace{}%
\AgdaFunction{DFT}\AgdaSymbol{)}\<%
\\
%
\>[2]\AgdaKeyword{open}\AgdaSpace{}%
\AgdaKeyword{import}\AgdaSpace{}%
\AgdaModule{Data.Nat.Base}\AgdaSpace{}%
\AgdaKeyword{using}\AgdaSpace{}%
\AgdaSymbol{()}\AgdaSpace{}%
\AgdaKeyword{renaming}\AgdaSpace{}%
\AgdaSymbol{(}\AgdaOperator{\AgdaPrimitive{\AgdaUnderscore{}*\AgdaUnderscore{}}}\AgdaSpace{}%
\AgdaSymbol{to}\AgdaSpace{}%
\AgdaOperator{\AgdaPrimitive{\AgdaUnderscore{}*ₙ\AgdaUnderscore{}}}\AgdaSymbol{)}\<%
\\
%
\>[2]\AgdaKeyword{open}\AgdaSpace{}%
\AgdaModule{Cplx}\AgdaSpace{}%
\AgdaBound{cplx}\AgdaSpace{}%
\AgdaKeyword{using}\AgdaSpace{}%
\AgdaSymbol{(}\AgdaField{ℂ}\AgdaSymbol{;}\AgdaSpace{}%
\AgdaOperator{\AgdaField{\AgdaUnderscore{}*\AgdaUnderscore{}}}\AgdaSymbol{;}\AgdaSpace{}%
\AgdaField{-ω}\AgdaSymbol{;}\AgdaSpace{}%
\AgdaOperator{\AgdaField{e\textasciicircum{}i\AgdaUnderscore{}}}\AgdaSymbol{;}\AgdaSpace{}%
\AgdaOperator{\AgdaField{\AgdaUnderscore{}+\AgdaUnderscore{}}}\AgdaSymbol{;}\AgdaSpace{}%
\AgdaFunction{0ℂ}\AgdaSymbol{;}\AgdaSpace{}%
\AgdaField{+-*-isCommutativeRing}\AgdaSymbol{)}\<%
\\
%
\\[\AgdaEmptyExtraSkip]%
%
\>[2]\AgdaKeyword{private}\<%
\\
\>[2][@{}l@{\AgdaIndent{0}}]%
\>[4]\AgdaKeyword{variable}\<%
\\
\>[4][@{}l@{\AgdaIndent{0}}]%
\>[6]\AgdaGeneralizable{s}\AgdaSpace{}%
\AgdaGeneralizable{p}\AgdaSpace{}%
\AgdaSymbol{:}\AgdaSpace{}%
\AgdaDatatype{Shape}\<%
\\
%
\>[6]\AgdaGeneralizable{r₁}\AgdaSpace{}%
\AgdaGeneralizable{r₂}\AgdaSpace{}%
\AgdaSymbol{:}\AgdaSpace{}%
\AgdaDatatype{ℕ}\<%
\\
%
\\[\AgdaEmptyExtraSkip]%
%
\>[2]\AgdaFunction{Ar∔}\AgdaSpace{}%
\AgdaSymbol{:}\AgdaSpace{}%
\AgdaDatatype{Shape}\AgdaSpace{}%
\AgdaSymbol{→}\AgdaSpace{}%
\AgdaPrimitive{Set}\AgdaSpace{}%
\AgdaSymbol{→}\AgdaSpace{}%
\AgdaPrimitive{Set}\<%
\\
%
\>[2]\AgdaFunction{Ar∔}\AgdaSpace{}%
\AgdaSymbol{=}\AgdaSpace{}%
\AgdaFunction{Ar}\<%
\end{code}
%TC:endignore
\begin{code}%
%
\>[2]\AgdaFunction{2D-twiddles}\AgdaSpace{}%
\AgdaSymbol{:}\AgdaSpace{}%
\AgdaFunction{Ar∔}\AgdaSpace{}%
\AgdaSymbol{(}\AgdaInductiveConstructor{ι}\AgdaSpace{}%
\AgdaGeneralizable{r₂}\AgdaSpace{}%
\AgdaOperator{\AgdaInductiveConstructor{⊗}}\AgdaSpace{}%
\AgdaInductiveConstructor{ι}\AgdaSpace{}%
\AgdaGeneralizable{r₁}\AgdaSymbol{)}\AgdaSpace{}%
\AgdaField{ℂ}\<%
\\
%
\>[2]\AgdaFunction{2D-twiddles}\AgdaSpace{}%
\AgdaSymbol{\{}\AgdaBound{r₁}\AgdaSymbol{\}}\AgdaSpace{}%
\AgdaSymbol{\{}\AgdaBound{r₂}\AgdaSymbol{\}}\AgdaSpace{}%
\AgdaSymbol{(}\AgdaBound{k₀}\AgdaSpace{}%
\AgdaOperator{\AgdaInductiveConstructor{⊗}}\AgdaSpace{}%
\AgdaBound{j₁}\AgdaSymbol{)}\AgdaSpace{}%
\AgdaSymbol{=}\AgdaSpace{}%
\AgdaField{-ω}\AgdaSpace{}%
\AgdaSymbol{(}\AgdaBound{r₂}\AgdaSpace{}%
\AgdaOperator{\AgdaPrimitive{*ₙ}}\AgdaSpace{}%
\AgdaBound{r₁}\AgdaSymbol{)}\AgdaSpace{}%
\AgdaSymbol{(}\AgdaFunction{iota}\AgdaSpace{}%
\AgdaBound{k₀}\AgdaSpace{}%
\AgdaOperator{\AgdaPrimitive{*ₙ}}\AgdaSpace{}%
\AgdaFunction{iota}\AgdaSpace{}%
\AgdaBound{j₁}\AgdaSpace{}%
\AgdaSymbol{)}\<%
\end{code}
%TC:ignore
\begin{code}[hide]%
\>[2][@{}l@{\AgdaIndent{1}}]%
\>[6]\AgdaKeyword{where}\<%
\\
\>[6][@{}l@{\AgdaIndent{0}}]%
\>[8]\AgdaKeyword{postulate}\<%
\\
\>[8][@{}l@{\AgdaIndent{0}}]%
\>[10]\AgdaKeyword{instance}\<%
\\
\>[10][@{}l@{\AgdaIndent{0}}]%
\>[12]\AgdaSymbol{\AgdaUnderscore{}}\AgdaSpace{}%
\AgdaSymbol{:}\AgdaSpace{}%
\AgdaRecord{NonZero}\AgdaSpace{}%
\AgdaSymbol{(}\AgdaBound{r₂}\AgdaSpace{}%
\AgdaOperator{\AgdaPrimitive{*ₙ}}\AgdaSpace{}%
\AgdaBound{r₁}\AgdaSymbol{)}\<%
\end{code}
%TC:endignore

It can be seen here that when computing these twiddle factors, the number of 
elements in the input vector is used as the base value.
It is defined previously that this base value cannot be zero, and so this 
step imposes the requirement that the FFT can only operate upon tensors with one or more elements.

Using this twiddle tensor, the definition for the two dimensional FFT is generated
by forming each section into its own step.
Of note in the definition below are the three uses of swap.
The first swap allows DFT′ to map over the columns of the input array,
while the next inverts this and allows map to be performed over the rows.
The final swap is performed because, given an input in row major order, the 
result of the FFT is produced in column major order. 
For this to be represented correctly when flatten, \AF{♭}, is applied the output
tensor must be transposed, this is performed over two dimensions with \AF{swap}.
Because of this third swap, the shape of the output tensor is transposed, as
indicated to in the function type by \AF{ᵗ}.

\begin{code}%
%
\>[2]\AgdaFunction{2D-FFT}\AgdaSpace{}%
\AgdaSymbol{:}\AgdaSpace{}%
\AgdaFunction{Ar∔}\AgdaSpace{}%
\AgdaSymbol{(}\AgdaInductiveConstructor{ι}\AgdaSpace{}%
\AgdaGeneralizable{r₁}\AgdaSpace{}%
\AgdaOperator{\AgdaInductiveConstructor{⊗}}\AgdaSpace{}%
\AgdaInductiveConstructor{ι}\AgdaSpace{}%
\AgdaGeneralizable{r₂}\AgdaSymbol{)}\AgdaSpace{}%
\AgdaField{ℂ}\AgdaSpace{}%
\AgdaSymbol{→}\AgdaSpace{}%
\AgdaFunction{Ar∔}\AgdaSpace{}%
\AgdaSymbol{((}\AgdaInductiveConstructor{ι}\AgdaSpace{}%
\AgdaGeneralizable{r₁}\AgdaSpace{}%
\AgdaOperator{\AgdaInductiveConstructor{⊗}}\AgdaSpace{}%
\AgdaInductiveConstructor{ι}\AgdaSpace{}%
\AgdaGeneralizable{r₂}\AgdaSymbol{)}\AgdaSpace{}%
\AgdaOperator{\AgdaFunction{ᵗ}}\AgdaSymbol{)}\AgdaSpace{}%
\AgdaField{ℂ}\<%
\\
%
\>[2]\AgdaFunction{2D-FFT}\AgdaSpace{}%
\AgdaSymbol{\{}\AgdaBound{r₁}\AgdaSymbol{\}}\AgdaSpace{}%
\AgdaSymbol{\{}\AgdaBound{r₂}\AgdaSymbol{\}}\AgdaSpace{}%
\AgdaBound{arr}\AgdaSpace{}%
\AgdaSymbol{=}\<%
\\
\>[2][@{}l@{\AgdaIndent{0}}]%
\>[6]\AgdaKeyword{let}\<%
\\
\>[6][@{}l@{\AgdaIndent{0}}]%
\>[10]\AgdaBound{innerDFTapplied}%
\>[33]\AgdaSymbol{=}\AgdaSpace{}%
\AgdaFunction{mapLeft}\AgdaSpace{}%
\AgdaSymbol{(}\AgdaFunction{DFT}\AgdaSpace{}%
\AgdaSymbol{\{}\AgdaBound{r₁}\AgdaSymbol{\})}\AgdaSpace{}%
\AgdaSymbol{(}\AgdaFunction{reshape}\AgdaSpace{}%
\AgdaInductiveConstructor{swap}\AgdaSpace{}%
\AgdaBound{arr}\AgdaSymbol{)}\<%
\\
%
\>[10]\AgdaBound{twiddleFactorsApplied}%
\>[33]\AgdaSymbol{=}\AgdaSpace{}%
\AgdaFunction{zipWith}\AgdaSpace{}%
\AgdaOperator{\AgdaField{\AgdaUnderscore{}*\AgdaUnderscore{}}}%
\>[49]\AgdaBound{innerDFTapplied}\AgdaSpace{}%
\AgdaFunction{2D-twiddles}\<%
\\
%
\>[10]\AgdaBound{outerDFTapplied}%
\>[33]\AgdaSymbol{=}\AgdaSpace{}%
\AgdaFunction{mapLeft}\AgdaSpace{}%
\AgdaSymbol{(}\AgdaFunction{DFT}\AgdaSpace{}%
\AgdaSymbol{\{}\AgdaBound{r₂}\AgdaSymbol{\})}\AgdaSpace{}%
\AgdaSymbol{(}\AgdaFunction{reshape}\AgdaSpace{}%
\AgdaInductiveConstructor{swap}\AgdaSpace{}%
\AgdaBound{twiddleFactorsApplied}\AgdaSymbol{)}\<%
\\
%
\>[6]\AgdaKeyword{in}%
\>[10]\AgdaFunction{reshape}\AgdaSpace{}%
\AgdaInductiveConstructor{swap}\AgdaSpace{}%
\AgdaBound{outerDFTapplied}\<%
\end{code}
%TC:ignore
\begin{code}[hide]%
%
\>[6]\AgdaKeyword{where}\<%
\\
\>[6][@{}l@{\AgdaIndent{0}}]%
\>[8]\AgdaKeyword{postulate}\<%
\\
\>[8][@{}l@{\AgdaIndent{0}}]%
\>[10]\AgdaKeyword{instance}\<%
\\
\>[10][@{}l@{\AgdaIndent{0}}]%
\>[12]\AgdaSymbol{\AgdaUnderscore{}}\AgdaSpace{}%
\AgdaSymbol{:}\AgdaSpace{}%
\AgdaRecord{NonZero}\AgdaSpace{}%
\AgdaBound{r₁}\<%
\\
%
\>[12]\AgdaSymbol{\AgdaUnderscore{}}\AgdaSpace{}%
\AgdaSymbol{:}\AgdaSpace{}%
\AgdaRecord{NonZero}\AgdaSpace{}%
\AgdaBound{r₂}\<%
\\
%
\>[12]\AgdaSymbol{\AgdaUnderscore{}}\AgdaSpace{}%
\AgdaSymbol{:}\AgdaSpace{}%
\AgdaDatatype{NonZeroₛ}\AgdaSpace{}%
\AgdaSymbol{(}\AgdaInductiveConstructor{ι}\AgdaSpace{}%
\AgdaBound{r₂}\AgdaSpace{}%
\AgdaOperator{\AgdaInductiveConstructor{⊗}}\AgdaSpace{}%
\AgdaInductiveConstructor{ι}\AgdaSpace{}%
\AgdaBound{r₁}\AgdaSymbol{)}\<%
\end{code}
%TC:endignore

\begin{AgdaAlign}

Given knowledge that the DFT should be equivalent to the FFT, the two dimensional
definition can then be improved by instead applying the FFT at each step.
This requires the slight modification of the 2D-FFT implementation such that it 
accepts a tensor of any shape \AF{s} as input.

The definition for the twiddle factors must also be redefined, such that twiddles
can be computed for any shape with more than two dimensions.
It is easy to see, that the previous base of the roots of unity, $r₁\times r₂$,
maps directly to the \AF{length} of any given tensor.
To calculate the power of the root of unity, we can define \AF{offset-prod}.
This flattens the values of \AF{k} and \AF{j}, before multiplying them together to
calculate the power.
\begin{code}%
%
\>[2]\AgdaFunction{offset-prod}\AgdaSpace{}%
\AgdaSymbol{:}\AgdaSpace{}%
\AgdaDatatype{Position}\AgdaSpace{}%
\AgdaSymbol{(}\AgdaGeneralizable{s}\AgdaSpace{}%
\AgdaOperator{\AgdaInductiveConstructor{⊗}}\AgdaSpace{}%
\AgdaGeneralizable{p}\AgdaSymbol{)}\AgdaSpace{}%
\AgdaSymbol{→}\AgdaSpace{}%
\AgdaDatatype{ℕ}\<%
\\
%
\>[2]\AgdaFunction{offset-prod}\AgdaSpace{}%
\AgdaSymbol{(}\AgdaBound{k}\AgdaSpace{}%
\AgdaOperator{\AgdaInductiveConstructor{⊗}}\AgdaSpace{}%
\AgdaBound{j}\AgdaSymbol{)}\AgdaSpace{}%
\AgdaSymbol{=}\AgdaSpace{}%
\AgdaFunction{iota}\AgdaSpace{}%
\AgdaSymbol{(}\AgdaBound{k}\AgdaSpace{}%
\AgdaOperator{\AgdaFunction{⟨}}\AgdaSpace{}%
\AgdaFunction{♯}\AgdaSpace{}%
\AgdaOperator{\AgdaFunction{⟩}}\AgdaSymbol{)}\AgdaSpace{}%
\AgdaOperator{\AgdaPrimitive{*ₙ}}\AgdaSpace{}%
\AgdaFunction{iota}\AgdaSpace{}%
\AgdaSymbol{(}\AgdaBound{j}\AgdaSpace{}%
\AgdaOperator{\AgdaFunction{⟨}}\AgdaSpace{}%
\AgdaFunction{♯}\AgdaSpace{}%
\AgdaOperator{\AgdaFunction{⟩}}\AgdaSymbol{)}\<%
\\
\>[0]\<%
\\
%
\>[2]\AgdaFunction{twiddles}\AgdaSpace{}%
\AgdaSymbol{:}\AgdaSpace{}%
\AgdaFunction{Ar∔}\AgdaSpace{}%
\AgdaSymbol{(}\AgdaGeneralizable{s}\AgdaSpace{}%
\AgdaOperator{\AgdaInductiveConstructor{⊗}}\AgdaSpace{}%
\AgdaGeneralizable{p}\AgdaSymbol{)}\AgdaSpace{}%
\AgdaField{ℂ}\<%
\\
%
\>[2]\AgdaFunction{twiddles}\AgdaSpace{}%
\AgdaSymbol{\{}\AgdaBound{s}\AgdaSymbol{\}}\AgdaSpace{}%
\AgdaSymbol{\{}\AgdaBound{p}\AgdaSymbol{\}}\AgdaSpace{}%
\AgdaBound{i}\AgdaSpace{}%
\AgdaSymbol{=}\AgdaSpace{}%
\AgdaField{-ω}\AgdaSpace{}%
\AgdaSymbol{(}\AgdaFunction{length}\AgdaSpace{}%
\AgdaSymbol{(}\AgdaBound{s}\AgdaSpace{}%
\AgdaOperator{\AgdaInductiveConstructor{⊗}}\AgdaSpace{}%
\AgdaBound{p}\AgdaSymbol{))}\AgdaSpace{}%
\AgdaSymbol{(}\AgdaFunction{offset-prod}\AgdaSpace{}%
\AgdaBound{i}\AgdaSymbol{)}\<%
\end{code}
%TC:ignore
\begin{code}[hide]%
\>[2][@{}l@{\AgdaIndent{1}}]%
\>[6]\AgdaKeyword{where}\<%
\\
\>[6][@{}l@{\AgdaIndent{0}}]%
\>[8]\AgdaKeyword{postulate}\<%
\\
\>[8][@{}l@{\AgdaIndent{0}}]%
\>[10]\AgdaKeyword{instance}\<%
\\
\>[10][@{}l@{\AgdaIndent{0}}]%
\>[12]\AgdaSymbol{\AgdaUnderscore{}}\AgdaSpace{}%
\AgdaSymbol{:}\AgdaSpace{}%
\AgdaRecord{NonZero}\AgdaSpace{}%
\AgdaSymbol{(}\AgdaFunction{length}\AgdaSpace{}%
\AgdaBound{s}\AgdaSpace{}%
\AgdaOperator{\AgdaPrimitive{*ₙ}}\AgdaSpace{}%
\AgdaFunction{length}\AgdaSpace{}%
\AgdaBound{p}\AgdaSymbol{)}\<%
\end{code}
%TC:endignore
\end{AgdaAlign}

The definition of this general twiddle tensor now allows for FFT to be defined
for an input of any shape.
The same problem of the output shape must then be dealt with again.
As the result of the FFT is in column major order, the result must be transposed
for flatten to represent it correctly.
This can be achieved through the application of \AF{swap} to \AF{outerDFTapplied}
before returning, as each sub tensor is the result of the application of the FFT and
will be transposed.

\begin{AgdaSuppressSpace}
\begin{code}%
%
\>[2]\AgdaFunction{FFT}\AgdaSpace{}%
\AgdaSymbol{:}\AgdaSpace{}%
\AgdaFunction{Ar∔}\AgdaSpace{}%
\AgdaGeneralizable{s}\AgdaSpace{}%
\AgdaField{ℂ}\AgdaSpace{}%
\AgdaSymbol{→}\AgdaSpace{}%
\AgdaFunction{Ar∔}\AgdaSpace{}%
\AgdaSymbol{(}\AgdaGeneralizable{s}\AgdaSpace{}%
\AgdaOperator{\AgdaFunction{ᵗ}}\AgdaSymbol{)}\AgdaSpace{}%
\AgdaField{ℂ}\<%
\\
%
\>[2]\AgdaFunction{FFT}\AgdaSpace{}%
\AgdaSymbol{\{}\AgdaInductiveConstructor{ι}\AgdaSpace{}%
\AgdaBound{N}\AgdaSymbol{\}}\AgdaSpace{}%
\AgdaBound{arr}\AgdaSpace{}%
\AgdaSymbol{=}\AgdaSpace{}%
\AgdaFunction{DFT}\AgdaSpace{}%
\AgdaBound{arr}\<%
\end{code}
%TC:ignore
\begin{code}[hide]%
\>[2][@{}l@{\AgdaIndent{1}}]%
\>[4]\AgdaKeyword{where}\<%
\\
\>[4][@{}l@{\AgdaIndent{0}}]%
\>[6]\AgdaKeyword{postulate}\<%
\\
\>[6][@{}l@{\AgdaIndent{0}}]%
\>[8]\AgdaKeyword{instance}\<%
\\
\>[8][@{}l@{\AgdaIndent{0}}]%
\>[10]\AgdaPostulate{\AgdaUnderscore{}}\AgdaSpace{}%
\AgdaSymbol{:}\AgdaSpace{}%
\AgdaRecord{NonZero}\AgdaSpace{}%
\AgdaBound{N}\<%
\end{code}
%TC:endignore
\begin{code}%
%
\>[2]\AgdaFunction{FFT}%
\>[1381I]\AgdaSymbol{\{}\AgdaBound{r₁}\AgdaSpace{}%
\AgdaOperator{\AgdaInductiveConstructor{⊗}}\AgdaSpace{}%
\AgdaBound{r₂}\AgdaSymbol{\}}\AgdaSpace{}%
\AgdaBound{arr}\AgdaSpace{}%
\AgdaSymbol{=}\<%
\\
\>[.][@{}l@{}]\<[1381I]%
\>[6]\AgdaKeyword{let}\<%
\\
\>[6][@{}l@{\AgdaIndent{0}}]%
\>[10]\AgdaBound{innerDFTapplied}%
\>[33]\AgdaSymbol{=}\AgdaSpace{}%
\AgdaFunction{mapLeft}\AgdaSpace{}%
\AgdaFunction{FFT}\AgdaSpace{}%
\AgdaSymbol{(}\AgdaFunction{reshape}\AgdaSpace{}%
\AgdaInductiveConstructor{swap}\AgdaSpace{}%
\AgdaBound{arr}\AgdaSymbol{)}\<%
\\
%
\>[10]\AgdaBound{twiddleFactorsApplied}%
\>[33]\AgdaSymbol{=}\AgdaSpace{}%
\AgdaFunction{zipWith}\AgdaSpace{}%
\AgdaOperator{\AgdaField{\AgdaUnderscore{}*\AgdaUnderscore{}}}%
\>[49]\AgdaBound{innerDFTapplied}\AgdaSpace{}%
\AgdaFunction{twiddles}\<%
\\
%
\>[10]\AgdaBound{outerDFTapplied}%
\>[33]\AgdaSymbol{=}\AgdaSpace{}%
\AgdaFunction{mapLeft}\AgdaSpace{}%
\AgdaFunction{FFT}\AgdaSpace{}%
\AgdaSymbol{(}\AgdaFunction{reshape}\AgdaSpace{}%
\AgdaInductiveConstructor{swap}\AgdaSpace{}%
\AgdaBound{twiddleFactorsApplied}\AgdaSymbol{)}\<%
\\
%
\>[6]\AgdaKeyword{in}%
\>[10]\AgdaFunction{reshape}\AgdaSpace{}%
\AgdaInductiveConstructor{swap}\AgdaSpace{}%
\AgdaBound{outerDFTapplied}\<%
\end{code}
%TC:ignore
\begin{code}[hide]%
%
\>[6]\AgdaKeyword{where}\<%
\\
\>[6][@{}l@{\AgdaIndent{0}}]%
\>[8]\AgdaKeyword{postulate}\<%
\\
\>[8][@{}l@{\AgdaIndent{0}}]%
\>[10]\AgdaKeyword{instance}\<%
\\
\>[10][@{}l@{\AgdaIndent{0}}]%
\>[12]\AgdaPostulate{\AgdaUnderscore{}}\AgdaSpace{}%
\AgdaSymbol{:}\AgdaSpace{}%
\AgdaDatatype{NonZeroₛ}\AgdaSpace{}%
\AgdaBound{r₁}\<%
\\
%
\>[12]\AgdaPostulate{\AgdaUnderscore{}}\AgdaSpace{}%
\AgdaSymbol{:}\AgdaSpace{}%
\AgdaDatatype{NonZeroₛ}\AgdaSpace{}%
\AgdaBound{r₂}\<%
\\
%
\>[12]\AgdaPostulate{\AgdaUnderscore{}}\AgdaSpace{}%
\AgdaSymbol{:}\AgdaSpace{}%
\AgdaDatatype{NonZeroₛ}\AgdaSpace{}%
\AgdaSymbol{(}\AgdaBound{r₂}\AgdaSpace{}%
\AgdaOperator{\AgdaInductiveConstructor{⊗}}\AgdaSpace{}%
\AgdaSymbol{(}\AgdaBound{r₁}\AgdaSpace{}%
\AgdaOperator{\AgdaFunction{ᵗ}}\AgdaSymbol{))}\<%
\end{code}
%TC:endignore
\end{AgdaSuppressSpace}

As time was invested at the start of the project into a the creation of a language 
on tensors and reshaping, every case of the Cooley Tukey algorithm can be 
represented within the three lines shown above. 
Given a proof of correctness, this generality makes way for further experiments 
into different radix sizes, and combination of radix sizes, to be easily undertaken.

If this was instead written in \verb|C|, or a \verb|C| style language, this level
of generality would be almost impossible.
Any such general, \verb|C| style implementation would require many, low level,
index manipulations.
Without structures such as those defined for here for position, these index manipulations 
become increasingly complex as the radix sizes, and levels of nesting, increase.
This complexity makes it difficult to reason upon any such implementation meaning
garuntees are more challenging to achive.


% Spend some time explaining why expressing the FFT in the way we did is very good - Its a family of cooley tukey algorithms in 3 lines
% - The reason we can do this because we have invsted in these arrays and combinators on them
% - Doing this in C would be hell - we need to spoon feed this to the reader, this FFT definition is very cool!





\begin{code}[hide]%
\>[0]\AgdaKeyword{module}\AgdaSpace{}%
\AgdaModule{ProofWriteup}\AgdaSpace{}%
\AgdaKeyword{where}\<%
\end{code}
\section{Proof of correctness}
Given the above defintion of the FFT, and our previous definition of the DFT, 
a proof of equality between the two can be formed.
This proof consists of two component parts, the first of which is the proposition.
The proposition describes the relationship between the FFT and DFT and is shown below.

\begin{code}[hide]%
\>[0]\AgdaKeyword{open}\AgdaSpace{}%
\AgdaKeyword{import}\AgdaSpace{}%
\AgdaModule{Real}\AgdaSpace{}%
\AgdaKeyword{using}\AgdaSpace{}%
\AgdaSymbol{(}\AgdaRecord{Real}\AgdaSymbol{)}\<%
\\
\>[0]\AgdaKeyword{open}\AgdaSpace{}%
\AgdaKeyword{import}\AgdaSpace{}%
\AgdaModule{Complex}\AgdaSpace{}%
\AgdaKeyword{using}\AgdaSpace{}%
\AgdaSymbol{(}\AgdaRecord{Cplx}\AgdaSymbol{)}\<%
\\
%
\\[\AgdaEmptyExtraSkip]%
\>[0]\AgdaKeyword{import}\AgdaSpace{}%
\AgdaModule{Algebra.Structures}\AgdaSpace{}%
\AgdaSymbol{as}\AgdaSpace{}%
\AgdaModule{AlgebraStructures}\<%
\\
\>[0]\AgdaKeyword{import}\AgdaSpace{}%
\AgdaModule{Algebra.Definitions}\AgdaSpace{}%
\AgdaSymbol{as}\AgdaSpace{}%
\AgdaModule{AlgebraDefinitions}\<%
\\
%
\\[\AgdaEmptyExtraSkip]%
\>[0]\AgdaKeyword{open}\AgdaSpace{}%
\AgdaKeyword{import}\AgdaSpace{}%
\AgdaModule{Relation.Nullary}\<%
\\
\>[0]\AgdaKeyword{import}\AgdaSpace{}%
\AgdaModule{Relation.Binary.PropositionalEquality}\AgdaSpace{}%
\AgdaSymbol{as}\AgdaSpace{}%
\AgdaModule{Eq}\<%
\\
\>[0]\AgdaKeyword{open}\AgdaSpace{}%
\AgdaModule{Eq}\AgdaSpace{}%
\AgdaKeyword{using}\AgdaSpace{}%
\AgdaSymbol{(}\AgdaOperator{\AgdaDatatype{\AgdaUnderscore{}≡\AgdaUnderscore{}}}\AgdaSymbol{;}\AgdaSpace{}%
\AgdaInductiveConstructor{refl}\AgdaSymbol{;}\AgdaSpace{}%
\AgdaFunction{cong}\AgdaSymbol{;}\AgdaSpace{}%
\AgdaFunction{trans}\AgdaSymbol{;}\AgdaSpace{}%
\AgdaFunction{sym}\AgdaSymbol{;}\AgdaSpace{}%
\AgdaFunction{cong₂}\AgdaSymbol{;}\AgdaSpace{}%
\AgdaFunction{subst}\AgdaSymbol{;}\AgdaSpace{}%
\AgdaFunction{cong-app}\AgdaSymbol{;}\AgdaSpace{}%
\AgdaFunction{cong′}\AgdaSymbol{;}\AgdaSpace{}%
\AgdaFunction{icong}\AgdaSymbol{)}\<%
\\
\>[0]\AgdaKeyword{open}\AgdaSpace{}%
\AgdaModule{Eq.≡-Reasoning}\<%
\\
%
\\[\AgdaEmptyExtraSkip]%
\>[0]\AgdaKeyword{module}\AgdaSpace{}%
\AgdaModule{Proposition}\AgdaSpace{}%
\AgdaSymbol{(}\AgdaBound{real}\AgdaSpace{}%
\AgdaSymbol{:}\AgdaSpace{}%
\AgdaRecord{Real}\AgdaSymbol{)}\AgdaSpace{}%
\AgdaSymbol{(}\AgdaBound{cplx}\AgdaSpace{}%
\AgdaSymbol{:}\AgdaSpace{}%
\AgdaRecord{Cplx}\AgdaSpace{}%
\AgdaBound{real}\AgdaSymbol{)}\AgdaSpace{}%
\AgdaKeyword{where}\<%
\\
%
\\[\AgdaEmptyExtraSkip]%
\>[0][@{}l@{\AgdaIndent{0}}]%
\>[2]\AgdaKeyword{open}\AgdaSpace{}%
\AgdaModule{Real.Real}\AgdaSpace{}%
\AgdaBound{real}\AgdaSpace{}%
\AgdaKeyword{using}\AgdaSpace{}%
\AgdaSymbol{(}\AgdaOperator{\AgdaField{\AgdaUnderscore{}ᵣ}}\AgdaSymbol{;}\AgdaSpace{}%
\AgdaField{ℝ}\AgdaSymbol{)}\<%
\\
\>[2][@{}l@{\AgdaIndent{0}}]%
\>[4]\AgdaKeyword{renaming}\AgdaSpace{}%
\AgdaSymbol{(}\AgdaOperator{\AgdaField{\AgdaUnderscore{}+\AgdaUnderscore{}}}\AgdaSpace{}%
\AgdaSymbol{to}\AgdaSpace{}%
\AgdaOperator{\AgdaField{\AgdaUnderscore{}+ᵣ\AgdaUnderscore{}}}\AgdaSymbol{;}\AgdaSpace{}%
\AgdaOperator{\AgdaField{\AgdaUnderscore{}-\AgdaUnderscore{}}}\AgdaSpace{}%
\AgdaSymbol{to}\AgdaSpace{}%
\AgdaOperator{\AgdaField{\AgdaUnderscore{}-ᵣ\AgdaUnderscore{}}}\AgdaSymbol{;}\AgdaSpace{}%
\AgdaOperator{\AgdaField{-\AgdaUnderscore{}}}\AgdaSpace{}%
\AgdaSymbol{to}\AgdaSpace{}%
\AgdaOperator{\AgdaField{-ᵣ\AgdaUnderscore{}}}\AgdaSymbol{;}\AgdaSpace{}%
\AgdaOperator{\AgdaField{\AgdaUnderscore{}/\AgdaUnderscore{}}}\AgdaSpace{}%
\AgdaSymbol{to}\AgdaSpace{}%
\AgdaOperator{\AgdaField{\AgdaUnderscore{}/ᵣ\AgdaUnderscore{}}}\AgdaSymbol{;}\AgdaSpace{}%
\AgdaOperator{\AgdaField{\AgdaUnderscore{}*\AgdaUnderscore{}}}\AgdaSpace{}%
\AgdaSymbol{to}\AgdaSpace{}%
\AgdaOperator{\AgdaField{\AgdaUnderscore{}*ᵣ\AgdaUnderscore{}}}\AgdaSymbol{)}\<%
\\
%
\>[2]\AgdaKeyword{open}\AgdaSpace{}%
\AgdaModule{Cplx}\AgdaSpace{}%
\AgdaBound{cplx}\AgdaSpace{}%
\AgdaKeyword{using}\AgdaSpace{}%
\AgdaSymbol{(}\AgdaField{ℂ}\AgdaSymbol{;}\AgdaSpace{}%
\AgdaOperator{\AgdaField{\AgdaUnderscore{}+\AgdaUnderscore{}}}\AgdaSymbol{;}\AgdaSpace{}%
\AgdaField{fromℝ}\AgdaSymbol{;}\AgdaSpace{}%
\AgdaOperator{\AgdaField{\AgdaUnderscore{}*\AgdaUnderscore{}}}\AgdaSymbol{;}\AgdaSpace{}%
\AgdaField{-ω}\AgdaSymbol{;}\AgdaSpace{}%
\AgdaFunction{0ℂ}\AgdaSymbol{;}\AgdaSpace{}%
\AgdaField{+-*-isCommutativeRing}\AgdaSymbol{;}\AgdaSpace{}%
\AgdaField{ω-r₁x-r₁y}\AgdaSymbol{;}\AgdaSpace{}%
\AgdaField{ω-N-mN}\AgdaSymbol{;}\AgdaSpace{}%
\AgdaField{ω-N-k₀+k₁}\AgdaSymbol{)}\<%
\\
%
\\[\AgdaEmptyExtraSkip]%
%
\>[2]\AgdaKeyword{open}\AgdaSpace{}%
\AgdaModule{AlgebraStructures}%
\>[26]\AgdaSymbol{\{}\AgdaArgument{A}\AgdaSpace{}%
\AgdaSymbol{=}\AgdaSpace{}%
\AgdaField{ℂ}\AgdaSymbol{\}}\AgdaSpace{}%
\AgdaOperator{\AgdaDatatype{\AgdaUnderscore{}≡\AgdaUnderscore{}}}\<%
\\
%
\>[2]\AgdaKeyword{open}\AgdaSpace{}%
\AgdaModule{AlgebraDefinitions}\AgdaSpace{}%
\AgdaSymbol{\{}\AgdaArgument{A}\AgdaSpace{}%
\AgdaSymbol{=}\AgdaSpace{}%
\AgdaField{ℂ}\AgdaSymbol{\}}\AgdaSpace{}%
\AgdaOperator{\AgdaDatatype{\AgdaUnderscore{}≡\AgdaUnderscore{}}}\<%
\\
%
\\[\AgdaEmptyExtraSkip]%
%
\>[2]\AgdaKeyword{open}\AgdaSpace{}%
\AgdaModule{IsCommutativeRing}\AgdaSpace{}%
\AgdaField{+-*-isCommutativeRing}\AgdaSpace{}%
\AgdaKeyword{using}\AgdaSpace{}%
\AgdaSymbol{(}\AgdaFunction{+-isCommutativeMonoid}\AgdaSymbol{;}\AgdaSpace{}%
\AgdaFunction{distribˡ}\AgdaSymbol{;}\AgdaSpace{}%
\AgdaField{*-comm}\AgdaSymbol{;}\AgdaSpace{}%
\AgdaFunction{zeroʳ}\AgdaSymbol{;}\AgdaSpace{}%
\AgdaFunction{zeroˡ}\AgdaSymbol{;}\AgdaSpace{}%
\AgdaFunction{*-identityʳ}\AgdaSymbol{;}\AgdaSpace{}%
\AgdaFunction{*-assoc}\AgdaSymbol{;}\AgdaSpace{}%
\AgdaFunction{+-identityʳ}\AgdaSymbol{;}\AgdaSpace{}%
\AgdaFunction{+-assoc}\AgdaSymbol{;}\AgdaSpace{}%
\AgdaFunction{+-comm}\AgdaSymbol{;}\AgdaSpace{}%
\AgdaFunction{+-identityˡ}\AgdaSymbol{)}\<%
\\
%
\\[\AgdaEmptyExtraSkip]%
%
\>[2]\AgdaKeyword{open}\AgdaSpace{}%
\AgdaKeyword{import}\AgdaSpace{}%
\AgdaModule{Data.Nat.Base}\AgdaSpace{}%
\AgdaKeyword{using}\AgdaSpace{}%
\AgdaSymbol{(}\AgdaDatatype{ℕ}\AgdaSymbol{;}\AgdaSpace{}%
\AgdaInductiveConstructor{zero}\AgdaSymbol{;}\AgdaSpace{}%
\AgdaInductiveConstructor{suc}\AgdaSymbol{;}\AgdaSpace{}%
\AgdaRecord{NonZero}\AgdaSymbol{;}\AgdaSpace{}%
\AgdaOperator{\AgdaPrimitive{\AgdaUnderscore{}≡ᵇ\AgdaUnderscore{}}}\AgdaSymbol{;}\AgdaSpace{}%
\AgdaFunction{nonZero}\AgdaSymbol{)}\AgdaSpace{}%
\AgdaKeyword{renaming}\AgdaSpace{}%
\AgdaSymbol{(}\AgdaOperator{\AgdaPrimitive{\AgdaUnderscore{}*\AgdaUnderscore{}}}\AgdaSpace{}%
\AgdaSymbol{to}\AgdaSpace{}%
\AgdaOperator{\AgdaPrimitive{\AgdaUnderscore{}*ₙ\AgdaUnderscore{}}}\AgdaSymbol{;}\AgdaSpace{}%
\AgdaOperator{\AgdaPrimitive{\AgdaUnderscore{}+\AgdaUnderscore{}}}\AgdaSpace{}%
\AgdaSymbol{to}\AgdaSpace{}%
\AgdaOperator{\AgdaPrimitive{\AgdaUnderscore{}+ₙ\AgdaUnderscore{}}}\AgdaSymbol{)}\<%
\\
%
\>[2]\AgdaKeyword{open}\AgdaSpace{}%
\AgdaKeyword{import}\AgdaSpace{}%
\AgdaModule{Data.Nat.Properties}\AgdaSpace{}%
\AgdaKeyword{using}\AgdaSpace{}%
\AgdaSymbol{(}\AgdaFunction{suc-injective}\AgdaSymbol{;}\AgdaSpace{}%
\AgdaFunction{m*n≢0}\AgdaSymbol{;}\AgdaSpace{}%
\AgdaFunction{m*n≢0⇒m≢0}\AgdaSymbol{;}\AgdaSpace{}%
\AgdaFunction{m*n≢0⇒n≢0}\AgdaSymbol{;}\AgdaSpace{}%
\AgdaFunction{nonZero?}\AgdaSymbol{)}\AgdaSpace{}%
\AgdaKeyword{renaming}\AgdaSpace{}%
\AgdaSymbol{(}\AgdaFunction{*-comm}\AgdaSpace{}%
\AgdaSymbol{to}\AgdaSpace{}%
\AgdaFunction{*ₙ-comm}\AgdaSymbol{;}\AgdaSpace{}%
\AgdaFunction{*-identityʳ}\AgdaSpace{}%
\AgdaSymbol{to}\AgdaSpace{}%
\AgdaFunction{*ₙ-identityʳ}\AgdaSymbol{;}\AgdaSpace{}%
\AgdaFunction{*-assoc}\AgdaSpace{}%
\AgdaSymbol{to}\AgdaSpace{}%
\AgdaFunction{*ₙ-assoc}\AgdaSymbol{;}\<%
\\
\>[2][@{}l@{\AgdaIndent{0}}]%
\>[4]\AgdaFunction{+-identityʳ}\AgdaSpace{}%
\AgdaSymbol{to}\AgdaSpace{}%
\AgdaFunction{+ₙ-identityʳ}\AgdaSymbol{;}\AgdaSpace{}%
\AgdaFunction{*-zeroˡ}\AgdaSpace{}%
\AgdaSymbol{to}\AgdaSpace{}%
\AgdaFunction{*ₙ-zeroˡ}\AgdaSymbol{;}\AgdaSpace{}%
\AgdaFunction{*-zeroʳ}\AgdaSpace{}%
\AgdaSymbol{to}\AgdaSpace{}%
\AgdaFunction{*ₙ-zeroʳ}\AgdaSymbol{)}\<%
\\
%
\>[2]\AgdaKeyword{open}\AgdaSpace{}%
\AgdaKeyword{import}\AgdaSpace{}%
\AgdaModule{Data.Nat.Solver}\AgdaSpace{}%
\AgdaKeyword{using}\AgdaSpace{}%
\AgdaSymbol{(}\AgdaKeyword{module}\AgdaSpace{}%
\AgdaModule{+-*-Solver}\AgdaSymbol{)}\<%
\\
%
\>[2]\AgdaKeyword{open}\AgdaSpace{}%
\AgdaModule{+-*-Solver}\AgdaSpace{}%
\AgdaKeyword{using}\AgdaSpace{}%
\AgdaSymbol{(}\AgdaFunction{solve}\AgdaSymbol{;}\AgdaSpace{}%
\AgdaOperator{\AgdaFunction{\AgdaUnderscore{}:*\AgdaUnderscore{}}}\AgdaSymbol{;}\AgdaSpace{}%
\AgdaOperator{\AgdaFunction{\AgdaUnderscore{}:+\AgdaUnderscore{}}}\AgdaSymbol{;}\AgdaSpace{}%
\AgdaInductiveConstructor{con}\AgdaSymbol{;}\AgdaSpace{}%
\AgdaOperator{\AgdaFunction{\AgdaUnderscore{}:=\AgdaUnderscore{}}}\AgdaSymbol{)}\<%
\\
%
\>[2]\AgdaKeyword{open}\AgdaSpace{}%
\AgdaKeyword{import}\AgdaSpace{}%
\AgdaModule{Data.Fin.Base}\AgdaSpace{}%
\AgdaKeyword{using}\AgdaSpace{}%
\AgdaSymbol{(}\AgdaDatatype{Fin}\AgdaSymbol{;}\AgdaSpace{}%
\AgdaFunction{quotRem}\AgdaSymbol{;}\AgdaSpace{}%
\AgdaFunction{toℕ}\AgdaSymbol{;}\AgdaSpace{}%
\AgdaFunction{combine}\AgdaSymbol{;}\AgdaSpace{}%
\AgdaFunction{remQuot}\AgdaSymbol{;}\AgdaSpace{}%
\AgdaFunction{quotient}\AgdaSymbol{;}\AgdaSpace{}%
\AgdaFunction{remainder}\AgdaSymbol{;}\AgdaSpace{}%
\AgdaFunction{cast}\AgdaSymbol{;}\AgdaSpace{}%
\AgdaFunction{fromℕ<}\AgdaSymbol{;}\AgdaSpace{}%
\AgdaOperator{\AgdaFunction{\AgdaUnderscore{}↑ˡ\AgdaUnderscore{}}}\AgdaSymbol{;}\AgdaSpace{}%
\AgdaOperator{\AgdaFunction{\AgdaUnderscore{}↑ʳ\AgdaUnderscore{}}}\AgdaSymbol{;}\AgdaSpace{}%
\AgdaFunction{splitAt}\AgdaSymbol{;}\AgdaSpace{}%
\AgdaFunction{join}\AgdaSymbol{)}\AgdaSpace{}%
\AgdaKeyword{renaming}\AgdaSpace{}%
\AgdaSymbol{(}\AgdaInductiveConstructor{zero}\AgdaSpace{}%
\AgdaSymbol{to}\AgdaSpace{}%
\AgdaInductiveConstructor{fzero}\AgdaSymbol{;}\AgdaSpace{}%
\AgdaInductiveConstructor{suc}\AgdaSpace{}%
\AgdaSymbol{to}\AgdaSpace{}%
\AgdaInductiveConstructor{fsuc}\AgdaSymbol{)}\<%
\\
%
\>[2]\AgdaKeyword{open}\AgdaSpace{}%
\AgdaKeyword{import}\AgdaSpace{}%
\AgdaModule{Data.Fin.Properties}\AgdaSpace{}%
\AgdaKeyword{using}\AgdaSpace{}%
\AgdaSymbol{(}\AgdaFunction{cast-is-id}\AgdaSymbol{;}\AgdaSpace{}%
\AgdaFunction{remQuot-combine}\AgdaSymbol{;}\AgdaSpace{}%
\AgdaFunction{splitAt-↑ˡ}\AgdaSymbol{;}\AgdaSpace{}%
\AgdaFunction{splitAt-↑ʳ}\AgdaSymbol{;}\AgdaSpace{}%
\AgdaFunction{toℕ-↑ˡ}\AgdaSymbol{;}\AgdaSpace{}%
\AgdaFunction{toℕ-↑ʳ}\AgdaSymbol{;}\AgdaSpace{}%
\AgdaFunction{toℕ-combine}\AgdaSymbol{;}\AgdaSpace{}%
\AgdaFunction{combine-remQuot}\AgdaSymbol{;}\AgdaSpace{}%
\AgdaFunction{combine-surjective}\AgdaSymbol{;}\AgdaSpace{}%
\AgdaFunction{toℕ-injective}\AgdaSymbol{;}\AgdaSpace{}%
\AgdaFunction{toℕ-cast}\AgdaSymbol{;}\AgdaSpace{}%
\AgdaFunction{cast-trans}\AgdaSymbol{)}\<%
\\
%
\>[2]\AgdaKeyword{open}\AgdaSpace{}%
\AgdaKeyword{import}\AgdaSpace{}%
\AgdaModule{Data.Bool}\AgdaSpace{}%
\AgdaKeyword{using}\AgdaSpace{}%
\AgdaSymbol{(}\AgdaDatatype{Bool}\AgdaSymbol{;}\AgdaSpace{}%
\AgdaInductiveConstructor{true}\AgdaSymbol{;}\AgdaSpace{}%
\AgdaInductiveConstructor{false}\AgdaSymbol{;}\AgdaSpace{}%
\AgdaFunction{not}\AgdaSymbol{)}\<%
\\
%
\>[2]\AgdaKeyword{open}\AgdaSpace{}%
\AgdaKeyword{import}\AgdaSpace{}%
\AgdaModule{Data.Bool}\AgdaSpace{}%
\AgdaKeyword{using}\AgdaSpace{}%
\AgdaSymbol{(}\AgdaFunction{T}\AgdaSymbol{)}\<%
\\
%
\>[2]\AgdaKeyword{open}\AgdaSpace{}%
\AgdaKeyword{import}\AgdaSpace{}%
\AgdaModule{Data.Empty}\<%
\\
%
\\[\AgdaEmptyExtraSkip]%
%
\>[2]\AgdaKeyword{open}\AgdaSpace{}%
\AgdaKeyword{import}\AgdaSpace{}%
\AgdaModule{Data.Product.Base}\AgdaSpace{}%
\AgdaKeyword{using}\AgdaSpace{}%
\AgdaSymbol{(}\AgdaFunction{∃}\AgdaSymbol{;}\AgdaSpace{}%
\AgdaFunction{∃₂}\AgdaSymbol{;}\AgdaSpace{}%
\AgdaOperator{\AgdaFunction{\AgdaUnderscore{}×\AgdaUnderscore{}}}\AgdaSymbol{;}\AgdaSpace{}%
\AgdaField{proj₁}\AgdaSymbol{;}\AgdaSpace{}%
\AgdaField{proj₂}\AgdaSymbol{;}\AgdaSpace{}%
\AgdaFunction{map₁}\AgdaSymbol{;}\AgdaSpace{}%
\AgdaFunction{map₂}\AgdaSymbol{;}\AgdaSpace{}%
\AgdaFunction{uncurry}\AgdaSymbol{)}\AgdaSpace{}%
\AgdaKeyword{renaming}\AgdaSpace{}%
\AgdaSymbol{(}\AgdaSpace{}%
\AgdaOperator{\AgdaInductiveConstructor{\AgdaUnderscore{},\AgdaUnderscore{}}}\AgdaSpace{}%
\AgdaSymbol{to}\AgdaSpace{}%
\AgdaOperator{\AgdaInductiveConstructor{⟨\AgdaUnderscore{},\AgdaUnderscore{}⟩}}\AgdaSymbol{)}\<%
\\
%
\>[2]\AgdaKeyword{open}\AgdaSpace{}%
\AgdaKeyword{import}\AgdaSpace{}%
\AgdaModule{Data.Sum.Base}\AgdaSpace{}%
\AgdaKeyword{using}\AgdaSpace{}%
\AgdaSymbol{(}\AgdaInductiveConstructor{inj₁}\AgdaSymbol{;}\AgdaSpace{}%
\AgdaInductiveConstructor{inj₂}\AgdaSymbol{;}\AgdaSpace{}%
\AgdaOperator{\AgdaDatatype{\AgdaUnderscore{}⊎\AgdaUnderscore{}}}\AgdaSymbol{)}\<%
\\
%
\>[2]\AgdaKeyword{open}\AgdaSpace{}%
\AgdaKeyword{import}\AgdaSpace{}%
\AgdaModule{Data.Unit}\AgdaSpace{}%
\AgdaKeyword{using}\AgdaSpace{}%
\AgdaSymbol{(}\AgdaRecord{⊤}\AgdaSymbol{;}\AgdaSpace{}%
\AgdaInductiveConstructor{tt}\AgdaSymbol{)}\<%
\\
%
\\[\AgdaEmptyExtraSkip]%
%
\>[2]\AgdaKeyword{open}\AgdaSpace{}%
\AgdaKeyword{import}\AgdaSpace{}%
\AgdaModule{Matrix}\AgdaSpace{}%
\AgdaKeyword{using}\AgdaSpace{}%
\AgdaSymbol{(}\AgdaFunction{Ar}\AgdaSymbol{;}\AgdaSpace{}%
\AgdaDatatype{Shape}\AgdaSymbol{;}\AgdaSpace{}%
\AgdaOperator{\AgdaInductiveConstructor{\AgdaUnderscore{}⊗\AgdaUnderscore{}}}\AgdaSymbol{;}\AgdaSpace{}%
\AgdaInductiveConstructor{ι}\AgdaSymbol{;}\AgdaSpace{}%
\AgdaDatatype{Position}\AgdaSymbol{;}\AgdaSpace{}%
\AgdaFunction{mapRows}\AgdaSymbol{;}\AgdaSpace{}%
\AgdaFunction{zipWith}\AgdaSymbol{;}\AgdaSpace{}%
\AgdaFunction{nest}\AgdaSymbol{;}\AgdaSpace{}%
\AgdaFunction{map}\AgdaSymbol{;}\AgdaSpace{}%
\AgdaFunction{unnest}\AgdaSymbol{;}\AgdaSpace{}%
\AgdaFunction{head₁}\AgdaSymbol{;}\AgdaSpace{}%
\AgdaFunction{tail₁}\AgdaSymbol{;}\AgdaSpace{}%
\AgdaFunction{zip}\AgdaSymbol{;}\AgdaSpace{}%
\AgdaFunction{iterate}\AgdaSymbol{;}\AgdaSpace{}%
\AgdaFunction{ι-cons}\AgdaSymbol{;}\AgdaSpace{}%
\AgdaFunction{nil}\AgdaSymbol{;}\AgdaSpace{}%
\AgdaFunction{length}\AgdaSymbol{;}\AgdaSpace{}%
\AgdaFunction{splitAr}\AgdaSymbol{;}\AgdaSpace{}%
\AgdaFunction{splitArₗ}\AgdaSymbol{;}\AgdaSpace{}%
\AgdaFunction{splitArᵣ}\AgdaSymbol{)}\<%
\\
%
\>[2]\AgdaKeyword{open}\AgdaSpace{}%
\AgdaKeyword{import}\AgdaSpace{}%
\AgdaModule{Matrix.Equality}\AgdaSpace{}%
\AgdaKeyword{using}\AgdaSpace{}%
\AgdaSymbol{(}\AgdaOperator{\AgdaFunction{\AgdaUnderscore{}≅\AgdaUnderscore{}}}\AgdaSymbol{;}\AgdaSpace{}%
\AgdaFunction{reduce-≅}\AgdaSymbol{;}\AgdaSpace{}%
\AgdaFunction{tail₁-cong}\AgdaSymbol{)}\<%
\\
%
\>[2]\AgdaKeyword{open}\AgdaSpace{}%
\AgdaKeyword{import}\AgdaSpace{}%
\AgdaModule{Matrix.Properties}\AgdaSpace{}%
\AgdaKeyword{using}\AgdaSpace{}%
\AgdaSymbol{(}\AgdaFunction{splitArᵣ-zero}\AgdaSymbol{;}\AgdaSpace{}%
\AgdaFunction{tail₁-const}\AgdaSymbol{;}\AgdaSpace{}%
\AgdaFunction{zipWith-congˡ}\AgdaSymbol{)}\<%
\\
%
\>[2]\AgdaKeyword{open}\AgdaSpace{}%
\AgdaKeyword{import}\AgdaSpace{}%
\AgdaModule{Matrix.NonZero}\AgdaSpace{}%
\AgdaKeyword{using}\AgdaSpace{}%
\AgdaSymbol{(}\AgdaDatatype{NonZeroₛ}\AgdaSymbol{;}\AgdaSpace{}%
\AgdaInductiveConstructor{ι}\AgdaSymbol{;}\AgdaSpace{}%
\AgdaOperator{\AgdaInductiveConstructor{\AgdaUnderscore{}⊗\AgdaUnderscore{}}}\AgdaSymbol{;}\AgdaSpace{}%
\AgdaFunction{nonZeroₛ-s⇒nonZero-s}\AgdaSymbol{;}\AgdaSpace{}%
\AgdaFunction{nonZeroDec}\AgdaSymbol{;}\AgdaSpace{}%
\AgdaFunction{nonZeroₛ-s⇒nonZeroₛ-sᵗ}\AgdaSymbol{;}\AgdaSpace{}%
\AgdaFunction{nonZeroₛ-s⇒nonZero-sᵗ}\AgdaSymbol{;}\AgdaSpace{}%
\AgdaFunction{¬nonZeroₛ-s⇒¬nonZero-sᵗ}\AgdaSymbol{;}\AgdaSpace{}%
\AgdaFunction{¬nonZero-N⇒PosN-irrelevant}\AgdaSymbol{;}\AgdaSpace{}%
\AgdaFunction{¬nonZero-sᵗ⇒¬nonZero-s}\AgdaSymbol{)}\<%
\\
%
\\[\AgdaEmptyExtraSkip]%
%
\>[2]\AgdaKeyword{import}\AgdaSpace{}%
\AgdaModule{Matrix.Sum}\AgdaSpace{}%
\AgdaSymbol{as}\AgdaSpace{}%
\AgdaModule{S}\<%
\\
%
\>[2]\AgdaKeyword{open}\AgdaSpace{}%
\AgdaModule{S}\AgdaSpace{}%
\AgdaOperator{\AgdaField{\AgdaUnderscore{}+\AgdaUnderscore{}}}\AgdaSpace{}%
\AgdaFunction{0ℂ}\AgdaSpace{}%
\AgdaFunction{+-isCommutativeMonoid}\AgdaSpace{}%
\AgdaKeyword{using}\AgdaSpace{}%
\AgdaSymbol{(}\AgdaFunction{merge-sum}\AgdaSymbol{;}\AgdaSpace{}%
\AgdaFunction{sum-reindex}\AgdaSymbol{;}\AgdaSpace{}%
\AgdaFunction{sum-swap}\AgdaSymbol{)}\<%
\\
%
\>[2]\AgdaFunction{sum}\AgdaSpace{}%
\AgdaSymbol{=}\AgdaSpace{}%
\AgdaFunction{S.sum}\AgdaSpace{}%
\AgdaOperator{\AgdaField{\AgdaUnderscore{}+\AgdaUnderscore{}}}\AgdaSpace{}%
\AgdaFunction{0ℂ}\AgdaSpace{}%
\AgdaFunction{+-isCommutativeMonoid}\<%
\\
%
\>[2]\AgdaSymbol{\{-\#}\AgdaSpace{}%
\AgdaKeyword{DISPLAY}\AgdaSpace{}%
\AgdaFunction{S.sum}\AgdaSpace{}%
\AgdaOperator{\AgdaBound{\AgdaUnderscore{}+\AgdaUnderscore{}}}\AgdaSpace{}%
\AgdaBound{0ℂ}\AgdaSpace{}%
\AgdaBound{+-isCommutativeMonoid}\AgdaSpace{}%
\AgdaPragma{=}\AgdaSpace{}%
\AgdaFunction{sum}\AgdaSpace{}%
\AgdaSymbol{\#-\}}\<%
\\
%
\>[2]\AgdaFunction{sum-cong}\AgdaSpace{}%
\AgdaSymbol{=}\AgdaSpace{}%
\AgdaFunction{S.sum-cong}\AgdaSpace{}%
\AgdaOperator{\AgdaField{\AgdaUnderscore{}+\AgdaUnderscore{}}}\AgdaSpace{}%
\AgdaFunction{0ℂ}\AgdaSpace{}%
\AgdaFunction{+-isCommutativeMonoid}\<%
\\
%
\\[\AgdaEmptyExtraSkip]%
%
\>[2]\AgdaKeyword{open}\AgdaSpace{}%
\AgdaKeyword{import}\AgdaSpace{}%
\AgdaModule{Matrix.Reshape}\AgdaSpace{}%
\AgdaKeyword{using}\AgdaSpace{}%
\AgdaSymbol{(}\AgdaFunction{reshape}\AgdaSymbol{;}\AgdaSpace{}%
\AgdaDatatype{Reshape}\AgdaSymbol{;}\AgdaSpace{}%
\AgdaInductiveConstructor{flat}\AgdaSymbol{;}\AgdaSpace{}%
\AgdaFunction{♭}\AgdaSymbol{;}\AgdaSpace{}%
\AgdaFunction{♯}\AgdaSymbol{;}\AgdaSpace{}%
\AgdaFunction{recursive-transpose}\AgdaSymbol{;}\AgdaSpace{}%
\AgdaFunction{recursive-transposeᵣ}\AgdaSymbol{;}\AgdaSpace{}%
\AgdaOperator{\AgdaInductiveConstructor{\AgdaUnderscore{}∙\AgdaUnderscore{}}}\AgdaSymbol{;}\AgdaSpace{}%
\AgdaFunction{rev}\AgdaSymbol{;}\AgdaSpace{}%
\AgdaOperator{\AgdaInductiveConstructor{\AgdaUnderscore{}⊕\AgdaUnderscore{}}}\AgdaSymbol{;}\AgdaSpace{}%
\AgdaInductiveConstructor{swap}\AgdaSymbol{;}\AgdaSpace{}%
\AgdaInductiveConstructor{eq}\AgdaSymbol{;}\AgdaSpace{}%
\AgdaInductiveConstructor{split}\AgdaSymbol{;}\AgdaSpace{}%
\AgdaOperator{\AgdaFunction{\AgdaUnderscore{}⟨\AgdaUnderscore{}⟩}}\AgdaSymbol{;}\AgdaSpace{}%
\AgdaFunction{reindex}\AgdaSymbol{;}\AgdaSpace{}%
\AgdaFunction{rev-eq}\AgdaSymbol{;}\AgdaSpace{}%
\AgdaFunction{flatten-reindex}\AgdaSymbol{;}\AgdaSpace{}%
\AgdaFunction{|s|≡|sᵗ|}\AgdaSymbol{;}\AgdaSpace{}%
\AgdaFunction{reindex-reindex}\AgdaSymbol{;}\AgdaSpace{}%
\AgdaFunction{recursive-transpose-inv}\AgdaSymbol{)}\<%
\\
%
\>[2]\AgdaKeyword{open}\AgdaSpace{}%
\AgdaKeyword{import}\AgdaSpace{}%
\AgdaModule{Function.Base}\AgdaSpace{}%
\AgdaKeyword{using}\AgdaSpace{}%
\AgdaSymbol{(}\AgdaOperator{\AgdaFunction{\AgdaUnderscore{}\$\AgdaUnderscore{}}}\AgdaSymbol{;}\AgdaSpace{}%
\AgdaFunction{id}\AgdaSymbol{;}\AgdaSpace{}%
\AgdaOperator{\AgdaFunction{\AgdaUnderscore{}∘\AgdaUnderscore{}}}\AgdaSymbol{;}\AgdaSpace{}%
\AgdaFunction{flip}\AgdaSymbol{;}\AgdaSpace{}%
\AgdaOperator{\AgdaFunction{\AgdaUnderscore{}∘₂\AgdaUnderscore{}}}\AgdaSymbol{)}\<%
\\
%
\\[\AgdaEmptyExtraSkip]%
%
\>[2]\AgdaKeyword{open}\AgdaSpace{}%
\AgdaKeyword{import}\AgdaSpace{}%
\AgdaModule{FFT}\AgdaSpace{}%
\AgdaBound{real}\AgdaSpace{}%
\AgdaBound{cplx}\AgdaSpace{}%
\AgdaKeyword{using}\AgdaSpace{}%
\AgdaSymbol{(}\AgdaFunction{DFT}\AgdaSymbol{;}\AgdaSpace{}%
\AgdaFunction{FFT}\AgdaSymbol{;}\AgdaSpace{}%
\AgdaFunction{DFT′}\AgdaSymbol{;}\AgdaSpace{}%
\AgdaFunction{FFT′}\AgdaSymbol{;}\AgdaSpace{}%
\AgdaFunction{offset-prod}\AgdaSymbol{;}\AgdaSpace{}%
\AgdaFunction{iota}\AgdaSymbol{;}\AgdaSpace{}%
\AgdaFunction{twiddles}\AgdaSymbol{)}\<%
\\
%
\\[\AgdaEmptyExtraSkip]%
%
\>[2]\AgdaKeyword{private}\<%
\\
\>[2][@{}l@{\AgdaIndent{0}}]%
\>[4]\AgdaKeyword{variable}\<%
\\
\>[4][@{}l@{\AgdaIndent{0}}]%
\>[6]\AgdaGeneralizable{s}\AgdaSpace{}%
\AgdaGeneralizable{p}\AgdaSpace{}%
\AgdaGeneralizable{r₁}\AgdaSpace{}%
\AgdaGeneralizable{r₂}\AgdaSpace{}%
\AgdaSymbol{:}\AgdaSpace{}%
\AgdaDatatype{Shape}\<%
\\
%
\>[6]\AgdaGeneralizable{N}\AgdaSpace{}%
\AgdaGeneralizable{M}\AgdaSpace{}%
\AgdaSymbol{:}\AgdaSpace{}%
\AgdaDatatype{ℕ}\<%
\\
%
\>[2]\AgdaKeyword{infixl}\AgdaSpace{}%
\AgdaNumber{5}\AgdaSpace{}%
\AgdaOperator{\AgdaFunction{\AgdaUnderscore{}⊡\AgdaUnderscore{}}}\<%
\\
%
\>[2]\AgdaOperator{\AgdaFunction{\AgdaUnderscore{}⊡\AgdaUnderscore{}}}\AgdaSpace{}%
\AgdaSymbol{=}\AgdaSpace{}%
\AgdaFunction{trans}\<%
\\
%
\\[\AgdaEmptyExtraSkip]%
%
\>[2]\AgdaKeyword{infix}\AgdaSpace{}%
\AgdaNumber{10}\AgdaSpace{}%
\AgdaOperator{\AgdaFunction{\#\AgdaUnderscore{}}}\<%
\\
%
\>[2]\AgdaOperator{\AgdaFunction{\#\AgdaUnderscore{}}}\AgdaSpace{}%
\AgdaSymbol{:}\AgdaSpace{}%
\AgdaDatatype{Shape}\AgdaSpace{}%
\AgdaSymbol{→}\AgdaSpace{}%
\AgdaDatatype{ℕ}\<%
\\
%
\>[2]\AgdaOperator{\AgdaFunction{\#\AgdaUnderscore{}}}\AgdaSpace{}%
\AgdaSymbol{=}\AgdaSpace{}%
\AgdaFunction{length}\<%
\\
%
\\[\AgdaEmptyExtraSkip]%
%
\>[2]\AgdaKeyword{infix}\AgdaSpace{}%
\AgdaNumber{11}\AgdaSpace{}%
\AgdaOperator{\AgdaFunction{\AgdaUnderscore{}ᵗ}}\<%
\\
%
\>[2]\AgdaOperator{\AgdaFunction{\AgdaUnderscore{}ᵗ}}\AgdaSpace{}%
\AgdaSymbol{:}\AgdaSpace{}%
\AgdaDatatype{Shape}\AgdaSpace{}%
\AgdaSymbol{→}\AgdaSpace{}%
\AgdaDatatype{Shape}\<%
\\
%
\>[2]\AgdaOperator{\AgdaFunction{\AgdaUnderscore{}ᵗ}}\AgdaSpace{}%
\AgdaSymbol{=}\AgdaSpace{}%
\AgdaFunction{recursive-transpose}\<%
\\
%
\\[\AgdaEmptyExtraSkip]%
%
\>[2]\AgdaFunction{nz-\#}\AgdaSpace{}%
\AgdaSymbol{:}\AgdaSpace{}%
\AgdaDatatype{NonZeroₛ}\AgdaSpace{}%
\AgdaGeneralizable{s}\AgdaSpace{}%
\AgdaSymbol{→}\AgdaSpace{}%
\AgdaRecord{NonZero}\AgdaSpace{}%
\AgdaSymbol{(}\AgdaFunction{length}\AgdaSpace{}%
\AgdaGeneralizable{s}\AgdaSymbol{)}\<%
\\
%
\>[2]\AgdaFunction{nz-\#}\AgdaSpace{}%
\AgdaSymbol{=}\AgdaSpace{}%
\AgdaFunction{nonZeroₛ-s⇒nonZero-s}\<%
\\
%
\\[\AgdaEmptyExtraSkip]%
%
\>[2]\AgdaFunction{nz-ι\#}\AgdaSpace{}%
\AgdaSymbol{:}\AgdaSpace{}%
\AgdaDatatype{NonZeroₛ}\AgdaSpace{}%
\AgdaGeneralizable{s}\AgdaSpace{}%
\AgdaSymbol{→}\AgdaSpace{}%
\AgdaDatatype{NonZeroₛ}\AgdaSpace{}%
\AgdaSymbol{(}\AgdaInductiveConstructor{ι}\AgdaSpace{}%
\AgdaSymbol{(}\AgdaFunction{length}\AgdaSpace{}%
\AgdaGeneralizable{s}\AgdaSymbol{))}\<%
\\
%
\>[2]\AgdaFunction{nz-ι\#}\AgdaSpace{}%
\AgdaSymbol{(}\AgdaInductiveConstructor{ι}\AgdaSpace{}%
\AgdaBound{x}\AgdaSymbol{)}\AgdaSpace{}%
\AgdaSymbol{=}\AgdaSpace{}%
\AgdaInductiveConstructor{ι}\AgdaSpace{}%
\AgdaBound{x}\<%
\\
%
\>[2]\AgdaFunction{nz-ι\#}\AgdaSpace{}%
\AgdaSymbol{\{}\AgdaBound{s}\AgdaSpace{}%
\AgdaOperator{\AgdaInductiveConstructor{⊗}}\AgdaSpace{}%
\AgdaBound{p}\AgdaSymbol{\}}\AgdaSpace{}%
\AgdaSymbol{(}\AgdaBound{nz-s}\AgdaSpace{}%
\AgdaOperator{\AgdaInductiveConstructor{⊗}}\AgdaSpace{}%
\AgdaBound{nz-p}\AgdaSymbol{)}\AgdaSpace{}%
\AgdaSymbol{=}\AgdaSpace{}%
\AgdaInductiveConstructor{ι}\AgdaSpace{}%
\AgdaSymbol{(}\AgdaFunction{m*n≢0}\AgdaSpace{}%
\AgdaSymbol{(}\AgdaOperator{\AgdaFunction{\#}}\AgdaSpace{}%
\AgdaBound{s}\AgdaSymbol{)}\AgdaSpace{}%
\AgdaSymbol{(}\AgdaOperator{\AgdaFunction{\#}}\AgdaSpace{}%
\AgdaBound{p}\AgdaSymbol{)}\AgdaSpace{}%
\AgdaSymbol{⦃}\AgdaSpace{}%
\AgdaFunction{nz-\#}\AgdaSpace{}%
\AgdaBound{nz-s}\AgdaSpace{}%
\AgdaSymbol{⦄}\AgdaSpace{}%
\AgdaSymbol{⦃}\AgdaSpace{}%
\AgdaFunction{nz-\#}\AgdaSpace{}%
\AgdaBound{nz-p}\AgdaSpace{}%
\AgdaSymbol{⦄}\AgdaSpace{}%
\AgdaSymbol{)}\<%
\\
%
\\[\AgdaEmptyExtraSkip]%
%
\>[2]\AgdaFunction{nzᵗ}\AgdaSpace{}%
\AgdaSymbol{:}\AgdaSpace{}%
\AgdaDatatype{NonZeroₛ}\AgdaSpace{}%
\AgdaGeneralizable{s}\AgdaSpace{}%
\AgdaSymbol{→}\AgdaSpace{}%
\AgdaDatatype{NonZeroₛ}\AgdaSpace{}%
\AgdaSymbol{(}\AgdaGeneralizable{s}\AgdaSpace{}%
\AgdaOperator{\AgdaFunction{ᵗ}}\AgdaSymbol{)}\<%
\\
%
\>[2]\AgdaFunction{nzᵗ}\AgdaSpace{}%
\AgdaSymbol{=}\AgdaSpace{}%
\AgdaFunction{nonZeroₛ-s⇒nonZeroₛ-sᵗ}\<%
\\
%
\\[\AgdaEmptyExtraSkip]%
%
\>[2]\AgdaFunction{rev-eq-applied}\AgdaSpace{}%
\AgdaSymbol{:}\AgdaSpace{}%
\AgdaSymbol{(}\AgdaBound{rshp}\AgdaSpace{}%
\AgdaSymbol{:}\AgdaSpace{}%
\AgdaDatatype{Reshape}\AgdaSpace{}%
\AgdaGeneralizable{r₂}\AgdaSpace{}%
\AgdaGeneralizable{r₁}\AgdaSymbol{)}\AgdaSpace{}%
\AgdaSymbol{(}\AgdaBound{arr}\AgdaSpace{}%
\AgdaSymbol{:}\AgdaSpace{}%
\AgdaFunction{Ar}\AgdaSpace{}%
\AgdaGeneralizable{r₁}\AgdaSpace{}%
\AgdaField{ℂ}\AgdaSymbol{)}\AgdaSpace{}%
\AgdaSymbol{→}\AgdaSpace{}%
\AgdaFunction{reshape}\AgdaSpace{}%
\AgdaSymbol{(}\AgdaBound{rshp}\AgdaSpace{}%
\AgdaOperator{\AgdaInductiveConstructor{∙}}\AgdaSpace{}%
\AgdaFunction{rev}\AgdaSpace{}%
\AgdaBound{rshp}\AgdaSymbol{)}\AgdaSpace{}%
\AgdaBound{arr}\AgdaSpace{}%
\AgdaOperator{\AgdaFunction{≅}}\AgdaSpace{}%
\AgdaBound{arr}\<%
\\
%
\>[2]\AgdaFunction{rev-eq-applied}\AgdaSpace{}%
\AgdaBound{rshp}\AgdaSpace{}%
\AgdaBound{arr}\AgdaSpace{}%
\AgdaBound{i}\AgdaSpace{}%
\AgdaSymbol{=}\AgdaSpace{}%
\AgdaFunction{cong}\AgdaSpace{}%
\AgdaBound{arr}\AgdaSpace{}%
\AgdaSymbol{(}\AgdaFunction{rev-eq}\AgdaSpace{}%
\AgdaBound{rshp}\AgdaSpace{}%
\AgdaBound{i}\AgdaSymbol{)}\<%
\\
%
\>[2]\AgdaKeyword{postulate}\<%
\end{code}
\begin{code}%
\>[2][@{}l@{\AgdaIndent{1}}]%
\>[4]\AgdaPostulate{fft′≅dft′}\AgdaSpace{}%
\AgdaSymbol{:}\<%
\\
\>[4][@{}l@{\AgdaIndent{0}}]%
\>[8]\AgdaSymbol{⦃}\AgdaSpace{}%
\AgdaBound{nz-s}%
\>[16]\AgdaSymbol{:}\AgdaSpace{}%
\AgdaDatatype{NonZeroₛ}\AgdaSpace{}%
\AgdaGeneralizable{s}\AgdaSpace{}%
\AgdaSymbol{⦄}\<%
\\
\>[4][@{}l@{\AgdaIndent{0}}]%
\>[6]\AgdaSymbol{→}\AgdaSpace{}%
\AgdaSymbol{∀}\AgdaSpace{}%
\AgdaSymbol{(}\AgdaBound{arr}\AgdaSpace{}%
\AgdaSymbol{:}\AgdaSpace{}%
\AgdaFunction{Ar}\AgdaSpace{}%
\AgdaGeneralizable{s}\AgdaSpace{}%
\AgdaField{ℂ}\AgdaSymbol{)}\<%
\\
%
\>[6]\AgdaSymbol{→}%
\>[472I]\AgdaFunction{FFT′}\AgdaSpace{}%
\AgdaBound{arr}\<%
\\
\>[.][@{}l@{}]\<[472I]%
\>[8]\AgdaOperator{\AgdaFunction{≅}}\<%
\\
%
\>[8]\AgdaSymbol{(}\AgdaSpace{}%
\AgdaSymbol{(}\AgdaFunction{reshape}\AgdaSpace{}%
\AgdaFunction{♯}\AgdaSymbol{)}\<%
\\
%
\>[8]\AgdaOperator{\AgdaFunction{∘}}\AgdaSpace{}%
\AgdaSymbol{(}\AgdaFunction{DFT′}\AgdaSpace{}%
\AgdaSymbol{⦃}\AgdaSpace{}%
\AgdaFunction{nz-\#}\AgdaSpace{}%
\AgdaSymbol{(}\AgdaFunction{nzᵗ}\AgdaSpace{}%
\AgdaBound{nz-s}\AgdaSymbol{)}\AgdaSpace{}%
\AgdaSymbol{⦄}\AgdaSpace{}%
\AgdaSymbol{)}\<%
\\
%
\>[8]\AgdaOperator{\AgdaFunction{∘}}\AgdaSpace{}%
\AgdaSymbol{(}\AgdaFunction{reshape}\AgdaSpace{}%
\AgdaSymbol{(}\AgdaFunction{reindex}\AgdaSpace{}%
\AgdaSymbol{(}\AgdaFunction{|s|≡|sᵗ|}\AgdaSpace{}%
\AgdaSymbol{\{}\AgdaGeneralizable{s}\AgdaSymbol{\})}\AgdaSpace{}%
\AgdaOperator{\AgdaInductiveConstructor{∙}}\AgdaSpace{}%
\AgdaFunction{♭}\AgdaSymbol{)))}\AgdaSpace{}%
\AgdaBound{arr}\<%
\end{code}

To define this relation, pointwise equality \AF{\_≅\_} is used.
This defines equality between two matrices of shape \AF{s} to hold when
\AF{∀ (i : Position s) → xs i ≡ ys i}.
This allows for proofs to be defined for a general position \AF{i}.

As the DFT operates on the vector form, reshape operations are used to
flatten the input matrix and unflatten the output for comparison.
Not mentioned previously, is the reindex operation.
As the output of the FFT must be read in column major order, it is of the
form \AF{recursive-transpose s}.
When flattened this gives a matrix of shape \AF{ι (length (recursive-transpose s))}.
Meanwhile, without the use of reindex, the output of the DFT is of shape
\AF{ι (length s)}.
Reindexing allows this to be modeled as \AF{ι (length (recursive-transpose s))}
without changing the ordering of the results in this matrix.
This allows us to make use of pointwise equality.

% https://q.uiver.app/#q=WzAsMTEsWzAsMCwiXFxvdmVyYnJhY2V7XFxiZWdpbntwbWF0cml4fSB4XzEgJiB4XzIgXFxcXCB4XzMgJiB4XzQgXFxcXCB4XzUgJiB4XzYgXFxcXCB4XzcgJiB4XzggXFxlbmR7cG1hdHJpeH19XntcXHZlcmJ8c3x9Il0sWzMsMCwiXFxvdmVyYnJhY2V7XFxiZWdpbntwbWF0cml4fSB4XzEgJiB4XzIgJiB4XzMgJiB4XzQgJiB4XzUgJiB4XzYgJiB4XzcgJiB4XzggXFxlbmR7cG1hdHJpeH19XntcXHZlcmJ8bGVuZ3RoIHN8fSJdLFsyLDBdLFswLDQsIiBcXG92ZXJicmFjZXtcXGJlZ2lue3BtYXRyaXh9IFhfMSAmIFhfMiAmIFhfMyAmIFhfNCBcXFxcIFhfNSAmIFhfNiAmIFhfNyAmIFhfOCBcXGVuZHtwbWF0cml4fX1ee1xcdmVyYnxyZWN1cnNpdmUtdHJhbnNwb3NlIHN8fSJdLFsxLDJdLFszLDIsIlxcb3ZlcmJyYWNle1xcYmVnaW57cG1hdHJpeH0gWF8xICYgWF8yICYgWF8zICYgWF80ICYgWF81ICYgWF82ICYgWF83ICYgWF84IFxcZW5ke3BtYXRyaXh9fV57XFx2ZXJifGxlbmd0aCBzfH0iXSxbMywxXSxbMyw0LCJcXGJlZ2lue3BtYXRyaXh9IFhfMSAmIFhfMiBcXFxcIFhfMyAmIFhfNCBcXFxcIFhfNSAmIFhfNiBcXFxcIFhfNyAmIFhfOCBcXGVuZHtwbWF0cml4fSJdLFs3LDAsIlxcb3ZlcmJyYWNle1xcYmVnaW57cG1hdHJpeH0geF8xICYgeF8yICYgeF8zICYgeF80ICYgeF81ICYgeF82ICYgeF83ICYgeF84IFxcZW5ke3BtYXRyaXh9fV57XFx2ZXJifGxlbmd0aCAocmVjdXJzaXZlLXRyYW5zcG9zZSBzKXx9Il0sWzcsMiwiXFxvdmVyYnJhY2V7XFxiZWdpbntwbWF0cml4fSBYXzEgJiBYXzIgJiBYXzMgJiBYXzQgJiBYXzUgJiBYXzYgJiBYXzcgJiBYXzggXFxlbmR7cG1hdHJpeH19XntcXHZlcmJ8bGVuZ3RoIChyZWN1cnNpdmUtdHJhbnNwb3NlIHMpfH0iXSxbNyw0LCIgXFxvdmVyYnJhY2V7XFxiZWdpbntwbWF0cml4fSBYXzEgJiBYXzIgJiBYXzMgJiBYXzQgXFxcXCBYXzUgJiBYXzYgJiBYXzcgJiBYXzggXFxlbmR7cG1hdHJpeH19XntcXHZlcmJ8cmVjdXJzaXZlLXRyYW5zcG9zZSBzfH0iXSxbMCwxLCJcXHZlcmJ8cmVzaGFwZSB8IFxcZmxhdCJdLFswLDMsIlxcdmVyYnxGRlR8IiwyXSxbMSw1LCJcXHZlcmJ8REZUfCJdLFs1LDcsIlxcdmVyYnxyZXNoYXBlIHxcXHNoYXJwIl0sWzMsNywiXFxub3RcXGNvbmcgXFx0ZXh0e2FzIH1cXHZlcmJ8cmVjdXJzaXZlLXRyYW5zcG9zZSBzfCBcXG5vdFxcZXF1aXYgcyIsMix7InN0eWxlIjp7InRhaWwiOnsibmFtZSI6ImFycm93aGVhZCJ9LCJib2R5Ijp7Im5hbWUiOiJkb3R0ZWQifX19XSxbMSw4LCJcXHZlcmJ8cmVzaGFwZSAocmVpbmRleCB8fHN8XFxlcXVpdnxzXnR8XFx2ZXJifCl8IiwyXSxbOCw5LCJcXHZlcmJ8REZUfCIsMl0sWzksMTAsIlxcdmVyYnxyZXNoYXBlIHwgXFxzaGFycCIsMl0sWzMsMTAsIlxcY29uZyBcXHRleHR7YXMgfVxcdmVyYnxyZWN1cnNpdmUtdHJhbnNwb3NlIHN8IFxcZXF1aXZcXHZlcmJ8cmVjdXJzaXZlLXRyYW5zcG9zZSBzfFxcdGV4dHsgYW5kIGV2ZXJ5IGVsZW1lbnQgaXMgZXF1YWx9IiwyLHsiY3VydmUiOjUsInN0eWxlIjp7InRhaWwiOnsibmFtZSI6ImFycm93aGVhZCJ9fX1dXQ==
% Made with quiver and then messed with to make it actually work....
\[
\resizebox{\linewidth}{!}{
\begin{tikzcd}[ampersand replacement=\&]
	\begin{array}{c} 
		\overbrace{\begin{pmatrix} x_1 & x_2 \\ x_3 & x_4 \\ x_5 & x_6 \\ x_7 & x_8 \end{pmatrix}}^{\text{\texttt{s}}} 
	\end{array} 
	\&\& {} \& 
	{\overbrace{\begin{pmatrix} x_1 & \cdots & x_8 \end{pmatrix}}^{\text{\texttt{length s}}}} 
	\&\&\&\& 
	{\overbrace{\begin{pmatrix} x_1 & \cdots & x_8 \end{pmatrix}}^{\text{\texttt{length (recursive-transpose s)}}}} 
	\\
	\&\&\& {} \\
	\& {} \&\& 
	{\overbrace{\begin{pmatrix} X_1 & \cdots & X_8 \end{pmatrix}}^{\text{\texttt{length s}}}} 
	\&\&\&\& 
	{\overbrace{\begin{pmatrix} X_1 & \cdots & X_8 \end{pmatrix}}^{\text{\texttt{length (recursive-transpose s)}}}} 
	\\
	\\
	\begin{array}{c}  
		\overbrace{\begin{pmatrix} X_1 & X_2 & X_3 & X_4 \\ X_5 & X_6 & X_7 & X_8 \end{pmatrix}}^{\text{\texttt{recursive-transpose s}}} 
	\end{array} 
	\&\&\& 
	\begin{array}{c} 
		\begin{pmatrix} X_1 & X_2 \\ X_3 & X_4 \\ X_5 & X_6 \\ X_7 & X_8 \end{pmatrix} 
	\end{array} 
	\&\&\&\& 
	\begin{array}{c}  
		\overbrace{\begin{pmatrix} X_1 & X_2 & X_3 & X_4 \\ X_5 & X_6 & X_7 & X_8 \end{pmatrix}}^{\text{\texttt{recursive-transpose s}}} 
	\end{array} 
	\\
	{} \& {} \& {} \& {} \& {} \& {} \& {} \& {}  % empty 5th row so references like 5-1 exist
	\arrow["{\texttt{reshape } \flat}", from=1-1, to=1-4]
	\arrow["{\texttt{FFT}}"', from=1-1, to=5-1]
	\arrow["{\texttt{reshape reindex} }", from=1-4, to=1-8]
	\arrow["{\texttt{DFT}}", dotted, from=1-4, to=3-4]
	\arrow["{\texttt{DFT}}"', from=1-8, to=3-8]
	\arrow["{\texttt{reshape } \sharp}", dotted, from=3-4, to=5-4]
	\arrow["{\texttt{reshape } \sharp}"', from=3-8, to=5-8]
	\arrow["{\not\cong \text{ as } \texttt{recursive-transpose s} \not\equiv s}"', dotted, tail reversed, from=5-1, to=5-4]
	\arrow["{\cong \text{ as } \texttt{recursive-transpose s} \equiv \texttt{recursive-transpose s} \text{ and every element is equal}}"', curve={height=60pt}, tail reversed, from=5-1, to=5-8]
\end{tikzcd}
}
\]
\subsection{Chain of Reasoning}
While the proposition defines what we wish to prove, the chain of reasoning is
used to justify that the proof holds.
The full proof can be found in the attached files, while the most important sections
are discussed here.
It is important to note that as proofs must hold every invariant, at every step
a large amount of complexity is held within this chain of reasoning.
As done previously to hide \AF{NonZero}, as much complexity as possible is hidden 
in the below extracts from the main chain of reasoning as to improve readability.
This complexity remains important, however, as it what allows the strict guarantees 
provided by Agda to hold.


\begin{code}[hide]%
\>[0]\AgdaKeyword{open}\AgdaSpace{}%
\AgdaKeyword{import}\AgdaSpace{}%
\AgdaModule{Real}\AgdaSpace{}%
\AgdaKeyword{using}\AgdaSpace{}%
\AgdaSymbol{(}\AgdaRecord{Real}\AgdaSymbol{)}\<%
\\
\>[0]\AgdaKeyword{open}\AgdaSpace{}%
\AgdaKeyword{import}\AgdaSpace{}%
\AgdaModule{Complex}\AgdaSpace{}%
\AgdaKeyword{using}\AgdaSpace{}%
\AgdaSymbol{(}\AgdaRecord{Cplx}\AgdaSymbol{)}\<%
\\
%
\\[\AgdaEmptyExtraSkip]%
\>[0]\AgdaKeyword{import}\AgdaSpace{}%
\AgdaModule{Algebra.Structures}\AgdaSpace{}%
\AgdaSymbol{as}\AgdaSpace{}%
\AgdaModule{AlgebraStructures}\<%
\\
\>[0]\AgdaKeyword{import}\AgdaSpace{}%
\AgdaModule{Algebra.Definitions}\AgdaSpace{}%
\AgdaSymbol{as}\AgdaSpace{}%
\AgdaModule{AlgebraDefinitions}\<%
\\
%
\\[\AgdaEmptyExtraSkip]%
\>[0]\AgdaKeyword{open}\AgdaSpace{}%
\AgdaKeyword{import}\AgdaSpace{}%
\AgdaModule{Relation.Nullary}\<%
\\
\>[0]\AgdaKeyword{import}\AgdaSpace{}%
\AgdaModule{Relation.Binary.PropositionalEquality}\AgdaSpace{}%
\AgdaSymbol{as}\AgdaSpace{}%
\AgdaModule{Eq}\<%
\\
\>[0]\AgdaKeyword{open}\AgdaSpace{}%
\AgdaModule{Eq}\AgdaSpace{}%
\AgdaKeyword{using}\AgdaSpace{}%
\AgdaSymbol{(}\AgdaOperator{\AgdaDatatype{\AgdaUnderscore{}≡\AgdaUnderscore{}}}\AgdaSymbol{;}\AgdaSpace{}%
\AgdaInductiveConstructor{refl}\AgdaSymbol{;}\AgdaSpace{}%
\AgdaFunction{cong}\AgdaSymbol{;}\AgdaSpace{}%
\AgdaFunction{trans}\AgdaSymbol{;}\AgdaSpace{}%
\AgdaFunction{sym}\AgdaSymbol{;}\AgdaSpace{}%
\AgdaFunction{cong₂}\AgdaSymbol{;}\AgdaSpace{}%
\AgdaFunction{subst}\AgdaSymbol{;}\AgdaSpace{}%
\AgdaFunction{cong-app}\AgdaSymbol{;}\AgdaSpace{}%
\AgdaFunction{cong′}\AgdaSymbol{;}\AgdaSpace{}%
\AgdaFunction{icong}\AgdaSymbol{)}\<%
\\
\>[0]\AgdaKeyword{open}\AgdaSpace{}%
\AgdaModule{Eq.≡-Reasoning}\<%
\\
%
\\[\AgdaEmptyExtraSkip]%
\>[0]\AgdaKeyword{module}\AgdaSpace{}%
\AgdaModule{Proof}\AgdaSpace{}%
\AgdaSymbol{(}\AgdaBound{real}\AgdaSpace{}%
\AgdaSymbol{:}\AgdaSpace{}%
\AgdaRecord{Real}\AgdaSymbol{)}\AgdaSpace{}%
\AgdaSymbol{(}\AgdaBound{cplx}\AgdaSpace{}%
\AgdaSymbol{:}\AgdaSpace{}%
\AgdaRecord{Cplx}\AgdaSpace{}%
\AgdaBound{real}\AgdaSymbol{)}\AgdaSpace{}%
\AgdaKeyword{where}\<%
\\
%
\\[\AgdaEmptyExtraSkip]%
\>[0][@{}l@{\AgdaIndent{0}}]%
\>[2]\AgdaKeyword{open}\AgdaSpace{}%
\AgdaModule{Real.Real}\AgdaSpace{}%
\AgdaBound{real}\AgdaSpace{}%
\AgdaKeyword{using}\AgdaSpace{}%
\AgdaSymbol{(}\AgdaOperator{\AgdaField{\AgdaUnderscore{}ᵣ}}\AgdaSymbol{;}\AgdaSpace{}%
\AgdaField{ℝ}\AgdaSymbol{)}\<%
\\
\>[2][@{}l@{\AgdaIndent{0}}]%
\>[4]\AgdaKeyword{renaming}\AgdaSpace{}%
\AgdaSymbol{(}\AgdaOperator{\AgdaField{\AgdaUnderscore{}+\AgdaUnderscore{}}}\AgdaSpace{}%
\AgdaSymbol{to}\AgdaSpace{}%
\AgdaOperator{\AgdaField{\AgdaUnderscore{}+ᵣ\AgdaUnderscore{}}}\AgdaSymbol{;}\AgdaSpace{}%
\AgdaOperator{\AgdaField{\AgdaUnderscore{}-\AgdaUnderscore{}}}\AgdaSpace{}%
\AgdaSymbol{to}\AgdaSpace{}%
\AgdaOperator{\AgdaField{\AgdaUnderscore{}-ᵣ\AgdaUnderscore{}}}\AgdaSymbol{;}\AgdaSpace{}%
\AgdaOperator{\AgdaField{-\AgdaUnderscore{}}}\AgdaSpace{}%
\AgdaSymbol{to}\AgdaSpace{}%
\AgdaOperator{\AgdaField{-ᵣ\AgdaUnderscore{}}}\AgdaSymbol{;}\AgdaSpace{}%
\AgdaOperator{\AgdaField{\AgdaUnderscore{}/\AgdaUnderscore{}}}\AgdaSpace{}%
\AgdaSymbol{to}\AgdaSpace{}%
\AgdaOperator{\AgdaField{\AgdaUnderscore{}/ᵣ\AgdaUnderscore{}}}\AgdaSymbol{;}\AgdaSpace{}%
\AgdaOperator{\AgdaField{\AgdaUnderscore{}*\AgdaUnderscore{}}}\AgdaSpace{}%
\AgdaSymbol{to}\AgdaSpace{}%
\AgdaOperator{\AgdaField{\AgdaUnderscore{}*ᵣ\AgdaUnderscore{}}}\AgdaSymbol{)}\<%
\\
%
\>[2]\AgdaKeyword{open}\AgdaSpace{}%
\AgdaModule{Cplx}\AgdaSpace{}%
\AgdaBound{cplx}\AgdaSpace{}%
\AgdaKeyword{using}\AgdaSpace{}%
\AgdaSymbol{(}\AgdaField{ℂ}\AgdaSymbol{;}\AgdaSpace{}%
\AgdaOperator{\AgdaField{\AgdaUnderscore{}+\AgdaUnderscore{}}}\AgdaSymbol{;}\AgdaSpace{}%
\AgdaField{fromℝ}\AgdaSymbol{;}\AgdaSpace{}%
\AgdaOperator{\AgdaField{\AgdaUnderscore{}*\AgdaUnderscore{}}}\AgdaSymbol{;}\AgdaSpace{}%
\AgdaField{-ω}\AgdaSymbol{;}\AgdaSpace{}%
\AgdaFunction{0ℂ}\AgdaSymbol{;}\AgdaSpace{}%
\AgdaField{+-*-isCommutativeRing}\AgdaSymbol{;}\AgdaSpace{}%
\AgdaField{ω-r₁x-r₁y}\AgdaSymbol{;}\AgdaSpace{}%
\AgdaField{ω-N-mN}\AgdaSymbol{;}\AgdaSpace{}%
\AgdaField{ω-N-k₀+k₁}\AgdaSymbol{)}\<%
\\
%
\\[\AgdaEmptyExtraSkip]%
%
\>[2]\AgdaKeyword{open}\AgdaSpace{}%
\AgdaModule{AlgebraStructures}%
\>[26]\AgdaSymbol{\{}\AgdaArgument{A}\AgdaSpace{}%
\AgdaSymbol{=}\AgdaSpace{}%
\AgdaField{ℂ}\AgdaSymbol{\}}\AgdaSpace{}%
\AgdaOperator{\AgdaDatatype{\AgdaUnderscore{}≡\AgdaUnderscore{}}}\<%
\\
%
\>[2]\AgdaKeyword{open}\AgdaSpace{}%
\AgdaModule{AlgebraDefinitions}\AgdaSpace{}%
\AgdaSymbol{\{}\AgdaArgument{A}\AgdaSpace{}%
\AgdaSymbol{=}\AgdaSpace{}%
\AgdaField{ℂ}\AgdaSymbol{\}}\AgdaSpace{}%
\AgdaOperator{\AgdaDatatype{\AgdaUnderscore{}≡\AgdaUnderscore{}}}\<%
\\
%
\\[\AgdaEmptyExtraSkip]%
%
\>[2]\AgdaKeyword{open}\AgdaSpace{}%
\AgdaModule{IsCommutativeRing}\AgdaSpace{}%
\AgdaField{+-*-isCommutativeRing}\AgdaSpace{}%
\AgdaKeyword{using}\AgdaSpace{}%
\AgdaSymbol{(}\AgdaFunction{+-isCommutativeMonoid}\AgdaSymbol{;}\AgdaSpace{}%
\AgdaFunction{distribˡ}\AgdaSymbol{;}\AgdaSpace{}%
\AgdaField{*-comm}\AgdaSymbol{;}\AgdaSpace{}%
\AgdaFunction{zeroʳ}\AgdaSymbol{;}\AgdaSpace{}%
\AgdaFunction{zeroˡ}\AgdaSymbol{;}\AgdaSpace{}%
\AgdaFunction{*-identityʳ}\AgdaSymbol{;}\AgdaSpace{}%
\AgdaFunction{*-assoc}\AgdaSymbol{;}\AgdaSpace{}%
\AgdaFunction{+-identityʳ}\AgdaSymbol{;}\AgdaSpace{}%
\AgdaFunction{+-assoc}\AgdaSymbol{;}\AgdaSpace{}%
\AgdaFunction{+-comm}\AgdaSymbol{;}\AgdaSpace{}%
\AgdaFunction{+-identityˡ}\AgdaSymbol{)}\<%
\\
%
\\[\AgdaEmptyExtraSkip]%
%
\>[2]\AgdaKeyword{open}\AgdaSpace{}%
\AgdaKeyword{import}\AgdaSpace{}%
\AgdaModule{Data.Nat.Base}\AgdaSpace{}%
\AgdaKeyword{using}\AgdaSpace{}%
\AgdaSymbol{(}\AgdaDatatype{ℕ}\AgdaSymbol{;}\AgdaSpace{}%
\AgdaInductiveConstructor{zero}\AgdaSymbol{;}\AgdaSpace{}%
\AgdaInductiveConstructor{suc}\AgdaSymbol{;}\AgdaSpace{}%
\AgdaRecord{NonZero}\AgdaSymbol{;}\AgdaSpace{}%
\AgdaOperator{\AgdaPrimitive{\AgdaUnderscore{}≡ᵇ\AgdaUnderscore{}}}\AgdaSymbol{;}\AgdaSpace{}%
\AgdaFunction{nonZero}\AgdaSymbol{)}\AgdaSpace{}%
\AgdaKeyword{renaming}\AgdaSpace{}%
\AgdaSymbol{(}\AgdaOperator{\AgdaPrimitive{\AgdaUnderscore{}*\AgdaUnderscore{}}}\AgdaSpace{}%
\AgdaSymbol{to}\AgdaSpace{}%
\AgdaOperator{\AgdaPrimitive{\AgdaUnderscore{}*ₙ\AgdaUnderscore{}}}\AgdaSymbol{;}\AgdaSpace{}%
\AgdaOperator{\AgdaPrimitive{\AgdaUnderscore{}+\AgdaUnderscore{}}}\AgdaSpace{}%
\AgdaSymbol{to}\AgdaSpace{}%
\AgdaOperator{\AgdaPrimitive{\AgdaUnderscore{}+ₙ\AgdaUnderscore{}}}\AgdaSymbol{)}\<%
\\
%
\>[2]\AgdaKeyword{open}\AgdaSpace{}%
\AgdaKeyword{import}\AgdaSpace{}%
\AgdaModule{Data.Nat.Properties}\AgdaSpace{}%
\AgdaKeyword{using}\AgdaSpace{}%
\AgdaSymbol{(}\AgdaFunction{suc-injective}\AgdaSymbol{;}\AgdaSpace{}%
\AgdaFunction{m*n≢0}\AgdaSymbol{;}\AgdaSpace{}%
\AgdaFunction{m*n≢0⇒m≢0}\AgdaSymbol{;}\AgdaSpace{}%
\AgdaFunction{m*n≢0⇒n≢0}\AgdaSymbol{;}\AgdaSpace{}%
\AgdaFunction{nonZero?}\AgdaSymbol{)}\AgdaSpace{}%
\AgdaKeyword{renaming}\AgdaSpace{}%
\AgdaSymbol{(}\AgdaFunction{*-comm}\AgdaSpace{}%
\AgdaSymbol{to}\AgdaSpace{}%
\AgdaFunction{*ₙ-comm}\AgdaSymbol{;}\AgdaSpace{}%
\AgdaFunction{*-identityʳ}\AgdaSpace{}%
\AgdaSymbol{to}\AgdaSpace{}%
\AgdaFunction{*ₙ-identityʳ}\AgdaSymbol{;}\AgdaSpace{}%
\AgdaFunction{*-assoc}\AgdaSpace{}%
\AgdaSymbol{to}\AgdaSpace{}%
\AgdaFunction{*ₙ-assoc}\AgdaSymbol{;}\<%
\\
\>[2][@{}l@{\AgdaIndent{0}}]%
\>[4]\AgdaFunction{+-identityʳ}\AgdaSpace{}%
\AgdaSymbol{to}\AgdaSpace{}%
\AgdaFunction{+ₙ-identityʳ}\AgdaSymbol{;}\AgdaSpace{}%
\AgdaFunction{*-zeroˡ}\AgdaSpace{}%
\AgdaSymbol{to}\AgdaSpace{}%
\AgdaFunction{*ₙ-zeroˡ}\AgdaSymbol{;}\AgdaSpace{}%
\AgdaFunction{*-zeroʳ}\AgdaSpace{}%
\AgdaSymbol{to}\AgdaSpace{}%
\AgdaFunction{*ₙ-zeroʳ}\AgdaSymbol{)}\<%
\\
%
\>[2]\AgdaKeyword{open}\AgdaSpace{}%
\AgdaKeyword{import}\AgdaSpace{}%
\AgdaModule{Data.Nat.Solver}\AgdaSpace{}%
\AgdaKeyword{using}\AgdaSpace{}%
\AgdaSymbol{(}\AgdaKeyword{module}\AgdaSpace{}%
\AgdaModule{+-*-Solver}\AgdaSymbol{)}\<%
\\
%
\>[2]\AgdaKeyword{open}\AgdaSpace{}%
\AgdaModule{+-*-Solver}\AgdaSpace{}%
\AgdaKeyword{using}\AgdaSpace{}%
\AgdaSymbol{(}\AgdaFunction{solve}\AgdaSymbol{;}\AgdaSpace{}%
\AgdaOperator{\AgdaFunction{\AgdaUnderscore{}:*\AgdaUnderscore{}}}\AgdaSymbol{;}\AgdaSpace{}%
\AgdaOperator{\AgdaFunction{\AgdaUnderscore{}:+\AgdaUnderscore{}}}\AgdaSymbol{;}\AgdaSpace{}%
\AgdaInductiveConstructor{con}\AgdaSymbol{;}\AgdaSpace{}%
\AgdaOperator{\AgdaFunction{\AgdaUnderscore{}:=\AgdaUnderscore{}}}\AgdaSymbol{)}\<%
\\
%
\>[2]\AgdaKeyword{open}\AgdaSpace{}%
\AgdaKeyword{import}\AgdaSpace{}%
\AgdaModule{Data.Fin.Base}\AgdaSpace{}%
\AgdaKeyword{using}\AgdaSpace{}%
\AgdaSymbol{(}\AgdaDatatype{Fin}\AgdaSymbol{;}\AgdaSpace{}%
\AgdaFunction{quotRem}\AgdaSymbol{;}\AgdaSpace{}%
\AgdaFunction{toℕ}\AgdaSymbol{;}\AgdaSpace{}%
\AgdaFunction{combine}\AgdaSymbol{;}\AgdaSpace{}%
\AgdaFunction{remQuot}\AgdaSymbol{;}\AgdaSpace{}%
\AgdaFunction{quotient}\AgdaSymbol{;}\AgdaSpace{}%
\AgdaFunction{remainder}\AgdaSymbol{;}\AgdaSpace{}%
\AgdaFunction{cast}\AgdaSymbol{;}\AgdaSpace{}%
\AgdaFunction{fromℕ<}\AgdaSymbol{;}\AgdaSpace{}%
\AgdaOperator{\AgdaFunction{\AgdaUnderscore{}↑ˡ\AgdaUnderscore{}}}\AgdaSymbol{;}\AgdaSpace{}%
\AgdaOperator{\AgdaFunction{\AgdaUnderscore{}↑ʳ\AgdaUnderscore{}}}\AgdaSymbol{;}\AgdaSpace{}%
\AgdaFunction{splitAt}\AgdaSymbol{;}\AgdaSpace{}%
\AgdaFunction{join}\AgdaSymbol{)}\AgdaSpace{}%
\AgdaKeyword{renaming}\AgdaSpace{}%
\AgdaSymbol{(}\AgdaInductiveConstructor{zero}\AgdaSpace{}%
\AgdaSymbol{to}\AgdaSpace{}%
\AgdaInductiveConstructor{fzero}\AgdaSymbol{;}\AgdaSpace{}%
\AgdaInductiveConstructor{suc}\AgdaSpace{}%
\AgdaSymbol{to}\AgdaSpace{}%
\AgdaInductiveConstructor{fsuc}\AgdaSymbol{)}\<%
\\
%
\>[2]\AgdaKeyword{open}\AgdaSpace{}%
\AgdaKeyword{import}\AgdaSpace{}%
\AgdaModule{Data.Fin.Properties}\AgdaSpace{}%
\AgdaKeyword{using}\AgdaSpace{}%
\AgdaSymbol{(}\AgdaFunction{cast-is-id}\AgdaSymbol{;}\AgdaSpace{}%
\AgdaFunction{remQuot-combine}\AgdaSymbol{;}\AgdaSpace{}%
\AgdaFunction{splitAt-↑ˡ}\AgdaSymbol{;}\AgdaSpace{}%
\AgdaFunction{splitAt-↑ʳ}\AgdaSymbol{;}\AgdaSpace{}%
\AgdaFunction{toℕ-↑ˡ}\AgdaSymbol{;}\AgdaSpace{}%
\AgdaFunction{toℕ-↑ʳ}\AgdaSymbol{;}\AgdaSpace{}%
\AgdaFunction{toℕ-combine}\AgdaSymbol{;}\AgdaSpace{}%
\AgdaFunction{combine-remQuot}\AgdaSymbol{;}\AgdaSpace{}%
\AgdaFunction{combine-surjective}\AgdaSymbol{;}\AgdaSpace{}%
\AgdaFunction{toℕ-injective}\AgdaSymbol{;}\AgdaSpace{}%
\AgdaFunction{toℕ-cast}\AgdaSymbol{;}\AgdaSpace{}%
\AgdaFunction{cast-trans}\AgdaSymbol{)}\<%
\\
%
\>[2]\AgdaKeyword{open}\AgdaSpace{}%
\AgdaKeyword{import}\AgdaSpace{}%
\AgdaModule{Data.Bool}\AgdaSpace{}%
\AgdaKeyword{using}\AgdaSpace{}%
\AgdaSymbol{(}\AgdaDatatype{Bool}\AgdaSymbol{;}\AgdaSpace{}%
\AgdaInductiveConstructor{true}\AgdaSymbol{;}\AgdaSpace{}%
\AgdaInductiveConstructor{false}\AgdaSymbol{;}\AgdaSpace{}%
\AgdaFunction{not}\AgdaSymbol{)}\<%
\\
%
\>[2]\AgdaKeyword{open}\AgdaSpace{}%
\AgdaKeyword{import}\AgdaSpace{}%
\AgdaModule{Data.Bool}\AgdaSpace{}%
\AgdaKeyword{using}\AgdaSpace{}%
\AgdaSymbol{(}\AgdaFunction{T}\AgdaSymbol{)}\<%
\\
%
\>[2]\AgdaKeyword{open}\AgdaSpace{}%
\AgdaKeyword{import}\AgdaSpace{}%
\AgdaModule{Data.Empty}\<%
\\
%
\\[\AgdaEmptyExtraSkip]%
%
\>[2]\AgdaKeyword{open}\AgdaSpace{}%
\AgdaKeyword{import}\AgdaSpace{}%
\AgdaModule{Data.Product.Base}\AgdaSpace{}%
\AgdaKeyword{using}\AgdaSpace{}%
\AgdaSymbol{(}\AgdaFunction{∃}\AgdaSymbol{;}\AgdaSpace{}%
\AgdaFunction{∃₂}\AgdaSymbol{;}\AgdaSpace{}%
\AgdaOperator{\AgdaFunction{\AgdaUnderscore{}×\AgdaUnderscore{}}}\AgdaSymbol{;}\AgdaSpace{}%
\AgdaField{proj₁}\AgdaSymbol{;}\AgdaSpace{}%
\AgdaField{proj₂}\AgdaSymbol{;}\AgdaSpace{}%
\AgdaFunction{map₁}\AgdaSymbol{;}\AgdaSpace{}%
\AgdaFunction{map₂}\AgdaSymbol{;}\AgdaSpace{}%
\AgdaFunction{uncurry}\AgdaSymbol{)}\AgdaSpace{}%
\AgdaKeyword{renaming}\AgdaSpace{}%
\AgdaSymbol{(}\AgdaSpace{}%
\AgdaOperator{\AgdaInductiveConstructor{\AgdaUnderscore{},\AgdaUnderscore{}}}\AgdaSpace{}%
\AgdaSymbol{to}\AgdaSpace{}%
\AgdaOperator{\AgdaInductiveConstructor{⟨\AgdaUnderscore{},\AgdaUnderscore{}⟩}}\AgdaSymbol{)}\<%
\\
%
\>[2]\AgdaKeyword{open}\AgdaSpace{}%
\AgdaKeyword{import}\AgdaSpace{}%
\AgdaModule{Data.Sum.Base}\AgdaSpace{}%
\AgdaKeyword{using}\AgdaSpace{}%
\AgdaSymbol{(}\AgdaInductiveConstructor{inj₁}\AgdaSymbol{;}\AgdaSpace{}%
\AgdaInductiveConstructor{inj₂}\AgdaSymbol{;}\AgdaSpace{}%
\AgdaOperator{\AgdaDatatype{\AgdaUnderscore{}⊎\AgdaUnderscore{}}}\AgdaSymbol{)}\<%
\\
%
\>[2]\AgdaKeyword{open}\AgdaSpace{}%
\AgdaKeyword{import}\AgdaSpace{}%
\AgdaModule{Data.Unit}\AgdaSpace{}%
\AgdaKeyword{using}\AgdaSpace{}%
\AgdaSymbol{(}\AgdaRecord{⊤}\AgdaSymbol{;}\AgdaSpace{}%
\AgdaInductiveConstructor{tt}\AgdaSymbol{)}\<%
\\
%
\\[\AgdaEmptyExtraSkip]%
%
\>[2]\AgdaKeyword{open}\AgdaSpace{}%
\AgdaKeyword{import}\AgdaSpace{}%
\AgdaModule{Matrix}\AgdaSpace{}%
\AgdaKeyword{using}\AgdaSpace{}%
\AgdaSymbol{(}\AgdaFunction{Ar}\AgdaSymbol{;}\AgdaSpace{}%
\AgdaDatatype{Shape}\AgdaSymbol{;}\AgdaSpace{}%
\AgdaOperator{\AgdaInductiveConstructor{\AgdaUnderscore{}⊗\AgdaUnderscore{}}}\AgdaSymbol{;}\AgdaSpace{}%
\AgdaInductiveConstructor{ι}\AgdaSymbol{;}\AgdaSpace{}%
\AgdaDatatype{Position}\AgdaSymbol{;}\AgdaSpace{}%
\AgdaFunction{mapRows}\AgdaSymbol{;}\AgdaSpace{}%
\AgdaFunction{zipWith}\AgdaSymbol{;}\AgdaSpace{}%
\AgdaFunction{nest}\AgdaSymbol{;}\AgdaSpace{}%
\AgdaFunction{map}\AgdaSymbol{;}\AgdaSpace{}%
\AgdaFunction{unnest}\AgdaSymbol{;}\AgdaSpace{}%
\AgdaFunction{head₁}\AgdaSymbol{;}\AgdaSpace{}%
\AgdaFunction{tail₁}\AgdaSymbol{;}\AgdaSpace{}%
\AgdaFunction{zip}\AgdaSymbol{;}\AgdaSpace{}%
\AgdaFunction{iterate}\AgdaSymbol{;}\AgdaSpace{}%
\AgdaFunction{ι-cons}\AgdaSymbol{;}\AgdaSpace{}%
\AgdaFunction{nil}\AgdaSymbol{;}\AgdaSpace{}%
\AgdaFunction{length}\AgdaSymbol{;}\AgdaSpace{}%
\AgdaFunction{splitAr}\AgdaSymbol{;}\AgdaSpace{}%
\AgdaFunction{splitArₗ}\AgdaSymbol{;}\AgdaSpace{}%
\AgdaFunction{splitArᵣ}\AgdaSymbol{)}\<%
\\
%
\>[2]\AgdaKeyword{open}\AgdaSpace{}%
\AgdaKeyword{import}\AgdaSpace{}%
\AgdaModule{Matrix.Equality}\AgdaSpace{}%
\AgdaKeyword{using}\AgdaSpace{}%
\AgdaSymbol{(}\AgdaOperator{\AgdaFunction{\AgdaUnderscore{}≅\AgdaUnderscore{}}}\AgdaSymbol{;}\AgdaSpace{}%
\AgdaFunction{reduce-≅}\AgdaSymbol{;}\AgdaSpace{}%
\AgdaFunction{tail₁-cong}\AgdaSymbol{)}\<%
\\
%
\>[2]\AgdaKeyword{open}\AgdaSpace{}%
\AgdaKeyword{import}\AgdaSpace{}%
\AgdaModule{Matrix.Properties}\AgdaSpace{}%
\AgdaKeyword{using}\AgdaSpace{}%
\AgdaSymbol{(}\AgdaFunction{splitArᵣ-zero}\AgdaSymbol{;}\AgdaSpace{}%
\AgdaFunction{tail₁-const}\AgdaSymbol{;}\AgdaSpace{}%
\AgdaFunction{zipWith-congˡ}\AgdaSymbol{)}\<%
\\
%
\>[2]\AgdaKeyword{open}\AgdaSpace{}%
\AgdaKeyword{import}\AgdaSpace{}%
\AgdaModule{Matrix.NonZero}\AgdaSpace{}%
\AgdaKeyword{using}\AgdaSpace{}%
\AgdaSymbol{(}\AgdaDatatype{NonZeroₛ}\AgdaSymbol{;}\AgdaSpace{}%
\AgdaInductiveConstructor{ι}\AgdaSymbol{;}\AgdaSpace{}%
\AgdaOperator{\AgdaInductiveConstructor{\AgdaUnderscore{}⊗\AgdaUnderscore{}}}\AgdaSymbol{;}\AgdaSpace{}%
\AgdaFunction{nonZeroₛ-s⇒nonZero-s}\AgdaSymbol{;}\AgdaSpace{}%
\AgdaFunction{nonZeroDec}\AgdaSymbol{;}\AgdaSpace{}%
\AgdaFunction{nonZeroₛ-s⇒nonZeroₛ-sᵗ}\AgdaSymbol{;}\AgdaSpace{}%
\AgdaFunction{nonZeroₛ-s⇒nonZero-sᵗ}\AgdaSymbol{;}\AgdaSpace{}%
\AgdaFunction{¬nonZeroₛ-s⇒¬nonZero-sᵗ}\AgdaSymbol{;}\AgdaSpace{}%
\AgdaFunction{¬nonZero-N⇒PosN-irrelevant}\AgdaSymbol{;}\AgdaSpace{}%
\AgdaFunction{¬nonZero-sᵗ⇒¬nonZero-s}\AgdaSymbol{)}\<%
\\
%
\\[\AgdaEmptyExtraSkip]%
%
\>[2]\AgdaKeyword{import}\AgdaSpace{}%
\AgdaModule{Matrix.Sum}\AgdaSpace{}%
\AgdaSymbol{as}\AgdaSpace{}%
\AgdaModule{S}\<%
\\
%
\>[2]\AgdaKeyword{open}\AgdaSpace{}%
\AgdaModule{S}\AgdaSpace{}%
\AgdaOperator{\AgdaField{\AgdaUnderscore{}+\AgdaUnderscore{}}}\AgdaSpace{}%
\AgdaFunction{0ℂ}\AgdaSpace{}%
\AgdaFunction{+-isCommutativeMonoid}\AgdaSpace{}%
\AgdaKeyword{using}\AgdaSpace{}%
\AgdaSymbol{(}\AgdaFunction{merge-sum}\AgdaSymbol{;}\AgdaSpace{}%
\AgdaFunction{sum-reindex}\AgdaSymbol{;}\AgdaSpace{}%
\AgdaFunction{sum-swap}\AgdaSymbol{)}\<%
\\
%
\>[2]\AgdaFunction{sum}\AgdaSpace{}%
\AgdaSymbol{=}\AgdaSpace{}%
\AgdaFunction{S.sum}\AgdaSpace{}%
\AgdaOperator{\AgdaField{\AgdaUnderscore{}+\AgdaUnderscore{}}}\AgdaSpace{}%
\AgdaFunction{0ℂ}\AgdaSpace{}%
\AgdaFunction{+-isCommutativeMonoid}\<%
\\
%
\>[2]\AgdaSymbol{\{-\#}\AgdaSpace{}%
\AgdaKeyword{DISPLAY}\AgdaSpace{}%
\AgdaFunction{S.sum}\AgdaSpace{}%
\AgdaOperator{\AgdaBound{\AgdaUnderscore{}+\AgdaUnderscore{}}}\AgdaSpace{}%
\AgdaBound{0ℂ}\AgdaSpace{}%
\AgdaBound{+-isCommutativeMonoid}\AgdaSpace{}%
\AgdaPragma{=}\AgdaSpace{}%
\AgdaFunction{sum}\AgdaSpace{}%
\AgdaSymbol{\#-\}}\<%
\\
%
\>[2]\AgdaFunction{sum-cong}\AgdaSpace{}%
\AgdaSymbol{=}\AgdaSpace{}%
\AgdaFunction{S.sum-cong}\AgdaSpace{}%
\AgdaOperator{\AgdaField{\AgdaUnderscore{}+\AgdaUnderscore{}}}\AgdaSpace{}%
\AgdaFunction{0ℂ}\AgdaSpace{}%
\AgdaFunction{+-isCommutativeMonoid}\<%
\\
%
\\[\AgdaEmptyExtraSkip]%
%
\>[2]\AgdaKeyword{open}\AgdaSpace{}%
\AgdaKeyword{import}\AgdaSpace{}%
\AgdaModule{Matrix.Reshape}\AgdaSpace{}%
\AgdaKeyword{using}\AgdaSpace{}%
\AgdaSymbol{(}\AgdaFunction{reshape}\AgdaSymbol{;}\AgdaSpace{}%
\AgdaDatatype{Reshape}\AgdaSymbol{;}\AgdaSpace{}%
\AgdaInductiveConstructor{flat}\AgdaSymbol{;}\AgdaSpace{}%
\AgdaFunction{♭}\AgdaSymbol{;}\AgdaSpace{}%
\AgdaFunction{♯}\AgdaSymbol{;}\AgdaSpace{}%
\AgdaFunction{recursive-transpose}\AgdaSymbol{;}\AgdaSpace{}%
\AgdaFunction{recursive-transposeᵣ}\AgdaSymbol{;}\AgdaSpace{}%
\AgdaOperator{\AgdaInductiveConstructor{\AgdaUnderscore{}∙\AgdaUnderscore{}}}\AgdaSymbol{;}\AgdaSpace{}%
\AgdaFunction{rev}\AgdaSymbol{;}\AgdaSpace{}%
\AgdaOperator{\AgdaInductiveConstructor{\AgdaUnderscore{}⊕\AgdaUnderscore{}}}\AgdaSymbol{;}\AgdaSpace{}%
\AgdaInductiveConstructor{swap}\AgdaSymbol{;}\AgdaSpace{}%
\AgdaInductiveConstructor{eq}\AgdaSymbol{;}\AgdaSpace{}%
\AgdaInductiveConstructor{split}\AgdaSymbol{;}\AgdaSpace{}%
\AgdaOperator{\AgdaFunction{\AgdaUnderscore{}⟨\AgdaUnderscore{}⟩}}\AgdaSymbol{;}\AgdaSpace{}%
\AgdaFunction{reindex}\AgdaSymbol{;}\AgdaSpace{}%
\AgdaFunction{rev-eq}\AgdaSymbol{;}\AgdaSpace{}%
\AgdaFunction{flatten-reindex}\AgdaSymbol{;}\AgdaSpace{}%
\AgdaFunction{|s|≡|sᵗ|}\AgdaSymbol{;}\AgdaSpace{}%
\AgdaFunction{reindex-reindex}\AgdaSymbol{;}\AgdaSpace{}%
\AgdaFunction{recursive-transpose-inv}\AgdaSymbol{)}\<%
\\
%
\>[2]\AgdaKeyword{open}\AgdaSpace{}%
\AgdaKeyword{import}\AgdaSpace{}%
\AgdaModule{Function.Base}\AgdaSpace{}%
\AgdaKeyword{using}\AgdaSpace{}%
\AgdaSymbol{(}\AgdaOperator{\AgdaFunction{\AgdaUnderscore{}\$\AgdaUnderscore{}}}\AgdaSymbol{;}\AgdaSpace{}%
\AgdaFunction{id}\AgdaSymbol{;}\AgdaSpace{}%
\AgdaOperator{\AgdaFunction{\AgdaUnderscore{}∘\AgdaUnderscore{}}}\AgdaSymbol{;}\AgdaSpace{}%
\AgdaFunction{flip}\AgdaSymbol{;}\AgdaSpace{}%
\AgdaOperator{\AgdaFunction{\AgdaUnderscore{}∘₂\AgdaUnderscore{}}}\AgdaSymbol{)}\<%
\\
%
\\[\AgdaEmptyExtraSkip]%
%
\>[2]\AgdaKeyword{open}\AgdaSpace{}%
\AgdaKeyword{import}\AgdaSpace{}%
\AgdaModule{FFT}\AgdaSpace{}%
\AgdaBound{real}\AgdaSpace{}%
\AgdaBound{cplx}\AgdaSpace{}%
\AgdaKeyword{using}\AgdaSpace{}%
\AgdaSymbol{(}\AgdaFunction{DFT}\AgdaSymbol{;}\AgdaSpace{}%
\AgdaFunction{FFT}\AgdaSymbol{;}\AgdaSpace{}%
\AgdaFunction{DFT′}\AgdaSymbol{;}\AgdaSpace{}%
\AgdaFunction{FFT′}\AgdaSymbol{;}\AgdaSpace{}%
\AgdaFunction{offset-prod}\AgdaSymbol{;}\AgdaSpace{}%
\AgdaFunction{iota}\AgdaSymbol{;}\AgdaSpace{}%
\AgdaFunction{twiddles}\AgdaSymbol{)}\<%
\\
%
\\[\AgdaEmptyExtraSkip]%
%
\>[2]\AgdaKeyword{private}\<%
\\
\>[2][@{}l@{\AgdaIndent{0}}]%
\>[4]\AgdaKeyword{variable}\<%
\\
\>[4][@{}l@{\AgdaIndent{0}}]%
\>[6]\AgdaGeneralizable{s}\AgdaSpace{}%
\AgdaGeneralizable{p}\AgdaSpace{}%
\AgdaGeneralizable{r₁}\AgdaSpace{}%
\AgdaGeneralizable{r₂}\AgdaSpace{}%
\AgdaSymbol{:}\AgdaSpace{}%
\AgdaDatatype{Shape}\<%
\\
%
\>[6]\AgdaGeneralizable{N}\AgdaSpace{}%
\AgdaGeneralizable{M}\AgdaSpace{}%
\AgdaSymbol{:}\AgdaSpace{}%
\AgdaDatatype{ℕ}\<%
\\
%
\\[\AgdaEmptyExtraSkip]%
%
\>[2]\AgdaComment{-----------------------------------------}\<%
\\
%
\>[2]\AgdaComment{---\ Shorthands\ to\ improve\ readability\ ---}\<%
\\
%
\>[2]\AgdaComment{-----------------------------------------}\<%
\\
%
\\[\AgdaEmptyExtraSkip]%
%
\>[2]\AgdaKeyword{infixl}\AgdaSpace{}%
\AgdaNumber{5}\AgdaSpace{}%
\AgdaOperator{\AgdaFunction{\AgdaUnderscore{}⊡\AgdaUnderscore{}}}\<%
\\
%
\>[2]\AgdaOperator{\AgdaFunction{\AgdaUnderscore{}⊡\AgdaUnderscore{}}}\AgdaSpace{}%
\AgdaSymbol{=}\AgdaSpace{}%
\AgdaFunction{trans}\<%
\\
%
\\[\AgdaEmptyExtraSkip]%
%
\>[2]\AgdaKeyword{infix}\AgdaSpace{}%
\AgdaNumber{10}\AgdaSpace{}%
\AgdaOperator{\AgdaFunction{\#\AgdaUnderscore{}}}\<%
\\
%
\>[2]\AgdaOperator{\AgdaFunction{\#\AgdaUnderscore{}}}\AgdaSpace{}%
\AgdaSymbol{:}\AgdaSpace{}%
\AgdaDatatype{Shape}\AgdaSpace{}%
\AgdaSymbol{→}\AgdaSpace{}%
\AgdaDatatype{ℕ}\<%
\\
%
\>[2]\AgdaOperator{\AgdaFunction{\#\AgdaUnderscore{}}}\AgdaSpace{}%
\AgdaSymbol{=}\AgdaSpace{}%
\AgdaFunction{length}\<%
\\
%
\\[\AgdaEmptyExtraSkip]%
%
\>[2]\AgdaKeyword{infix}\AgdaSpace{}%
\AgdaNumber{11}\AgdaSpace{}%
\AgdaOperator{\AgdaFunction{\AgdaUnderscore{}ᵗ}}\<%
\\
%
\>[2]\AgdaOperator{\AgdaFunction{\AgdaUnderscore{}ᵗ}}\AgdaSpace{}%
\AgdaSymbol{:}\AgdaSpace{}%
\AgdaDatatype{Shape}\AgdaSpace{}%
\AgdaSymbol{→}\AgdaSpace{}%
\AgdaDatatype{Shape}\<%
\\
%
\>[2]\AgdaOperator{\AgdaFunction{\AgdaUnderscore{}ᵗ}}\AgdaSpace{}%
\AgdaSymbol{=}\AgdaSpace{}%
\AgdaFunction{recursive-transpose}\<%
\\
%
\\[\AgdaEmptyExtraSkip]%
%
\>[2]\AgdaFunction{nz-\#}\AgdaSpace{}%
\AgdaSymbol{:}\AgdaSpace{}%
\AgdaDatatype{NonZeroₛ}\AgdaSpace{}%
\AgdaGeneralizable{s}\AgdaSpace{}%
\AgdaSymbol{→}\AgdaSpace{}%
\AgdaRecord{NonZero}\AgdaSpace{}%
\AgdaSymbol{(}\AgdaFunction{length}\AgdaSpace{}%
\AgdaGeneralizable{s}\AgdaSymbol{)}\<%
\\
%
\>[2]\AgdaFunction{nz-\#}\AgdaSpace{}%
\AgdaSymbol{=}\AgdaSpace{}%
\AgdaFunction{nonZeroₛ-s⇒nonZero-s}\<%
\\
%
\\[\AgdaEmptyExtraSkip]%
%
\>[2]\AgdaFunction{nz-ι\#}\AgdaSpace{}%
\AgdaSymbol{:}\AgdaSpace{}%
\AgdaDatatype{NonZeroₛ}\AgdaSpace{}%
\AgdaGeneralizable{s}\AgdaSpace{}%
\AgdaSymbol{→}\AgdaSpace{}%
\AgdaDatatype{NonZeroₛ}\AgdaSpace{}%
\AgdaSymbol{(}\AgdaInductiveConstructor{ι}\AgdaSpace{}%
\AgdaSymbol{(}\AgdaFunction{length}\AgdaSpace{}%
\AgdaGeneralizable{s}\AgdaSymbol{))}\<%
\\
%
\>[2]\AgdaFunction{nz-ι\#}\AgdaSpace{}%
\AgdaSymbol{(}\AgdaInductiveConstructor{ι}\AgdaSpace{}%
\AgdaBound{x}\AgdaSymbol{)}\AgdaSpace{}%
\AgdaSymbol{=}\AgdaSpace{}%
\AgdaInductiveConstructor{ι}\AgdaSpace{}%
\AgdaBound{x}\<%
\\
%
\>[2]\AgdaFunction{nz-ι\#}\AgdaSpace{}%
\AgdaSymbol{\{}\AgdaBound{s}\AgdaSpace{}%
\AgdaOperator{\AgdaInductiveConstructor{⊗}}\AgdaSpace{}%
\AgdaBound{p}\AgdaSymbol{\}}\AgdaSpace{}%
\AgdaSymbol{(}\AgdaBound{nz-s}\AgdaSpace{}%
\AgdaOperator{\AgdaInductiveConstructor{⊗}}\AgdaSpace{}%
\AgdaBound{nz-p}\AgdaSymbol{)}\AgdaSpace{}%
\AgdaSymbol{=}\AgdaSpace{}%
\AgdaInductiveConstructor{ι}\AgdaSpace{}%
\AgdaSymbol{(}\AgdaFunction{m*n≢0}\AgdaSpace{}%
\AgdaSymbol{(}\AgdaOperator{\AgdaFunction{\#}}\AgdaSpace{}%
\AgdaBound{s}\AgdaSymbol{)}\AgdaSpace{}%
\AgdaSymbol{(}\AgdaOperator{\AgdaFunction{\#}}\AgdaSpace{}%
\AgdaBound{p}\AgdaSymbol{)}\AgdaSpace{}%
\AgdaSymbol{⦃}\AgdaSpace{}%
\AgdaFunction{nz-\#}\AgdaSpace{}%
\AgdaBound{nz-s}\AgdaSpace{}%
\AgdaSymbol{⦄}\AgdaSpace{}%
\AgdaSymbol{⦃}\AgdaSpace{}%
\AgdaFunction{nz-\#}\AgdaSpace{}%
\AgdaBound{nz-p}\AgdaSpace{}%
\AgdaSymbol{⦄}\AgdaSpace{}%
\AgdaSymbol{)}\<%
\\
%
\\[\AgdaEmptyExtraSkip]%
%
\>[2]\AgdaFunction{nzᵗ}\AgdaSpace{}%
\AgdaSymbol{:}\AgdaSpace{}%
\AgdaDatatype{NonZeroₛ}\AgdaSpace{}%
\AgdaGeneralizable{s}\AgdaSpace{}%
\AgdaSymbol{→}\AgdaSpace{}%
\AgdaDatatype{NonZeroₛ}\AgdaSpace{}%
\AgdaSymbol{(}\AgdaGeneralizable{s}\AgdaSpace{}%
\AgdaOperator{\AgdaFunction{ᵗ}}\AgdaSymbol{)}\<%
\\
%
\>[2]\AgdaFunction{nzᵗ}\AgdaSpace{}%
\AgdaSymbol{=}\AgdaSpace{}%
\AgdaFunction{nonZeroₛ-s⇒nonZeroₛ-sᵗ}\<%
\\
%
\\[\AgdaEmptyExtraSkip]%
%
\>[2]\AgdaFunction{rev-eq-applied}\AgdaSpace{}%
\AgdaSymbol{:}\AgdaSpace{}%
\AgdaSymbol{(}\AgdaBound{rshp}\AgdaSpace{}%
\AgdaSymbol{:}\AgdaSpace{}%
\AgdaDatatype{Reshape}\AgdaSpace{}%
\AgdaGeneralizable{r₂}\AgdaSpace{}%
\AgdaGeneralizable{r₁}\AgdaSymbol{)}\AgdaSpace{}%
\AgdaSymbol{(}\AgdaBound{arr}\AgdaSpace{}%
\AgdaSymbol{:}\AgdaSpace{}%
\AgdaFunction{Ar}\AgdaSpace{}%
\AgdaGeneralizable{r₁}\AgdaSpace{}%
\AgdaField{ℂ}\AgdaSymbol{)}\AgdaSpace{}%
\AgdaSymbol{→}\AgdaSpace{}%
\AgdaFunction{reshape}\AgdaSpace{}%
\AgdaSymbol{(}\AgdaBound{rshp}\AgdaSpace{}%
\AgdaOperator{\AgdaInductiveConstructor{∙}}\AgdaSpace{}%
\AgdaFunction{rev}\AgdaSpace{}%
\AgdaBound{rshp}\AgdaSymbol{)}\AgdaSpace{}%
\AgdaBound{arr}\AgdaSpace{}%
\AgdaOperator{\AgdaFunction{≅}}\AgdaSpace{}%
\AgdaBound{arr}\<%
\\
%
\>[2]\AgdaFunction{rev-eq-applied}\AgdaSpace{}%
\AgdaBound{rshp}\AgdaSpace{}%
\AgdaBound{arr}\AgdaSpace{}%
\AgdaBound{i}\AgdaSpace{}%
\AgdaSymbol{=}\AgdaSpace{}%
\AgdaFunction{cong}\AgdaSpace{}%
\AgdaBound{arr}\AgdaSpace{}%
\AgdaSymbol{(}\AgdaFunction{rev-eq}\AgdaSpace{}%
\AgdaBound{rshp}\AgdaSpace{}%
\AgdaBound{i}\AgdaSymbol{)}\<%
\\
%
\\[\AgdaEmptyExtraSkip]%
%
\>[2]\AgdaComment{-------------------------------------------}\<%
\\
%
\>[2]\AgdaComment{---\ 4\ way\ associativity\ helper\ function\ ---}\<%
\\
%
\>[2]\AgdaComment{-------------------------------------------}\<%
\\
%
\\[\AgdaEmptyExtraSkip]%
%
\>[2]\AgdaFunction{assoc₄}\AgdaSpace{}%
\AgdaSymbol{:}\AgdaSpace{}%
\AgdaSymbol{(}\AgdaBound{a}\AgdaSpace{}%
\AgdaBound{b}\AgdaSpace{}%
\AgdaBound{c}\AgdaSpace{}%
\AgdaBound{d}\AgdaSpace{}%
\AgdaSymbol{:}\AgdaSpace{}%
\AgdaField{ℂ}\AgdaSymbol{)}\AgdaSpace{}%
\AgdaSymbol{→}\AgdaSpace{}%
\AgdaBound{a}\AgdaSpace{}%
\AgdaOperator{\AgdaField{*}}\AgdaSpace{}%
\AgdaBound{b}\AgdaSpace{}%
\AgdaOperator{\AgdaField{*}}\AgdaSpace{}%
\AgdaBound{c}\AgdaSpace{}%
\AgdaOperator{\AgdaField{*}}\AgdaSpace{}%
\AgdaBound{d}\AgdaSpace{}%
\AgdaOperator{\AgdaDatatype{≡}}\AgdaSpace{}%
\AgdaBound{a}\AgdaSpace{}%
\AgdaOperator{\AgdaField{*}}\AgdaSpace{}%
\AgdaSymbol{(}\AgdaBound{b}\AgdaSpace{}%
\AgdaOperator{\AgdaField{*}}\AgdaSpace{}%
\AgdaBound{c}\AgdaSpace{}%
\AgdaOperator{\AgdaField{*}}\AgdaSpace{}%
\AgdaBound{d}\AgdaSymbol{)}\<%
\\
%
\>[2]\AgdaFunction{assoc₄}\AgdaSpace{}%
\AgdaBound{a}\AgdaSpace{}%
\AgdaBound{b}\AgdaSpace{}%
\AgdaBound{c}\AgdaSpace{}%
\AgdaBound{d}\AgdaSpace{}%
\AgdaKeyword{rewrite}\<%
\\
\>[2][@{}l@{\AgdaIndent{0}}]%
\>[6]\AgdaFunction{*-assoc}\AgdaSpace{}%
\AgdaBound{a}\AgdaSpace{}%
\AgdaBound{b}\AgdaSpace{}%
\AgdaBound{c}\<%
\\
\>[2][@{}l@{\AgdaIndent{0}}]%
\>[4]\AgdaSymbol{|}\AgdaSpace{}%
\AgdaFunction{*-assoc}\AgdaSpace{}%
\AgdaBound{a}\AgdaSpace{}%
\AgdaSymbol{(}\AgdaBound{b}\AgdaSpace{}%
\AgdaOperator{\AgdaField{*}}\AgdaSpace{}%
\AgdaBound{c}\AgdaSymbol{)}\AgdaSpace{}%
\AgdaBound{d}\<%
\\
%
\>[4]\AgdaSymbol{=}\AgdaSpace{}%
\AgdaInductiveConstructor{refl}\<%
\\
%
\\[\AgdaEmptyExtraSkip]%
%
\>[2]\AgdaComment{--------------------------}\<%
\\
%
\>[2]\AgdaComment{---\ Properties\ of\ iota\ ---}\<%
\\
%
\>[2]\AgdaComment{--------------------------}\<%
\\
%
\\[\AgdaEmptyExtraSkip]%
%
\>[2]\AgdaFunction{iota-reindex}\AgdaSpace{}%
\AgdaSymbol{:}\AgdaSpace{}%
\AgdaSymbol{∀}\AgdaSpace{}%
\AgdaSymbol{\{}\AgdaBound{i}\AgdaSpace{}%
\AgdaSymbol{:}\AgdaSpace{}%
\AgdaDatatype{Position}\AgdaSpace{}%
\AgdaSymbol{(}\AgdaInductiveConstructor{ι}\AgdaSpace{}%
\AgdaGeneralizable{N}\AgdaSymbol{)\}}\AgdaSpace{}%
\AgdaSymbol{→}\AgdaSpace{}%
\AgdaSymbol{(}\AgdaBound{prf}\AgdaSpace{}%
\AgdaSymbol{:}\AgdaSpace{}%
\AgdaGeneralizable{M}\AgdaSpace{}%
\AgdaOperator{\AgdaDatatype{≡}}\AgdaSpace{}%
\AgdaGeneralizable{N}\AgdaSymbol{)}\AgdaSpace{}%
\AgdaSymbol{→}\AgdaSpace{}%
\AgdaFunction{iota}\AgdaSpace{}%
\AgdaSymbol{(}\AgdaBound{i}\AgdaSpace{}%
\AgdaOperator{\AgdaFunction{⟨}}\AgdaSpace{}%
\AgdaFunction{reindex}\AgdaSpace{}%
\AgdaBound{prf}\AgdaSpace{}%
\AgdaOperator{\AgdaFunction{⟩}}\AgdaSymbol{)}\AgdaSpace{}%
\AgdaOperator{\AgdaDatatype{≡}}\AgdaSpace{}%
\AgdaFunction{iota}\AgdaSpace{}%
\AgdaBound{i}\<%
\\
%
\>[2]\AgdaFunction{iota-reindex}\AgdaSpace{}%
\AgdaInductiveConstructor{refl}\AgdaSpace{}%
\AgdaSymbol{=}\AgdaSpace{}%
\AgdaInductiveConstructor{refl}\<%
\\
%
\\[\AgdaEmptyExtraSkip]%
%
\>[2]\AgdaFunction{iota-split}\AgdaSpace{}%
\AgdaSymbol{:}\AgdaSpace{}%
\AgdaSymbol{∀}\<%
\\
\>[2][@{}l@{\AgdaIndent{0}}]%
\>[5]\AgdaSymbol{(}\AgdaBound{k₀}%
\>[11]\AgdaSymbol{:}\AgdaSpace{}%
\AgdaDatatype{Position}\AgdaSpace{}%
\AgdaSymbol{(}\AgdaInductiveConstructor{ι}\AgdaSpace{}%
\AgdaSymbol{(}\AgdaOperator{\AgdaFunction{\#}}\AgdaSpace{}%
\AgdaGeneralizable{r₂}\AgdaSymbol{)))}\<%
\\
%
\>[5]\AgdaSymbol{(}\AgdaBound{k₁}%
\>[11]\AgdaSymbol{:}\AgdaSpace{}%
\AgdaDatatype{Position}\AgdaSpace{}%
\AgdaSymbol{(}\AgdaInductiveConstructor{ι}\AgdaSpace{}%
\AgdaSymbol{(}\AgdaOperator{\AgdaFunction{\#}}\AgdaSpace{}%
\AgdaGeneralizable{r₁}\AgdaSymbol{)))}\<%
\\
%
\>[5]\AgdaSymbol{→}\AgdaSpace{}%
\AgdaFunction{iota}\AgdaSpace{}%
\AgdaSymbol{((}\AgdaBound{k₁}\AgdaSpace{}%
\AgdaOperator{\AgdaInductiveConstructor{⊗}}\AgdaSpace{}%
\AgdaBound{k₀}\AgdaSymbol{)}\AgdaSpace{}%
\AgdaOperator{\AgdaFunction{⟨}}\AgdaSpace{}%
\AgdaInductiveConstructor{split}\AgdaSpace{}%
\AgdaOperator{\AgdaFunction{⟩}}\AgdaSymbol{)}\AgdaSpace{}%
\AgdaOperator{\AgdaDatatype{≡}}\AgdaSpace{}%
\AgdaSymbol{(}\AgdaOperator{\AgdaFunction{\#}}\AgdaSpace{}%
\AgdaGeneralizable{r₂}\AgdaSpace{}%
\AgdaOperator{\AgdaPrimitive{*ₙ}}\AgdaSpace{}%
\AgdaFunction{iota}\AgdaSpace{}%
\AgdaBound{k₁}\AgdaSymbol{)}\AgdaSpace{}%
\AgdaOperator{\AgdaPrimitive{+ₙ}}\AgdaSpace{}%
\AgdaFunction{iota}\AgdaSpace{}%
\AgdaBound{k₀}\<%
\\
%
\>[2]\AgdaFunction{iota-split}\AgdaSpace{}%
\AgdaSymbol{(}\AgdaInductiveConstructor{ι}\AgdaSpace{}%
\AgdaBound{k₀}\AgdaSymbol{)}\AgdaSpace{}%
\AgdaSymbol{(}\AgdaInductiveConstructor{ι}\AgdaSpace{}%
\AgdaBound{k₁}\AgdaSymbol{)}\AgdaSpace{}%
\AgdaKeyword{rewrite}\AgdaSpace{}%
\AgdaFunction{toℕ-combine}\AgdaSpace{}%
\AgdaBound{k₁}\AgdaSpace{}%
\AgdaBound{k₀}\AgdaSpace{}%
\AgdaSymbol{=}\AgdaSpace{}%
\AgdaInductiveConstructor{refl}\<%
\\
%
\\[\AgdaEmptyExtraSkip]%
%
\>[2]\AgdaComment{-----------------------------}\<%
\\
%
\>[2]\AgdaComment{---\ Congurance\ properties\ ---}\<%
\\
%
\>[2]\AgdaComment{-----------------------------}\<%
\\
%
\\[\AgdaEmptyExtraSkip]%
%
\>[2]\AgdaFunction{-ω-cong₂}\AgdaSpace{}%
\AgdaSymbol{:}\<%
\\
\>[2][@{}l@{\AgdaIndent{0}}]%
\>[4]\AgdaSymbol{∀}\AgdaSpace{}%
\AgdaSymbol{\{}\AgdaBound{n}\AgdaSpace{}%
\AgdaBound{m}\AgdaSpace{}%
\AgdaSymbol{:}\AgdaSpace{}%
\AgdaDatatype{ℕ}\AgdaSymbol{\}}\<%
\\
%
\>[4]\AgdaSymbol{→}\AgdaSpace{}%
\AgdaSymbol{⦃}\AgdaSpace{}%
\AgdaBound{nonZero-n}\AgdaSpace{}%
\AgdaSymbol{:}\AgdaSpace{}%
\AgdaRecord{NonZero}\AgdaSpace{}%
\AgdaBound{n}\AgdaSpace{}%
\AgdaSymbol{⦄}\<%
\\
%
\>[4]\AgdaSymbol{→}\AgdaSpace{}%
\AgdaSymbol{⦃}\AgdaSpace{}%
\AgdaBound{nonZero-m}\AgdaSpace{}%
\AgdaSymbol{:}\AgdaSpace{}%
\AgdaRecord{NonZero}\AgdaSpace{}%
\AgdaBound{m}\AgdaSpace{}%
\AgdaSymbol{⦄}\<%
\\
%
\>[4]\AgdaSymbol{→}\AgdaSpace{}%
\AgdaSymbol{∀}\AgdaSpace{}%
\AgdaSymbol{\{}\AgdaBound{k}\AgdaSpace{}%
\AgdaBound{j}\AgdaSpace{}%
\AgdaSymbol{:}\AgdaSpace{}%
\AgdaDatatype{ℕ}\AgdaSymbol{\}}\<%
\\
%
\>[4]\AgdaSymbol{→}\AgdaSpace{}%
\AgdaSymbol{(}\AgdaBound{prfₗ}\AgdaSpace{}%
\AgdaSymbol{:}\AgdaSpace{}%
\AgdaBound{n}\AgdaSpace{}%
\AgdaOperator{\AgdaDatatype{≡}}\AgdaSpace{}%
\AgdaBound{m}\AgdaSymbol{)}\<%
\\
%
\>[4]\AgdaSymbol{→}\AgdaSpace{}%
\AgdaBound{k}\AgdaSpace{}%
\AgdaOperator{\AgdaDatatype{≡}}\AgdaSpace{}%
\AgdaBound{j}\<%
\\
%
\>[4]\AgdaSymbol{→}\AgdaSpace{}%
\AgdaField{-ω}\AgdaSpace{}%
\AgdaBound{n}\AgdaSpace{}%
\AgdaSymbol{⦃}\AgdaSpace{}%
\AgdaBound{nonZero-n}\AgdaSpace{}%
\AgdaSymbol{⦄}\AgdaSpace{}%
\AgdaBound{k}\AgdaSpace{}%
\AgdaOperator{\AgdaDatatype{≡}}\AgdaSpace{}%
\AgdaField{-ω}\AgdaSpace{}%
\AgdaBound{m}\AgdaSpace{}%
\AgdaSymbol{⦃}\AgdaSpace{}%
\AgdaBound{nonZero-m}\AgdaSpace{}%
\AgdaSymbol{⦄}\AgdaSpace{}%
\AgdaBound{j}\<%
\\
%
\>[2]\AgdaFunction{-ω-cong₂}\AgdaSpace{}%
\AgdaInductiveConstructor{refl}\AgdaSpace{}%
\AgdaInductiveConstructor{refl}\AgdaSpace{}%
\AgdaSymbol{=}\AgdaSpace{}%
\AgdaInductiveConstructor{refl}\<%
\\
%
\\[\AgdaEmptyExtraSkip]%
%
\>[2]\AgdaFunction{DFT′-cong}\AgdaSpace{}%
\AgdaSymbol{:}\AgdaSpace{}%
\AgdaSymbol{∀}\AgdaSpace{}%
\AgdaSymbol{\{}\AgdaBound{xs}\AgdaSpace{}%
\AgdaBound{ys}\AgdaSpace{}%
\AgdaSymbol{:}\AgdaSpace{}%
\AgdaFunction{Ar}\AgdaSpace{}%
\AgdaSymbol{(}\AgdaInductiveConstructor{ι}\AgdaSpace{}%
\AgdaGeneralizable{N}\AgdaSymbol{)}\AgdaSpace{}%
\AgdaField{ℂ}\AgdaSymbol{\}}\AgdaSpace{}%
\AgdaSymbol{→}\AgdaSpace{}%
\AgdaSymbol{⦃}\AgdaSpace{}%
\AgdaBound{nonZero-N}\AgdaSpace{}%
\AgdaSymbol{:}\AgdaSpace{}%
\AgdaRecord{NonZero}\AgdaSpace{}%
\AgdaGeneralizable{N}\AgdaSpace{}%
\AgdaSymbol{⦄}\AgdaSpace{}%
\AgdaSymbol{→}\AgdaSpace{}%
\AgdaBound{xs}\AgdaSpace{}%
\AgdaOperator{\AgdaFunction{≅}}\AgdaSpace{}%
\AgdaBound{ys}\AgdaSpace{}%
\AgdaSymbol{→}\AgdaSpace{}%
\AgdaFunction{DFT′}\AgdaSpace{}%
\AgdaBound{xs}\AgdaSpace{}%
\AgdaOperator{\AgdaFunction{≅}}\AgdaSpace{}%
\AgdaFunction{DFT′}\AgdaSpace{}%
\AgdaBound{ys}\<%
\\
%
\>[2]\AgdaFunction{DFT′-cong}\AgdaSpace{}%
\AgdaSymbol{\{}\AgdaInductiveConstructor{suc}\AgdaSpace{}%
\AgdaBound{N}\AgdaSymbol{\}}\AgdaSpace{}%
\AgdaSymbol{⦃}\AgdaSpace{}%
\AgdaBound{nonZero-N}\AgdaSpace{}%
\AgdaSymbol{⦄}\AgdaSpace{}%
\AgdaBound{prf}\AgdaSpace{}%
\AgdaSymbol{(}\AgdaInductiveConstructor{ι}\AgdaSpace{}%
\AgdaBound{j}\AgdaSymbol{)}\AgdaSpace{}%
\AgdaSymbol{=}\AgdaSpace{}%
\AgdaFunction{sum-cong}\AgdaSpace{}%
\AgdaSymbol{\{}\AgdaInductiveConstructor{suc}\AgdaSpace{}%
\AgdaBound{N}\AgdaSymbol{\}}\AgdaSpace{}%
\AgdaSymbol{(λ}\AgdaSpace{}%
\AgdaBound{i}\AgdaSpace{}%
\AgdaSymbol{→}\AgdaSpace{}%
\AgdaFunction{cong₂}\AgdaSpace{}%
\AgdaOperator{\AgdaField{\AgdaUnderscore{}*\AgdaUnderscore{}}}\AgdaSpace{}%
\AgdaSymbol{(}\AgdaBound{prf}\AgdaSpace{}%
\AgdaBound{i}\AgdaSymbol{)}\AgdaSpace{}%
\AgdaInductiveConstructor{refl}\AgdaSymbol{)}\<%
\\
%
\\[\AgdaEmptyExtraSkip]%
%
\>[2]\AgdaFunction{FFT′-cong}\AgdaSpace{}%
\AgdaSymbol{:}\AgdaSpace{}%
\AgdaSymbol{∀}\AgdaSpace{}%
\AgdaSymbol{\{}\AgdaBound{s}\AgdaSpace{}%
\AgdaSymbol{:}\AgdaSpace{}%
\AgdaDatatype{Shape}\AgdaSymbol{\}}\AgdaSpace{}%
\AgdaSymbol{\{}\AgdaBound{xs}\AgdaSpace{}%
\AgdaBound{ys}\AgdaSpace{}%
\AgdaSymbol{:}\AgdaSpace{}%
\AgdaFunction{Ar}\AgdaSpace{}%
\AgdaBound{s}\AgdaSpace{}%
\AgdaField{ℂ}\AgdaSymbol{\}}\AgdaSpace{}%
\AgdaSymbol{→}\AgdaSpace{}%
\AgdaSymbol{⦃}\AgdaSpace{}%
\AgdaBound{nonZeroₛ-s}\AgdaSpace{}%
\AgdaSymbol{:}\AgdaSpace{}%
\AgdaDatatype{NonZeroₛ}\AgdaSpace{}%
\AgdaBound{s}\AgdaSpace{}%
\AgdaSymbol{⦄}\AgdaSpace{}%
\AgdaSymbol{→}\AgdaSpace{}%
\AgdaBound{xs}\AgdaSpace{}%
\AgdaOperator{\AgdaFunction{≅}}\AgdaSpace{}%
\AgdaBound{ys}\AgdaSpace{}%
\AgdaSymbol{→}\AgdaSpace{}%
\AgdaFunction{FFT′}\AgdaSpace{}%
\AgdaBound{xs}\AgdaSpace{}%
\AgdaOperator{\AgdaFunction{≅}}\AgdaSpace{}%
\AgdaFunction{FFT′}\AgdaSpace{}%
\AgdaBound{ys}\<%
\\
%
\>[2]\AgdaFunction{FFT′-cong}\AgdaSpace{}%
\AgdaSymbol{\{}\AgdaInductiveConstructor{ι}\AgdaSpace{}%
\AgdaBound{N}\AgdaSymbol{\}}\AgdaSpace{}%
\AgdaSymbol{⦃}\AgdaSpace{}%
\AgdaInductiveConstructor{ι}\AgdaSpace{}%
\AgdaBound{nonZero-N}\AgdaSpace{}%
\AgdaSymbol{⦄}\AgdaSpace{}%
\AgdaSymbol{=}\AgdaSpace{}%
\AgdaFunction{DFT′-cong}\AgdaSpace{}%
\AgdaSymbol{⦃}\AgdaSpace{}%
\AgdaBound{nonZero-N}\AgdaSpace{}%
\AgdaSymbol{⦄}\<%
\\
%
\>[2]\AgdaFunction{FFT′-cong}\AgdaSpace{}%
\AgdaSymbol{\{}\AgdaBound{r₁}\AgdaSpace{}%
\AgdaOperator{\AgdaInductiveConstructor{⊗}}\AgdaSpace{}%
\AgdaBound{r₂}\AgdaSymbol{\}}\AgdaSpace{}%
\AgdaSymbol{\{}\AgdaBound{xs}\AgdaSymbol{\}}\AgdaSpace{}%
\AgdaSymbol{\{}\AgdaBound{ys}\AgdaSymbol{\}}\AgdaSpace{}%
\AgdaSymbol{⦃}\AgdaSpace{}%
\AgdaBound{nonZero-r₁}\AgdaSpace{}%
\AgdaOperator{\AgdaInductiveConstructor{⊗}}\AgdaSpace{}%
\AgdaBound{nonZero-r₂}\AgdaSpace{}%
\AgdaSymbol{⦄}\AgdaSpace{}%
\AgdaBound{prf}\AgdaSpace{}%
\AgdaSymbol{(}\AgdaBound{j₁}\AgdaSpace{}%
\AgdaOperator{\AgdaInductiveConstructor{⊗}}\AgdaSpace{}%
\AgdaBound{j₀}\AgdaSymbol{)}\AgdaSpace{}%
\AgdaSymbol{=}\<%
\\
\>[2][@{}l@{\AgdaIndent{0}}]%
\>[4]\AgdaKeyword{let}\AgdaSpace{}%
\AgdaKeyword{instance}\<%
\\
\>[4][@{}l@{\AgdaIndent{0}}]%
\>[6]\AgdaBound{\AgdaUnderscore{}}\AgdaSpace{}%
\AgdaSymbol{:}\AgdaSpace{}%
\AgdaDatatype{NonZeroₛ}\AgdaSpace{}%
\AgdaBound{r₁}\<%
\\
%
\>[6]\AgdaSymbol{\AgdaUnderscore{}}\AgdaSpace{}%
\AgdaSymbol{=}\AgdaSpace{}%
\AgdaBound{nonZero-r₁}\<%
\\
%
\>[6]\AgdaBound{\AgdaUnderscore{}}\AgdaSpace{}%
\AgdaSymbol{:}\AgdaSpace{}%
\AgdaDatatype{NonZeroₛ}\AgdaSpace{}%
\AgdaBound{r₂}\<%
\\
%
\>[6]\AgdaSymbol{\AgdaUnderscore{}}\AgdaSpace{}%
\AgdaSymbol{=}\AgdaSpace{}%
\AgdaBound{nonZero-r₂}\<%
\\
%
\>[4]\AgdaKeyword{in}%
\>[8]\AgdaSymbol{(}\AgdaFunction{FFT′-cong}\AgdaSpace{}%
\AgdaSymbol{\{}\AgdaBound{r₂}\AgdaSymbol{\}}\AgdaSpace{}%
\AgdaSymbol{λ\{}\AgdaSpace{}%
\AgdaBound{k₀}\AgdaSpace{}%
\AgdaSymbol{→}\<%
\\
\>[8][@{}l@{\AgdaIndent{0}}]%
\>[10]\AgdaSymbol{(}\AgdaFunction{cong₂}\AgdaSpace{}%
\AgdaOperator{\AgdaField{\AgdaUnderscore{}*\AgdaUnderscore{}}}\<%
\\
\>[10][@{}l@{\AgdaIndent{0}}]%
\>[12]\AgdaSymbol{((}\AgdaFunction{FFT′-cong}\AgdaSpace{}%
\AgdaSymbol{\{}\AgdaBound{r₁}\AgdaSymbol{\}}\AgdaSpace{}%
\AgdaSymbol{λ\{}\AgdaBound{k₁}\AgdaSpace{}%
\AgdaSymbol{→}\<%
\\
\>[12][@{}l@{\AgdaIndent{0}}]%
\>[14]\AgdaBound{prf}\AgdaSpace{}%
\AgdaSymbol{(}\AgdaBound{k₁}\AgdaSpace{}%
\AgdaOperator{\AgdaInductiveConstructor{⊗}}\AgdaSpace{}%
\AgdaBound{k₀}\AgdaSymbol{)}\<%
\\
%
\>[12]\AgdaSymbol{\})}\AgdaSpace{}%
\AgdaBound{j₀}\AgdaSpace{}%
\AgdaSymbol{)}\<%
\\
%
\>[12]\AgdaInductiveConstructor{refl}\<%
\\
%
\>[10]\AgdaSymbol{)}\<%
\\
%
\>[8]\AgdaSymbol{\})}\AgdaSpace{}%
\AgdaBound{j₁}\<%
\\
%
\\[\AgdaEmptyExtraSkip]%
%
\>[2]\AgdaComment{-------------------------}\<%
\\
%
\>[2]\AgdaComment{---\ Properties\ of\ Sum\ ---}\<%
\\
%
\>[2]\AgdaComment{-------------------------}\<%
\\
%
\\[\AgdaEmptyExtraSkip]%
%
\>[2]\AgdaComment{--\ Of\ note\ here\ is\ that\ I\ cannot\ put\ this\ in\ the\ sum\ module\ as\ it\ requires\ knowledge\ of\ \AgdaUnderscore{}+\AgdaUnderscore{}\ as\ well\ as\ its\ relation\ with\ \AgdaUnderscore{}*\AgdaUnderscore{}}\<%
\\
%
\\[\AgdaEmptyExtraSkip]%
%
\>[2]\AgdaFunction{*-distribˡ-sum}\AgdaSpace{}%
\AgdaSymbol{:}\AgdaSpace{}%
\AgdaSymbol{\{}\AgdaBound{xs}\AgdaSpace{}%
\AgdaSymbol{:}\AgdaSpace{}%
\AgdaFunction{Ar}\AgdaSpace{}%
\AgdaSymbol{(}\AgdaInductiveConstructor{ι}\AgdaSpace{}%
\AgdaGeneralizable{N}\AgdaSymbol{)}\AgdaSpace{}%
\AgdaField{ℂ}\AgdaSymbol{\}}\AgdaSpace{}%
\AgdaSymbol{(}\AgdaBound{x}\AgdaSpace{}%
\AgdaSymbol{:}\AgdaSpace{}%
\AgdaField{ℂ}\AgdaSymbol{)}\AgdaSpace{}%
\AgdaSymbol{→}\AgdaSpace{}%
\AgdaBound{x}\AgdaSpace{}%
\AgdaOperator{\AgdaField{*}}\AgdaSpace{}%
\AgdaSymbol{(}\AgdaFunction{sum}\AgdaSpace{}%
\AgdaBound{xs}\AgdaSymbol{)}\AgdaSpace{}%
\AgdaOperator{\AgdaDatatype{≡}}\AgdaSpace{}%
\AgdaFunction{sum}\AgdaSpace{}%
\AgdaSymbol{(}\AgdaFunction{map}\AgdaSpace{}%
\AgdaSymbol{(}\AgdaBound{x}\AgdaSpace{}%
\AgdaOperator{\AgdaField{*\AgdaUnderscore{}}}\AgdaSymbol{)}\AgdaSpace{}%
\AgdaBound{xs}\AgdaSymbol{)}\<%
\\
%
\>[2]\AgdaFunction{*-distribˡ-sum}\AgdaSpace{}%
\AgdaSymbol{\{}\AgdaInductiveConstructor{zero}\AgdaSymbol{\}}\AgdaSpace{}%
\AgdaSymbol{\{}\AgdaBound{xs}\AgdaSymbol{\}}\AgdaSpace{}%
\AgdaBound{x}\AgdaSpace{}%
\AgdaSymbol{=}\AgdaSpace{}%
\AgdaFunction{zeroʳ}\AgdaSpace{}%
\AgdaBound{x}\<%
\\
%
\>[2]\AgdaFunction{*-distribˡ-sum}\AgdaSpace{}%
\AgdaSymbol{\{}\AgdaInductiveConstructor{suc}\AgdaSpace{}%
\AgdaBound{N}\AgdaSymbol{\}}\AgdaSpace{}%
\AgdaSymbol{\{}\AgdaBound{xs}\AgdaSymbol{\}}\AgdaSpace{}%
\AgdaBound{x}\AgdaSpace{}%
\AgdaKeyword{rewrite}\<%
\\
\>[2][@{}l@{\AgdaIndent{0}}]%
\>[6]\AgdaFunction{distribˡ}\AgdaSpace{}%
\AgdaBound{x}\AgdaSpace{}%
\AgdaSymbol{(}\AgdaBound{xs}\AgdaSpace{}%
\AgdaSymbol{(}\AgdaInductiveConstructor{ι}\AgdaSpace{}%
\AgdaInductiveConstructor{fzero}\AgdaSymbol{))}\AgdaSpace{}%
\AgdaSymbol{(}\AgdaFunction{sum}\AgdaSpace{}%
\AgdaSymbol{(}\AgdaFunction{tail₁}\AgdaSpace{}%
\AgdaBound{xs}\AgdaSymbol{))}\<%
\\
\>[2][@{}l@{\AgdaIndent{0}}]%
\>[4]\AgdaSymbol{|}\AgdaSpace{}%
\AgdaFunction{*-distribˡ-sum}\AgdaSpace{}%
\AgdaSymbol{\{}\AgdaBound{N}\AgdaSymbol{\}}\AgdaSpace{}%
\AgdaSymbol{\{}\AgdaFunction{tail₁}\AgdaSpace{}%
\AgdaBound{xs}\AgdaSymbol{\}}\AgdaSpace{}%
\AgdaBound{x}\<%
\\
%
\>[4]\AgdaSymbol{=}\AgdaSpace{}%
\AgdaFunction{cong₂}\AgdaSpace{}%
\AgdaOperator{\AgdaField{\AgdaUnderscore{}+\AgdaUnderscore{}}}\AgdaSpace{}%
\AgdaInductiveConstructor{refl}\AgdaSpace{}%
\AgdaSymbol{(}\AgdaFunction{sum-cong}\AgdaSpace{}%
\AgdaSymbol{\{}\AgdaBound{N}\AgdaSymbol{\}}\AgdaSpace{}%
\AgdaSymbol{(λ\{(}\AgdaInductiveConstructor{ι}\AgdaSpace{}%
\AgdaBound{i}\AgdaSymbol{)}\AgdaSpace{}%
\AgdaSymbol{→}\AgdaSpace{}%
\AgdaInductiveConstructor{refl}\AgdaSpace{}%
\AgdaSymbol{\}))}\<%
\\
%
\\[\AgdaEmptyExtraSkip]%
%
\>[2]\AgdaFunction{*-distribʳ-sum}\AgdaSpace{}%
\AgdaSymbol{:}\AgdaSpace{}%
\AgdaSymbol{\{}\AgdaBound{xs}\AgdaSpace{}%
\AgdaSymbol{:}\AgdaSpace{}%
\AgdaFunction{Ar}\AgdaSpace{}%
\AgdaSymbol{(}\AgdaInductiveConstructor{ι}\AgdaSpace{}%
\AgdaGeneralizable{N}\AgdaSymbol{)}\AgdaSpace{}%
\AgdaField{ℂ}\AgdaSymbol{\}}\AgdaSpace{}%
\AgdaSymbol{(}\AgdaBound{x}\AgdaSpace{}%
\AgdaSymbol{:}\AgdaSpace{}%
\AgdaField{ℂ}\AgdaSymbol{)}\AgdaSpace{}%
\AgdaSymbol{→}\AgdaSpace{}%
\AgdaSymbol{(}\AgdaFunction{sum}\AgdaSpace{}%
\AgdaBound{xs}\AgdaSymbol{)}\AgdaSpace{}%
\AgdaOperator{\AgdaField{*}}\AgdaSpace{}%
\AgdaBound{x}\AgdaSpace{}%
\AgdaOperator{\AgdaDatatype{≡}}\AgdaSpace{}%
\AgdaFunction{sum}\AgdaSpace{}%
\AgdaSymbol{(}\AgdaFunction{map}\AgdaSpace{}%
\AgdaSymbol{(}\AgdaOperator{\AgdaField{\AgdaUnderscore{}*}}\AgdaSpace{}%
\AgdaBound{x}\AgdaSymbol{)}\AgdaSpace{}%
\AgdaBound{xs}\AgdaSymbol{)}\<%
\\
%
\>[2]\AgdaFunction{*-distribʳ-sum}\AgdaSpace{}%
\AgdaSymbol{\{}\AgdaBound{N}\AgdaSymbol{\}}\AgdaSpace{}%
\AgdaSymbol{\{}\AgdaBound{xs}\AgdaSymbol{\}}\AgdaSpace{}%
\AgdaBound{x}\AgdaSpace{}%
\AgdaKeyword{rewrite}\<%
\\
\>[2][@{}l@{\AgdaIndent{0}}]%
\>[6]\AgdaFunction{*-comm}\AgdaSpace{}%
\AgdaSymbol{(}\AgdaFunction{sum}\AgdaSpace{}%
\AgdaBound{xs}\AgdaSymbol{)}\AgdaSpace{}%
\AgdaBound{x}\<%
\\
\>[2][@{}l@{\AgdaIndent{0}}]%
\>[4]\AgdaSymbol{|}\AgdaSpace{}%
\AgdaFunction{*-distribˡ-sum}\AgdaSpace{}%
\AgdaSymbol{\{}\AgdaBound{N}\AgdaSymbol{\}}\AgdaSpace{}%
\AgdaSymbol{\{}\AgdaBound{xs}\AgdaSymbol{\}}\AgdaSpace{}%
\AgdaBound{x}\<%
\\
%
\>[4]\AgdaSymbol{=}\AgdaSpace{}%
\AgdaFunction{sum-cong}\AgdaSpace{}%
\AgdaSymbol{\{}\AgdaBound{N}\AgdaSymbol{\}}\AgdaSpace{}%
\AgdaSymbol{λ}\AgdaSpace{}%
\AgdaBound{i}\AgdaSpace{}%
\AgdaSymbol{→}\AgdaSpace{}%
\AgdaFunction{*-comm}\AgdaSpace{}%
\AgdaBound{x}\AgdaSpace{}%
\AgdaSymbol{(}\AgdaBound{xs}\AgdaSpace{}%
\AgdaBound{i}\AgdaSymbol{)}\<%
\\
%
\\[\AgdaEmptyExtraSkip]%
%
\>[2]\AgdaComment{------------------------------------}\<%
\\
%
\>[2]\AgdaComment{---\ Rearanging\ of\ roots\ of\ unity\ ---}\<%
\\
%
\>[2]\AgdaComment{------------------------------------}\<%
\\
%
\>[2]\AgdaFunction{cooley-tukey-derivation}\AgdaSpace{}%
\AgdaSymbol{:}\<%
\\
\>[2][@{}l@{\AgdaIndent{0}}]%
\>[4]\AgdaSymbol{∀}\AgdaSpace{}%
\AgdaSymbol{(}\AgdaBound{r₁}\AgdaSpace{}%
\AgdaBound{r₂}\AgdaSpace{}%
\AgdaBound{k₀}\AgdaSpace{}%
\AgdaBound{k₁}\AgdaSpace{}%
\AgdaBound{j₀}\AgdaSpace{}%
\AgdaBound{j₁}\AgdaSpace{}%
\AgdaSymbol{:}\AgdaSpace{}%
\AgdaDatatype{ℕ}\AgdaSymbol{)}\<%
\\
%
\>[4]\AgdaSymbol{→}\AgdaSpace{}%
\AgdaSymbol{⦃}\AgdaSpace{}%
\AgdaBound{nonZero-r₁}%
\>[21]\AgdaSymbol{:}\AgdaSpace{}%
\AgdaRecord{NonZero}\AgdaSpace{}%
\AgdaBound{r₁}\AgdaSpace{}%
\AgdaSymbol{⦄}\<%
\\
%
\>[4]\AgdaSymbol{→}\AgdaSpace{}%
\AgdaSymbol{⦃}\AgdaSpace{}%
\AgdaBound{nonZero-r₂}%
\>[21]\AgdaSymbol{:}\AgdaSpace{}%
\AgdaRecord{NonZero}\AgdaSpace{}%
\AgdaBound{r₂}\AgdaSpace{}%
\AgdaSymbol{⦄}\<%
\\
%
\>[4]\AgdaSymbol{→}\<%
\\
\>[4][@{}l@{\AgdaIndent{0}}]%
\>[14]\AgdaField{-ω}\<%
\\
\>[14][@{}l@{\AgdaIndent{0}}]%
\>[16]\AgdaSymbol{(}\AgdaBound{r₂}\AgdaSpace{}%
\AgdaOperator{\AgdaPrimitive{*ₙ}}\AgdaSpace{}%
\AgdaBound{r₁}\AgdaSymbol{)}\<%
\\
%
\>[16]\AgdaSymbol{⦃}\AgdaSpace{}%
\AgdaFunction{m*n≢0}\AgdaSpace{}%
\AgdaBound{r₂}\AgdaSpace{}%
\AgdaBound{r₁}\AgdaSpace{}%
\AgdaSymbol{⦄}\<%
\\
%
\>[16]\AgdaSymbol{(}\<%
\\
\>[16][@{}l@{\AgdaIndent{0}}]%
\>[18]\AgdaSymbol{(}\AgdaBound{r₂}\AgdaSpace{}%
\AgdaOperator{\AgdaPrimitive{*ₙ}}\AgdaSpace{}%
\AgdaBound{k₁}\AgdaSpace{}%
\AgdaOperator{\AgdaPrimitive{+ₙ}}\AgdaSpace{}%
\AgdaBound{k₀}\AgdaSymbol{)}\<%
\\
%
\>[18]\AgdaOperator{\AgdaPrimitive{*ₙ}}\<%
\\
%
\>[18]\AgdaSymbol{(}\AgdaBound{r₁}\AgdaSpace{}%
\AgdaOperator{\AgdaPrimitive{*ₙ}}\AgdaSpace{}%
\AgdaBound{j₁}\AgdaSpace{}%
\AgdaOperator{\AgdaPrimitive{+ₙ}}\AgdaSpace{}%
\AgdaBound{j₀}\AgdaSymbol{)}\<%
\\
%
\>[16]\AgdaSymbol{)}\<%
\\
%
\>[14]\AgdaOperator{\AgdaDatatype{≡}}\<%
\\
\>[14][@{}l@{\AgdaIndent{0}}]%
\>[16]\AgdaField{-ω}\AgdaSpace{}%
\AgdaSymbol{(}\AgdaBound{r₁}\AgdaSymbol{)}\AgdaSpace{}%
\AgdaSymbol{(}\AgdaBound{k₁}\AgdaSpace{}%
\AgdaOperator{\AgdaPrimitive{*ₙ}}\AgdaSpace{}%
\AgdaBound{j₀}\AgdaSymbol{)}\<%
\\
%
\>[14]\AgdaOperator{\AgdaField{*}}\AgdaSpace{}%
\AgdaField{-ω}\AgdaSpace{}%
\AgdaSymbol{(}\AgdaBound{r₂}\AgdaSpace{}%
\AgdaOperator{\AgdaPrimitive{*ₙ}}\AgdaSpace{}%
\AgdaBound{r₁}\AgdaSymbol{)}\AgdaSpace{}%
\AgdaSymbol{⦃}\AgdaSpace{}%
\AgdaFunction{m*n≢0}\AgdaSpace{}%
\AgdaBound{r₂}\AgdaSpace{}%
\AgdaBound{r₁}\AgdaSpace{}%
\AgdaSymbol{⦄}\AgdaSpace{}%
\AgdaSymbol{(}\AgdaBound{k₀}\AgdaSpace{}%
\AgdaOperator{\AgdaPrimitive{*ₙ}}\AgdaSpace{}%
\AgdaBound{j₀}\AgdaSymbol{)}\<%
\\
%
\>[14]\AgdaOperator{\AgdaField{*}}\AgdaSpace{}%
\AgdaField{-ω}\AgdaSpace{}%
\AgdaSymbol{(}\AgdaBound{r₂}\AgdaSymbol{)}\AgdaSpace{}%
\AgdaSymbol{(}\AgdaBound{k₀}\AgdaSpace{}%
\AgdaOperator{\AgdaPrimitive{*ₙ}}\AgdaSpace{}%
\AgdaBound{j₁}\AgdaSymbol{)}\<%
\\
%
\>[2]\AgdaFunction{cooley-tukey-derivation}\AgdaSpace{}%
\AgdaBound{r₁}\AgdaSpace{}%
\AgdaBound{r₂}\AgdaSpace{}%
\AgdaBound{k₀}\AgdaSpace{}%
\AgdaBound{k₁}\AgdaSpace{}%
\AgdaBound{j₀}\AgdaSpace{}%
\AgdaBound{j₁}\AgdaSpace{}%
\AgdaSymbol{⦃}\AgdaSpace{}%
\AgdaBound{nonZero-r₁}\AgdaSpace{}%
\AgdaSymbol{⦄}\AgdaSpace{}%
\AgdaSymbol{⦃}\AgdaSpace{}%
\AgdaBound{nonZero-r₂}\AgdaSpace{}%
\AgdaSymbol{⦄}\<%
\\
\>[2][@{}l@{\AgdaIndent{0}}]%
\>[4]\AgdaSymbol{=}\AgdaSpace{}%
\AgdaFunction{rearrange-ω-power}\<%
\\
%
\>[4]\AgdaOperator{\AgdaFunction{⊡}}\AgdaSpace{}%
\AgdaFunction{split-ω}\<%
\\
%
\>[4]\AgdaOperator{\AgdaFunction{⊡}}\AgdaSpace{}%
\AgdaFunction{remove-constant-term}\<%
\\
%
\>[4]\AgdaOperator{\AgdaFunction{⊡}}\AgdaSpace{}%
\AgdaFunction{simplify-bases}\<%
\\
%
\>[4]\AgdaKeyword{where}\<%
\\
\>[4][@{}l@{\AgdaIndent{0}}]%
\>[6]\AgdaKeyword{instance}\<%
\\
\>[6][@{}l@{\AgdaIndent{0}}]%
\>[8]\AgdaFunction{\AgdaUnderscore{}}\AgdaSpace{}%
\AgdaSymbol{:}\AgdaSpace{}%
\AgdaRecord{NonZero}\AgdaSpace{}%
\AgdaSymbol{(}\AgdaBound{r₁}\AgdaSpace{}%
\AgdaOperator{\AgdaPrimitive{*ₙ}}\AgdaSpace{}%
\AgdaBound{r₂}\AgdaSymbol{)}\<%
\\
%
\>[8]\AgdaSymbol{\AgdaUnderscore{}}\AgdaSpace{}%
\AgdaSymbol{=}\AgdaSpace{}%
\AgdaFunction{m*n≢0}\AgdaSpace{}%
\AgdaBound{r₁}\AgdaSpace{}%
\AgdaBound{r₂}\<%
\\
%
\>[8]\AgdaFunction{\AgdaUnderscore{}}\AgdaSpace{}%
\AgdaSymbol{:}\AgdaSpace{}%
\AgdaRecord{NonZero}\AgdaSpace{}%
\AgdaSymbol{(}\AgdaBound{r₂}\AgdaSpace{}%
\AgdaOperator{\AgdaPrimitive{*ₙ}}\AgdaSpace{}%
\AgdaBound{r₁}\AgdaSymbol{)}\<%
\\
%
\>[8]\AgdaSymbol{\AgdaUnderscore{}}\AgdaSpace{}%
\AgdaSymbol{=}\AgdaSpace{}%
\AgdaFunction{m*n≢0}\AgdaSpace{}%
\AgdaBound{r₂}\AgdaSpace{}%
\AgdaBound{r₁}\<%
\\
%
\>[6]\AgdaFunction{simplify-bases}\AgdaSpace{}%
\AgdaSymbol{:}\<%
\\
\>[6][@{}l@{\AgdaIndent{0}}]%
\>[12]\AgdaField{-ω}\AgdaSpace{}%
\AgdaSymbol{(}\AgdaBound{r₂}\AgdaSpace{}%
\AgdaOperator{\AgdaPrimitive{*ₙ}}\AgdaSpace{}%
\AgdaBound{r₁}\AgdaSymbol{)}\AgdaSpace{}%
\AgdaSymbol{⦃}\AgdaSpace{}%
\AgdaFunction{m*n≢0}\AgdaSpace{}%
\AgdaBound{r₂}\AgdaSpace{}%
\AgdaBound{r₁}\AgdaSpace{}%
\AgdaSymbol{⦄}\AgdaSpace{}%
\AgdaSymbol{(}\AgdaBound{r₂}\AgdaSpace{}%
\AgdaOperator{\AgdaPrimitive{*ₙ}}\AgdaSpace{}%
\AgdaSymbol{(}\AgdaBound{k₁}\AgdaSpace{}%
\AgdaOperator{\AgdaPrimitive{*ₙ}}\AgdaSpace{}%
\AgdaBound{j₀}\AgdaSymbol{))}\<%
\\
\>[6][@{}l@{\AgdaIndent{0}}]%
\>[10]\AgdaOperator{\AgdaField{*}}\AgdaSpace{}%
\AgdaField{-ω}\AgdaSpace{}%
\AgdaSymbol{(}\AgdaBound{r₂}\AgdaSpace{}%
\AgdaOperator{\AgdaPrimitive{*ₙ}}\AgdaSpace{}%
\AgdaBound{r₁}\AgdaSymbol{)}\AgdaSpace{}%
\AgdaSymbol{⦃}\AgdaSpace{}%
\AgdaFunction{m*n≢0}\AgdaSpace{}%
\AgdaBound{r₂}\AgdaSpace{}%
\AgdaBound{r₁}\AgdaSpace{}%
\AgdaSymbol{⦄}\AgdaSpace{}%
\AgdaSymbol{(}\AgdaBound{k₀}\AgdaSpace{}%
\AgdaOperator{\AgdaPrimitive{*ₙ}}\AgdaSpace{}%
\AgdaBound{j₀}\AgdaSymbol{)}\<%
\\
%
\>[10]\AgdaOperator{\AgdaField{*}}\AgdaSpace{}%
\AgdaField{-ω}\AgdaSpace{}%
\AgdaSymbol{(}\AgdaBound{r₂}\AgdaSpace{}%
\AgdaOperator{\AgdaPrimitive{*ₙ}}\AgdaSpace{}%
\AgdaBound{r₁}\AgdaSymbol{)}\AgdaSpace{}%
\AgdaSymbol{⦃}\AgdaSpace{}%
\AgdaFunction{m*n≢0}\AgdaSpace{}%
\AgdaBound{r₂}\AgdaSpace{}%
\AgdaBound{r₁}\AgdaSpace{}%
\AgdaSymbol{⦄}\AgdaSpace{}%
\AgdaSymbol{(}\AgdaBound{r₁}\AgdaSpace{}%
\AgdaOperator{\AgdaPrimitive{*ₙ}}\AgdaSpace{}%
\AgdaSymbol{(}\AgdaBound{k₀}\AgdaSpace{}%
\AgdaOperator{\AgdaPrimitive{*ₙ}}\AgdaSpace{}%
\AgdaBound{j₁}\AgdaSymbol{))}\<%
\\
\>[6][@{}l@{\AgdaIndent{0}}]%
\>[8]\AgdaOperator{\AgdaDatatype{≡}}\<%
\\
\>[8][@{}l@{\AgdaIndent{0}}]%
\>[12]\AgdaField{-ω}\AgdaSpace{}%
\AgdaSymbol{(}\AgdaBound{r₁}\AgdaSymbol{)}\AgdaSpace{}%
\AgdaSymbol{(}\AgdaBound{k₁}\AgdaSpace{}%
\AgdaOperator{\AgdaPrimitive{*ₙ}}\AgdaSpace{}%
\AgdaBound{j₀}\AgdaSymbol{)}\<%
\\
\>[8][@{}l@{\AgdaIndent{0}}]%
\>[10]\AgdaOperator{\AgdaField{*}}\AgdaSpace{}%
\AgdaField{-ω}\AgdaSpace{}%
\AgdaSymbol{(}\AgdaBound{r₂}\AgdaSpace{}%
\AgdaOperator{\AgdaPrimitive{*ₙ}}\AgdaSpace{}%
\AgdaBound{r₁}\AgdaSymbol{)}\AgdaSpace{}%
\AgdaSymbol{⦃}\AgdaSpace{}%
\AgdaFunction{m*n≢0}\AgdaSpace{}%
\AgdaBound{r₂}\AgdaSpace{}%
\AgdaBound{r₁}\AgdaSpace{}%
\AgdaSymbol{⦄}\AgdaSpace{}%
\AgdaSymbol{(}\AgdaBound{k₀}\AgdaSpace{}%
\AgdaOperator{\AgdaPrimitive{*ₙ}}\AgdaSpace{}%
\AgdaBound{j₀}\AgdaSymbol{)}\<%
\\
%
\>[10]\AgdaOperator{\AgdaField{*}}\AgdaSpace{}%
\AgdaField{-ω}\AgdaSpace{}%
\AgdaSymbol{(}\AgdaBound{r₂}\AgdaSymbol{)}\AgdaSpace{}%
\AgdaSymbol{(}\AgdaBound{k₀}\AgdaSpace{}%
\AgdaOperator{\AgdaPrimitive{*ₙ}}\AgdaSpace{}%
\AgdaBound{j₁}\AgdaSymbol{)}\<%
\\
%
\>[6]\AgdaFunction{simplify-bases}\AgdaSpace{}%
\AgdaSymbol{=}\<%
\\
\>[6][@{}l@{\AgdaIndent{0}}]%
\>[10]\AgdaFunction{cong₂}\<%
\\
\>[10][@{}l@{\AgdaIndent{0}}]%
\>[12]\AgdaOperator{\AgdaField{\AgdaUnderscore{}*\AgdaUnderscore{}}}\<%
\\
%
\>[12]\AgdaSymbol{(}%
\>[1466I]\AgdaFunction{cong₂}\<%
\\
\>[1466I][@{}l@{\AgdaIndent{0}}]%
\>[18]\AgdaOperator{\AgdaField{\AgdaUnderscore{}*\AgdaUnderscore{}}}\<%
\\
%
\>[18]\AgdaSymbol{(}\AgdaField{ω-r₁x-r₁y}\AgdaSpace{}%
\AgdaBound{r₂}\AgdaSpace{}%
\AgdaBound{r₁}\AgdaSpace{}%
\AgdaSymbol{(}\AgdaBound{k₁}\AgdaSpace{}%
\AgdaOperator{\AgdaPrimitive{*ₙ}}\AgdaSpace{}%
\AgdaBound{j₀}\AgdaSymbol{))}\<%
\\
%
\>[18]\AgdaInductiveConstructor{refl}\<%
\\
%
\>[12]\AgdaSymbol{)}\<%
\\
%
\>[12]\AgdaSymbol{(}%
\>[16]\AgdaFunction{-ω-cong₂}\AgdaSpace{}%
\AgdaSymbol{(}\AgdaFunction{*ₙ-comm}\AgdaSpace{}%
\AgdaBound{r₂}\AgdaSpace{}%
\AgdaBound{r₁}\AgdaSymbol{)}\AgdaSpace{}%
\AgdaInductiveConstructor{refl}\<%
\\
\>[12][@{}l@{\AgdaIndent{0}}]%
\>[14]\AgdaOperator{\AgdaFunction{⊡}}\AgdaSpace{}%
\AgdaSymbol{(}\AgdaField{ω-r₁x-r₁y}\AgdaSpace{}%
\AgdaBound{r₁}\AgdaSpace{}%
\AgdaBound{r₂}\AgdaSpace{}%
\AgdaSymbol{(}\AgdaBound{k₀}\AgdaSpace{}%
\AgdaOperator{\AgdaPrimitive{*ₙ}}\AgdaSpace{}%
\AgdaBound{j₁}\AgdaSymbol{))}\<%
\\
%
\>[12]\AgdaSymbol{)}\<%
\\
%
\>[6]\AgdaFunction{remove-constant-term}\AgdaSpace{}%
\AgdaSymbol{:}\<%
\\
\>[6][@{}l@{\AgdaIndent{0}}]%
\>[12]\AgdaField{-ω}\AgdaSpace{}%
\AgdaSymbol{(}\AgdaBound{r₂}\AgdaSpace{}%
\AgdaOperator{\AgdaPrimitive{*ₙ}}\AgdaSpace{}%
\AgdaBound{r₁}\AgdaSymbol{)}\AgdaSpace{}%
\AgdaSymbol{⦃}\AgdaSpace{}%
\AgdaFunction{m*n≢0}\AgdaSpace{}%
\AgdaBound{r₂}\AgdaSpace{}%
\AgdaBound{r₁}\AgdaSpace{}%
\AgdaSymbol{⦄}\AgdaSpace{}%
\AgdaSymbol{(}\AgdaBound{r₂}\AgdaSpace{}%
\AgdaOperator{\AgdaPrimitive{*ₙ}}\AgdaSpace{}%
\AgdaSymbol{(}\AgdaBound{k₁}\AgdaSpace{}%
\AgdaOperator{\AgdaPrimitive{*ₙ}}\AgdaSpace{}%
\AgdaBound{j₀}\AgdaSymbol{))}\<%
\\
\>[6][@{}l@{\AgdaIndent{0}}]%
\>[10]\AgdaOperator{\AgdaField{*}}\AgdaSpace{}%
\AgdaField{-ω}\AgdaSpace{}%
\AgdaSymbol{(}\AgdaBound{r₂}\AgdaSpace{}%
\AgdaOperator{\AgdaPrimitive{*ₙ}}\AgdaSpace{}%
\AgdaBound{r₁}\AgdaSymbol{)}\AgdaSpace{}%
\AgdaSymbol{⦃}\AgdaSpace{}%
\AgdaFunction{m*n≢0}\AgdaSpace{}%
\AgdaBound{r₂}\AgdaSpace{}%
\AgdaBound{r₁}\AgdaSpace{}%
\AgdaSymbol{⦄}\AgdaSpace{}%
\AgdaSymbol{(}\AgdaBound{k₀}\AgdaSpace{}%
\AgdaOperator{\AgdaPrimitive{*ₙ}}\AgdaSpace{}%
\AgdaBound{j₀}\AgdaSymbol{)}\<%
\\
%
\>[10]\AgdaOperator{\AgdaField{*}}\AgdaSpace{}%
\AgdaField{-ω}\AgdaSpace{}%
\AgdaSymbol{(}\AgdaBound{r₂}\AgdaSpace{}%
\AgdaOperator{\AgdaPrimitive{*ₙ}}\AgdaSpace{}%
\AgdaBound{r₁}\AgdaSymbol{)}\AgdaSpace{}%
\AgdaSymbol{⦃}\AgdaSpace{}%
\AgdaFunction{m*n≢0}\AgdaSpace{}%
\AgdaBound{r₂}\AgdaSpace{}%
\AgdaBound{r₁}\AgdaSpace{}%
\AgdaSymbol{⦄}\AgdaSpace{}%
\AgdaSymbol{(}\AgdaBound{r₁}\AgdaSpace{}%
\AgdaOperator{\AgdaPrimitive{*ₙ}}\AgdaSpace{}%
\AgdaSymbol{(}\AgdaBound{k₀}\AgdaSpace{}%
\AgdaOperator{\AgdaPrimitive{*ₙ}}\AgdaSpace{}%
\AgdaBound{j₁}\AgdaSymbol{))}\<%
\\
%
\>[10]\AgdaOperator{\AgdaField{*}}\AgdaSpace{}%
\AgdaField{-ω}\AgdaSpace{}%
\AgdaSymbol{(}\AgdaBound{r₂}\AgdaSpace{}%
\AgdaOperator{\AgdaPrimitive{*ₙ}}\AgdaSpace{}%
\AgdaBound{r₁}\AgdaSymbol{)}\AgdaSpace{}%
\AgdaSymbol{⦃}\AgdaSpace{}%
\AgdaFunction{m*n≢0}\AgdaSpace{}%
\AgdaBound{r₂}\AgdaSpace{}%
\AgdaBound{r₁}\AgdaSpace{}%
\AgdaSymbol{⦄}\AgdaSpace{}%
\AgdaSymbol{((}\AgdaBound{r₂}\AgdaSpace{}%
\AgdaOperator{\AgdaPrimitive{*ₙ}}\AgdaSpace{}%
\AgdaBound{r₁}\AgdaSymbol{)}\AgdaSpace{}%
\AgdaOperator{\AgdaPrimitive{*ₙ}}\AgdaSpace{}%
\AgdaSymbol{(}\AgdaBound{j₁}\AgdaSpace{}%
\AgdaOperator{\AgdaPrimitive{*ₙ}}\AgdaSpace{}%
\AgdaBound{k₁}\AgdaSymbol{))}\<%
\\
\>[6][@{}l@{\AgdaIndent{0}}]%
\>[8]\AgdaOperator{\AgdaDatatype{≡}}\<%
\\
\>[8][@{}l@{\AgdaIndent{0}}]%
\>[12]\AgdaField{-ω}\AgdaSpace{}%
\AgdaSymbol{(}\AgdaBound{r₂}\AgdaSpace{}%
\AgdaOperator{\AgdaPrimitive{*ₙ}}\AgdaSpace{}%
\AgdaBound{r₁}\AgdaSymbol{)}\AgdaSpace{}%
\AgdaSymbol{⦃}\AgdaSpace{}%
\AgdaFunction{m*n≢0}\AgdaSpace{}%
\AgdaBound{r₂}\AgdaSpace{}%
\AgdaBound{r₁}\AgdaSpace{}%
\AgdaSymbol{⦄}\AgdaSpace{}%
\AgdaSymbol{(}\AgdaBound{r₂}\AgdaSpace{}%
\AgdaOperator{\AgdaPrimitive{*ₙ}}\AgdaSpace{}%
\AgdaSymbol{(}\AgdaBound{k₁}\AgdaSpace{}%
\AgdaOperator{\AgdaPrimitive{*ₙ}}\AgdaSpace{}%
\AgdaBound{j₀}\AgdaSymbol{))}\<%
\\
\>[8][@{}l@{\AgdaIndent{0}}]%
\>[10]\AgdaOperator{\AgdaField{*}}\AgdaSpace{}%
\AgdaField{-ω}\AgdaSpace{}%
\AgdaSymbol{(}\AgdaBound{r₂}\AgdaSpace{}%
\AgdaOperator{\AgdaPrimitive{*ₙ}}\AgdaSpace{}%
\AgdaBound{r₁}\AgdaSymbol{)}\AgdaSpace{}%
\AgdaSymbol{⦃}\AgdaSpace{}%
\AgdaFunction{m*n≢0}\AgdaSpace{}%
\AgdaBound{r₂}\AgdaSpace{}%
\AgdaBound{r₁}\AgdaSpace{}%
\AgdaSymbol{⦄}\AgdaSpace{}%
\AgdaSymbol{(}\AgdaBound{k₀}\AgdaSpace{}%
\AgdaOperator{\AgdaPrimitive{*ₙ}}\AgdaSpace{}%
\AgdaBound{j₀}\AgdaSymbol{)}\<%
\\
%
\>[10]\AgdaOperator{\AgdaField{*}}\AgdaSpace{}%
\AgdaField{-ω}\AgdaSpace{}%
\AgdaSymbol{(}\AgdaBound{r₂}\AgdaSpace{}%
\AgdaOperator{\AgdaPrimitive{*ₙ}}\AgdaSpace{}%
\AgdaBound{r₁}\AgdaSymbol{)}\AgdaSpace{}%
\AgdaSymbol{⦃}\AgdaSpace{}%
\AgdaFunction{m*n≢0}\AgdaSpace{}%
\AgdaBound{r₂}\AgdaSpace{}%
\AgdaBound{r₁}\AgdaSpace{}%
\AgdaSymbol{⦄}\AgdaSpace{}%
\AgdaSymbol{(}\AgdaBound{r₁}\AgdaSpace{}%
\AgdaOperator{\AgdaPrimitive{*ₙ}}\AgdaSpace{}%
\AgdaSymbol{(}\AgdaBound{k₀}\AgdaSpace{}%
\AgdaOperator{\AgdaPrimitive{*ₙ}}\AgdaSpace{}%
\AgdaBound{j₁}\AgdaSymbol{))}\<%
\\
%
\>[6]\AgdaFunction{remove-constant-term}\AgdaSpace{}%
\AgdaSymbol{=}\<%
\\
\>[6][@{}l@{\AgdaIndent{0}}]%
\>[10]\AgdaFunction{cong₂}\AgdaSpace{}%
\AgdaOperator{\AgdaField{\AgdaUnderscore{}*\AgdaUnderscore{}}}\AgdaSpace{}%
\AgdaInductiveConstructor{refl}\AgdaSpace{}%
\AgdaSymbol{(}\AgdaField{ω-N-mN}\AgdaSpace{}%
\AgdaSymbol{\{}\AgdaBound{r₂}\AgdaSpace{}%
\AgdaOperator{\AgdaPrimitive{*ₙ}}\AgdaSpace{}%
\AgdaBound{r₁}\AgdaSymbol{\}}\AgdaSpace{}%
\AgdaSymbol{\{}\AgdaBound{j₁}\AgdaSpace{}%
\AgdaOperator{\AgdaPrimitive{*ₙ}}\AgdaSpace{}%
\AgdaBound{k₁}\AgdaSymbol{\})}\<%
\\
\>[6][@{}l@{\AgdaIndent{0}}]%
\>[8]\AgdaOperator{\AgdaFunction{⊡}}\AgdaSpace{}%
\AgdaFunction{*-identityʳ}\AgdaSpace{}%
\AgdaSymbol{\AgdaUnderscore{}}\<%
\\
%
\>[6]\AgdaFunction{rearrange-ω-power}\AgdaSpace{}%
\AgdaSymbol{:}\<%
\\
\>[6][@{}l@{\AgdaIndent{0}}]%
\>[10]\AgdaField{-ω}\<%
\\
\>[10][@{}l@{\AgdaIndent{0}}]%
\>[12]\AgdaSymbol{(}\AgdaBound{r₂}\AgdaSpace{}%
\AgdaOperator{\AgdaPrimitive{*ₙ}}\AgdaSpace{}%
\AgdaBound{r₁}\AgdaSymbol{)}\<%
\\
%
\>[12]\AgdaSymbol{(}%
\>[15]\AgdaSymbol{(}\AgdaBound{r₂}\AgdaSpace{}%
\AgdaOperator{\AgdaPrimitive{*ₙ}}\AgdaSpace{}%
\AgdaBound{k₁}\AgdaSpace{}%
\AgdaOperator{\AgdaPrimitive{+ₙ}}\AgdaSpace{}%
\AgdaBound{k₀}\AgdaSymbol{)}\<%
\\
%
\>[12]\AgdaOperator{\AgdaPrimitive{*ₙ}}\AgdaSpace{}%
\AgdaSymbol{(}\AgdaBound{r₁}\AgdaSpace{}%
\AgdaOperator{\AgdaPrimitive{*ₙ}}\AgdaSpace{}%
\AgdaBound{j₁}\AgdaSpace{}%
\AgdaOperator{\AgdaPrimitive{+ₙ}}\AgdaSpace{}%
\AgdaBound{j₀}\AgdaSymbol{)}\<%
\\
%
\>[12]\AgdaSymbol{)}\<%
\\
\>[0]\<%
\\
\>[6][@{}l@{\AgdaIndent{0}}]%
\>[8]\AgdaOperator{\AgdaDatatype{≡}}\<%
\\
\>[8][@{}l@{\AgdaIndent{0}}]%
\>[10]\AgdaField{-ω}\<%
\\
\>[10][@{}l@{\AgdaIndent{0}}]%
\>[12]\AgdaSymbol{(}\AgdaBound{r₂}\AgdaSpace{}%
\AgdaOperator{\AgdaPrimitive{*ₙ}}\AgdaSpace{}%
\AgdaBound{r₁}\AgdaSymbol{)}\<%
\\
%
\>[12]\AgdaSymbol{(}\AgdaSpace{}%
\AgdaBound{r₂}\AgdaSpace{}%
\AgdaOperator{\AgdaPrimitive{*ₙ}}\AgdaSpace{}%
\AgdaSymbol{(}\AgdaBound{k₁}\AgdaSpace{}%
\AgdaOperator{\AgdaPrimitive{*ₙ}}\AgdaSpace{}%
\AgdaBound{j₀}\AgdaSymbol{)}\<%
\\
%
\>[12]\AgdaOperator{\AgdaPrimitive{+ₙ}}\AgdaSpace{}%
\AgdaBound{k₀}\AgdaSpace{}%
\AgdaOperator{\AgdaPrimitive{*ₙ}}\AgdaSpace{}%
\AgdaBound{j₀}\<%
\\
%
\>[12]\AgdaOperator{\AgdaPrimitive{+ₙ}}\AgdaSpace{}%
\AgdaBound{r₁}\AgdaSpace{}%
\AgdaOperator{\AgdaPrimitive{*ₙ}}\AgdaSpace{}%
\AgdaSymbol{(}\AgdaBound{k₀}\AgdaSpace{}%
\AgdaOperator{\AgdaPrimitive{*ₙ}}\AgdaSpace{}%
\AgdaBound{j₁}\AgdaSymbol{)}\<%
\\
%
\>[12]\AgdaOperator{\AgdaPrimitive{+ₙ}}\AgdaSpace{}%
\AgdaBound{r₂}\AgdaSpace{}%
\AgdaOperator{\AgdaPrimitive{*ₙ}}\AgdaSpace{}%
\AgdaSymbol{(}\AgdaBound{r₁}\AgdaSpace{}%
\AgdaOperator{\AgdaPrimitive{*ₙ}}\AgdaSpace{}%
\AgdaSymbol{(}\AgdaBound{j₁}\AgdaSpace{}%
\AgdaOperator{\AgdaPrimitive{*ₙ}}\AgdaSpace{}%
\AgdaBound{k₁}\AgdaSymbol{))}\<%
\\
%
\>[12]\AgdaSymbol{)}\<%
\\
%
\>[6]\AgdaFunction{rearrange-ω-power}\AgdaSpace{}%
\AgdaSymbol{=}\<%
\\
\>[6][@{}l@{\AgdaIndent{0}}]%
\>[8]\AgdaFunction{-ω-cong₂}\<%
\\
\>[8][@{}l@{\AgdaIndent{0}}]%
\>[10]\AgdaInductiveConstructor{refl}\<%
\\
%
\>[10]\AgdaSymbol{(}\AgdaFunction{solve}\<%
\\
\>[10][@{}l@{\AgdaIndent{0}}]%
\>[12]\AgdaNumber{6}\<%
\\
%
\>[12]\AgdaSymbol{(λ}\AgdaSpace{}%
\AgdaBound{r₁ℕ}\AgdaSpace{}%
\AgdaBound{r₂ℕ}\AgdaSpace{}%
\AgdaBound{k₀ℕ}\AgdaSpace{}%
\AgdaBound{k₁ℕ}\AgdaSpace{}%
\AgdaBound{j₀ℕ}\AgdaSpace{}%
\AgdaBound{j₁ℕ}\AgdaSpace{}%
\AgdaSymbol{→}\AgdaSpace{}%
\AgdaSymbol{(}\AgdaBound{r₂ℕ}\AgdaSpace{}%
\AgdaOperator{\AgdaFunction{:*}}\AgdaSpace{}%
\AgdaBound{k₁ℕ}\AgdaSpace{}%
\AgdaOperator{\AgdaFunction{:+}}\AgdaSpace{}%
\AgdaBound{k₀ℕ}\AgdaSymbol{)}\AgdaSpace{}%
\AgdaOperator{\AgdaFunction{:*}}\AgdaSpace{}%
\AgdaSymbol{(}\AgdaBound{r₁ℕ}\AgdaSpace{}%
\AgdaOperator{\AgdaFunction{:*}}\AgdaSpace{}%
\AgdaBound{j₁ℕ}\AgdaSpace{}%
\AgdaOperator{\AgdaFunction{:+}}\AgdaSpace{}%
\AgdaBound{j₀ℕ}\AgdaSymbol{)}\AgdaSpace{}%
\AgdaOperator{\AgdaFunction{:=}}\AgdaSpace{}%
\AgdaBound{r₂ℕ}\AgdaSpace{}%
\AgdaOperator{\AgdaFunction{:*}}\AgdaSpace{}%
\AgdaSymbol{(}\AgdaBound{k₁ℕ}\AgdaSpace{}%
\AgdaOperator{\AgdaFunction{:*}}\AgdaSpace{}%
\AgdaBound{j₀ℕ}\AgdaSymbol{)}\AgdaSpace{}%
\AgdaOperator{\AgdaFunction{:+}}\AgdaSpace{}%
\AgdaBound{k₀ℕ}\AgdaSpace{}%
\AgdaOperator{\AgdaFunction{:*}}\AgdaSpace{}%
\AgdaBound{j₀ℕ}\AgdaSpace{}%
\AgdaOperator{\AgdaFunction{:+}}\AgdaSpace{}%
\AgdaBound{r₁ℕ}\AgdaSpace{}%
\AgdaOperator{\AgdaFunction{:*}}\AgdaSpace{}%
\AgdaSymbol{(}\AgdaBound{k₀ℕ}\AgdaSpace{}%
\AgdaOperator{\AgdaFunction{:*}}\AgdaSpace{}%
\AgdaBound{j₁ℕ}\AgdaSymbol{)}\AgdaSpace{}%
\AgdaOperator{\AgdaFunction{:+}}\AgdaSpace{}%
\AgdaBound{r₂ℕ}\AgdaSpace{}%
\AgdaOperator{\AgdaFunction{:*}}\AgdaSpace{}%
\AgdaSymbol{(}\AgdaBound{r₁ℕ}\AgdaSpace{}%
\AgdaOperator{\AgdaFunction{:*}}\AgdaSpace{}%
\AgdaSymbol{(}\AgdaBound{j₁ℕ}\AgdaSpace{}%
\AgdaOperator{\AgdaFunction{:*}}\AgdaSpace{}%
\AgdaBound{k₁ℕ}\AgdaSymbol{)))}\<%
\\
%
\>[12]\AgdaInductiveConstructor{refl}\<%
\\
%
\>[12]\AgdaBound{r₁}\AgdaSpace{}%
\AgdaBound{r₂}\AgdaSpace{}%
\AgdaBound{k₀}\AgdaSpace{}%
\AgdaBound{k₁}\AgdaSpace{}%
\AgdaBound{j₀}\AgdaSpace{}%
\AgdaBound{j₁}\<%
\\
%
\>[10]\AgdaSymbol{)}\<%
\\
%
\>[6]\AgdaFunction{split-ω}\AgdaSpace{}%
\AgdaSymbol{:}\<%
\\
\>[6][@{}l@{\AgdaIndent{0}}]%
\>[10]\AgdaField{-ω}\<%
\\
\>[10][@{}l@{\AgdaIndent{0}}]%
\>[12]\AgdaSymbol{(}\AgdaBound{r₂}\AgdaSpace{}%
\AgdaOperator{\AgdaPrimitive{*ₙ}}\AgdaSpace{}%
\AgdaBound{r₁}\AgdaSymbol{)}\<%
\\
%
\>[12]\AgdaSymbol{(}\AgdaSpace{}%
\AgdaBound{r₂}\AgdaSpace{}%
\AgdaOperator{\AgdaPrimitive{*ₙ}}\AgdaSpace{}%
\AgdaSymbol{(}\AgdaBound{k₁}\AgdaSpace{}%
\AgdaOperator{\AgdaPrimitive{*ₙ}}\AgdaSpace{}%
\AgdaBound{j₀}\AgdaSymbol{)}\<%
\\
%
\>[12]\AgdaOperator{\AgdaPrimitive{+ₙ}}\AgdaSpace{}%
\AgdaBound{k₀}\AgdaSpace{}%
\AgdaOperator{\AgdaPrimitive{*ₙ}}\AgdaSpace{}%
\AgdaBound{j₀}\<%
\\
%
\>[12]\AgdaOperator{\AgdaPrimitive{+ₙ}}\AgdaSpace{}%
\AgdaBound{r₁}\AgdaSpace{}%
\AgdaOperator{\AgdaPrimitive{*ₙ}}\AgdaSpace{}%
\AgdaSymbol{(}\AgdaBound{k₀}\AgdaSpace{}%
\AgdaOperator{\AgdaPrimitive{*ₙ}}\AgdaSpace{}%
\AgdaBound{j₁}\AgdaSymbol{)}\<%
\\
%
\>[12]\AgdaOperator{\AgdaPrimitive{+ₙ}}\AgdaSpace{}%
\AgdaBound{r₂}\AgdaSpace{}%
\AgdaOperator{\AgdaPrimitive{*ₙ}}\AgdaSpace{}%
\AgdaSymbol{(}\AgdaBound{r₁}\AgdaSpace{}%
\AgdaOperator{\AgdaPrimitive{*ₙ}}\AgdaSpace{}%
\AgdaSymbol{(}\AgdaBound{j₁}\AgdaSpace{}%
\AgdaOperator{\AgdaPrimitive{*ₙ}}\AgdaSpace{}%
\AgdaBound{k₁}\AgdaSymbol{))}\<%
\\
%
\>[12]\AgdaSymbol{)}\<%
\\
%
\>[10]\AgdaOperator{\AgdaDatatype{≡}}\<%
\\
\>[10][@{}l@{\AgdaIndent{0}}]%
\>[12]\AgdaField{-ω}\AgdaSpace{}%
\AgdaSymbol{(}\AgdaBound{r₂}\AgdaSpace{}%
\AgdaOperator{\AgdaPrimitive{*ₙ}}\AgdaSpace{}%
\AgdaBound{r₁}\AgdaSymbol{)}\AgdaSpace{}%
\AgdaSymbol{⦃}\AgdaSpace{}%
\AgdaFunction{m*n≢0}\AgdaSpace{}%
\AgdaBound{r₂}\AgdaSpace{}%
\AgdaBound{r₁}\AgdaSpace{}%
\AgdaSymbol{⦄}\AgdaSpace{}%
\AgdaSymbol{(}\AgdaBound{r₂}\AgdaSpace{}%
\AgdaOperator{\AgdaPrimitive{*ₙ}}\AgdaSpace{}%
\AgdaSymbol{(}\AgdaBound{k₁}\AgdaSpace{}%
\AgdaOperator{\AgdaPrimitive{*ₙ}}\AgdaSpace{}%
\AgdaBound{j₀}\AgdaSymbol{))}\<%
\\
%
\>[10]\AgdaOperator{\AgdaField{*}}\AgdaSpace{}%
\AgdaField{-ω}\AgdaSpace{}%
\AgdaSymbol{(}\AgdaBound{r₂}\AgdaSpace{}%
\AgdaOperator{\AgdaPrimitive{*ₙ}}\AgdaSpace{}%
\AgdaBound{r₁}\AgdaSymbol{)}\AgdaSpace{}%
\AgdaSymbol{⦃}\AgdaSpace{}%
\AgdaFunction{m*n≢0}\AgdaSpace{}%
\AgdaBound{r₂}\AgdaSpace{}%
\AgdaBound{r₁}\AgdaSpace{}%
\AgdaSymbol{⦄}\AgdaSpace{}%
\AgdaSymbol{(}\AgdaBound{k₀}\AgdaSpace{}%
\AgdaOperator{\AgdaPrimitive{*ₙ}}\AgdaSpace{}%
\AgdaBound{j₀}\AgdaSymbol{)}\<%
\\
%
\>[10]\AgdaOperator{\AgdaField{*}}\AgdaSpace{}%
\AgdaField{-ω}\AgdaSpace{}%
\AgdaSymbol{(}\AgdaBound{r₂}\AgdaSpace{}%
\AgdaOperator{\AgdaPrimitive{*ₙ}}\AgdaSpace{}%
\AgdaBound{r₁}\AgdaSymbol{)}\AgdaSpace{}%
\AgdaSymbol{⦃}\AgdaSpace{}%
\AgdaFunction{m*n≢0}\AgdaSpace{}%
\AgdaBound{r₂}\AgdaSpace{}%
\AgdaBound{r₁}\AgdaSpace{}%
\AgdaSymbol{⦄}\AgdaSpace{}%
\AgdaSymbol{(}\AgdaBound{r₁}\AgdaSpace{}%
\AgdaOperator{\AgdaPrimitive{*ₙ}}\AgdaSpace{}%
\AgdaSymbol{(}\AgdaBound{k₀}\AgdaSpace{}%
\AgdaOperator{\AgdaPrimitive{*ₙ}}\AgdaSpace{}%
\AgdaBound{j₁}\AgdaSymbol{))}\<%
\\
%
\>[10]\AgdaOperator{\AgdaField{*}}\AgdaSpace{}%
\AgdaField{-ω}\AgdaSpace{}%
\AgdaSymbol{(}\AgdaBound{r₂}\AgdaSpace{}%
\AgdaOperator{\AgdaPrimitive{*ₙ}}\AgdaSpace{}%
\AgdaBound{r₁}\AgdaSymbol{)}\AgdaSpace{}%
\AgdaSymbol{⦃}\AgdaSpace{}%
\AgdaFunction{m*n≢0}\AgdaSpace{}%
\AgdaBound{r₂}\AgdaSpace{}%
\AgdaBound{r₁}\AgdaSpace{}%
\AgdaSymbol{⦄}\AgdaSpace{}%
\AgdaSymbol{((}\AgdaBound{r₂}\AgdaSpace{}%
\AgdaOperator{\AgdaPrimitive{*ₙ}}\AgdaSpace{}%
\AgdaBound{r₁}\AgdaSymbol{)}\AgdaSpace{}%
\AgdaOperator{\AgdaPrimitive{*ₙ}}\AgdaSpace{}%
\AgdaSymbol{(}\AgdaBound{j₁}\AgdaSpace{}%
\AgdaOperator{\AgdaPrimitive{*ₙ}}\AgdaSpace{}%
\AgdaBound{k₁}\AgdaSymbol{))}\<%
\\
%
\>[6]\AgdaFunction{split-ω}\AgdaSpace{}%
\AgdaSymbol{=}\<%
\\
\>[6][@{}l@{\AgdaIndent{0}}]%
\>[12]\AgdaSymbol{(}\AgdaField{ω-N-k₀+k₁}\AgdaSpace{}%
\AgdaSymbol{\{}\AgdaBound{r₂}\AgdaSpace{}%
\AgdaOperator{\AgdaPrimitive{*ₙ}}\AgdaSpace{}%
\AgdaBound{r₁}\AgdaSymbol{\}}\AgdaSpace{}%
\AgdaSymbol{\{}\AgdaBound{r₂}\AgdaSpace{}%
\AgdaOperator{\AgdaPrimitive{*ₙ}}\AgdaSpace{}%
\AgdaSymbol{(}\AgdaBound{k₁}\AgdaSpace{}%
\AgdaOperator{\AgdaPrimitive{*ₙ}}\AgdaSpace{}%
\AgdaBound{j₀}\AgdaSymbol{)}\AgdaSpace{}%
\AgdaOperator{\AgdaPrimitive{+ₙ}}\AgdaSpace{}%
\AgdaBound{k₀}\AgdaSpace{}%
\AgdaOperator{\AgdaPrimitive{*ₙ}}\AgdaSpace{}%
\AgdaBound{j₀}\AgdaSpace{}%
\AgdaOperator{\AgdaPrimitive{+ₙ}}\AgdaSpace{}%
\AgdaBound{r₁}\AgdaSpace{}%
\AgdaOperator{\AgdaPrimitive{*ₙ}}\AgdaSpace{}%
\AgdaSymbol{(}\AgdaBound{k₀}\AgdaSpace{}%
\AgdaOperator{\AgdaPrimitive{*ₙ}}\AgdaSpace{}%
\AgdaBound{j₁}\AgdaSymbol{)\}}\AgdaSpace{}%
\AgdaSymbol{\{}\AgdaBound{r₂}\AgdaSpace{}%
\AgdaOperator{\AgdaPrimitive{*ₙ}}\AgdaSpace{}%
\AgdaSymbol{(}\AgdaBound{r₁}\AgdaSpace{}%
\AgdaOperator{\AgdaPrimitive{*ₙ}}\AgdaSpace{}%
\AgdaSymbol{(}\AgdaBound{j₁}\AgdaSpace{}%
\AgdaOperator{\AgdaPrimitive{*ₙ}}\AgdaSpace{}%
\AgdaBound{k₁}\AgdaSymbol{))\}}\AgdaSpace{}%
\AgdaSymbol{)}\<%
\\
\>[6][@{}l@{\AgdaIndent{0}}]%
\>[10]\AgdaOperator{\AgdaFunction{⊡}}\AgdaSpace{}%
\AgdaSymbol{(}\AgdaFunction{flip}\AgdaSpace{}%
\AgdaOperator{\AgdaFunction{\$}}\AgdaSpace{}%
\AgdaFunction{cong₂}\AgdaSpace{}%
\AgdaOperator{\AgdaField{\AgdaUnderscore{}*\AgdaUnderscore{}}}\AgdaSymbol{)}\AgdaSpace{}%
\AgdaSymbol{(}\AgdaFunction{-ω-cong₂}\AgdaSpace{}%
\AgdaInductiveConstructor{refl}\AgdaSpace{}%
\AgdaOperator{\AgdaFunction{\$}}\AgdaSpace{}%
\AgdaFunction{sym}\AgdaSpace{}%
\AgdaOperator{\AgdaFunction{\$}}\AgdaSpace{}%
\AgdaFunction{*ₙ-assoc}\AgdaSpace{}%
\AgdaBound{r₂}\AgdaSpace{}%
\AgdaBound{r₁}\AgdaSpace{}%
\AgdaSymbol{(}\AgdaBound{j₁}\AgdaSpace{}%
\AgdaOperator{\AgdaPrimitive{*ₙ}}\AgdaSpace{}%
\AgdaBound{k₁}\AgdaSymbol{))}\<%
\\
%
\>[10]\AgdaSymbol{(}\AgdaSpace{}%
\AgdaSymbol{(}\AgdaField{ω-N-k₀+k₁}\AgdaSpace{}%
\AgdaSymbol{\{}\AgdaBound{r₂}\AgdaSpace{}%
\AgdaOperator{\AgdaPrimitive{*ₙ}}\AgdaSpace{}%
\AgdaBound{r₁}\AgdaSymbol{\}}\AgdaSpace{}%
\AgdaSymbol{\{}\AgdaBound{r₂}\AgdaSpace{}%
\AgdaOperator{\AgdaPrimitive{*ₙ}}\AgdaSpace{}%
\AgdaSymbol{(}\AgdaBound{k₁}\AgdaSpace{}%
\AgdaOperator{\AgdaPrimitive{*ₙ}}\AgdaSpace{}%
\AgdaBound{j₀}\AgdaSymbol{)}\AgdaSpace{}%
\AgdaOperator{\AgdaPrimitive{+ₙ}}\AgdaSpace{}%
\AgdaBound{k₀}\AgdaSpace{}%
\AgdaOperator{\AgdaPrimitive{*ₙ}}\AgdaSpace{}%
\AgdaBound{j₀}%
\>[83]\AgdaSymbol{\}}\AgdaSpace{}%
\AgdaSymbol{\{}\AgdaBound{r₁}\AgdaSpace{}%
\AgdaOperator{\AgdaPrimitive{*ₙ}}\AgdaSpace{}%
\AgdaSymbol{(}\AgdaBound{k₀}\AgdaSpace{}%
\AgdaOperator{\AgdaPrimitive{*ₙ}}\AgdaSpace{}%
\AgdaBound{j₁}\AgdaSymbol{)}%
\>[110]\AgdaSymbol{\}}\AgdaSpace{}%
\AgdaSymbol{)}\<%
\\
%
\>[10]\AgdaOperator{\AgdaFunction{⊡}}%
\>[1810I]\AgdaSymbol{(}\AgdaFunction{flip}\AgdaSpace{}%
\AgdaOperator{\AgdaFunction{\$}}\AgdaSpace{}%
\AgdaFunction{cong₂}\AgdaSpace{}%
\AgdaOperator{\AgdaField{\AgdaUnderscore{}*\AgdaUnderscore{}}}\AgdaSymbol{)}\AgdaSpace{}%
\AgdaInductiveConstructor{refl}\<%
\\
\>[.][@{}l@{}]\<[1810I]%
\>[12]\AgdaSymbol{(}\AgdaField{ω-N-k₀+k₁}\AgdaSpace{}%
\AgdaSymbol{\{}\AgdaBound{r₂}\AgdaSpace{}%
\AgdaOperator{\AgdaPrimitive{*ₙ}}\AgdaSpace{}%
\AgdaBound{r₁}\AgdaSymbol{\}}\AgdaSpace{}%
\AgdaSymbol{\{}\AgdaBound{r₂}\AgdaSpace{}%
\AgdaOperator{\AgdaPrimitive{*ₙ}}\AgdaSpace{}%
\AgdaSymbol{(}\AgdaBound{k₁}\AgdaSpace{}%
\AgdaOperator{\AgdaPrimitive{*ₙ}}\AgdaSpace{}%
\AgdaBound{j₀}\AgdaSymbol{)}%
\>[83]\AgdaSymbol{\}}\AgdaSpace{}%
\AgdaSymbol{\{}\AgdaBound{k₀}\AgdaSpace{}%
\AgdaOperator{\AgdaPrimitive{*ₙ}}\AgdaSpace{}%
\AgdaBound{j₀}%
\>[110]\AgdaSymbol{\}}\AgdaSpace{}%
\AgdaSymbol{)}\<%
\\
%
\>[10]\AgdaSymbol{)}\<%
\\
%
\\[\AgdaEmptyExtraSkip]%
%
\>[2]\AgdaFunction{cooley-tukey-derivation-application}\AgdaSpace{}%
\AgdaSymbol{:}\AgdaSpace{}%
\AgdaSymbol{∀}\<%
\\
\>[2][@{}l@{\AgdaIndent{0}}]%
\>[5]\AgdaSymbol{(}\AgdaBound{j₁}%
\>[11]\AgdaSymbol{:}\AgdaSpace{}%
\AgdaDatatype{Position}\AgdaSpace{}%
\AgdaSymbol{(}\AgdaGeneralizable{r₂}\AgdaSpace{}%
\AgdaOperator{\AgdaFunction{ᵗ}}\AgdaSymbol{))}\<%
\\
%
\>[5]\AgdaSymbol{(}\AgdaBound{j₀}%
\>[11]\AgdaSymbol{:}\AgdaSpace{}%
\AgdaDatatype{Position}\AgdaSpace{}%
\AgdaSymbol{(}\AgdaGeneralizable{r₁}\AgdaSpace{}%
\AgdaOperator{\AgdaFunction{ᵗ}}\AgdaSymbol{))}\<%
\\
%
\>[5]\AgdaSymbol{(}\AgdaBound{k₀}%
\>[11]\AgdaSymbol{:}\AgdaSpace{}%
\AgdaDatatype{Position}\AgdaSpace{}%
\AgdaSymbol{(}\AgdaInductiveConstructor{ι}\AgdaSpace{}%
\AgdaSymbol{(}\AgdaOperator{\AgdaFunction{\#}}\AgdaSpace{}%
\AgdaGeneralizable{r₂}\AgdaSpace{}%
\AgdaOperator{\AgdaFunction{ᵗ}}\AgdaSymbol{)))}\<%
\\
%
\>[5]\AgdaSymbol{(}\AgdaBound{k₁}%
\>[11]\AgdaSymbol{:}\AgdaSpace{}%
\AgdaDatatype{Position}\AgdaSpace{}%
\AgdaSymbol{(}\AgdaInductiveConstructor{ι}\AgdaSpace{}%
\AgdaSymbol{(}\AgdaOperator{\AgdaFunction{\#}}\AgdaSpace{}%
\AgdaGeneralizable{r₁}\AgdaSpace{}%
\AgdaOperator{\AgdaFunction{ᵗ}}\AgdaSymbol{)))}\<%
\\
%
\>[5]\AgdaSymbol{→}\AgdaSpace{}%
\AgdaSymbol{(}\AgdaBound{nz-r₁}\AgdaSpace{}%
\AgdaSymbol{:}\AgdaSpace{}%
\AgdaDatatype{NonZeroₛ}\AgdaSpace{}%
\AgdaGeneralizable{r₁}\AgdaSymbol{)}\<%
\\
%
\>[5]\AgdaSymbol{→}\AgdaSpace{}%
\AgdaSymbol{(}\AgdaBound{nz-r₂}\AgdaSpace{}%
\AgdaSymbol{:}\AgdaSpace{}%
\AgdaDatatype{NonZeroₛ}\AgdaSpace{}%
\AgdaGeneralizable{r₂}\AgdaSymbol{)}\<%
\\
%
\>[5]\AgdaSymbol{→}\<%
\\
\>[5][@{}l@{\AgdaIndent{0}}]%
\>[10]\AgdaField{-ω}\<%
\\
\>[10][@{}l@{\AgdaIndent{0}}]%
\>[12]\AgdaSymbol{(}\AgdaOperator{\AgdaFunction{\#}}\AgdaSpace{}%
\AgdaGeneralizable{r₁}\AgdaSpace{}%
\AgdaOperator{\AgdaFunction{ᵗ}}\AgdaSymbol{)}\<%
\\
%
\>[12]\AgdaSymbol{⦃}\AgdaSpace{}%
\AgdaFunction{nz-\#}\AgdaSpace{}%
\AgdaSymbol{(}\AgdaFunction{nzᵗ}\AgdaSpace{}%
\AgdaBound{nz-r₁}\AgdaSymbol{)}\AgdaSpace{}%
\AgdaSymbol{⦄}\<%
\\
%
\>[12]\AgdaSymbol{(}\AgdaFunction{iota}\AgdaSpace{}%
\AgdaBound{k₁}\AgdaSpace{}%
\AgdaOperator{\AgdaPrimitive{*ₙ}}\AgdaSpace{}%
\AgdaFunction{iota}\AgdaSpace{}%
\AgdaSymbol{(}\AgdaBound{j₀}\AgdaSpace{}%
\AgdaOperator{\AgdaFunction{⟨}}\AgdaSpace{}%
\AgdaFunction{rev}\AgdaSpace{}%
\AgdaFunction{♭}\AgdaSpace{}%
\AgdaOperator{\AgdaFunction{⟩}}\AgdaSymbol{))}\<%
\\
\>[5][@{}l@{\AgdaIndent{0}}]%
\>[8]\AgdaOperator{\AgdaField{*}}%
\>[1867I]\AgdaField{-ω}\<%
\\
\>[1867I][@{}l@{\AgdaIndent{0}}]%
\>[12]\AgdaSymbol{(}\AgdaOperator{\AgdaFunction{\#}}\AgdaSpace{}%
\AgdaGeneralizable{r₂}\AgdaSpace{}%
\AgdaOperator{\AgdaPrimitive{*ₙ}}\AgdaSpace{}%
\AgdaOperator{\AgdaFunction{\#}}\AgdaSpace{}%
\AgdaGeneralizable{r₁}\AgdaSpace{}%
\AgdaOperator{\AgdaFunction{ᵗ}}\AgdaSymbol{)}\<%
\\
%
\>[12]\AgdaSymbol{⦃}\AgdaSpace{}%
\AgdaFunction{m*n≢0}\AgdaSpace{}%
\AgdaSymbol{(}\AgdaOperator{\AgdaFunction{\#}}\AgdaSpace{}%
\AgdaGeneralizable{r₂}\AgdaSymbol{)}\AgdaSpace{}%
\AgdaSymbol{(}\AgdaOperator{\AgdaFunction{\#}}\AgdaSpace{}%
\AgdaGeneralizable{r₁}\AgdaSpace{}%
\AgdaOperator{\AgdaFunction{ᵗ}}\AgdaSymbol{)}\AgdaSpace{}%
\AgdaSymbol{⦃}\AgdaSpace{}%
\AgdaFunction{nz-\#}\AgdaSpace{}%
\AgdaBound{nz-r₂}\AgdaSpace{}%
\AgdaSymbol{⦄}\AgdaSpace{}%
\AgdaSymbol{⦃}\AgdaSpace{}%
\AgdaFunction{nz-\#}\AgdaSpace{}%
\AgdaSymbol{(}\AgdaFunction{nzᵗ}\AgdaSpace{}%
\AgdaBound{nz-r₁}\AgdaSymbol{)}\AgdaSpace{}%
\AgdaSymbol{⦄}\AgdaSpace{}%
\AgdaSymbol{⦄}\<%
\\
%
\>[12]\AgdaSymbol{(}\AgdaFunction{iota}\AgdaSpace{}%
\AgdaSymbol{(((}\AgdaBound{k₀}\AgdaSpace{}%
\AgdaOperator{\AgdaFunction{⟨}}\AgdaSpace{}%
\AgdaFunction{reindex}\AgdaSpace{}%
\AgdaSymbol{(}\AgdaFunction{|s|≡|sᵗ|}\AgdaSpace{}%
\AgdaSymbol{\{}\AgdaGeneralizable{r₂}\AgdaSymbol{\})}\AgdaSpace{}%
\AgdaOperator{\AgdaFunction{⟩}}\AgdaSymbol{)}\AgdaSpace{}%
\AgdaOperator{\AgdaFunction{⟨}}\AgdaSpace{}%
\AgdaFunction{♭}\AgdaSpace{}%
\AgdaOperator{\AgdaFunction{⟩}}\AgdaSymbol{)}\AgdaSpace{}%
\AgdaOperator{\AgdaFunction{⟨}}\AgdaSpace{}%
\AgdaFunction{rev}\AgdaSpace{}%
\AgdaSymbol{(}\AgdaFunction{♭}\AgdaSpace{}%
\AgdaSymbol{\{}\AgdaGeneralizable{r₂}\AgdaSymbol{\})}\AgdaSpace{}%
\AgdaOperator{\AgdaFunction{⟩}}\AgdaSymbol{)}\AgdaSpace{}%
\AgdaOperator{\AgdaPrimitive{*ₙ}}\AgdaSpace{}%
\AgdaFunction{iota}\AgdaSpace{}%
\AgdaSymbol{(}\AgdaBound{j₀}\AgdaSpace{}%
\AgdaOperator{\AgdaFunction{⟨}}\AgdaSpace{}%
\AgdaFunction{rev}\AgdaSpace{}%
\AgdaFunction{♭}\AgdaSpace{}%
\AgdaOperator{\AgdaFunction{⟩}}\AgdaSymbol{))}\<%
\\
%
\>[8]\AgdaOperator{\AgdaField{*}}%
\>[1910I]\AgdaField{-ω}\<%
\\
\>[1910I][@{}l@{\AgdaIndent{0}}]%
\>[12]\AgdaSymbol{(}\AgdaOperator{\AgdaFunction{\#}}\AgdaSpace{}%
\AgdaGeneralizable{r₂}\AgdaSpace{}%
\AgdaOperator{\AgdaFunction{ᵗ}}\AgdaSymbol{)}\<%
\\
%
\>[12]\AgdaSymbol{⦃}\AgdaSpace{}%
\AgdaFunction{nz-\#}\AgdaSpace{}%
\AgdaSymbol{(}\AgdaFunction{nzᵗ}\AgdaSpace{}%
\AgdaBound{nz-r₂}\AgdaSymbol{)}\AgdaSpace{}%
\AgdaSymbol{⦄}\<%
\\
%
\>[12]\AgdaSymbol{(}\AgdaFunction{iota}\AgdaSpace{}%
\AgdaBound{k₀}\AgdaSpace{}%
\AgdaOperator{\AgdaPrimitive{*ₙ}}\AgdaSpace{}%
\AgdaFunction{iota}\AgdaSpace{}%
\AgdaSymbol{(}\AgdaBound{j₁}\AgdaSpace{}%
\AgdaOperator{\AgdaFunction{⟨}}\AgdaSpace{}%
\AgdaFunction{rev}\AgdaSpace{}%
\AgdaFunction{♭}\AgdaSpace{}%
\AgdaOperator{\AgdaFunction{⟩}}\AgdaSymbol{))}\<%
\\
%
\>[8]\AgdaOperator{\AgdaDatatype{≡}}\<%
\\
\>[8][@{}l@{\AgdaIndent{0}}]%
\>[10]\AgdaField{-ω}\<%
\\
\>[10][@{}l@{\AgdaIndent{0}}]%
\>[12]\AgdaSymbol{(}\AgdaOperator{\AgdaFunction{\#}}\AgdaSpace{}%
\AgdaGeneralizable{r₂}\AgdaSpace{}%
\AgdaOperator{\AgdaFunction{ᵗ}}\AgdaSpace{}%
\AgdaOperator{\AgdaPrimitive{*ₙ}}\AgdaSpace{}%
\AgdaOperator{\AgdaFunction{\#}}\AgdaSpace{}%
\AgdaGeneralizable{r₁}\AgdaSpace{}%
\AgdaOperator{\AgdaFunction{ᵗ}}\AgdaSymbol{)}\<%
\\
%
\>[12]\AgdaSymbol{⦃}\AgdaSpace{}%
\AgdaFunction{m*n≢0}\AgdaSpace{}%
\AgdaSymbol{(}\AgdaOperator{\AgdaFunction{\#}}\AgdaSpace{}%
\AgdaGeneralizable{r₂}\AgdaSpace{}%
\AgdaOperator{\AgdaFunction{ᵗ}}\AgdaSymbol{)}\AgdaSpace{}%
\AgdaSymbol{(}\AgdaOperator{\AgdaFunction{\#}}\AgdaSpace{}%
\AgdaGeneralizable{r₁}\AgdaSpace{}%
\AgdaOperator{\AgdaFunction{ᵗ}}\AgdaSymbol{)}\AgdaSpace{}%
\AgdaSymbol{⦃}\AgdaSpace{}%
\AgdaFunction{nz-\#}\AgdaSpace{}%
\AgdaSymbol{(}\AgdaFunction{nzᵗ}\AgdaSpace{}%
\AgdaBound{nz-r₂}\AgdaSymbol{)}\AgdaSpace{}%
\AgdaSymbol{⦄}\AgdaSpace{}%
\AgdaSymbol{⦃}\AgdaSpace{}%
\AgdaFunction{nz-\#}\AgdaSpace{}%
\AgdaSymbol{(}\AgdaFunction{nzᵗ}\AgdaSpace{}%
\AgdaBound{nz-r₁}\AgdaSymbol{)}\AgdaSpace{}%
\AgdaSymbol{⦄}\AgdaSpace{}%
\AgdaSymbol{⦄}\<%
\\
%
\>[12]\AgdaSymbol{(}\AgdaFunction{iota}\AgdaSpace{}%
\AgdaSymbol{(((}\AgdaBound{k₁}\AgdaSpace{}%
\AgdaOperator{\AgdaFunction{⟨}}\AgdaSpace{}%
\AgdaFunction{reindex}\AgdaSpace{}%
\AgdaSymbol{(}\AgdaFunction{|s|≡|sᵗ|}\AgdaSpace{}%
\AgdaSymbol{\{}\AgdaGeneralizable{r₁}\AgdaSymbol{\})}\AgdaSpace{}%
\AgdaOperator{\AgdaFunction{⟩}}\AgdaSymbol{)}\AgdaSpace{}%
\AgdaOperator{\AgdaInductiveConstructor{⊗}}\AgdaSpace{}%
\AgdaSymbol{(}\AgdaBound{k₀}\AgdaSpace{}%
\AgdaOperator{\AgdaFunction{⟨}}\AgdaSpace{}%
\AgdaFunction{reindex}\AgdaSpace{}%
\AgdaSymbol{(}\AgdaFunction{|s|≡|sᵗ|}\AgdaSpace{}%
\AgdaSymbol{\{}\AgdaGeneralizable{r₂}\AgdaSymbol{\})}\AgdaSpace{}%
\AgdaOperator{\AgdaFunction{⟩}}\AgdaSymbol{))}\AgdaSpace{}%
\AgdaOperator{\AgdaFunction{⟨}}\AgdaSpace{}%
\AgdaInductiveConstructor{split}\AgdaSpace{}%
\AgdaOperator{\AgdaFunction{⟩}}\AgdaSymbol{)}\AgdaSpace{}%
\AgdaOperator{\AgdaPrimitive{*ₙ}}\AgdaSpace{}%
\AgdaFunction{iota}\AgdaSpace{}%
\AgdaSymbol{(((}\AgdaBound{j₁}\AgdaSpace{}%
\AgdaOperator{\AgdaFunction{⟨}}\AgdaSpace{}%
\AgdaFunction{rev}\AgdaSpace{}%
\AgdaFunction{♭}\AgdaSpace{}%
\AgdaOperator{\AgdaFunction{⟩}}\AgdaSymbol{)}\AgdaSpace{}%
\AgdaOperator{\AgdaInductiveConstructor{⊗}}\AgdaSpace{}%
\AgdaSymbol{(}\AgdaBound{j₀}\AgdaSpace{}%
\AgdaOperator{\AgdaFunction{⟨}}\AgdaSpace{}%
\AgdaFunction{rev}\AgdaSpace{}%
\AgdaFunction{♭}\AgdaSpace{}%
\AgdaOperator{\AgdaFunction{⟩}}\AgdaSymbol{))}\AgdaSpace{}%
\AgdaOperator{\AgdaFunction{⟨}}\AgdaSpace{}%
\AgdaInductiveConstructor{split}\AgdaSpace{}%
\AgdaOperator{\AgdaFunction{⟩}}\AgdaSymbol{))}\<%
\\
%
\>[2]\AgdaFunction{cooley-tukey-derivation-application}\AgdaSpace{}%
\AgdaSymbol{\{}\AgdaBound{r₂}\AgdaSymbol{\}}\AgdaSpace{}%
\AgdaSymbol{\{}\AgdaBound{r₁}\AgdaSymbol{\}}\AgdaSpace{}%
\AgdaBound{j₁}\AgdaSpace{}%
\AgdaBound{j₀}\AgdaSpace{}%
\AgdaBound{k₀}\AgdaSpace{}%
\AgdaBound{k₁}\AgdaSpace{}%
\AgdaBound{nz-r₁}\AgdaSpace{}%
\AgdaBound{nz-r₂}\AgdaSpace{}%
\AgdaSymbol{=}\<%
\\
\>[2][@{}l@{\AgdaIndent{0}}]%
\>[4]\AgdaKeyword{let}\AgdaSpace{}%
\AgdaKeyword{instance}\<%
\\
\>[4][@{}l@{\AgdaIndent{0}}]%
\>[6]\AgdaBound{\AgdaUnderscore{}}\AgdaSpace{}%
\AgdaSymbol{:}\AgdaSpace{}%
\AgdaRecord{NonZero}\AgdaSpace{}%
\AgdaSymbol{(}\AgdaOperator{\AgdaFunction{\#}}\AgdaSpace{}%
\AgdaBound{r₁}\AgdaSymbol{)}\<%
\\
%
\>[6]\AgdaSymbol{\AgdaUnderscore{}}\AgdaSpace{}%
\AgdaSymbol{=}\AgdaSpace{}%
\AgdaFunction{nz-\#}\AgdaSpace{}%
\AgdaBound{nz-r₁}\<%
\\
%
\>[6]\AgdaBound{\AgdaUnderscore{}}\AgdaSpace{}%
\AgdaSymbol{:}\AgdaSpace{}%
\AgdaRecord{NonZero}\AgdaSpace{}%
\AgdaSymbol{(}\AgdaOperator{\AgdaFunction{\#}}\AgdaSpace{}%
\AgdaBound{r₂}\AgdaSymbol{)}\<%
\\
%
\>[6]\AgdaSymbol{\AgdaUnderscore{}}\AgdaSpace{}%
\AgdaSymbol{=}\AgdaSpace{}%
\AgdaFunction{nz-\#}\AgdaSpace{}%
\AgdaBound{nz-r₂}\<%
\\
%
\>[6]\AgdaBound{\AgdaUnderscore{}}\AgdaSpace{}%
\AgdaSymbol{:}\AgdaSpace{}%
\AgdaRecord{NonZero}\AgdaSpace{}%
\AgdaSymbol{(}\AgdaOperator{\AgdaFunction{\#}}\AgdaSpace{}%
\AgdaBound{r₁}\AgdaSpace{}%
\AgdaOperator{\AgdaFunction{ᵗ}}\AgdaSymbol{)}\<%
\\
%
\>[6]\AgdaSymbol{\AgdaUnderscore{}}\AgdaSpace{}%
\AgdaSymbol{=}\AgdaSpace{}%
\AgdaFunction{nz-\#}\AgdaSpace{}%
\AgdaSymbol{(}\AgdaFunction{nzᵗ}\AgdaSpace{}%
\AgdaBound{nz-r₁}\AgdaSymbol{)}\<%
\\
%
\>[6]\AgdaBound{\AgdaUnderscore{}}\AgdaSpace{}%
\AgdaSymbol{:}\AgdaSpace{}%
\AgdaRecord{NonZero}\AgdaSpace{}%
\AgdaSymbol{(}\AgdaOperator{\AgdaFunction{\#}}\AgdaSpace{}%
\AgdaBound{r₂}\AgdaSpace{}%
\AgdaOperator{\AgdaFunction{ᵗ}}\AgdaSymbol{)}\<%
\\
%
\>[6]\AgdaSymbol{\AgdaUnderscore{}}\AgdaSpace{}%
\AgdaSymbol{=}\AgdaSpace{}%
\AgdaFunction{nz-\#}\AgdaSpace{}%
\AgdaSymbol{(}\AgdaFunction{nzᵗ}\AgdaSpace{}%
\AgdaBound{nz-r₂}\AgdaSymbol{)}\<%
\\
%
\>[6]\AgdaBound{\AgdaUnderscore{}}\AgdaSpace{}%
\AgdaSymbol{:}\AgdaSpace{}%
\AgdaRecord{NonZero}\AgdaSpace{}%
\AgdaSymbol{(}\AgdaOperator{\AgdaFunction{\#}}\AgdaSpace{}%
\AgdaBound{r₂}\AgdaSpace{}%
\AgdaOperator{\AgdaPrimitive{*ₙ}}\AgdaSpace{}%
\AgdaOperator{\AgdaFunction{\#}}\AgdaSpace{}%
\AgdaBound{r₁}\AgdaSpace{}%
\AgdaOperator{\AgdaFunction{ᵗ}}\AgdaSymbol{)}\<%
\\
%
\>[6]\AgdaSymbol{\AgdaUnderscore{}}\AgdaSpace{}%
\AgdaSymbol{=}\AgdaSpace{}%
\AgdaFunction{m*n≢0}\AgdaSpace{}%
\AgdaSymbol{(}\AgdaOperator{\AgdaFunction{\#}}\AgdaSpace{}%
\AgdaBound{r₂}\AgdaSymbol{)}\AgdaSpace{}%
\AgdaSymbol{(}\AgdaOperator{\AgdaFunction{\#}}\AgdaSpace{}%
\AgdaBound{r₁}\AgdaSpace{}%
\AgdaOperator{\AgdaFunction{ᵗ}}\AgdaSymbol{)}\<%
\\
%
\>[6]\AgdaBound{\AgdaUnderscore{}}\AgdaSpace{}%
\AgdaSymbol{:}\AgdaSpace{}%
\AgdaRecord{NonZero}\AgdaSpace{}%
\AgdaSymbol{(}\AgdaOperator{\AgdaFunction{\#}}\AgdaSpace{}%
\AgdaBound{r₂}\AgdaSpace{}%
\AgdaOperator{\AgdaFunction{ᵗ}}\AgdaSpace{}%
\AgdaOperator{\AgdaPrimitive{*ₙ}}\AgdaSpace{}%
\AgdaOperator{\AgdaFunction{\#}}\AgdaSpace{}%
\AgdaBound{r₁}\AgdaSpace{}%
\AgdaOperator{\AgdaFunction{ᵗ}}\AgdaSymbol{)}\<%
\\
%
\>[6]\AgdaSymbol{\AgdaUnderscore{}}\AgdaSpace{}%
\AgdaSymbol{=}\AgdaSpace{}%
\AgdaFunction{m*n≢0}\AgdaSpace{}%
\AgdaSymbol{(}\AgdaOperator{\AgdaFunction{\#}}\AgdaSpace{}%
\AgdaBound{r₂}\AgdaSpace{}%
\AgdaOperator{\AgdaFunction{ᵗ}}\AgdaSymbol{)}\AgdaSpace{}%
\AgdaSymbol{(}\AgdaOperator{\AgdaFunction{\#}}\AgdaSpace{}%
\AgdaBound{r₁}\AgdaSpace{}%
\AgdaOperator{\AgdaFunction{ᵗ}}\AgdaSymbol{)}\<%
\\
%
\>[4]\AgdaKeyword{in}\AgdaSpace{}%
\AgdaOperator{\AgdaFunction{begin}}\<%
\\
%
\>[4]\AgdaSymbol{\AgdaUnderscore{}}%
\>[2056I]\AgdaFunction{≡⟨}%
\>[10]\AgdaFunction{cong₂}\AgdaSpace{}%
\AgdaOperator{\AgdaField{\AgdaUnderscore{}*\AgdaUnderscore{}}}\<%
\\
\>[10][@{}l@{\AgdaIndent{0}}]%
\>[12]\AgdaSymbol{(}%
\>[2058I]\AgdaFunction{cong₂}\AgdaSpace{}%
\AgdaOperator{\AgdaField{\AgdaUnderscore{}*\AgdaUnderscore{}}}\<%
\\
\>[2058I][@{}l@{\AgdaIndent{0}}]%
\>[16]\AgdaInductiveConstructor{refl}\<%
\\
%
\>[16]\AgdaSymbol{(}%
\>[2060I]\AgdaFunction{-ω-cong₂}\<%
\\
\>[2060I][@{}l@{\AgdaIndent{0}}]%
\>[20]\AgdaSymbol{(}\AgdaSpace{}%
\AgdaFunction{cong₂}\AgdaSpace{}%
\AgdaOperator{\AgdaPrimitive{\AgdaUnderscore{}*ₙ\AgdaUnderscore{}}}\AgdaSpace{}%
\AgdaSymbol{(}\AgdaFunction{|s|≡|sᵗ|}\AgdaSpace{}%
\AgdaSymbol{\{}\AgdaBound{r₂}\AgdaSymbol{\})}\AgdaSpace{}%
\AgdaInductiveConstructor{refl}\AgdaSymbol{)}\<%
\\
%
\>[20]\AgdaSymbol{(}%
\>[2066I]\AgdaFunction{cong₂}\AgdaSpace{}%
\AgdaOperator{\AgdaPrimitive{\AgdaUnderscore{}*ₙ\AgdaUnderscore{}}}\<%
\\
\>[2066I][@{}l@{\AgdaIndent{0}}]%
\>[24]\AgdaSymbol{(}%
\>[2068I]\AgdaSymbol{(}\AgdaFunction{cong}\<%
\\
\>[2068I][@{}l@{\AgdaIndent{0}}]%
\>[30]\AgdaFunction{iota}\<%
\\
%
\>[30]\AgdaSymbol{(}\AgdaFunction{rev-eq}\AgdaSpace{}%
\AgdaSymbol{\{}\AgdaBound{r₂}\AgdaSymbol{\}}\AgdaSpace{}%
\AgdaFunction{♭}\AgdaSpace{}%
\AgdaSymbol{(}\AgdaBound{k₀}\AgdaSpace{}%
\AgdaOperator{\AgdaFunction{⟨}}\AgdaSpace{}%
\AgdaFunction{reindex}\AgdaSpace{}%
\AgdaSymbol{(}\AgdaFunction{|s|≡|sᵗ|}\AgdaSpace{}%
\AgdaSymbol{\{}\AgdaBound{r₂}\AgdaSymbol{\})}\AgdaSpace{}%
\AgdaOperator{\AgdaFunction{⟩}}\AgdaSymbol{))}\<%
\\
\>[.][@{}l@{}]\<[2068I]%
\>[26]\AgdaSymbol{)}\<%
\\
%
\>[24]\AgdaOperator{\AgdaFunction{⊡}}\AgdaSpace{}%
\AgdaSymbol{(}\AgdaFunction{iota-reindex}\AgdaSpace{}%
\AgdaSymbol{(}\AgdaFunction{|s|≡|sᵗ|}\AgdaSpace{}%
\AgdaSymbol{\{}\AgdaBound{r₂}\AgdaSymbol{\}))}\<%
\\
%
\>[24]\AgdaSymbol{)}\<%
\\
%
\>[24]\AgdaInductiveConstructor{refl}\<%
\\
%
\>[20]\AgdaSymbol{)}\<%
\\
%
\>[16]\AgdaSymbol{)}\<%
\\
%
\>[12]\AgdaSymbol{)}\<%
\\
%
\>[12]\AgdaInductiveConstructor{refl}\<%
\\
\>[.][@{}l@{}]\<[2056I]%
\>[6]\AgdaFunction{⟩}\<%
\\
%
\>[4]\AgdaSymbol{\AgdaUnderscore{}}%
\>[2080I]\AgdaFunction{≡⟨}%
\>[2081I]\AgdaFunction{sym}\AgdaSpace{}%
\AgdaSymbol{(}\AgdaFunction{cooley-tukey-derivation}\<%
\\
\>[2081I][@{}l@{\AgdaIndent{0}}]%
\>[10]\AgdaSymbol{(}\AgdaOperator{\AgdaFunction{\#}}\AgdaSpace{}%
\AgdaBound{r₁}\AgdaSpace{}%
\AgdaOperator{\AgdaFunction{ᵗ}}\AgdaSymbol{)}\<%
\\
%
\>[10]\AgdaSymbol{(}\AgdaOperator{\AgdaFunction{\#}}\AgdaSpace{}%
\AgdaBound{r₂}\AgdaSpace{}%
\AgdaOperator{\AgdaFunction{ᵗ}}\AgdaSymbol{)}\<%
\\
%
\>[10]\AgdaSymbol{(}\AgdaFunction{iota}\AgdaSpace{}%
\AgdaBound{k₀}\AgdaSymbol{)}\<%
\\
%
\>[10]\AgdaSymbol{(}\AgdaFunction{iota}\AgdaSpace{}%
\AgdaBound{k₁}\AgdaSymbol{)}\<%
\\
%
\>[10]\AgdaSymbol{(}\AgdaFunction{iota}\AgdaSpace{}%
\AgdaSymbol{(}\AgdaBound{j₀}\AgdaSpace{}%
\AgdaOperator{\AgdaFunction{⟨}}\AgdaSpace{}%
\AgdaFunction{rev}\AgdaSpace{}%
\AgdaFunction{♭}\AgdaSpace{}%
\AgdaOperator{\AgdaFunction{⟩}}\AgdaSymbol{))}\<%
\\
%
\>[10]\AgdaSymbol{(}\AgdaFunction{iota}\AgdaSpace{}%
\AgdaSymbol{(}\AgdaBound{j₁}\AgdaSpace{}%
\AgdaOperator{\AgdaFunction{⟨}}\AgdaSpace{}%
\AgdaFunction{rev}\AgdaSpace{}%
\AgdaFunction{♭}\AgdaSpace{}%
\AgdaOperator{\AgdaFunction{⟩}}\AgdaSymbol{))}\<%
\\
\>[.][@{}l@{}]\<[2081I]%
\>[9]\AgdaSymbol{)}\<%
\\
\>[2080I][@{}l@{\AgdaIndent{0}}]%
\>[7]\AgdaFunction{⟩}\<%
\\
%
\>[4]\AgdaSymbol{\AgdaUnderscore{}}\AgdaSpace{}%
\AgdaFunction{≡⟨}%
\>[2100I]\AgdaFunction{sym}\AgdaSpace{}%
\AgdaSymbol{(}\AgdaFunction{-ω-cong₂}\AgdaSpace{}%
\AgdaInductiveConstructor{refl}\<%
\\
\>[2100I][@{}l@{\AgdaIndent{0}}]%
\>[12]\AgdaSymbol{(}\AgdaFunction{cong}\AgdaSpace{}%
\AgdaSymbol{(}\AgdaOperator{\AgdaPrimitive{\AgdaUnderscore{}*ₙ}}\AgdaSpace{}%
\AgdaSymbol{(}\AgdaOperator{\AgdaFunction{\#}}\AgdaSpace{}%
\AgdaBound{r₁}\AgdaSpace{}%
\AgdaOperator{\AgdaFunction{ᵗ}}\AgdaSpace{}%
\AgdaOperator{\AgdaPrimitive{*ₙ}}\AgdaSpace{}%
\AgdaFunction{iota}\AgdaSpace{}%
\AgdaSymbol{(}\AgdaBound{j₁}\AgdaSpace{}%
\AgdaOperator{\AgdaFunction{⟨}}\AgdaSpace{}%
\AgdaFunction{rev}\AgdaSpace{}%
\AgdaFunction{♭}\AgdaSpace{}%
\AgdaOperator{\AgdaFunction{⟩}}\AgdaSymbol{)}\AgdaSpace{}%
\AgdaOperator{\AgdaPrimitive{+ₙ}}\AgdaSpace{}%
\AgdaFunction{iota}\AgdaSpace{}%
\AgdaSymbol{(}\AgdaBound{j₀}\AgdaSpace{}%
\AgdaOperator{\AgdaFunction{⟨}}\AgdaSpace{}%
\AgdaFunction{rev}\AgdaSpace{}%
\AgdaFunction{♭}\AgdaSpace{}%
\AgdaOperator{\AgdaFunction{⟩}}\AgdaSymbol{)))}\<%
\\
\>[12][@{}l@{\AgdaIndent{0}}]%
\>[14]\AgdaSymbol{(}\AgdaFunction{cong}\AgdaSpace{}%
\AgdaSymbol{((}\AgdaOperator{\AgdaFunction{\#}}\AgdaSpace{}%
\AgdaBound{r₂}\AgdaSpace{}%
\AgdaOperator{\AgdaFunction{ᵗ}}\AgdaSpace{}%
\AgdaOperator{\AgdaPrimitive{*ₙ}}\AgdaSpace{}%
\AgdaFunction{iota}\AgdaSpace{}%
\AgdaBound{k₁}\AgdaSpace{}%
\AgdaOperator{\AgdaPrimitive{+ₙ\AgdaUnderscore{}}}\AgdaSymbol{))}\<%
\\
\>[14][@{}l@{\AgdaIndent{0}}]%
\>[16]\AgdaSymbol{(}\AgdaFunction{iota-reindex}\AgdaSpace{}%
\AgdaSymbol{(}\AgdaFunction{|s|≡|sᵗ|}\AgdaSpace{}%
\AgdaSymbol{\{}\AgdaBound{r₂}\AgdaSymbol{\}))}\<%
\\
%
\>[14]\AgdaSymbol{)}\<%
\\
%
\>[12]\AgdaSymbol{)}\<%
\\
\>[2100I][@{}l@{\AgdaIndent{0}}]%
\>[11]\AgdaSymbol{)}\<%
\\
\>[4][@{}l@{\AgdaIndent{0}}]%
\>[5]\AgdaFunction{⟩}\<%
\\
%
\>[4]\AgdaSymbol{\AgdaUnderscore{}}\AgdaSpace{}%
\AgdaFunction{≡⟨}%
\>[2131I]\AgdaFunction{sym}\AgdaSpace{}%
\AgdaSymbol{(}\AgdaFunction{-ω-cong₂}\AgdaSpace{}%
\AgdaInductiveConstructor{refl}\<%
\\
\>[2131I][@{}l@{\AgdaIndent{0}}]%
\>[12]\AgdaSymbol{(}\AgdaFunction{cong}\AgdaSpace{}%
\AgdaSymbol{(}\AgdaOperator{\AgdaPrimitive{\AgdaUnderscore{}*ₙ}}\AgdaSpace{}%
\AgdaSymbol{(}\AgdaOperator{\AgdaFunction{\#}}\AgdaSpace{}%
\AgdaBound{r₁}\AgdaSpace{}%
\AgdaOperator{\AgdaFunction{ᵗ}}\AgdaSpace{}%
\AgdaOperator{\AgdaPrimitive{*ₙ}}\AgdaSpace{}%
\AgdaFunction{iota}\AgdaSpace{}%
\AgdaSymbol{(}\AgdaBound{j₁}\AgdaSpace{}%
\AgdaOperator{\AgdaFunction{⟨}}\AgdaSpace{}%
\AgdaFunction{rev}\AgdaSpace{}%
\AgdaFunction{♭}\AgdaSpace{}%
\AgdaOperator{\AgdaFunction{⟩}}\AgdaSymbol{)}\AgdaSpace{}%
\AgdaOperator{\AgdaPrimitive{+ₙ}}\AgdaSpace{}%
\AgdaFunction{iota}\AgdaSpace{}%
\AgdaSymbol{(}\AgdaBound{j₀}\AgdaSpace{}%
\AgdaOperator{\AgdaFunction{⟨}}\AgdaSpace{}%
\AgdaFunction{rev}\AgdaSpace{}%
\AgdaFunction{♭}\AgdaSpace{}%
\AgdaOperator{\AgdaFunction{⟩}}\AgdaSymbol{)))}\<%
\\
\>[12][@{}l@{\AgdaIndent{0}}]%
\>[14]\AgdaSymbol{(}\AgdaFunction{cong}\AgdaSpace{}%
\AgdaSymbol{(}\AgdaOperator{\AgdaPrimitive{\AgdaUnderscore{}+ₙ}}\AgdaSpace{}%
\AgdaFunction{iota}\AgdaSpace{}%
\AgdaSymbol{(}\AgdaBound{k₀}\AgdaSpace{}%
\AgdaOperator{\AgdaFunction{⟨}}\AgdaSpace{}%
\AgdaFunction{reindex}\AgdaSpace{}%
\AgdaSymbol{(}\AgdaFunction{|s|≡|sᵗ|}\AgdaSpace{}%
\AgdaSymbol{\{}\AgdaBound{r₂}\AgdaSymbol{\})}\AgdaSpace{}%
\AgdaOperator{\AgdaFunction{⟩}}\AgdaSymbol{))}\<%
\\
\>[14][@{}l@{\AgdaIndent{0}}]%
\>[16]\AgdaSymbol{(}\AgdaFunction{cong}\AgdaSpace{}%
\AgdaSymbol{(}\AgdaOperator{\AgdaFunction{\#}}\AgdaSpace{}%
\AgdaBound{r₂}\AgdaSpace{}%
\AgdaOperator{\AgdaFunction{ᵗ}}\AgdaSpace{}%
\AgdaOperator{\AgdaPrimitive{*ₙ\AgdaUnderscore{}}}\AgdaSymbol{)}\AgdaSpace{}%
\AgdaSymbol{(}\AgdaFunction{iota-reindex}\AgdaSpace{}%
\AgdaSymbol{\{}\AgdaOperator{\AgdaFunction{\#}}\AgdaSpace{}%
\AgdaBound{r₁}\AgdaSpace{}%
\AgdaOperator{\AgdaFunction{ᵗ}}\AgdaSymbol{\}}\AgdaSpace{}%
\AgdaSymbol{\{}\AgdaOperator{\AgdaFunction{\#}}\AgdaSpace{}%
\AgdaBound{r₁}\AgdaSymbol{\}}\AgdaSpace{}%
\AgdaSymbol{\{}\AgdaBound{k₁}\AgdaSymbol{\}}\AgdaSpace{}%
\AgdaSymbol{(}\AgdaFunction{|s|≡|sᵗ|}\AgdaSpace{}%
\AgdaSymbol{\{}\AgdaBound{r₁}\AgdaSymbol{\})))}\<%
\\
%
\>[14]\AgdaSymbol{)}\<%
\\
%
\>[12]\AgdaSymbol{)}\<%
\\
\>[2131I][@{}l@{\AgdaIndent{0}}]%
\>[11]\AgdaSymbol{)}\<%
\\
\>[4][@{}l@{\AgdaIndent{0}}]%
\>[5]\AgdaFunction{⟩}\<%
\\
%
\>[4]\AgdaSymbol{\AgdaUnderscore{}}\AgdaSpace{}%
\AgdaFunction{≡⟨}\AgdaSpace{}%
\AgdaFunction{sym}\AgdaSpace{}%
\AgdaSymbol{(}\AgdaFunction{-ω-cong₂}\AgdaSpace{}%
\AgdaInductiveConstructor{refl}\AgdaSpace{}%
\AgdaSymbol{(}\AgdaFunction{cong₂}\AgdaSpace{}%
\AgdaOperator{\AgdaPrimitive{\AgdaUnderscore{}*ₙ\AgdaUnderscore{}}}\AgdaSpace{}%
\AgdaSymbol{(}\AgdaFunction{cong₂}\AgdaSpace{}%
\AgdaOperator{\AgdaPrimitive{\AgdaUnderscore{}+ₙ\AgdaUnderscore{}}}\AgdaSpace{}%
\AgdaSymbol{(}\AgdaFunction{cong₂}\AgdaSpace{}%
\AgdaOperator{\AgdaPrimitive{\AgdaUnderscore{}*ₙ\AgdaUnderscore{}}}\AgdaSpace{}%
\AgdaSymbol{(}\AgdaFunction{|s|≡|sᵗ|}\AgdaSpace{}%
\AgdaSymbol{\{}\AgdaBound{r₂}\AgdaSymbol{\})}\AgdaSpace{}%
\AgdaInductiveConstructor{refl}\AgdaSymbol{)}\AgdaSpace{}%
\AgdaInductiveConstructor{refl}\AgdaSymbol{)}\AgdaSpace{}%
\AgdaInductiveConstructor{refl}\AgdaSymbol{))}\AgdaSpace{}%
\AgdaFunction{⟩}\<%
\\
%
\>[4]\AgdaSymbol{\AgdaUnderscore{}}%
\>[2189I]\AgdaFunction{≡⟨}\AgdaSpace{}%
\AgdaFunction{sym}%
\>[2191I]\AgdaSymbol{(}\AgdaFunction{-ω-cong₂}\AgdaSpace{}%
\AgdaInductiveConstructor{refl}\<%
\\
\>[2191I][@{}l@{\AgdaIndent{0}}]%
\>[16]\AgdaSymbol{(}\AgdaFunction{cong₂}\AgdaSpace{}%
\AgdaOperator{\AgdaPrimitive{\AgdaUnderscore{}*ₙ\AgdaUnderscore{}}}\<%
\\
\>[16][@{}l@{\AgdaIndent{0}}]%
\>[20]\AgdaSymbol{(}\AgdaFunction{iota-split}\<%
\\
\>[20][@{}l@{\AgdaIndent{0}}]%
\>[22]\AgdaSymbol{\{}\AgdaBound{r₂}\AgdaSymbol{\}}\<%
\\
%
\>[22]\AgdaSymbol{\{}\AgdaBound{r₁}\AgdaSymbol{\}}\<%
\\
%
\>[22]\AgdaSymbol{(}\AgdaBound{k₀}\AgdaSpace{}%
\AgdaOperator{\AgdaFunction{⟨}}\AgdaSpace{}%
\AgdaFunction{reindex}\AgdaSpace{}%
\AgdaSymbol{(}\AgdaFunction{|s|≡|sᵗ|}\AgdaSpace{}%
\AgdaSymbol{\{}\AgdaBound{r₂}\AgdaSymbol{\})}\AgdaSpace{}%
\AgdaOperator{\AgdaFunction{⟩}}\AgdaSymbol{)}\<%
\\
%
\>[22]\AgdaSymbol{(}\AgdaBound{k₁}\AgdaSpace{}%
\AgdaOperator{\AgdaFunction{⟨}}\AgdaSpace{}%
\AgdaFunction{reindex}\AgdaSpace{}%
\AgdaSymbol{(}\AgdaFunction{|s|≡|sᵗ|}\AgdaSpace{}%
\AgdaSymbol{\{}\AgdaBound{r₁}\AgdaSymbol{\})}\AgdaSpace{}%
\AgdaOperator{\AgdaFunction{⟩}}\AgdaSymbol{)}\<%
\\
%
\>[20]\AgdaSymbol{)}\<%
\\
%
\>[20]\AgdaSymbol{(}\AgdaFunction{iota-split}\<%
\\
\>[20][@{}l@{\AgdaIndent{0}}]%
\>[22]\AgdaSymbol{\{}\AgdaInductiveConstructor{ι}\AgdaSpace{}%
\AgdaSymbol{(}\AgdaOperator{\AgdaFunction{\#}}\AgdaSpace{}%
\AgdaBound{r₁}\AgdaSpace{}%
\AgdaOperator{\AgdaFunction{ᵗ}}\AgdaSymbol{)\}}\<%
\\
%
\>[22]\AgdaSymbol{\{}\AgdaInductiveConstructor{ι}\AgdaSpace{}%
\AgdaSymbol{(}\AgdaOperator{\AgdaFunction{\#}}\AgdaSpace{}%
\AgdaBound{r₂}\AgdaSpace{}%
\AgdaOperator{\AgdaFunction{ᵗ}}\AgdaSymbol{)\}}\<%
\\
%
\>[22]\AgdaSymbol{(}\AgdaBound{j₀}\AgdaSpace{}%
\AgdaOperator{\AgdaFunction{⟨}}\AgdaSpace{}%
\AgdaFunction{rev}\AgdaSpace{}%
\AgdaFunction{♭}\AgdaSpace{}%
\AgdaOperator{\AgdaFunction{⟩}}\AgdaSymbol{)}\<%
\\
%
\>[22]\AgdaSymbol{(}\AgdaBound{j₁}\AgdaSpace{}%
\AgdaOperator{\AgdaFunction{⟨}}\AgdaSpace{}%
\AgdaFunction{rev}\AgdaSpace{}%
\AgdaFunction{♭}\AgdaSpace{}%
\AgdaOperator{\AgdaFunction{⟩}}\AgdaSymbol{)}\<%
\\
%
\>[20]\AgdaSymbol{)}\<%
\\
%
\>[16]\AgdaSymbol{)}\<%
\\
\>[2191I][@{}l@{\AgdaIndent{0}}]%
\>[14]\AgdaSymbol{)}\<%
\\
\>[2189I][@{}l@{\AgdaIndent{0}}]%
\>[8]\AgdaFunction{⟩}\<%
\\
%
\>[4]\AgdaSymbol{\AgdaUnderscore{}}\AgdaSpace{}%
\AgdaOperator{\AgdaFunction{∎}}\<%
\\
\>[0]\<%
\\
%
\>[2]\AgdaComment{-----------------}\<%
\\
%
\>[2]\AgdaComment{---\ FFT\ ≡\ DFT\ ---}\<%
\\
%
\>[2]\AgdaComment{-----------------}\<%
\\
%
\\[\AgdaEmptyExtraSkip]%
%
\>[2]\AgdaFunction{fft′≅dft′}\AgdaSpace{}%
\AgdaSymbol{:}\<%
\\
\>[2][@{}l@{\AgdaIndent{0}}]%
\>[6]\AgdaSymbol{⦃}\AgdaSpace{}%
\AgdaBound{nz-s}%
\>[14]\AgdaSymbol{:}\AgdaSpace{}%
\AgdaDatatype{NonZeroₛ}\AgdaSpace{}%
\AgdaGeneralizable{s}\AgdaSpace{}%
\AgdaSymbol{⦄}\<%
\\
\>[2][@{}l@{\AgdaIndent{0}}]%
\>[4]\AgdaSymbol{→}\AgdaSpace{}%
\AgdaSymbol{∀}\AgdaSpace{}%
\AgdaSymbol{(}\AgdaBound{arr}\AgdaSpace{}%
\AgdaSymbol{:}\AgdaSpace{}%
\AgdaFunction{Ar}\AgdaSpace{}%
\AgdaGeneralizable{s}\AgdaSpace{}%
\AgdaField{ℂ}\AgdaSymbol{)}\<%
\\
%
\>[4]\AgdaSymbol{→}%
\>[2230I]\AgdaFunction{FFT′}\AgdaSpace{}%
\AgdaBound{arr}\<%
\\
\>[.][@{}l@{}]\<[2230I]%
\>[6]\AgdaOperator{\AgdaFunction{≅}}\<%
\\
%
\>[6]\AgdaSymbol{(}\AgdaSpace{}%
\AgdaSymbol{(}\AgdaFunction{reshape}\AgdaSpace{}%
\AgdaFunction{♯}\AgdaSymbol{)}\<%
\\
%
\>[6]\AgdaOperator{\AgdaFunction{∘}}\AgdaSpace{}%
\AgdaSymbol{(}\AgdaFunction{DFT′}\AgdaSpace{}%
\AgdaSymbol{⦃}\AgdaSpace{}%
\AgdaFunction{nz-\#}\AgdaSpace{}%
\AgdaSymbol{(}\AgdaFunction{nzᵗ}\AgdaSpace{}%
\AgdaBound{nz-s}\AgdaSymbol{)}\AgdaSpace{}%
\AgdaSymbol{⦄}\AgdaSpace{}%
\AgdaSymbol{)}\<%
\\
%
\>[6]\AgdaOperator{\AgdaFunction{∘}}\AgdaSpace{}%
\AgdaSymbol{(}\AgdaFunction{reshape}\AgdaSpace{}%
\AgdaFunction{flatten-reindex}\AgdaSymbol{))}\AgdaSpace{}%
\AgdaBound{arr}\<%
\end{code}
\subsubsection{Inductive Step}
The core of this proof which allows its application to any shape is the inductive step,
this is also the first section of the proof.
\begin{AgdaMultiCode}
\begin{code}[number=code:fft′≡dft′-ιN]%
%
\>[2]\AgdaFunction{fft′≅dft′}\AgdaSpace{}%
\AgdaSymbol{\{}\AgdaInductiveConstructor{ι}\AgdaSpace{}%
\AgdaBound{N}\AgdaSymbol{\}}\AgdaSpace{}%
\AgdaSymbol{⦃}\AgdaSpace{}%
\AgdaInductiveConstructor{ι}\AgdaSpace{}%
\AgdaBound{nz-N}\AgdaSpace{}%
\AgdaSymbol{⦄}\AgdaSpace{}%
\AgdaBound{arr}\AgdaSpace{}%
\AgdaBound{i}\AgdaSpace{}%
\AgdaSymbol{=}\AgdaSpace{}%
\AgdaInductiveConstructor{refl}\<%
\end{code}
\begin{code}[number=code:fft′≡dft′-r₁⊗r₂]%
%
\>[2]\AgdaFunction{fft′≅dft′}\AgdaSpace{}%
\AgdaSymbol{\{}\AgdaBound{r₁}\AgdaSpace{}%
\AgdaOperator{\AgdaInductiveConstructor{⊗}}\AgdaSpace{}%
\AgdaBound{r₂}\AgdaSymbol{\}}\AgdaSpace{}%
\AgdaSymbol{⦃}\AgdaSpace{}%
\AgdaBound{nz-r₁}\AgdaSpace{}%
\AgdaOperator{\AgdaInductiveConstructor{⊗}}\AgdaSpace{}%
\AgdaBound{nz-r₂}\AgdaSpace{}%
\AgdaSymbol{⦄}\AgdaSpace{}%
\AgdaBound{arr}\AgdaSpace{}%
\AgdaSymbol{(}\AgdaBound{j₁}\AgdaSpace{}%
\AgdaOperator{\AgdaInductiveConstructor{⊗}}\AgdaSpace{}%
\AgdaBound{j₀}\AgdaSymbol{)}\AgdaSpace{}%
\AgdaSymbol{=}\<%
\end{code}
\begin{code}[hide]%
\>[2][@{}l@{\AgdaIndent{1}}]%
\>[4]\AgdaKeyword{let}\AgdaSpace{}%
\AgdaKeyword{instance}\<%
\\
\>[4][@{}l@{\AgdaIndent{0}}]%
\>[6]\AgdaBound{\AgdaUnderscore{}}\AgdaSpace{}%
\AgdaSymbol{:}\AgdaSpace{}%
\AgdaDatatype{NonZeroₛ}\AgdaSpace{}%
\AgdaBound{r₁}\<%
\\
%
\>[6]\AgdaSymbol{\AgdaUnderscore{}}\AgdaSpace{}%
\AgdaSymbol{=}\AgdaSpace{}%
\AgdaBound{nz-r₁}\<%
\\
%
\>[6]\AgdaBound{\AgdaUnderscore{}}\AgdaSpace{}%
\AgdaSymbol{:}\AgdaSpace{}%
\AgdaDatatype{NonZeroₛ}\AgdaSpace{}%
\AgdaBound{r₂}\<%
\\
%
\>[6]\AgdaSymbol{\AgdaUnderscore{}}\AgdaSpace{}%
\AgdaSymbol{=}\AgdaSpace{}%
\AgdaBound{nz-r₂}\<%
\\
%
\>[6]\AgdaBound{\AgdaUnderscore{}}\AgdaSpace{}%
\AgdaSymbol{:}\AgdaSpace{}%
\AgdaRecord{NonZero}\AgdaSpace{}%
\AgdaSymbol{(}\AgdaOperator{\AgdaFunction{\#}}\AgdaSpace{}%
\AgdaBound{r₁}\AgdaSymbol{)}\<%
\\
%
\>[6]\AgdaSymbol{\AgdaUnderscore{}}\AgdaSpace{}%
\AgdaSymbol{=}\AgdaSpace{}%
\AgdaSymbol{(}\AgdaFunction{nz-\#}\AgdaSpace{}%
\AgdaBound{nz-r₁}\AgdaSymbol{)}\<%
\\
%
\>[6]\AgdaBound{\AgdaUnderscore{}}\AgdaSpace{}%
\AgdaSymbol{:}\AgdaSpace{}%
\AgdaRecord{NonZero}\AgdaSpace{}%
\AgdaSymbol{(}\AgdaOperator{\AgdaFunction{\#}}\AgdaSpace{}%
\AgdaBound{r₂}\AgdaSymbol{)}\<%
\\
%
\>[6]\AgdaSymbol{\AgdaUnderscore{}}\AgdaSpace{}%
\AgdaSymbol{=}\AgdaSpace{}%
\AgdaSymbol{(}\AgdaFunction{nz-\#}\AgdaSpace{}%
\AgdaBound{nz-r₂}\AgdaSymbol{)}\<%
\\
%
\>[6]\AgdaBound{\AgdaUnderscore{}}\AgdaSpace{}%
\AgdaSymbol{:}\AgdaSpace{}%
\AgdaRecord{NonZero}\AgdaSpace{}%
\AgdaSymbol{(}\AgdaOperator{\AgdaFunction{\#}}\AgdaSpace{}%
\AgdaBound{r₁}\AgdaSpace{}%
\AgdaOperator{\AgdaFunction{ᵗ}}\AgdaSymbol{)}\<%
\\
%
\>[6]\AgdaSymbol{\AgdaUnderscore{}}\AgdaSpace{}%
\AgdaSymbol{=}\AgdaSpace{}%
\AgdaSymbol{(}\AgdaFunction{nz-\#}\AgdaSpace{}%
\AgdaSymbol{(}\AgdaFunction{nzᵗ}\AgdaSpace{}%
\AgdaBound{nz-r₁}\AgdaSymbol{))}\<%
\\
%
\>[6]\AgdaBound{\AgdaUnderscore{}}\AgdaSpace{}%
\AgdaSymbol{:}\AgdaSpace{}%
\AgdaRecord{NonZero}\AgdaSpace{}%
\AgdaSymbol{(}\AgdaOperator{\AgdaFunction{\#}}\AgdaSpace{}%
\AgdaBound{r₂}\AgdaSpace{}%
\AgdaOperator{\AgdaFunction{ᵗ}}\AgdaSymbol{)}\<%
\\
%
\>[6]\AgdaSymbol{\AgdaUnderscore{}}\AgdaSpace{}%
\AgdaSymbol{=}\AgdaSpace{}%
\AgdaSymbol{(}\AgdaFunction{nz-\#}\AgdaSpace{}%
\AgdaSymbol{(}\AgdaFunction{nzᵗ}\AgdaSpace{}%
\AgdaBound{nz-r₂}\AgdaSymbol{))}\<%
\\
%
\>[6]\AgdaBound{\AgdaUnderscore{}}\AgdaSpace{}%
\AgdaSymbol{:}\AgdaSpace{}%
\AgdaRecord{NonZero}\AgdaSpace{}%
\AgdaSymbol{(}\AgdaOperator{\AgdaFunction{\#}}\AgdaSpace{}%
\AgdaBound{r₂}%
\>[26]\AgdaOperator{\AgdaPrimitive{*ₙ}}\AgdaSpace{}%
\AgdaOperator{\AgdaFunction{\#}}\AgdaSpace{}%
\AgdaBound{r₁}\AgdaSpace{}%
\AgdaOperator{\AgdaFunction{ᵗ}}\AgdaSymbol{)}\<%
\\
%
\>[6]\AgdaSymbol{\AgdaUnderscore{}}\AgdaSpace{}%
\AgdaSymbol{=}\AgdaSpace{}%
\AgdaFunction{m*n≢0}\AgdaSpace{}%
\AgdaSymbol{(}\AgdaOperator{\AgdaFunction{\#}}\AgdaSpace{}%
\AgdaBound{r₂}\AgdaSymbol{)}\AgdaSpace{}%
\AgdaSymbol{(}\AgdaOperator{\AgdaFunction{\#}}\AgdaSpace{}%
\AgdaBound{r₁}\AgdaSpace{}%
\AgdaOperator{\AgdaFunction{ᵗ}}\AgdaSymbol{)}\<%
\\
%
\>[6]\AgdaBound{\AgdaUnderscore{}}\AgdaSpace{}%
\AgdaSymbol{:}\AgdaSpace{}%
\AgdaRecord{NonZero}\AgdaSpace{}%
\AgdaSymbol{(}\AgdaOperator{\AgdaFunction{\#}}\AgdaSpace{}%
\AgdaBound{r₂}\AgdaSpace{}%
\AgdaOperator{\AgdaFunction{ᵗ}}\AgdaSpace{}%
\AgdaOperator{\AgdaPrimitive{*ₙ}}\AgdaSpace{}%
\AgdaOperator{\AgdaFunction{\#}}\AgdaSpace{}%
\AgdaBound{r₁}\AgdaSpace{}%
\AgdaOperator{\AgdaFunction{ᵗ}}\AgdaSymbol{)}\<%
\\
%
\>[6]\AgdaSymbol{\AgdaUnderscore{}}\AgdaSpace{}%
\AgdaSymbol{=}\AgdaSpace{}%
\AgdaFunction{m*n≢0}\AgdaSpace{}%
\AgdaSymbol{(}\AgdaOperator{\AgdaFunction{\#}}\AgdaSpace{}%
\AgdaBound{r₂}\AgdaSpace{}%
\AgdaOperator{\AgdaFunction{ᵗ}}\AgdaSymbol{)}\AgdaSpace{}%
\AgdaSymbol{(}\AgdaOperator{\AgdaFunction{\#}}\AgdaSpace{}%
\AgdaBound{r₁}\AgdaSpace{}%
\AgdaOperator{\AgdaFunction{ᵗ}}\AgdaSymbol{)}\<%
\\
%
\>[4]\AgdaKeyword{in}\<%
\end{code}
These first two lines of this chain of reasoning split the proof on the shape 
of the input matrix.
\ref{code:fft′≡dft′-ιN} pattern matches the case where the shape is one dimensional, 
as FFT on such a shape is equal by definition to the DFT, no chain of reasoning 
is required to prove this case.
This is the base case of the induction.
\ref{code:fft′≡dft′-r₁⊗r₂} pattern matches on the alternate case, and precedes 
the remainder of the proof, where \AF{r₁} and \AF{r₂} represent the left and right
sub shapes.

\begin{code}%
\>[4][@{}l@{\AgdaIndent{1}}]%
\>[6]\AgdaOperator{\AgdaFunction{begin}}\<%
\end{code}
\begin{code}[number=code:fft′≡dft′-lhs]%
\>[6][@{}l@{\AgdaIndent{1}}]%
\>[8]\AgdaFunction{FFT′}\AgdaSpace{}%
\AgdaSymbol{\{}\AgdaBound{r₂}\AgdaSymbol{\}}\AgdaSpace{}%
\AgdaSymbol{(λ}\AgdaSpace{}%
\AgdaBound{k₀}\AgdaSpace{}%
\AgdaSymbol{→}\<%
\\
\>[8][@{}l@{\AgdaIndent{0}}]%
\>[12]\AgdaFunction{FFT′}\AgdaSpace{}%
\AgdaSymbol{\{}\AgdaBound{r₁}\AgdaSymbol{\}}\AgdaSpace{}%
\AgdaSymbol{(λ}\AgdaSpace{}%
\AgdaBound{k₁}\AgdaSpace{}%
\AgdaSymbol{→}\AgdaSpace{}%
\AgdaSymbol{\AgdaUnderscore{})}\AgdaSpace{}%
\AgdaBound{j₀}\AgdaSpace{}%
\AgdaOperator{\AgdaField{*}}\AgdaSpace{}%
\AgdaSymbol{\AgdaUnderscore{}}\<%
\\
%
\>[8]\AgdaSymbol{)}\AgdaSpace{}%
\AgdaBound{j₁}\<%
\end{code}
\begin{code}[number=code:fft′≡dft′-inductive-step-1]%
%
\>[6]\AgdaFunction{≡⟨}\AgdaSpace{}%
\AgdaFunction{fft′≅dft′}\AgdaSpace{}%
\AgdaSymbol{\AgdaUnderscore{}}\AgdaSpace{}%
\AgdaBound{j₁}\AgdaSpace{}%
\AgdaFunction{⟩}\<%
\\
\>[6][@{}l@{\AgdaIndent{0}}]%
\>[8]\AgdaFunction{DFT′}\AgdaSpace{}%
\AgdaSymbol{\{}\AgdaOperator{\AgdaFunction{\#}}\AgdaSpace{}%
\AgdaBound{r₂}\AgdaSpace{}%
\AgdaOperator{\AgdaFunction{ᵗ}}\AgdaSymbol{\}}\AgdaSpace{}%
\AgdaSymbol{(λ}\AgdaSpace{}%
\AgdaBound{k₀}\AgdaSpace{}%
\AgdaSymbol{→}\<%
\\
\>[8][@{}l@{\AgdaIndent{0}}]%
\>[12]\AgdaFunction{FFT′}\AgdaSpace{}%
\AgdaSymbol{\{}\AgdaBound{r₁}\AgdaSymbol{\}}\AgdaSpace{}%
\AgdaSymbol{(λ}\AgdaSpace{}%
\AgdaBound{k₁}\AgdaSpace{}%
\AgdaSymbol{→}\AgdaSpace{}%
\AgdaSymbol{\AgdaUnderscore{})}\AgdaSpace{}%
\AgdaBound{j₀}\AgdaSpace{}%
\AgdaOperator{\AgdaField{*}}\AgdaSpace{}%
\AgdaSymbol{\AgdaUnderscore{}}\<%
\\
%
\>[8]\AgdaSymbol{)}\AgdaSpace{}%
\AgdaSymbol{(}\AgdaBound{j₁}\AgdaSpace{}%
\AgdaOperator{\AgdaFunction{⟨}}\AgdaSpace{}%
\AgdaFunction{♯}\AgdaSpace{}%
\AgdaOperator{\AgdaFunction{⟩}}\AgdaSymbol{)}\<%
\end{code}
\begin{code}[number=code:fft′≡dft′-inductive-step-2]%
%
\>[6]\AgdaFunction{≡⟨}\AgdaSpace{}%
\AgdaFunction{DFT′-cong}\AgdaSpace{}%
\AgdaSymbol{(λ}\AgdaSpace{}%
\AgdaBound{x}\AgdaSpace{}%
\AgdaSymbol{→}\AgdaSpace{}%
\AgdaFunction{cong₂}\AgdaSpace{}%
\AgdaOperator{\AgdaField{\AgdaUnderscore{}*\AgdaUnderscore{}}}\AgdaSpace{}%
\AgdaSymbol{(}\AgdaFunction{fft′≅dft′}\AgdaSpace{}%
\AgdaSymbol{\AgdaUnderscore{}}\AgdaSpace{}%
\AgdaBound{j₀}\AgdaSymbol{)}\AgdaSpace{}%
\AgdaInductiveConstructor{refl}\AgdaSymbol{)}\AgdaSpace{}%
\AgdaSymbol{(}\AgdaBound{j₁}\AgdaSpace{}%
\AgdaOperator{\AgdaFunction{⟨}}\AgdaSpace{}%
\AgdaFunction{♯}\AgdaSpace{}%
\AgdaOperator{\AgdaFunction{⟩}}\AgdaSpace{}%
\AgdaSymbol{)}\AgdaSpace{}%
\AgdaFunction{⟩}\<%
\\
\>[6][@{}l@{\AgdaIndent{0}}]%
\>[8]\AgdaFunction{DFT′}\AgdaSpace{}%
\AgdaSymbol{\{}\AgdaOperator{\AgdaFunction{\#}}\AgdaSpace{}%
\AgdaBound{r₂}\AgdaSpace{}%
\AgdaOperator{\AgdaFunction{ᵗ}}\AgdaSymbol{\}}\AgdaSpace{}%
\AgdaSymbol{(λ}\AgdaSpace{}%
\AgdaBound{k₀}\AgdaSpace{}%
\AgdaSymbol{→}\<%
\\
\>[8][@{}l@{\AgdaIndent{0}}]%
\>[12]\AgdaFunction{DFT′}\AgdaSpace{}%
\AgdaSymbol{\{}\AgdaOperator{\AgdaFunction{\#}}\AgdaSpace{}%
\AgdaBound{r₁}\AgdaSpace{}%
\AgdaOperator{\AgdaFunction{ᵗ}}\AgdaSymbol{\}}\AgdaSpace{}%
\AgdaSymbol{(λ}\AgdaSpace{}%
\AgdaBound{k₁}\AgdaSpace{}%
\AgdaSymbol{→}\AgdaSpace{}%
\AgdaSymbol{\AgdaUnderscore{})}\AgdaSpace{}%
\AgdaSymbol{(}\AgdaBound{j₀}\AgdaSpace{}%
\AgdaOperator{\AgdaFunction{⟨}}\AgdaSpace{}%
\AgdaFunction{♯}\AgdaSpace{}%
\AgdaOperator{\AgdaFunction{⟩}}\AgdaSymbol{)}\AgdaSpace{}%
\AgdaOperator{\AgdaField{*}}\AgdaSpace{}%
\AgdaSymbol{\AgdaUnderscore{}}\<%
\\
%
\>[8]\AgdaSymbol{)}\AgdaSpace{}%
\AgdaSymbol{(}\AgdaBound{j₁}\AgdaSpace{}%
\AgdaOperator{\AgdaFunction{⟨}}\AgdaSpace{}%
\AgdaFunction{♯}\AgdaSpace{}%
\AgdaOperator{\AgdaFunction{⟩}}\AgdaSymbol{)}\<%
\\
%
\>[6]\AgdaComment{--\ ...}\<%
\end{code}
Splitting upon the shape allows the left hand side to take 
the form shown in step \ref{code:fft′≡dft′-lhs}.
Step \ref{code:fft′≡dft′-inductive-step-1} and \ref{code:fft′≡dft′-inductive-step-2}
are then able to apply the proposition currently being proven to the outer and
inner instances of FFT′.
This allows both instances to be represented as DFT′, which in turn allows for
the representation of the left hand side in sum form.
\begin{code}%
%
\>[6]\AgdaFunction{≡⟨⟩}\<%
\\
\>[6][@{}l@{\AgdaIndent{0}}]%
\>[8]\AgdaFunction{sum}%
\>[2415I]\AgdaSymbol{\{}\AgdaOperator{\AgdaFunction{\#}}\AgdaSpace{}%
\AgdaBound{r₂}\AgdaSpace{}%
\AgdaOperator{\AgdaFunction{ᵗ}}\AgdaSymbol{\}}\AgdaSpace{}%
\AgdaSymbol{(λ}\AgdaSpace{}%
\AgdaBound{k}\AgdaSpace{}%
\AgdaSymbol{→}\<%
\\
\>[.][@{}l@{}]\<[2415I]%
\>[12]\AgdaFunction{sum}%
\>[2421I]\AgdaSymbol{\{}\AgdaOperator{\AgdaFunction{\#}}\AgdaSpace{}%
\AgdaBound{r₁}\AgdaSpace{}%
\AgdaOperator{\AgdaFunction{ᵗ}}\AgdaSymbol{\}}\AgdaSpace{}%
\AgdaSymbol{(λ}\AgdaSpace{}%
\AgdaBound{k₁}\AgdaSpace{}%
\AgdaSymbol{→}\<%
\\
\>[.][@{}l@{}]\<[2421I]%
\>[16]\AgdaBound{arr}\AgdaSpace{}%
\AgdaSymbol{\AgdaUnderscore{}}\<%
\\
\>[12][@{}l@{\AgdaIndent{0}}]%
\>[14]\AgdaOperator{\AgdaField{*}}\<%
\\
\>[14][@{}l@{\AgdaIndent{0}}]%
\>[16]\AgdaField{-ω}\AgdaSpace{}%
\AgdaSymbol{\AgdaUnderscore{}}\AgdaSpace{}%
\AgdaSymbol{\AgdaUnderscore{}}%
\>[46]\AgdaComment{--\ Inner\ DFT\ -ω}\<%
\\
%
\>[12]\AgdaSymbol{)}\<%
\\
\>[8][@{}l@{\AgdaIndent{0}}]%
\>[10]\AgdaOperator{\AgdaField{*}}\<%
\\
\>[10][@{}l@{\AgdaIndent{0}}]%
\>[12]\AgdaField{-ω}\AgdaSpace{}%
\AgdaSymbol{\AgdaUnderscore{}}\AgdaSpace{}%
\AgdaSymbol{\AgdaUnderscore{}}%
\>[46]\AgdaComment{--\ Twiddle\ Factor\ -ω}\<%
\\
%
\>[10]\AgdaOperator{\AgdaField{*}}\<%
\\
\>[10][@{}l@{\AgdaIndent{0}}]%
\>[12]\AgdaField{-ω}\AgdaSpace{}%
\AgdaSymbol{\AgdaUnderscore{}}\AgdaSpace{}%
\AgdaSymbol{\AgdaUnderscore{}}%
\>[46]\AgdaComment{--\ Outer\ DFT\ -ω}\<%
\\
%
\>[8]\AgdaSymbol{)}\<%
\end{code}
\begin{code}[hide]%
%
\>[8]\AgdaFunction{≡⟨}%
\>[12]\AgdaFunction{sum-cong}\AgdaSpace{}%
\AgdaSymbol{\{}%
\>[24]\AgdaOperator{\AgdaFunction{\#}}\AgdaSpace{}%
\AgdaBound{r₂}\AgdaSpace{}%
\AgdaOperator{\AgdaFunction{ᵗ}}\AgdaSpace{}%
\AgdaSymbol{\}}\<%
\\
\>[12][@{}l@{\AgdaIndent{0}}]%
\>[14]\AgdaSymbol{(λ}%
\>[2438I]\AgdaBound{k₀}\AgdaSpace{}%
\AgdaSymbol{→}\<%
\\
\>[2438I][@{}l@{\AgdaIndent{0}}]%
\>[18]\AgdaFunction{cong₂}\AgdaSpace{}%
\AgdaOperator{\AgdaField{\AgdaUnderscore{}*\AgdaUnderscore{}}}\AgdaSpace{}%
\AgdaSymbol{(}\AgdaFunction{*-distribʳ-sum}\AgdaSpace{}%
\AgdaSymbol{\{}\AgdaOperator{\AgdaFunction{\#}}\AgdaSpace{}%
\AgdaBound{r₁}\AgdaSpace{}%
\AgdaOperator{\AgdaFunction{ᵗ}}\AgdaSymbol{\}}\AgdaSpace{}%
\AgdaSymbol{\AgdaUnderscore{})}\AgdaSpace{}%
\AgdaInductiveConstructor{refl}\<%
\\
\>[14][@{}l@{\AgdaIndent{0}}]%
\>[16]\AgdaOperator{\AgdaFunction{⊡}}\AgdaSpace{}%
\AgdaFunction{*-distribʳ-sum}\AgdaSpace{}%
\AgdaSymbol{\{}\AgdaOperator{\AgdaFunction{\#}}\AgdaSpace{}%
\AgdaBound{r₁}\AgdaSpace{}%
\AgdaOperator{\AgdaFunction{ᵗ}}\AgdaSymbol{\}}\AgdaSpace{}%
\AgdaSymbol{\AgdaUnderscore{}}\<%
\\
%
\>[16]\AgdaOperator{\AgdaFunction{⊡}}%
\>[2452I]\AgdaFunction{sum-cong}\AgdaSpace{}%
\AgdaSymbol{\{}\AgdaOperator{\AgdaFunction{\#}}\AgdaSpace{}%
\AgdaBound{r₁}\AgdaSpace{}%
\AgdaOperator{\AgdaFunction{ᵗ}}\AgdaSpace{}%
\AgdaSymbol{\}}\<%
\\
\>[2452I][@{}l@{\AgdaIndent{0}}]%
\>[20]\AgdaSymbol{(λ}%
\>[2457I]\AgdaBound{k₁}\AgdaSpace{}%
\AgdaSymbol{→}\<%
\\
\>[2457I][@{}l@{\AgdaIndent{0}}]%
\>[24]\AgdaFunction{assoc₄}\AgdaSpace{}%
\AgdaSymbol{\AgdaUnderscore{}}\AgdaSpace{}%
\AgdaSymbol{\AgdaUnderscore{}}\AgdaSpace{}%
\AgdaSymbol{\AgdaUnderscore{}}\AgdaSpace{}%
\AgdaSymbol{\AgdaUnderscore{}}\<%
\\
\>[20][@{}l@{\AgdaIndent{0}}]%
\>[22]\AgdaOperator{\AgdaFunction{⊡}}%
\>[2463I]\AgdaFunction{cong₂}\AgdaSpace{}%
\AgdaOperator{\AgdaField{\AgdaUnderscore{}*\AgdaUnderscore{}}}\<%
\\
\>[2463I][@{}l@{\AgdaIndent{0}}]%
\>[26]\AgdaSymbol{(}\AgdaFunction{sym}\AgdaSpace{}%
\AgdaSymbol{((}\AgdaFunction{rev-eq-applied}\AgdaSpace{}%
\AgdaInductiveConstructor{split}\AgdaSpace{}%
\AgdaSymbol{(}\AgdaFunction{reshape}\AgdaSpace{}%
\AgdaSymbol{(}\AgdaFunction{♭}\AgdaSpace{}%
\AgdaOperator{\AgdaInductiveConstructor{⊕}}\AgdaSpace{}%
\AgdaFunction{♭}\AgdaSymbol{)}\AgdaSpace{}%
\AgdaBound{arr}\AgdaSymbol{))}\AgdaSpace{}%
\AgdaSymbol{(\AgdaUnderscore{}}\AgdaSpace{}%
\AgdaOperator{\AgdaInductiveConstructor{⊗}}\AgdaSpace{}%
\AgdaSymbol{\AgdaUnderscore{})))}\<%
\\
%
\>[26]\AgdaInductiveConstructor{refl}\<%
\\
%
\>[20]\AgdaSymbol{)}\<%
\\
%
\>[14]\AgdaSymbol{)}\<%
\\
\>[8][@{}l@{\AgdaIndent{0}}]%
\>[10]\AgdaFunction{⟩}\<%
\\
%
\>[6]\AgdaSymbol{\AgdaUnderscore{}}%
\>[2475I]\AgdaFunction{≡⟨}%
\>[12]\AgdaFunction{sum-cong}\AgdaSpace{}%
\AgdaSymbol{\{}%
\>[24]\AgdaOperator{\AgdaFunction{\#}}\AgdaSpace{}%
\AgdaBound{r₂}\AgdaSpace{}%
\AgdaOperator{\AgdaFunction{ᵗ}}\AgdaSpace{}%
\AgdaSymbol{\}}\AgdaSpace{}%
\AgdaSymbol{(λ}\AgdaSpace{}%
\AgdaBound{k₀}\AgdaSpace{}%
\AgdaSymbol{→}\<%
\\
\>[12][@{}l@{\AgdaIndent{0}}]%
\>[14]\AgdaFunction{sum-cong}\AgdaSpace{}%
\AgdaSymbol{\{}\AgdaOperator{\AgdaFunction{\#}}\AgdaSpace{}%
\AgdaBound{r₁}\AgdaSpace{}%
\AgdaOperator{\AgdaFunction{ᵗ}}\AgdaSpace{}%
\AgdaSymbol{\}}\AgdaSpace{}%
\AgdaSymbol{(λ}\AgdaSpace{}%
\AgdaBound{k₁}\AgdaSpace{}%
\AgdaSymbol{→}\<%
\\
\>[14][@{}l@{\AgdaIndent{0}}]%
\>[16]\AgdaFunction{cong₂}\AgdaSpace{}%
\AgdaOperator{\AgdaField{\AgdaUnderscore{}*\AgdaUnderscore{}}}\AgdaSpace{}%
\AgdaInductiveConstructor{refl}\<%
\\
\>[16][@{}l@{\AgdaIndent{0}}]%
\>[18]\AgdaSymbol{(}\AgdaFunction{cooley-tukey-derivation-application}\AgdaSpace{}%
\AgdaBound{j₁}\AgdaSpace{}%
\AgdaBound{j₀}\AgdaSpace{}%
\AgdaBound{k₀}\AgdaSpace{}%
\AgdaBound{k₁}\AgdaSpace{}%
\AgdaBound{nz-r₁}\AgdaSpace{}%
\AgdaBound{nz-r₂}\AgdaSymbol{)}\<%
\\
%
\>[14]\AgdaSymbol{)}\<%
\\
%
\>[12]\AgdaSymbol{)}\<%
\\
\>[2475I][@{}l@{\AgdaIndent{0}}]%
\>[10]\AgdaFunction{⟩}\<%
\\
%
\>[6]\AgdaSymbol{\AgdaUnderscore{}}%
\>[2498I]\AgdaFunction{≡⟨}%
\>[12]\AgdaFunction{sum-cong}\AgdaSpace{}%
\AgdaSymbol{\{}\AgdaOperator{\AgdaFunction{\#}}\AgdaSpace{}%
\AgdaBound{r₂}\AgdaSpace{}%
\AgdaOperator{\AgdaFunction{ᵗ}}\AgdaSymbol{\}}\AgdaSpace{}%
\AgdaSymbol{(λ}\AgdaSpace{}%
\AgdaBound{k₀}\AgdaSpace{}%
\AgdaSymbol{→}\AgdaSpace{}%
\AgdaFunction{sym}\AgdaSpace{}%
\AgdaSymbol{(}\AgdaFunction{sum-reindex}\AgdaSpace{}%
\AgdaSymbol{(}\AgdaFunction{|s|≡|sᵗ|}\AgdaSpace{}%
\AgdaSymbol{\{}\AgdaBound{r₁}\AgdaSymbol{\})))}\<%
\\
\>[2498I][@{}l@{\AgdaIndent{0}}]%
\>[10]\AgdaOperator{\AgdaFunction{⊡}}\AgdaSpace{}%
\AgdaFunction{sym}\AgdaSpace{}%
\AgdaSymbol{(}\AgdaFunction{sum-reindex}\AgdaSpace{}%
\AgdaSymbol{(}\AgdaFunction{|s|≡|sᵗ|}\AgdaSpace{}%
\AgdaSymbol{\{}\AgdaBound{r₂}\AgdaSymbol{\}))}\<%
\\
\>[2498I][@{}l@{\AgdaIndent{0}}]%
\>[9]\AgdaFunction{⟩}\<%
\\
%
\>[6]\AgdaSymbol{\AgdaUnderscore{}}%
\>[2513I]\AgdaFunction{≡⟨}%
\>[12]\AgdaFunction{sum-swap}\AgdaSpace{}%
\AgdaSymbol{\{}\AgdaOperator{\AgdaFunction{\#}}\AgdaSpace{}%
\AgdaBound{r₂}\AgdaSymbol{\}}\AgdaSpace{}%
\AgdaSymbol{\{}\AgdaOperator{\AgdaFunction{\#}}\AgdaSpace{}%
\AgdaBound{r₁}\AgdaSymbol{\}}\AgdaSpace{}%
\AgdaSymbol{\AgdaUnderscore{}}\<%
\\
\>[2513I][@{}l@{\AgdaIndent{0}}]%
\>[10]\AgdaOperator{\AgdaFunction{⊡}}\AgdaSpace{}%
\AgdaFunction{merge-sum}\AgdaSpace{}%
\AgdaSymbol{\{}\AgdaOperator{\AgdaFunction{\#}}\AgdaSpace{}%
\AgdaBound{r₁}\AgdaSymbol{\}}\AgdaSpace{}%
\AgdaSymbol{\{}\AgdaOperator{\AgdaFunction{\#}}\AgdaSpace{}%
\AgdaBound{r₂}\AgdaSymbol{\}}\AgdaSpace{}%
\AgdaSymbol{\AgdaUnderscore{}}\<%
\\
\>[2513I][@{}l@{\AgdaIndent{0}}]%
\>[9]\AgdaFunction{⟩}\<%
\\
\>[9][@{}l@{\AgdaIndent{0}}]%
\>[12]\AgdaFunction{sum}\AgdaSpace{}%
\AgdaSymbol{\{}\AgdaSpace{}%
\AgdaOperator{\AgdaFunction{\#}}\AgdaSpace{}%
\AgdaSymbol{(}\AgdaBound{r₁}\AgdaSpace{}%
\AgdaOperator{\AgdaInductiveConstructor{⊗}}\AgdaSpace{}%
\AgdaBound{r₂}\AgdaSymbol{)}\AgdaSpace{}%
\AgdaSymbol{\}}\<%
\\
\>[12][@{}l@{\AgdaIndent{0}}]%
\>[14]\AgdaSymbol{(λ}%
\>[2531I]\AgdaBound{k}%
\>[2532I]\AgdaSymbol{→}\<%
\\
\>[.][@{}l@{}]\<[2532I]%
\>[19]\AgdaBound{arr}\AgdaSpace{}%
\AgdaSymbol{(((}\AgdaBound{k}\AgdaSymbol{)}\AgdaSpace{}%
\AgdaOperator{\AgdaFunction{⟨}}\AgdaSpace{}%
\AgdaInductiveConstructor{flat}\AgdaSpace{}%
\AgdaOperator{\AgdaFunction{⟩}}\AgdaSymbol{)}\AgdaSpace{}%
\AgdaOperator{\AgdaFunction{⟨}}\AgdaSpace{}%
\AgdaFunction{♭}\AgdaSpace{}%
\AgdaOperator{\AgdaInductiveConstructor{⊕}}\AgdaSpace{}%
\AgdaFunction{♭}\AgdaSpace{}%
\AgdaOperator{\AgdaFunction{⟩}}\AgdaSymbol{)}\<%
\\
\>[.][@{}l@{}]\<[2531I]%
\>[17]\AgdaOperator{\AgdaField{*}}\<%
\\
\>[17][@{}l@{\AgdaIndent{0}}]%
\>[19]\AgdaField{-ω}\<%
\\
\>[19][@{}l@{\AgdaIndent{0}}]%
\>[21]\AgdaSymbol{(}\AgdaOperator{\AgdaFunction{\#}}\AgdaSpace{}%
\AgdaBound{r₂}\AgdaSpace{}%
\AgdaOperator{\AgdaFunction{ᵗ}}\AgdaSpace{}%
\AgdaOperator{\AgdaPrimitive{*ₙ}}\AgdaSpace{}%
\AgdaOperator{\AgdaFunction{\#}}\AgdaSpace{}%
\AgdaBound{r₁}\AgdaSpace{}%
\AgdaOperator{\AgdaFunction{ᵗ}}\AgdaSymbol{)}\<%
\\
%
\>[21]\AgdaSymbol{(}\AgdaFunction{iota}\AgdaSpace{}%
\AgdaBound{k}\AgdaSpace{}%
\AgdaOperator{\AgdaPrimitive{*ₙ}}\AgdaSpace{}%
\AgdaFunction{iota}\AgdaSpace{}%
\AgdaSymbol{(((}\AgdaBound{j₁}\AgdaSpace{}%
\AgdaOperator{\AgdaFunction{⟨}}\AgdaSpace{}%
\AgdaFunction{rev}\AgdaSpace{}%
\AgdaFunction{♭}\AgdaSpace{}%
\AgdaOperator{\AgdaFunction{⟩}}\AgdaSymbol{)}\AgdaSpace{}%
\AgdaOperator{\AgdaInductiveConstructor{⊗}}\AgdaSpace{}%
\AgdaSymbol{(}\AgdaBound{j₀}\AgdaSpace{}%
\AgdaOperator{\AgdaFunction{⟨}}\AgdaSpace{}%
\AgdaFunction{rev}\AgdaSpace{}%
\AgdaFunction{♭}\AgdaSpace{}%
\AgdaOperator{\AgdaFunction{⟩}}\AgdaSymbol{))}\AgdaSpace{}%
\AgdaOperator{\AgdaFunction{⟨}}\AgdaSpace{}%
\AgdaInductiveConstructor{split}\AgdaSpace{}%
\AgdaOperator{\AgdaFunction{⟩}}\AgdaSymbol{))}\<%
\\
%
\>[14]\AgdaSymbol{)}\<%
\\
\>[.][@{}l@{}]\<[2513I]%
\>[8]\AgdaFunction{≡⟨}\AgdaSpace{}%
\AgdaFunction{sum-reindex}\AgdaSpace{}%
\AgdaSymbol{(}\AgdaFunction{|s|≡|sᵗ|}\AgdaSpace{}%
\AgdaSymbol{\{}\AgdaBound{r₁}\AgdaSpace{}%
\AgdaOperator{\AgdaInductiveConstructor{⊗}}\AgdaSpace{}%
\AgdaBound{r₂}\AgdaSymbol{\})}\AgdaSpace{}%
\AgdaFunction{⟩}\<%
\\
\>[8][@{}l@{\AgdaIndent{0}}]%
\>[10]\AgdaFunction{sum}\AgdaSpace{}%
\AgdaSymbol{\{}\AgdaOperator{\AgdaFunction{\#}}\AgdaSpace{}%
\AgdaSymbol{(}\AgdaBound{r₁}\AgdaSpace{}%
\AgdaOperator{\AgdaInductiveConstructor{⊗}}\AgdaSpace{}%
\AgdaBound{r₂}\AgdaSymbol{)}\AgdaSpace{}%
\AgdaOperator{\AgdaFunction{ᵗ}}\AgdaSymbol{\}}\AgdaSpace{}%
\AgdaSymbol{\AgdaUnderscore{}}\<%
\\
%
\>[8]\AgdaFunction{≡⟨}\AgdaSpace{}%
\AgdaFunction{sum-cong}\<%
\\
\>[8][@{}l@{\AgdaIndent{0}}]%
\>[10]\AgdaSymbol{\{}\AgdaOperator{\AgdaFunction{\#}}\AgdaSpace{}%
\AgdaSymbol{(}\AgdaBound{r₁}\AgdaSpace{}%
\AgdaOperator{\AgdaInductiveConstructor{⊗}}\AgdaSpace{}%
\AgdaBound{r₂}\AgdaSymbol{)}\AgdaSpace{}%
\AgdaOperator{\AgdaFunction{ᵗ}}\AgdaSymbol{\}}\<%
\\
%
\>[10]\AgdaSymbol{(λ}%
\>[2582I]\AgdaBound{k}\AgdaSpace{}%
\AgdaSymbol{→}\<%
\\
\>[2582I][@{}l@{\AgdaIndent{0}}]%
\>[14]\AgdaFunction{cong₂}\AgdaSpace{}%
\AgdaOperator{\AgdaField{\AgdaUnderscore{}*\AgdaUnderscore{}}}\<%
\\
\>[14][@{}l@{\AgdaIndent{0}}]%
\>[16]\AgdaInductiveConstructor{refl}\<%
\\
%
\>[16]\AgdaSymbol{(}\AgdaFunction{-ω-cong₂}\<%
\\
\>[16][@{}l@{\AgdaIndent{0}}]%
\>[18]\AgdaInductiveConstructor{refl}\<%
\\
%
\>[18]\AgdaSymbol{(}\AgdaFunction{cong₂}\AgdaSpace{}%
\AgdaOperator{\AgdaPrimitive{\AgdaUnderscore{}*ₙ\AgdaUnderscore{}}}\AgdaSpace{}%
\AgdaSymbol{(}\AgdaFunction{iota-reindex}\AgdaSpace{}%
\AgdaSymbol{(}\AgdaFunction{|s|≡|sᵗ|}\AgdaSpace{}%
\AgdaSymbol{\{}\AgdaBound{r₁}\AgdaSpace{}%
\AgdaOperator{\AgdaInductiveConstructor{⊗}}\AgdaSpace{}%
\AgdaBound{r₂}\AgdaSymbol{\}))}\AgdaSpace{}%
\AgdaInductiveConstructor{refl}\AgdaSymbol{)}\<%
\\
%
\>[16]\AgdaSymbol{)}\<%
\\
%
\>[10]\AgdaSymbol{)}\<%
\\
%
\>[8]\AgdaFunction{⟩}\<%
\\
%
\>[8]\AgdaSymbol{(}\AgdaFunction{reshape}\AgdaSpace{}%
\AgdaFunction{♯}\AgdaSpace{}%
\AgdaOperator{\AgdaFunction{∘}}\AgdaSpace{}%
\AgdaSymbol{(}\AgdaFunction{DFT′}\AgdaSpace{}%
\AgdaSymbol{\{}\AgdaFunction{length}\AgdaSpace{}%
\AgdaSymbol{(}\AgdaFunction{recursive-transpose}\AgdaSpace{}%
\AgdaSymbol{(}\AgdaBound{r₁}\AgdaSpace{}%
\AgdaOperator{\AgdaInductiveConstructor{⊗}}\AgdaSpace{}%
\AgdaBound{r₂}\AgdaSymbol{))\})}\AgdaSpace{}%
\AgdaOperator{\AgdaFunction{∘}}\AgdaSpace{}%
\AgdaFunction{reshape}\AgdaSpace{}%
\AgdaFunction{flatten-reindex}\AgdaSymbol{)}\AgdaSpace{}%
\AgdaBound{arr}\AgdaSpace{}%
\AgdaSymbol{(}\AgdaBound{j₁}\AgdaSpace{}%
\AgdaOperator{\AgdaInductiveConstructor{⊗}}\AgdaSpace{}%
\AgdaBound{j₀}\AgdaSymbol{)}\<%
\\
%
\>[6]\AgdaOperator{\AgdaFunction{∎}}\<%
\end{code}
\end{AgdaMultiCode}
\subsubsection{Cooley Tukey Derivation}
\begin{code}[hide]%
\>[0]\AgdaKeyword{module}\AgdaSpace{}%
\AgdaModule{ct-derivation}\AgdaSpace{}%
\AgdaSymbol{(}\AgdaBound{real}\AgdaSpace{}%
\AgdaSymbol{:}\AgdaSpace{}%
\AgdaRecord{Real}\AgdaSymbol{)}\AgdaSpace{}%
\AgdaSymbol{(}\AgdaBound{cplx}\AgdaSpace{}%
\AgdaSymbol{:}\AgdaSpace{}%
\AgdaRecord{Cplx}\AgdaSpace{}%
\AgdaBound{real}\AgdaSymbol{)}\AgdaSpace{}%
\AgdaKeyword{where}\<%
\\
\>[0][@{}l@{\AgdaIndent{0}}]%
\>[2]\AgdaKeyword{open}\AgdaSpace{}%
\AgdaModule{Real.Real}\AgdaSpace{}%
\AgdaBound{real}\AgdaSpace{}%
\AgdaKeyword{using}\AgdaSpace{}%
\AgdaSymbol{(}\AgdaOperator{\AgdaField{\AgdaUnderscore{}ᵣ}}\AgdaSymbol{;}\AgdaSpace{}%
\AgdaField{ℝ}\AgdaSymbol{)}\<%
\\
\>[2][@{}l@{\AgdaIndent{0}}]%
\>[4]\AgdaKeyword{renaming}\AgdaSpace{}%
\AgdaSymbol{(}\AgdaOperator{\AgdaField{\AgdaUnderscore{}+\AgdaUnderscore{}}}\AgdaSpace{}%
\AgdaSymbol{to}\AgdaSpace{}%
\AgdaOperator{\AgdaField{\AgdaUnderscore{}+ᵣ\AgdaUnderscore{}}}\AgdaSymbol{;}\AgdaSpace{}%
\AgdaOperator{\AgdaField{\AgdaUnderscore{}-\AgdaUnderscore{}}}\AgdaSpace{}%
\AgdaSymbol{to}\AgdaSpace{}%
\AgdaOperator{\AgdaField{\AgdaUnderscore{}-ᵣ\AgdaUnderscore{}}}\AgdaSymbol{;}\AgdaSpace{}%
\AgdaOperator{\AgdaField{-\AgdaUnderscore{}}}\AgdaSpace{}%
\AgdaSymbol{to}\AgdaSpace{}%
\AgdaOperator{\AgdaField{-ᵣ\AgdaUnderscore{}}}\AgdaSymbol{;}\AgdaSpace{}%
\AgdaOperator{\AgdaField{\AgdaUnderscore{}/\AgdaUnderscore{}}}\AgdaSpace{}%
\AgdaSymbol{to}\AgdaSpace{}%
\AgdaOperator{\AgdaField{\AgdaUnderscore{}/ᵣ\AgdaUnderscore{}}}\AgdaSymbol{;}\AgdaSpace{}%
\AgdaOperator{\AgdaField{\AgdaUnderscore{}*\AgdaUnderscore{}}}\AgdaSpace{}%
\AgdaSymbol{to}\AgdaSpace{}%
\AgdaOperator{\AgdaField{\AgdaUnderscore{}*ᵣ\AgdaUnderscore{}}}\AgdaSymbol{)}\<%
\\
%
\>[2]\AgdaKeyword{open}\AgdaSpace{}%
\AgdaModule{Cplx}\AgdaSpace{}%
\AgdaBound{cplx}\AgdaSpace{}%
\AgdaKeyword{using}\AgdaSpace{}%
\AgdaSymbol{(}\AgdaField{ℂ}\AgdaSymbol{;}\AgdaSpace{}%
\AgdaOperator{\AgdaField{\AgdaUnderscore{}+\AgdaUnderscore{}}}\AgdaSymbol{;}\AgdaSpace{}%
\AgdaField{fromℝ}\AgdaSymbol{;}\AgdaSpace{}%
\AgdaOperator{\AgdaField{\AgdaUnderscore{}*\AgdaUnderscore{}}}\AgdaSymbol{;}\AgdaSpace{}%
\AgdaField{-ω}\AgdaSymbol{;}\AgdaSpace{}%
\AgdaFunction{0ℂ}\AgdaSymbol{;}\AgdaSpace{}%
\AgdaField{+-*-isCommutativeRing}\AgdaSymbol{;}\AgdaSpace{}%
\AgdaField{ω-r₁x-r₁y}\AgdaSymbol{;}\AgdaSpace{}%
\AgdaField{ω-N-mN}\AgdaSymbol{;}\AgdaSpace{}%
\AgdaField{ω-N-k₀+k₁}\AgdaSymbol{)}\<%
\\
%
\\[\AgdaEmptyExtraSkip]%
%
\>[2]\AgdaKeyword{open}\AgdaSpace{}%
\AgdaModule{AlgebraStructures}%
\>[26]\AgdaSymbol{\{}\AgdaArgument{A}\AgdaSpace{}%
\AgdaSymbol{=}\AgdaSpace{}%
\AgdaField{ℂ}\AgdaSymbol{\}}\AgdaSpace{}%
\AgdaOperator{\AgdaDatatype{\AgdaUnderscore{}≡\AgdaUnderscore{}}}\<%
\\
%
\>[2]\AgdaKeyword{open}\AgdaSpace{}%
\AgdaModule{AlgebraDefinitions}\AgdaSpace{}%
\AgdaSymbol{\{}\AgdaArgument{A}\AgdaSpace{}%
\AgdaSymbol{=}\AgdaSpace{}%
\AgdaField{ℂ}\AgdaSymbol{\}}\AgdaSpace{}%
\AgdaOperator{\AgdaDatatype{\AgdaUnderscore{}≡\AgdaUnderscore{}}}\<%
\\
%
\\[\AgdaEmptyExtraSkip]%
%
\>[2]\AgdaKeyword{open}\AgdaSpace{}%
\AgdaModule{IsCommutativeRing}\AgdaSpace{}%
\AgdaField{+-*-isCommutativeRing}\AgdaSpace{}%
\AgdaKeyword{using}\AgdaSpace{}%
\AgdaSymbol{(}\AgdaFunction{+-isCommutativeMonoid}\AgdaSymbol{;}\AgdaSpace{}%
\AgdaFunction{distribˡ}\AgdaSymbol{;}\AgdaSpace{}%
\AgdaField{*-comm}\AgdaSymbol{;}\AgdaSpace{}%
\AgdaFunction{zeroʳ}\AgdaSymbol{;}\AgdaSpace{}%
\AgdaFunction{zeroˡ}\AgdaSymbol{;}\AgdaSpace{}%
\AgdaFunction{*-identityʳ}\AgdaSymbol{;}\AgdaSpace{}%
\AgdaFunction{*-assoc}\AgdaSymbol{;}\AgdaSpace{}%
\AgdaFunction{+-identityʳ}\AgdaSymbol{;}\AgdaSpace{}%
\AgdaFunction{+-assoc}\AgdaSymbol{;}\AgdaSpace{}%
\AgdaFunction{+-comm}\AgdaSymbol{;}\AgdaSpace{}%
\AgdaFunction{+-identityˡ}\AgdaSymbol{)}\<%
\\
%
\\[\AgdaEmptyExtraSkip]%
%
\>[2]\AgdaKeyword{open}\AgdaSpace{}%
\AgdaKeyword{import}\AgdaSpace{}%
\AgdaModule{Data.Nat.Base}\AgdaSpace{}%
\AgdaKeyword{using}\AgdaSpace{}%
\AgdaSymbol{(}\AgdaDatatype{ℕ}\AgdaSymbol{;}\AgdaSpace{}%
\AgdaInductiveConstructor{zero}\AgdaSymbol{;}\AgdaSpace{}%
\AgdaInductiveConstructor{suc}\AgdaSymbol{;}\AgdaSpace{}%
\AgdaRecord{NonZero}\AgdaSymbol{;}\AgdaSpace{}%
\AgdaOperator{\AgdaPrimitive{\AgdaUnderscore{}≡ᵇ\AgdaUnderscore{}}}\AgdaSymbol{;}\AgdaSpace{}%
\AgdaFunction{nonZero}\AgdaSymbol{)}\AgdaSpace{}%
\AgdaKeyword{renaming}\AgdaSpace{}%
\AgdaSymbol{(}\AgdaOperator{\AgdaPrimitive{\AgdaUnderscore{}*\AgdaUnderscore{}}}\AgdaSpace{}%
\AgdaSymbol{to}\AgdaSpace{}%
\AgdaOperator{\AgdaPrimitive{\AgdaUnderscore{}*ₙ\AgdaUnderscore{}}}\AgdaSymbol{;}\AgdaSpace{}%
\AgdaOperator{\AgdaPrimitive{\AgdaUnderscore{}+\AgdaUnderscore{}}}\AgdaSpace{}%
\AgdaSymbol{to}\AgdaSpace{}%
\AgdaOperator{\AgdaPrimitive{\AgdaUnderscore{}+ₙ\AgdaUnderscore{}}}\AgdaSymbol{)}\<%
\\
%
\>[2]\AgdaKeyword{open}\AgdaSpace{}%
\AgdaKeyword{import}\AgdaSpace{}%
\AgdaModule{Data.Nat.Properties}\AgdaSpace{}%
\AgdaKeyword{using}\AgdaSpace{}%
\AgdaSymbol{(}\AgdaFunction{suc-injective}\AgdaSymbol{;}\AgdaSpace{}%
\AgdaFunction{m*n≢0}\AgdaSymbol{;}\AgdaSpace{}%
\AgdaFunction{m*n≢0⇒m≢0}\AgdaSymbol{;}\AgdaSpace{}%
\AgdaFunction{m*n≢0⇒n≢0}\AgdaSymbol{;}\AgdaSpace{}%
\AgdaFunction{nonZero?}\AgdaSymbol{)}\AgdaSpace{}%
\AgdaKeyword{renaming}\AgdaSpace{}%
\AgdaSymbol{(}\AgdaFunction{*-comm}\AgdaSpace{}%
\AgdaSymbol{to}\AgdaSpace{}%
\AgdaFunction{*ₙ-comm}\AgdaSymbol{;}\AgdaSpace{}%
\AgdaFunction{*-identityʳ}\AgdaSpace{}%
\AgdaSymbol{to}\AgdaSpace{}%
\AgdaFunction{*ₙ-identityʳ}\AgdaSymbol{;}\AgdaSpace{}%
\AgdaFunction{*-assoc}\AgdaSpace{}%
\AgdaSymbol{to}\AgdaSpace{}%
\AgdaFunction{*ₙ-assoc}\AgdaSymbol{;}\<%
\\
\>[2][@{}l@{\AgdaIndent{0}}]%
\>[4]\AgdaFunction{+-identityʳ}\AgdaSpace{}%
\AgdaSymbol{to}\AgdaSpace{}%
\AgdaFunction{+ₙ-identityʳ}\AgdaSymbol{;}\AgdaSpace{}%
\AgdaFunction{*-zeroˡ}\AgdaSpace{}%
\AgdaSymbol{to}\AgdaSpace{}%
\AgdaFunction{*ₙ-zeroˡ}\AgdaSymbol{;}\AgdaSpace{}%
\AgdaFunction{*-zeroʳ}\AgdaSpace{}%
\AgdaSymbol{to}\AgdaSpace{}%
\AgdaFunction{*ₙ-zeroʳ}\AgdaSymbol{)}\<%
\\
%
\>[2]\AgdaKeyword{open}\AgdaSpace{}%
\AgdaKeyword{import}\AgdaSpace{}%
\AgdaModule{Data.Nat.Solver}\AgdaSpace{}%
\AgdaKeyword{using}\AgdaSpace{}%
\AgdaSymbol{(}\AgdaKeyword{module}\AgdaSpace{}%
\AgdaModule{+-*-Solver}\AgdaSymbol{)}\<%
\\
%
\>[2]\AgdaKeyword{open}\AgdaSpace{}%
\AgdaModule{+-*-Solver}\AgdaSpace{}%
\AgdaKeyword{using}\AgdaSpace{}%
\AgdaSymbol{(}\AgdaFunction{solve}\AgdaSymbol{;}\AgdaSpace{}%
\AgdaOperator{\AgdaFunction{\AgdaUnderscore{}:*\AgdaUnderscore{}}}\AgdaSymbol{;}\AgdaSpace{}%
\AgdaOperator{\AgdaFunction{\AgdaUnderscore{}:+\AgdaUnderscore{}}}\AgdaSymbol{;}\AgdaSpace{}%
\AgdaInductiveConstructor{con}\AgdaSymbol{;}\AgdaSpace{}%
\AgdaOperator{\AgdaFunction{\AgdaUnderscore{}:=\AgdaUnderscore{}}}\AgdaSymbol{)}\<%
\\
%
\>[2]\AgdaKeyword{open}\AgdaSpace{}%
\AgdaKeyword{import}\AgdaSpace{}%
\AgdaModule{Data.Fin.Base}\AgdaSpace{}%
\AgdaKeyword{using}\AgdaSpace{}%
\AgdaSymbol{(}\AgdaDatatype{Fin}\AgdaSymbol{;}\AgdaSpace{}%
\AgdaFunction{quotRem}\AgdaSymbol{;}\AgdaSpace{}%
\AgdaFunction{toℕ}\AgdaSymbol{;}\AgdaSpace{}%
\AgdaFunction{combine}\AgdaSymbol{;}\AgdaSpace{}%
\AgdaFunction{remQuot}\AgdaSymbol{;}\AgdaSpace{}%
\AgdaFunction{quotient}\AgdaSymbol{;}\AgdaSpace{}%
\AgdaFunction{remainder}\AgdaSymbol{;}\AgdaSpace{}%
\AgdaFunction{cast}\AgdaSymbol{;}\AgdaSpace{}%
\AgdaFunction{fromℕ<}\AgdaSymbol{;}\AgdaSpace{}%
\AgdaOperator{\AgdaFunction{\AgdaUnderscore{}↑ˡ\AgdaUnderscore{}}}\AgdaSymbol{;}\AgdaSpace{}%
\AgdaOperator{\AgdaFunction{\AgdaUnderscore{}↑ʳ\AgdaUnderscore{}}}\AgdaSymbol{;}\AgdaSpace{}%
\AgdaFunction{splitAt}\AgdaSymbol{;}\AgdaSpace{}%
\AgdaFunction{join}\AgdaSymbol{)}\AgdaSpace{}%
\AgdaKeyword{renaming}\AgdaSpace{}%
\AgdaSymbol{(}\AgdaInductiveConstructor{zero}\AgdaSpace{}%
\AgdaSymbol{to}\AgdaSpace{}%
\AgdaInductiveConstructor{fzero}\AgdaSymbol{;}\AgdaSpace{}%
\AgdaInductiveConstructor{suc}\AgdaSpace{}%
\AgdaSymbol{to}\AgdaSpace{}%
\AgdaInductiveConstructor{fsuc}\AgdaSymbol{)}\<%
\\
%
\>[2]\AgdaKeyword{open}\AgdaSpace{}%
\AgdaKeyword{import}\AgdaSpace{}%
\AgdaModule{Data.Fin.Properties}\AgdaSpace{}%
\AgdaKeyword{using}\AgdaSpace{}%
\AgdaSymbol{(}\AgdaFunction{cast-is-id}\AgdaSymbol{;}\AgdaSpace{}%
\AgdaFunction{remQuot-combine}\AgdaSymbol{;}\AgdaSpace{}%
\AgdaFunction{splitAt-↑ˡ}\AgdaSymbol{;}\AgdaSpace{}%
\AgdaFunction{splitAt-↑ʳ}\AgdaSymbol{;}\AgdaSpace{}%
\AgdaFunction{toℕ-↑ˡ}\AgdaSymbol{;}\AgdaSpace{}%
\AgdaFunction{toℕ-↑ʳ}\AgdaSymbol{;}\AgdaSpace{}%
\AgdaFunction{toℕ-combine}\AgdaSymbol{;}\AgdaSpace{}%
\AgdaFunction{combine-remQuot}\AgdaSymbol{;}\AgdaSpace{}%
\AgdaFunction{combine-surjective}\AgdaSymbol{;}\AgdaSpace{}%
\AgdaFunction{toℕ-injective}\AgdaSymbol{;}\AgdaSpace{}%
\AgdaFunction{toℕ-cast}\AgdaSymbol{;}\AgdaSpace{}%
\AgdaFunction{cast-trans}\AgdaSymbol{)}\<%
\\
%
\>[2]\AgdaKeyword{open}\AgdaSpace{}%
\AgdaKeyword{import}\AgdaSpace{}%
\AgdaModule{Data.Bool}\AgdaSpace{}%
\AgdaKeyword{using}\AgdaSpace{}%
\AgdaSymbol{(}\AgdaDatatype{Bool}\AgdaSymbol{;}\AgdaSpace{}%
\AgdaInductiveConstructor{true}\AgdaSymbol{;}\AgdaSpace{}%
\AgdaInductiveConstructor{false}\AgdaSymbol{;}\AgdaSpace{}%
\AgdaFunction{not}\AgdaSymbol{)}\<%
\\
%
\>[2]\AgdaKeyword{open}\AgdaSpace{}%
\AgdaKeyword{import}\AgdaSpace{}%
\AgdaModule{Data.Bool}\AgdaSpace{}%
\AgdaKeyword{using}\AgdaSpace{}%
\AgdaSymbol{(}\AgdaFunction{T}\AgdaSymbol{)}\<%
\\
%
\>[2]\AgdaKeyword{open}\AgdaSpace{}%
\AgdaKeyword{import}\AgdaSpace{}%
\AgdaModule{Data.Empty}\<%
\\
%
\\[\AgdaEmptyExtraSkip]%
%
\>[2]\AgdaKeyword{open}\AgdaSpace{}%
\AgdaKeyword{import}\AgdaSpace{}%
\AgdaModule{Data.Product.Base}\AgdaSpace{}%
\AgdaKeyword{using}\AgdaSpace{}%
\AgdaSymbol{(}\AgdaFunction{∃}\AgdaSymbol{;}\AgdaSpace{}%
\AgdaFunction{∃₂}\AgdaSymbol{;}\AgdaSpace{}%
\AgdaOperator{\AgdaFunction{\AgdaUnderscore{}×\AgdaUnderscore{}}}\AgdaSymbol{;}\AgdaSpace{}%
\AgdaField{proj₁}\AgdaSymbol{;}\AgdaSpace{}%
\AgdaField{proj₂}\AgdaSymbol{;}\AgdaSpace{}%
\AgdaFunction{map₁}\AgdaSymbol{;}\AgdaSpace{}%
\AgdaFunction{map₂}\AgdaSymbol{;}\AgdaSpace{}%
\AgdaFunction{uncurry}\AgdaSymbol{)}\AgdaSpace{}%
\AgdaKeyword{renaming}\AgdaSpace{}%
\AgdaSymbol{(}\AgdaSpace{}%
\AgdaOperator{\AgdaInductiveConstructor{\AgdaUnderscore{},\AgdaUnderscore{}}}\AgdaSpace{}%
\AgdaSymbol{to}\AgdaSpace{}%
\AgdaOperator{\AgdaInductiveConstructor{⟨\AgdaUnderscore{},\AgdaUnderscore{}⟩}}\AgdaSymbol{)}\<%
\\
%
\>[2]\AgdaKeyword{open}\AgdaSpace{}%
\AgdaKeyword{import}\AgdaSpace{}%
\AgdaModule{Data.Sum.Base}\AgdaSpace{}%
\AgdaKeyword{using}\AgdaSpace{}%
\AgdaSymbol{(}\AgdaInductiveConstructor{inj₁}\AgdaSymbol{;}\AgdaSpace{}%
\AgdaInductiveConstructor{inj₂}\AgdaSymbol{;}\AgdaSpace{}%
\AgdaOperator{\AgdaDatatype{\AgdaUnderscore{}⊎\AgdaUnderscore{}}}\AgdaSymbol{)}\<%
\\
%
\>[2]\AgdaKeyword{open}\AgdaSpace{}%
\AgdaKeyword{import}\AgdaSpace{}%
\AgdaModule{Data.Unit}\AgdaSpace{}%
\AgdaKeyword{using}\AgdaSpace{}%
\AgdaSymbol{(}\AgdaRecord{⊤}\AgdaSymbol{;}\AgdaSpace{}%
\AgdaInductiveConstructor{tt}\AgdaSymbol{)}\<%
\\
%
\\[\AgdaEmptyExtraSkip]%
%
\>[2]\AgdaKeyword{open}\AgdaSpace{}%
\AgdaKeyword{import}\AgdaSpace{}%
\AgdaModule{Matrix}\AgdaSpace{}%
\AgdaKeyword{using}\AgdaSpace{}%
\AgdaSymbol{(}\AgdaFunction{Ar}\AgdaSymbol{;}\AgdaSpace{}%
\AgdaDatatype{Shape}\AgdaSymbol{;}\AgdaSpace{}%
\AgdaOperator{\AgdaInductiveConstructor{\AgdaUnderscore{}⊗\AgdaUnderscore{}}}\AgdaSymbol{;}\AgdaSpace{}%
\AgdaInductiveConstructor{ι}\AgdaSymbol{;}\AgdaSpace{}%
\AgdaDatatype{Position}\AgdaSymbol{;}\AgdaSpace{}%
\AgdaFunction{mapRows}\AgdaSymbol{;}\AgdaSpace{}%
\AgdaFunction{zipWith}\AgdaSymbol{;}\AgdaSpace{}%
\AgdaFunction{nest}\AgdaSymbol{;}\AgdaSpace{}%
\AgdaFunction{map}\AgdaSymbol{;}\AgdaSpace{}%
\AgdaFunction{unnest}\AgdaSymbol{;}\AgdaSpace{}%
\AgdaFunction{head₁}\AgdaSymbol{;}\AgdaSpace{}%
\AgdaFunction{tail₁}\AgdaSymbol{;}\AgdaSpace{}%
\AgdaFunction{zip}\AgdaSymbol{;}\AgdaSpace{}%
\AgdaFunction{iterate}\AgdaSymbol{;}\AgdaSpace{}%
\AgdaFunction{ι-cons}\AgdaSymbol{;}\AgdaSpace{}%
\AgdaFunction{nil}\AgdaSymbol{;}\AgdaSpace{}%
\AgdaFunction{length}\AgdaSymbol{;}\AgdaSpace{}%
\AgdaFunction{splitAr}\AgdaSymbol{;}\AgdaSpace{}%
\AgdaFunction{splitArₗ}\AgdaSymbol{;}\AgdaSpace{}%
\AgdaFunction{splitArᵣ}\AgdaSymbol{)}\<%
\\
%
\>[2]\AgdaKeyword{open}\AgdaSpace{}%
\AgdaKeyword{import}\AgdaSpace{}%
\AgdaModule{Matrix.Equality}\AgdaSpace{}%
\AgdaKeyword{using}\AgdaSpace{}%
\AgdaSymbol{(}\AgdaOperator{\AgdaFunction{\AgdaUnderscore{}≅\AgdaUnderscore{}}}\AgdaSymbol{;}\AgdaSpace{}%
\AgdaFunction{reduce-≅}\AgdaSymbol{;}\AgdaSpace{}%
\AgdaFunction{tail₁-cong}\AgdaSymbol{)}\<%
\\
%
\>[2]\AgdaKeyword{open}\AgdaSpace{}%
\AgdaKeyword{import}\AgdaSpace{}%
\AgdaModule{Matrix.Properties}\AgdaSpace{}%
\AgdaKeyword{using}\AgdaSpace{}%
\AgdaSymbol{(}\AgdaFunction{splitArᵣ-zero}\AgdaSymbol{;}\AgdaSpace{}%
\AgdaFunction{tail₁-const}\AgdaSymbol{;}\AgdaSpace{}%
\AgdaFunction{zipWith-congˡ}\AgdaSymbol{)}\<%
\\
%
\>[2]\AgdaKeyword{open}\AgdaSpace{}%
\AgdaKeyword{import}\AgdaSpace{}%
\AgdaModule{Matrix.NonZero}\AgdaSpace{}%
\AgdaKeyword{using}\AgdaSpace{}%
\AgdaSymbol{(}\AgdaDatatype{NonZeroₛ}\AgdaSymbol{;}\AgdaSpace{}%
\AgdaInductiveConstructor{ι}\AgdaSymbol{;}\AgdaSpace{}%
\AgdaOperator{\AgdaInductiveConstructor{\AgdaUnderscore{}⊗\AgdaUnderscore{}}}\AgdaSymbol{;}\AgdaSpace{}%
\AgdaFunction{nonZeroₛ-s⇒nonZero-s}\AgdaSymbol{;}\AgdaSpace{}%
\AgdaFunction{nonZeroDec}\AgdaSymbol{;}\AgdaSpace{}%
\AgdaFunction{nonZeroₛ-s⇒nonZeroₛ-sᵗ}\AgdaSymbol{;}\AgdaSpace{}%
\AgdaFunction{nonZeroₛ-s⇒nonZero-sᵗ}\AgdaSymbol{;}\AgdaSpace{}%
\AgdaFunction{¬nonZeroₛ-s⇒¬nonZero-sᵗ}\AgdaSymbol{;}\AgdaSpace{}%
\AgdaFunction{¬nonZero-N⇒PosN-irrelevant}\AgdaSymbol{;}\AgdaSpace{}%
\AgdaFunction{¬nonZero-sᵗ⇒¬nonZero-s}\AgdaSymbol{)}\<%
\\
%
\\[\AgdaEmptyExtraSkip]%
%
\>[2]\AgdaKeyword{import}\AgdaSpace{}%
\AgdaModule{Matrix.Sum}\AgdaSpace{}%
\AgdaSymbol{as}\AgdaSpace{}%
\AgdaModule{S}\<%
\\
%
\>[2]\AgdaKeyword{open}\AgdaSpace{}%
\AgdaModule{S}\AgdaSpace{}%
\AgdaOperator{\AgdaField{\AgdaUnderscore{}+\AgdaUnderscore{}}}\AgdaSpace{}%
\AgdaFunction{0ℂ}\AgdaSpace{}%
\AgdaFunction{+-isCommutativeMonoid}\AgdaSpace{}%
\AgdaKeyword{using}\AgdaSpace{}%
\AgdaSymbol{(}\AgdaFunction{merge-sum}\AgdaSymbol{;}\AgdaSpace{}%
\AgdaFunction{sum-reindex}\AgdaSymbol{;}\AgdaSpace{}%
\AgdaFunction{sum-swap}\AgdaSymbol{)}\<%
\\
%
\>[2]\AgdaFunction{sum}\AgdaSpace{}%
\AgdaSymbol{=}\AgdaSpace{}%
\AgdaFunction{S.sum}\AgdaSpace{}%
\AgdaOperator{\AgdaField{\AgdaUnderscore{}+\AgdaUnderscore{}}}\AgdaSpace{}%
\AgdaFunction{0ℂ}\AgdaSpace{}%
\AgdaFunction{+-isCommutativeMonoid}\<%
\\
%
\>[2]\AgdaSymbol{\{-\#}\AgdaSpace{}%
\AgdaKeyword{DISPLAY}\AgdaSpace{}%
\AgdaFunction{S.sum}\AgdaSpace{}%
\AgdaOperator{\AgdaBound{\AgdaUnderscore{}+\AgdaUnderscore{}}}\AgdaSpace{}%
\AgdaBound{0ℂ}\AgdaSpace{}%
\AgdaBound{+-isCommutativeMonoid}\AgdaSpace{}%
\AgdaPragma{=}\AgdaSpace{}%
\AgdaFunction{sum}\AgdaSpace{}%
\AgdaSymbol{\#-\}}\<%
\\
%
\>[2]\AgdaFunction{sum-cong}\AgdaSpace{}%
\AgdaSymbol{=}\AgdaSpace{}%
\AgdaFunction{S.sum-cong}\AgdaSpace{}%
\AgdaOperator{\AgdaField{\AgdaUnderscore{}+\AgdaUnderscore{}}}\AgdaSpace{}%
\AgdaFunction{0ℂ}\AgdaSpace{}%
\AgdaFunction{+-isCommutativeMonoid}\<%
\\
%
\\[\AgdaEmptyExtraSkip]%
%
\>[2]\AgdaKeyword{open}\AgdaSpace{}%
\AgdaKeyword{import}\AgdaSpace{}%
\AgdaModule{Matrix.Reshape}\AgdaSpace{}%
\AgdaKeyword{using}\AgdaSpace{}%
\AgdaSymbol{(}\AgdaFunction{reshape}\AgdaSymbol{;}\AgdaSpace{}%
\AgdaDatatype{Reshape}\AgdaSymbol{;}\AgdaSpace{}%
\AgdaInductiveConstructor{flat}\AgdaSymbol{;}\AgdaSpace{}%
\AgdaFunction{♭}\AgdaSymbol{;}\AgdaSpace{}%
\AgdaFunction{♯}\AgdaSymbol{;}\AgdaSpace{}%
\AgdaFunction{recursive-transpose}\AgdaSymbol{;}\AgdaSpace{}%
\AgdaFunction{recursive-transposeᵣ}\AgdaSymbol{;}\AgdaSpace{}%
\AgdaOperator{\AgdaInductiveConstructor{\AgdaUnderscore{}∙\AgdaUnderscore{}}}\AgdaSymbol{;}\AgdaSpace{}%
\AgdaFunction{rev}\AgdaSymbol{;}\AgdaSpace{}%
\AgdaOperator{\AgdaInductiveConstructor{\AgdaUnderscore{}⊕\AgdaUnderscore{}}}\AgdaSymbol{;}\AgdaSpace{}%
\AgdaInductiveConstructor{swap}\AgdaSymbol{;}\AgdaSpace{}%
\AgdaInductiveConstructor{eq}\AgdaSymbol{;}\AgdaSpace{}%
\AgdaInductiveConstructor{split}\AgdaSymbol{;}\AgdaSpace{}%
\AgdaOperator{\AgdaFunction{\AgdaUnderscore{}⟨\AgdaUnderscore{}⟩}}\AgdaSymbol{;}\AgdaSpace{}%
\AgdaFunction{reindex}\AgdaSymbol{;}\AgdaSpace{}%
\AgdaFunction{rev-eq}\AgdaSymbol{;}\AgdaSpace{}%
\AgdaFunction{flatten-reindex}\AgdaSymbol{;}\AgdaSpace{}%
\AgdaFunction{|s|≡|sᵗ|}\AgdaSymbol{;}\AgdaSpace{}%
\AgdaFunction{reindex-reindex}\AgdaSymbol{;}\AgdaSpace{}%
\AgdaFunction{recursive-transpose-inv}\AgdaSymbol{)}\<%
\\
%
\>[2]\AgdaKeyword{open}\AgdaSpace{}%
\AgdaKeyword{import}\AgdaSpace{}%
\AgdaModule{Function.Base}\AgdaSpace{}%
\AgdaKeyword{using}\AgdaSpace{}%
\AgdaSymbol{(}\AgdaOperator{\AgdaFunction{\AgdaUnderscore{}\$\AgdaUnderscore{}}}\AgdaSymbol{;}\AgdaSpace{}%
\AgdaFunction{id}\AgdaSymbol{;}\AgdaSpace{}%
\AgdaOperator{\AgdaFunction{\AgdaUnderscore{}∘\AgdaUnderscore{}}}\AgdaSymbol{;}\AgdaSpace{}%
\AgdaFunction{flip}\AgdaSymbol{;}\AgdaSpace{}%
\AgdaOperator{\AgdaFunction{\AgdaUnderscore{}∘₂\AgdaUnderscore{}}}\AgdaSymbol{)}\<%
\\
%
\\[\AgdaEmptyExtraSkip]%
%
\>[2]\AgdaKeyword{open}\AgdaSpace{}%
\AgdaKeyword{import}\AgdaSpace{}%
\AgdaModule{FFT}\AgdaSpace{}%
\AgdaBound{real}\AgdaSpace{}%
\AgdaBound{cplx}\AgdaSpace{}%
\AgdaKeyword{using}\AgdaSpace{}%
\AgdaSymbol{(}\AgdaFunction{DFT}\AgdaSymbol{;}\AgdaSpace{}%
\AgdaFunction{FFT}\AgdaSymbol{;}\AgdaSpace{}%
\AgdaFunction{DFT′}\AgdaSymbol{;}\AgdaSpace{}%
\AgdaFunction{FFT′}\AgdaSymbol{;}\AgdaSpace{}%
\AgdaFunction{offset-prod}\AgdaSymbol{;}\AgdaSpace{}%
\AgdaFunction{iota}\AgdaSymbol{;}\AgdaSpace{}%
\AgdaFunction{twiddles}\AgdaSymbol{)}\<%
\\
%
\\[\AgdaEmptyExtraSkip]%
%
\>[2]\AgdaKeyword{private}\<%
\\
\>[2][@{}l@{\AgdaIndent{0}}]%
\>[4]\AgdaKeyword{variable}\<%
\\
\>[4][@{}l@{\AgdaIndent{0}}]%
\>[6]\AgdaGeneralizable{s}\AgdaSpace{}%
\AgdaGeneralizable{p}\AgdaSpace{}%
\AgdaGeneralizable{r₁}\AgdaSpace{}%
\AgdaGeneralizable{r₂}\AgdaSpace{}%
\AgdaSymbol{:}\AgdaSpace{}%
\AgdaDatatype{Shape}\<%
\\
%
\>[6]\AgdaGeneralizable{N}\AgdaSpace{}%
\AgdaGeneralizable{M}\AgdaSpace{}%
\AgdaSymbol{:}\AgdaSpace{}%
\AgdaDatatype{ℕ}\<%
\\
%
\\[\AgdaEmptyExtraSkip]%
%
\>[2]\AgdaComment{-----------------------------------------}\<%
\\
%
\>[2]\AgdaComment{---\ Shorthands\ to\ improve\ readability\ ---}\<%
\\
%
\>[2]\AgdaComment{-----------------------------------------}\<%
\\
%
\\[\AgdaEmptyExtraSkip]%
%
\>[2]\AgdaKeyword{infixl}\AgdaSpace{}%
\AgdaNumber{5}\AgdaSpace{}%
\AgdaOperator{\AgdaFunction{\AgdaUnderscore{}⊡\AgdaUnderscore{}}}\<%
\\
%
\>[2]\AgdaOperator{\AgdaFunction{\AgdaUnderscore{}⊡\AgdaUnderscore{}}}\AgdaSpace{}%
\AgdaSymbol{=}\AgdaSpace{}%
\AgdaFunction{trans}\<%
\\
%
\\[\AgdaEmptyExtraSkip]%
%
\>[2]\AgdaKeyword{infix}\AgdaSpace{}%
\AgdaNumber{10}\AgdaSpace{}%
\AgdaOperator{\AgdaFunction{\#\AgdaUnderscore{}}}\<%
\\
%
\>[2]\AgdaOperator{\AgdaFunction{\#\AgdaUnderscore{}}}\AgdaSpace{}%
\AgdaSymbol{:}\AgdaSpace{}%
\AgdaDatatype{Shape}\AgdaSpace{}%
\AgdaSymbol{→}\AgdaSpace{}%
\AgdaDatatype{ℕ}\<%
\\
%
\>[2]\AgdaOperator{\AgdaFunction{\#\AgdaUnderscore{}}}\AgdaSpace{}%
\AgdaSymbol{=}\AgdaSpace{}%
\AgdaFunction{length}\<%
\\
%
\\[\AgdaEmptyExtraSkip]%
%
\>[2]\AgdaKeyword{infix}\AgdaSpace{}%
\AgdaNumber{11}\AgdaSpace{}%
\AgdaOperator{\AgdaFunction{\AgdaUnderscore{}ᵗ}}\<%
\\
%
\>[2]\AgdaOperator{\AgdaFunction{\AgdaUnderscore{}ᵗ}}\AgdaSpace{}%
\AgdaSymbol{:}\AgdaSpace{}%
\AgdaDatatype{Shape}\AgdaSpace{}%
\AgdaSymbol{→}\AgdaSpace{}%
\AgdaDatatype{Shape}\<%
\\
%
\>[2]\AgdaOperator{\AgdaFunction{\AgdaUnderscore{}ᵗ}}\AgdaSpace{}%
\AgdaSymbol{=}\AgdaSpace{}%
\AgdaFunction{recursive-transpose}\<%
\\
%
\\[\AgdaEmptyExtraSkip]%
%
\>[2]\AgdaFunction{nz-\#}\AgdaSpace{}%
\AgdaSymbol{:}\AgdaSpace{}%
\AgdaDatatype{NonZeroₛ}\AgdaSpace{}%
\AgdaGeneralizable{s}\AgdaSpace{}%
\AgdaSymbol{→}\AgdaSpace{}%
\AgdaRecord{NonZero}\AgdaSpace{}%
\AgdaSymbol{(}\AgdaFunction{length}\AgdaSpace{}%
\AgdaGeneralizable{s}\AgdaSymbol{)}\<%
\\
%
\>[2]\AgdaFunction{nz-\#}\AgdaSpace{}%
\AgdaSymbol{=}\AgdaSpace{}%
\AgdaFunction{nonZeroₛ-s⇒nonZero-s}\<%
\\
%
\\[\AgdaEmptyExtraSkip]%
%
\>[2]\AgdaFunction{nz-ι\#}\AgdaSpace{}%
\AgdaSymbol{:}\AgdaSpace{}%
\AgdaDatatype{NonZeroₛ}\AgdaSpace{}%
\AgdaGeneralizable{s}\AgdaSpace{}%
\AgdaSymbol{→}\AgdaSpace{}%
\AgdaDatatype{NonZeroₛ}\AgdaSpace{}%
\AgdaSymbol{(}\AgdaInductiveConstructor{ι}\AgdaSpace{}%
\AgdaSymbol{(}\AgdaFunction{length}\AgdaSpace{}%
\AgdaGeneralizable{s}\AgdaSymbol{))}\<%
\\
%
\>[2]\AgdaFunction{nz-ι\#}\AgdaSpace{}%
\AgdaSymbol{(}\AgdaInductiveConstructor{ι}\AgdaSpace{}%
\AgdaBound{x}\AgdaSymbol{)}\AgdaSpace{}%
\AgdaSymbol{=}\AgdaSpace{}%
\AgdaInductiveConstructor{ι}\AgdaSpace{}%
\AgdaBound{x}\<%
\\
%
\>[2]\AgdaFunction{nz-ι\#}\AgdaSpace{}%
\AgdaSymbol{\{}\AgdaBound{s}\AgdaSpace{}%
\AgdaOperator{\AgdaInductiveConstructor{⊗}}\AgdaSpace{}%
\AgdaBound{p}\AgdaSymbol{\}}\AgdaSpace{}%
\AgdaSymbol{(}\AgdaBound{nz-s}\AgdaSpace{}%
\AgdaOperator{\AgdaInductiveConstructor{⊗}}\AgdaSpace{}%
\AgdaBound{nz-p}\AgdaSymbol{)}\AgdaSpace{}%
\AgdaSymbol{=}\AgdaSpace{}%
\AgdaInductiveConstructor{ι}\AgdaSpace{}%
\AgdaSymbol{(}\AgdaFunction{m*n≢0}\AgdaSpace{}%
\AgdaSymbol{(}\AgdaOperator{\AgdaFunction{\#}}\AgdaSpace{}%
\AgdaBound{s}\AgdaSymbol{)}\AgdaSpace{}%
\AgdaSymbol{(}\AgdaOperator{\AgdaFunction{\#}}\AgdaSpace{}%
\AgdaBound{p}\AgdaSymbol{)}\AgdaSpace{}%
\AgdaSymbol{⦃}\AgdaSpace{}%
\AgdaFunction{nz-\#}\AgdaSpace{}%
\AgdaBound{nz-s}\AgdaSpace{}%
\AgdaSymbol{⦄}\AgdaSpace{}%
\AgdaSymbol{⦃}\AgdaSpace{}%
\AgdaFunction{nz-\#}\AgdaSpace{}%
\AgdaBound{nz-p}\AgdaSpace{}%
\AgdaSymbol{⦄}\AgdaSpace{}%
\AgdaSymbol{)}\<%
\\
%
\\[\AgdaEmptyExtraSkip]%
%
\>[2]\AgdaFunction{-ω-cong₂}\AgdaSpace{}%
\AgdaSymbol{:}\<%
\\
\>[2][@{}l@{\AgdaIndent{0}}]%
\>[4]\AgdaSymbol{∀}\AgdaSpace{}%
\AgdaSymbol{\{}\AgdaBound{n}\AgdaSpace{}%
\AgdaBound{m}\AgdaSpace{}%
\AgdaSymbol{:}\AgdaSpace{}%
\AgdaDatatype{ℕ}\AgdaSymbol{\}}\<%
\\
%
\>[4]\AgdaSymbol{→}\AgdaSpace{}%
\AgdaSymbol{⦃}\AgdaSpace{}%
\AgdaBound{nonZero-n}\AgdaSpace{}%
\AgdaSymbol{:}\AgdaSpace{}%
\AgdaRecord{NonZero}\AgdaSpace{}%
\AgdaBound{n}\AgdaSpace{}%
\AgdaSymbol{⦄}\<%
\\
%
\>[4]\AgdaSymbol{→}\AgdaSpace{}%
\AgdaSymbol{⦃}\AgdaSpace{}%
\AgdaBound{nonZero-m}\AgdaSpace{}%
\AgdaSymbol{:}\AgdaSpace{}%
\AgdaRecord{NonZero}\AgdaSpace{}%
\AgdaBound{m}\AgdaSpace{}%
\AgdaSymbol{⦄}\<%
\\
%
\>[4]\AgdaSymbol{→}\AgdaSpace{}%
\AgdaSymbol{∀}\AgdaSpace{}%
\AgdaSymbol{\{}\AgdaBound{k}\AgdaSpace{}%
\AgdaBound{j}\AgdaSpace{}%
\AgdaSymbol{:}\AgdaSpace{}%
\AgdaDatatype{ℕ}\AgdaSymbol{\}}\<%
\\
%
\>[4]\AgdaSymbol{→}\AgdaSpace{}%
\AgdaSymbol{(}\AgdaBound{prfₗ}\AgdaSpace{}%
\AgdaSymbol{:}\AgdaSpace{}%
\AgdaBound{n}\AgdaSpace{}%
\AgdaOperator{\AgdaDatatype{≡}}\AgdaSpace{}%
\AgdaBound{m}\AgdaSymbol{)}\<%
\\
%
\>[4]\AgdaSymbol{→}\AgdaSpace{}%
\AgdaBound{k}\AgdaSpace{}%
\AgdaOperator{\AgdaDatatype{≡}}\AgdaSpace{}%
\AgdaBound{j}\<%
\\
%
\>[4]\AgdaSymbol{→}\AgdaSpace{}%
\AgdaField{-ω}\AgdaSpace{}%
\AgdaBound{n}\AgdaSpace{}%
\AgdaSymbol{⦃}\AgdaSpace{}%
\AgdaBound{nonZero-n}\AgdaSpace{}%
\AgdaSymbol{⦄}\AgdaSpace{}%
\AgdaBound{k}\AgdaSpace{}%
\AgdaOperator{\AgdaDatatype{≡}}\AgdaSpace{}%
\AgdaField{-ω}\AgdaSpace{}%
\AgdaBound{m}\AgdaSpace{}%
\AgdaSymbol{⦃}\AgdaSpace{}%
\AgdaBound{nonZero-m}\AgdaSpace{}%
\AgdaSymbol{⦄}\AgdaSpace{}%
\AgdaBound{j}\<%
\\
%
\>[2]\AgdaFunction{-ω-cong₂}\AgdaSpace{}%
\AgdaInductiveConstructor{refl}\AgdaSpace{}%
\AgdaInductiveConstructor{refl}\AgdaSpace{}%
\AgdaSymbol{=}\AgdaSpace{}%
\AgdaInductiveConstructor{refl}\<%
\\
\>[0]\<%
\end{code}
Using the rule that $c\times\sum_{i=0}^n{x_i}\equiv\sum_{i=0}^n{cx_i}$, the two
instances of -ω in the outer sum, can be moved into the inner sum.
With all instances of \AF{-ω} gathered, the rules of the roots of unity can be
used, following the inverse of the initial Cooley Tukey derivation to represent 
all roots of unity as one root of unity.
\todo[color=green]{Written like a six year old with their pen license, reword}
% \ref{sec:complex_numbers}

\begin{code}%
\>[0][@{}l@{\AgdaIndent{1}}]%
\>[2]\AgdaFunction{cooley-tukey-derivation}\AgdaSpace{}%
\AgdaSymbol{:}\<%
\\
\>[2][@{}l@{\AgdaIndent{0}}]%
\>[4]\AgdaSymbol{∀}\AgdaSpace{}%
\AgdaSymbol{(}\AgdaBound{r₁}\AgdaSpace{}%
\AgdaBound{r₂}\AgdaSpace{}%
\AgdaBound{k₀}\AgdaSpace{}%
\AgdaBound{k₁}\AgdaSpace{}%
\AgdaBound{j₀}\AgdaSpace{}%
\AgdaBound{j₁}\AgdaSpace{}%
\AgdaSymbol{:}\AgdaSpace{}%
\AgdaDatatype{ℕ}\AgdaSymbol{)}\<%
\\
%
\>[4]\AgdaSymbol{→}\AgdaSpace{}%
\AgdaSymbol{⦃}\AgdaSpace{}%
\AgdaBound{nonZero-r₁}%
\>[21]\AgdaSymbol{:}\AgdaSpace{}%
\AgdaRecord{NonZero}\AgdaSpace{}%
\AgdaBound{r₁}\AgdaSpace{}%
\AgdaSymbol{⦄}\<%
\\
%
\>[4]\AgdaSymbol{→}\AgdaSpace{}%
\AgdaSymbol{⦃}\AgdaSpace{}%
\AgdaBound{nonZero-r₂}%
\>[21]\AgdaSymbol{:}\AgdaSpace{}%
\AgdaRecord{NonZero}\AgdaSpace{}%
\AgdaBound{r₂}\AgdaSpace{}%
\AgdaSymbol{⦄}\<%
\\
%
\>[4]\AgdaSymbol{→}\<%
\\
\>[4][@{}l@{\AgdaIndent{0}}]%
\>[14]\AgdaField{-ω}\<%
\\
\>[14][@{}l@{\AgdaIndent{0}}]%
\>[16]\AgdaSymbol{(}\AgdaBound{r₂}\AgdaSpace{}%
\AgdaOperator{\AgdaPrimitive{*ₙ}}\AgdaSpace{}%
\AgdaBound{r₁}\AgdaSymbol{)}\<%
\\
%
\>[16]\AgdaSymbol{⦃}\AgdaSpace{}%
\AgdaFunction{m*n≢0}\AgdaSpace{}%
\AgdaBound{r₂}\AgdaSpace{}%
\AgdaBound{r₁}\AgdaSpace{}%
\AgdaSymbol{⦄}\<%
\\
%
\>[16]\AgdaSymbol{(}\<%
\\
\>[16][@{}l@{\AgdaIndent{0}}]%
\>[18]\AgdaSymbol{(}\AgdaBound{r₂}\AgdaSpace{}%
\AgdaOperator{\AgdaPrimitive{*ₙ}}\AgdaSpace{}%
\AgdaBound{k₁}\AgdaSpace{}%
\AgdaOperator{\AgdaPrimitive{+ₙ}}\AgdaSpace{}%
\AgdaBound{k₀}\AgdaSymbol{)}\<%
\\
%
\>[18]\AgdaOperator{\AgdaPrimitive{*ₙ}}\<%
\\
%
\>[18]\AgdaSymbol{(}\AgdaBound{r₁}\AgdaSpace{}%
\AgdaOperator{\AgdaPrimitive{*ₙ}}\AgdaSpace{}%
\AgdaBound{j₁}\AgdaSpace{}%
\AgdaOperator{\AgdaPrimitive{+ₙ}}\AgdaSpace{}%
\AgdaBound{j₀}\AgdaSymbol{)}\<%
\\
%
\>[16]\AgdaSymbol{)}\<%
\\
%
\>[14]\AgdaOperator{\AgdaDatatype{≡}}\<%
\\
\>[14][@{}l@{\AgdaIndent{0}}]%
\>[16]\AgdaField{-ω}\AgdaSpace{}%
\AgdaSymbol{(}\AgdaBound{r₁}\AgdaSymbol{)}\AgdaSpace{}%
\AgdaSymbol{(}\AgdaBound{k₁}\AgdaSpace{}%
\AgdaOperator{\AgdaPrimitive{*ₙ}}\AgdaSpace{}%
\AgdaBound{j₀}\AgdaSymbol{)}\<%
\\
%
\>[14]\AgdaOperator{\AgdaField{*}}\AgdaSpace{}%
\AgdaField{-ω}\AgdaSpace{}%
\AgdaSymbol{(}\AgdaBound{r₂}\AgdaSpace{}%
\AgdaOperator{\AgdaPrimitive{*ₙ}}\AgdaSpace{}%
\AgdaBound{r₁}\AgdaSymbol{)}\AgdaSpace{}%
\AgdaSymbol{⦃}\AgdaSpace{}%
\AgdaFunction{m*n≢0}\AgdaSpace{}%
\AgdaBound{r₂}\AgdaSpace{}%
\AgdaBound{r₁}\AgdaSpace{}%
\AgdaSymbol{⦄}\AgdaSpace{}%
\AgdaSymbol{(}\AgdaBound{k₀}\AgdaSpace{}%
\AgdaOperator{\AgdaPrimitive{*ₙ}}\AgdaSpace{}%
\AgdaBound{j₀}\AgdaSymbol{)}\<%
\\
%
\>[14]\AgdaOperator{\AgdaField{*}}\AgdaSpace{}%
\AgdaField{-ω}\AgdaSpace{}%
\AgdaSymbol{(}\AgdaBound{r₂}\AgdaSymbol{)}\AgdaSpace{}%
\AgdaSymbol{(}\AgdaBound{k₀}\AgdaSpace{}%
\AgdaOperator{\AgdaPrimitive{*ₙ}}\AgdaSpace{}%
\AgdaBound{j₁}\AgdaSymbol{)}\<%
\\
%
\>[2]\AgdaFunction{cooley-tukey-derivation}\AgdaSpace{}%
\AgdaBound{r₁}\AgdaSpace{}%
\AgdaBound{r₂}\AgdaSpace{}%
\AgdaBound{k₀}\AgdaSpace{}%
\AgdaBound{k₁}\AgdaSpace{}%
\AgdaBound{j₀}\AgdaSpace{}%
\AgdaBound{j₁}\AgdaSpace{}%
\AgdaSymbol{⦃}\AgdaSpace{}%
\AgdaBound{nonZero-r₁}\AgdaSpace{}%
\AgdaSymbol{⦄}\AgdaSpace{}%
\AgdaSymbol{⦃}\AgdaSpace{}%
\AgdaBound{nonZero-r₂}\AgdaSpace{}%
\AgdaSymbol{⦄}\<%
\\
\>[2][@{}l@{\AgdaIndent{0}}]%
\>[4]\AgdaSymbol{=}\AgdaSpace{}%
\AgdaFunction{rearrange-ω-power}\<%
\\
%
\>[4]\AgdaOperator{\AgdaFunction{⊡}}\AgdaSpace{}%
\AgdaFunction{split-ω}\<%
\\
%
\>[4]\AgdaOperator{\AgdaFunction{⊡}}\AgdaSpace{}%
\AgdaFunction{remove-constant-term}\<%
\\
%
\>[4]\AgdaOperator{\AgdaFunction{⊡}}\AgdaSpace{}%
\AgdaFunction{simplify-bases}\<%
\end{code}
\begin{code}[hide]%
%
\>[4]\AgdaKeyword{where}\<%
\\
\>[4][@{}l@{\AgdaIndent{0}}]%
\>[6]\AgdaKeyword{instance}\<%
\\
\>[6][@{}l@{\AgdaIndent{0}}]%
\>[8]\AgdaFunction{\AgdaUnderscore{}}\AgdaSpace{}%
\AgdaSymbol{:}\AgdaSpace{}%
\AgdaRecord{NonZero}\AgdaSpace{}%
\AgdaSymbol{(}\AgdaBound{r₁}\AgdaSpace{}%
\AgdaOperator{\AgdaPrimitive{*ₙ}}\AgdaSpace{}%
\AgdaBound{r₂}\AgdaSymbol{)}\<%
\\
%
\>[8]\AgdaSymbol{\AgdaUnderscore{}}\AgdaSpace{}%
\AgdaSymbol{=}\AgdaSpace{}%
\AgdaFunction{m*n≢0}\AgdaSpace{}%
\AgdaBound{r₁}\AgdaSpace{}%
\AgdaBound{r₂}\<%
\\
%
\>[8]\AgdaFunction{\AgdaUnderscore{}}\AgdaSpace{}%
\AgdaSymbol{:}\AgdaSpace{}%
\AgdaRecord{NonZero}\AgdaSpace{}%
\AgdaSymbol{(}\AgdaBound{r₂}\AgdaSpace{}%
\AgdaOperator{\AgdaPrimitive{*ₙ}}\AgdaSpace{}%
\AgdaBound{r₁}\AgdaSymbol{)}\<%
\\
%
\>[8]\AgdaSymbol{\AgdaUnderscore{}}\AgdaSpace{}%
\AgdaSymbol{=}\AgdaSpace{}%
\AgdaFunction{m*n≢0}\AgdaSpace{}%
\AgdaBound{r₂}\AgdaSpace{}%
\AgdaBound{r₁}\<%
\\
%
\>[6]\AgdaFunction{simplify-bases}\AgdaSpace{}%
\AgdaSymbol{:}\<%
\\
\>[6][@{}l@{\AgdaIndent{0}}]%
\>[12]\AgdaField{-ω}\AgdaSpace{}%
\AgdaSymbol{(}\AgdaBound{r₂}\AgdaSpace{}%
\AgdaOperator{\AgdaPrimitive{*ₙ}}\AgdaSpace{}%
\AgdaBound{r₁}\AgdaSymbol{)}\AgdaSpace{}%
\AgdaSymbol{⦃}\AgdaSpace{}%
\AgdaFunction{m*n≢0}\AgdaSpace{}%
\AgdaBound{r₂}\AgdaSpace{}%
\AgdaBound{r₁}\AgdaSpace{}%
\AgdaSymbol{⦄}\AgdaSpace{}%
\AgdaSymbol{(}\AgdaBound{r₂}\AgdaSpace{}%
\AgdaOperator{\AgdaPrimitive{*ₙ}}\AgdaSpace{}%
\AgdaSymbol{(}\AgdaBound{k₁}\AgdaSpace{}%
\AgdaOperator{\AgdaPrimitive{*ₙ}}\AgdaSpace{}%
\AgdaBound{j₀}\AgdaSymbol{))}\<%
\\
\>[6][@{}l@{\AgdaIndent{0}}]%
\>[10]\AgdaOperator{\AgdaField{*}}\AgdaSpace{}%
\AgdaField{-ω}\AgdaSpace{}%
\AgdaSymbol{(}\AgdaBound{r₂}\AgdaSpace{}%
\AgdaOperator{\AgdaPrimitive{*ₙ}}\AgdaSpace{}%
\AgdaBound{r₁}\AgdaSymbol{)}\AgdaSpace{}%
\AgdaSymbol{⦃}\AgdaSpace{}%
\AgdaFunction{m*n≢0}\AgdaSpace{}%
\AgdaBound{r₂}\AgdaSpace{}%
\AgdaBound{r₁}\AgdaSpace{}%
\AgdaSymbol{⦄}\AgdaSpace{}%
\AgdaSymbol{(}\AgdaBound{k₀}\AgdaSpace{}%
\AgdaOperator{\AgdaPrimitive{*ₙ}}\AgdaSpace{}%
\AgdaBound{j₀}\AgdaSymbol{)}\<%
\\
%
\>[10]\AgdaOperator{\AgdaField{*}}\AgdaSpace{}%
\AgdaField{-ω}\AgdaSpace{}%
\AgdaSymbol{(}\AgdaBound{r₂}\AgdaSpace{}%
\AgdaOperator{\AgdaPrimitive{*ₙ}}\AgdaSpace{}%
\AgdaBound{r₁}\AgdaSymbol{)}\AgdaSpace{}%
\AgdaSymbol{⦃}\AgdaSpace{}%
\AgdaFunction{m*n≢0}\AgdaSpace{}%
\AgdaBound{r₂}\AgdaSpace{}%
\AgdaBound{r₁}\AgdaSpace{}%
\AgdaSymbol{⦄}\AgdaSpace{}%
\AgdaSymbol{(}\AgdaBound{r₁}\AgdaSpace{}%
\AgdaOperator{\AgdaPrimitive{*ₙ}}\AgdaSpace{}%
\AgdaSymbol{(}\AgdaBound{k₀}\AgdaSpace{}%
\AgdaOperator{\AgdaPrimitive{*ₙ}}\AgdaSpace{}%
\AgdaBound{j₁}\AgdaSymbol{))}\<%
\\
\>[6][@{}l@{\AgdaIndent{0}}]%
\>[8]\AgdaOperator{\AgdaDatatype{≡}}\<%
\\
\>[8][@{}l@{\AgdaIndent{0}}]%
\>[12]\AgdaField{-ω}\AgdaSpace{}%
\AgdaSymbol{(}\AgdaBound{r₁}\AgdaSymbol{)}\AgdaSpace{}%
\AgdaSymbol{(}\AgdaBound{k₁}\AgdaSpace{}%
\AgdaOperator{\AgdaPrimitive{*ₙ}}\AgdaSpace{}%
\AgdaBound{j₀}\AgdaSymbol{)}\<%
\\
\>[8][@{}l@{\AgdaIndent{0}}]%
\>[10]\AgdaOperator{\AgdaField{*}}\AgdaSpace{}%
\AgdaField{-ω}\AgdaSpace{}%
\AgdaSymbol{(}\AgdaBound{r₂}\AgdaSpace{}%
\AgdaOperator{\AgdaPrimitive{*ₙ}}\AgdaSpace{}%
\AgdaBound{r₁}\AgdaSymbol{)}\AgdaSpace{}%
\AgdaSymbol{⦃}\AgdaSpace{}%
\AgdaFunction{m*n≢0}\AgdaSpace{}%
\AgdaBound{r₂}\AgdaSpace{}%
\AgdaBound{r₁}\AgdaSpace{}%
\AgdaSymbol{⦄}\AgdaSpace{}%
\AgdaSymbol{(}\AgdaBound{k₀}\AgdaSpace{}%
\AgdaOperator{\AgdaPrimitive{*ₙ}}\AgdaSpace{}%
\AgdaBound{j₀}\AgdaSymbol{)}\<%
\\
%
\>[10]\AgdaOperator{\AgdaField{*}}\AgdaSpace{}%
\AgdaField{-ω}\AgdaSpace{}%
\AgdaSymbol{(}\AgdaBound{r₂}\AgdaSymbol{)}\AgdaSpace{}%
\AgdaSymbol{(}\AgdaBound{k₀}\AgdaSpace{}%
\AgdaOperator{\AgdaPrimitive{*ₙ}}\AgdaSpace{}%
\AgdaBound{j₁}\AgdaSymbol{)}\<%
\\
%
\>[6]\AgdaFunction{simplify-bases}\AgdaSpace{}%
\AgdaSymbol{=}\<%
\\
\>[6][@{}l@{\AgdaIndent{0}}]%
\>[10]\AgdaFunction{cong₂}\<%
\\
\>[10][@{}l@{\AgdaIndent{0}}]%
\>[12]\AgdaOperator{\AgdaField{\AgdaUnderscore{}*\AgdaUnderscore{}}}\<%
\\
%
\>[12]\AgdaSymbol{(}%
\>[3193I]\AgdaFunction{cong₂}\<%
\\
\>[3193I][@{}l@{\AgdaIndent{0}}]%
\>[18]\AgdaOperator{\AgdaField{\AgdaUnderscore{}*\AgdaUnderscore{}}}\<%
\\
%
\>[18]\AgdaSymbol{(}\AgdaField{ω-r₁x-r₁y}\AgdaSpace{}%
\AgdaBound{r₂}\AgdaSpace{}%
\AgdaBound{r₁}\AgdaSpace{}%
\AgdaSymbol{(}\AgdaBound{k₁}\AgdaSpace{}%
\AgdaOperator{\AgdaPrimitive{*ₙ}}\AgdaSpace{}%
\AgdaBound{j₀}\AgdaSymbol{))}\<%
\\
%
\>[18]\AgdaInductiveConstructor{refl}\<%
\\
%
\>[12]\AgdaSymbol{)}\<%
\\
%
\>[12]\AgdaSymbol{(}%
\>[16]\AgdaFunction{-ω-cong₂}\AgdaSpace{}%
\AgdaSymbol{(}\AgdaFunction{*ₙ-comm}\AgdaSpace{}%
\AgdaBound{r₂}\AgdaSpace{}%
\AgdaBound{r₁}\AgdaSymbol{)}\AgdaSpace{}%
\AgdaInductiveConstructor{refl}\<%
\\
\>[12][@{}l@{\AgdaIndent{0}}]%
\>[14]\AgdaOperator{\AgdaFunction{⊡}}\AgdaSpace{}%
\AgdaSymbol{(}\AgdaField{ω-r₁x-r₁y}\AgdaSpace{}%
\AgdaBound{r₁}\AgdaSpace{}%
\AgdaBound{r₂}\AgdaSpace{}%
\AgdaSymbol{(}\AgdaBound{k₀}\AgdaSpace{}%
\AgdaOperator{\AgdaPrimitive{*ₙ}}\AgdaSpace{}%
\AgdaBound{j₁}\AgdaSymbol{))}\<%
\\
%
\>[12]\AgdaSymbol{)}\<%
\\
%
\>[6]\AgdaFunction{remove-constant-term}\AgdaSpace{}%
\AgdaSymbol{:}\<%
\\
\>[6][@{}l@{\AgdaIndent{0}}]%
\>[12]\AgdaField{-ω}\AgdaSpace{}%
\AgdaSymbol{(}\AgdaBound{r₂}\AgdaSpace{}%
\AgdaOperator{\AgdaPrimitive{*ₙ}}\AgdaSpace{}%
\AgdaBound{r₁}\AgdaSymbol{)}\AgdaSpace{}%
\AgdaSymbol{⦃}\AgdaSpace{}%
\AgdaFunction{m*n≢0}\AgdaSpace{}%
\AgdaBound{r₂}\AgdaSpace{}%
\AgdaBound{r₁}\AgdaSpace{}%
\AgdaSymbol{⦄}\AgdaSpace{}%
\AgdaSymbol{(}\AgdaBound{r₂}\AgdaSpace{}%
\AgdaOperator{\AgdaPrimitive{*ₙ}}\AgdaSpace{}%
\AgdaSymbol{(}\AgdaBound{k₁}\AgdaSpace{}%
\AgdaOperator{\AgdaPrimitive{*ₙ}}\AgdaSpace{}%
\AgdaBound{j₀}\AgdaSymbol{))}\<%
\\
\>[6][@{}l@{\AgdaIndent{0}}]%
\>[10]\AgdaOperator{\AgdaField{*}}\AgdaSpace{}%
\AgdaField{-ω}\AgdaSpace{}%
\AgdaSymbol{(}\AgdaBound{r₂}\AgdaSpace{}%
\AgdaOperator{\AgdaPrimitive{*ₙ}}\AgdaSpace{}%
\AgdaBound{r₁}\AgdaSymbol{)}\AgdaSpace{}%
\AgdaSymbol{⦃}\AgdaSpace{}%
\AgdaFunction{m*n≢0}\AgdaSpace{}%
\AgdaBound{r₂}\AgdaSpace{}%
\AgdaBound{r₁}\AgdaSpace{}%
\AgdaSymbol{⦄}\AgdaSpace{}%
\AgdaSymbol{(}\AgdaBound{k₀}\AgdaSpace{}%
\AgdaOperator{\AgdaPrimitive{*ₙ}}\AgdaSpace{}%
\AgdaBound{j₀}\AgdaSymbol{)}\<%
\\
%
\>[10]\AgdaOperator{\AgdaField{*}}\AgdaSpace{}%
\AgdaField{-ω}\AgdaSpace{}%
\AgdaSymbol{(}\AgdaBound{r₂}\AgdaSpace{}%
\AgdaOperator{\AgdaPrimitive{*ₙ}}\AgdaSpace{}%
\AgdaBound{r₁}\AgdaSymbol{)}\AgdaSpace{}%
\AgdaSymbol{⦃}\AgdaSpace{}%
\AgdaFunction{m*n≢0}\AgdaSpace{}%
\AgdaBound{r₂}\AgdaSpace{}%
\AgdaBound{r₁}\AgdaSpace{}%
\AgdaSymbol{⦄}\AgdaSpace{}%
\AgdaSymbol{(}\AgdaBound{r₁}\AgdaSpace{}%
\AgdaOperator{\AgdaPrimitive{*ₙ}}\AgdaSpace{}%
\AgdaSymbol{(}\AgdaBound{k₀}\AgdaSpace{}%
\AgdaOperator{\AgdaPrimitive{*ₙ}}\AgdaSpace{}%
\AgdaBound{j₁}\AgdaSymbol{))}\<%
\\
%
\>[10]\AgdaOperator{\AgdaField{*}}\AgdaSpace{}%
\AgdaField{-ω}\AgdaSpace{}%
\AgdaSymbol{(}\AgdaBound{r₂}\AgdaSpace{}%
\AgdaOperator{\AgdaPrimitive{*ₙ}}\AgdaSpace{}%
\AgdaBound{r₁}\AgdaSymbol{)}\AgdaSpace{}%
\AgdaSymbol{⦃}\AgdaSpace{}%
\AgdaFunction{m*n≢0}\AgdaSpace{}%
\AgdaBound{r₂}\AgdaSpace{}%
\AgdaBound{r₁}\AgdaSpace{}%
\AgdaSymbol{⦄}\AgdaSpace{}%
\AgdaSymbol{((}\AgdaBound{r₂}\AgdaSpace{}%
\AgdaOperator{\AgdaPrimitive{*ₙ}}\AgdaSpace{}%
\AgdaBound{r₁}\AgdaSymbol{)}\AgdaSpace{}%
\AgdaOperator{\AgdaPrimitive{*ₙ}}\AgdaSpace{}%
\AgdaSymbol{(}\AgdaBound{j₁}\AgdaSpace{}%
\AgdaOperator{\AgdaPrimitive{*ₙ}}\AgdaSpace{}%
\AgdaBound{k₁}\AgdaSymbol{))}\<%
\\
\>[6][@{}l@{\AgdaIndent{0}}]%
\>[8]\AgdaOperator{\AgdaDatatype{≡}}\<%
\\
\>[8][@{}l@{\AgdaIndent{0}}]%
\>[12]\AgdaField{-ω}\AgdaSpace{}%
\AgdaSymbol{(}\AgdaBound{r₂}\AgdaSpace{}%
\AgdaOperator{\AgdaPrimitive{*ₙ}}\AgdaSpace{}%
\AgdaBound{r₁}\AgdaSymbol{)}\AgdaSpace{}%
\AgdaSymbol{⦃}\AgdaSpace{}%
\AgdaFunction{m*n≢0}\AgdaSpace{}%
\AgdaBound{r₂}\AgdaSpace{}%
\AgdaBound{r₁}\AgdaSpace{}%
\AgdaSymbol{⦄}\AgdaSpace{}%
\AgdaSymbol{(}\AgdaBound{r₂}\AgdaSpace{}%
\AgdaOperator{\AgdaPrimitive{*ₙ}}\AgdaSpace{}%
\AgdaSymbol{(}\AgdaBound{k₁}\AgdaSpace{}%
\AgdaOperator{\AgdaPrimitive{*ₙ}}\AgdaSpace{}%
\AgdaBound{j₀}\AgdaSymbol{))}\<%
\\
\>[8][@{}l@{\AgdaIndent{0}}]%
\>[10]\AgdaOperator{\AgdaField{*}}\AgdaSpace{}%
\AgdaField{-ω}\AgdaSpace{}%
\AgdaSymbol{(}\AgdaBound{r₂}\AgdaSpace{}%
\AgdaOperator{\AgdaPrimitive{*ₙ}}\AgdaSpace{}%
\AgdaBound{r₁}\AgdaSymbol{)}\AgdaSpace{}%
\AgdaSymbol{⦃}\AgdaSpace{}%
\AgdaFunction{m*n≢0}\AgdaSpace{}%
\AgdaBound{r₂}\AgdaSpace{}%
\AgdaBound{r₁}\AgdaSpace{}%
\AgdaSymbol{⦄}\AgdaSpace{}%
\AgdaSymbol{(}\AgdaBound{k₀}\AgdaSpace{}%
\AgdaOperator{\AgdaPrimitive{*ₙ}}\AgdaSpace{}%
\AgdaBound{j₀}\AgdaSymbol{)}\<%
\\
%
\>[10]\AgdaOperator{\AgdaField{*}}\AgdaSpace{}%
\AgdaField{-ω}\AgdaSpace{}%
\AgdaSymbol{(}\AgdaBound{r₂}\AgdaSpace{}%
\AgdaOperator{\AgdaPrimitive{*ₙ}}\AgdaSpace{}%
\AgdaBound{r₁}\AgdaSymbol{)}\AgdaSpace{}%
\AgdaSymbol{⦃}\AgdaSpace{}%
\AgdaFunction{m*n≢0}\AgdaSpace{}%
\AgdaBound{r₂}\AgdaSpace{}%
\AgdaBound{r₁}\AgdaSpace{}%
\AgdaSymbol{⦄}\AgdaSpace{}%
\AgdaSymbol{(}\AgdaBound{r₁}\AgdaSpace{}%
\AgdaOperator{\AgdaPrimitive{*ₙ}}\AgdaSpace{}%
\AgdaSymbol{(}\AgdaBound{k₀}\AgdaSpace{}%
\AgdaOperator{\AgdaPrimitive{*ₙ}}\AgdaSpace{}%
\AgdaBound{j₁}\AgdaSymbol{))}\<%
\\
%
\>[6]\AgdaFunction{remove-constant-term}\AgdaSpace{}%
\AgdaSymbol{=}\<%
\\
\>[6][@{}l@{\AgdaIndent{0}}]%
\>[10]\AgdaFunction{cong₂}\AgdaSpace{}%
\AgdaOperator{\AgdaField{\AgdaUnderscore{}*\AgdaUnderscore{}}}\AgdaSpace{}%
\AgdaInductiveConstructor{refl}\AgdaSpace{}%
\AgdaSymbol{(}\AgdaField{ω-N-mN}\AgdaSpace{}%
\AgdaSymbol{\{}\AgdaBound{r₂}\AgdaSpace{}%
\AgdaOperator{\AgdaPrimitive{*ₙ}}\AgdaSpace{}%
\AgdaBound{r₁}\AgdaSymbol{\}}\AgdaSpace{}%
\AgdaSymbol{\{}\AgdaBound{j₁}\AgdaSpace{}%
\AgdaOperator{\AgdaPrimitive{*ₙ}}\AgdaSpace{}%
\AgdaBound{k₁}\AgdaSymbol{\})}\<%
\\
\>[6][@{}l@{\AgdaIndent{0}}]%
\>[8]\AgdaOperator{\AgdaFunction{⊡}}\AgdaSpace{}%
\AgdaFunction{*-identityʳ}\AgdaSpace{}%
\AgdaSymbol{\AgdaUnderscore{}}\<%
\\
%
\>[6]\AgdaFunction{rearrange-ω-power}\AgdaSpace{}%
\AgdaSymbol{:}\<%
\\
\>[6][@{}l@{\AgdaIndent{0}}]%
\>[10]\AgdaField{-ω}\<%
\\
\>[10][@{}l@{\AgdaIndent{0}}]%
\>[12]\AgdaSymbol{(}\AgdaBound{r₂}\AgdaSpace{}%
\AgdaOperator{\AgdaPrimitive{*ₙ}}\AgdaSpace{}%
\AgdaBound{r₁}\AgdaSymbol{)}\<%
\\
%
\>[12]\AgdaSymbol{(}%
\>[15]\AgdaSymbol{(}\AgdaBound{r₂}\AgdaSpace{}%
\AgdaOperator{\AgdaPrimitive{*ₙ}}\AgdaSpace{}%
\AgdaBound{k₁}\AgdaSpace{}%
\AgdaOperator{\AgdaPrimitive{+ₙ}}\AgdaSpace{}%
\AgdaBound{k₀}\AgdaSymbol{)}\<%
\\
%
\>[12]\AgdaOperator{\AgdaPrimitive{*ₙ}}\AgdaSpace{}%
\AgdaSymbol{(}\AgdaBound{r₁}\AgdaSpace{}%
\AgdaOperator{\AgdaPrimitive{*ₙ}}\AgdaSpace{}%
\AgdaBound{j₁}\AgdaSpace{}%
\AgdaOperator{\AgdaPrimitive{+ₙ}}\AgdaSpace{}%
\AgdaBound{j₀}\AgdaSymbol{)}\<%
\\
%
\>[12]\AgdaSymbol{)}\<%
\\
\>[0]\<%
\\
\>[6][@{}l@{\AgdaIndent{0}}]%
\>[8]\AgdaOperator{\AgdaDatatype{≡}}\<%
\\
\>[8][@{}l@{\AgdaIndent{0}}]%
\>[10]\AgdaField{-ω}\<%
\\
\>[10][@{}l@{\AgdaIndent{0}}]%
\>[12]\AgdaSymbol{(}\AgdaBound{r₂}\AgdaSpace{}%
\AgdaOperator{\AgdaPrimitive{*ₙ}}\AgdaSpace{}%
\AgdaBound{r₁}\AgdaSymbol{)}\<%
\\
%
\>[12]\AgdaSymbol{(}\AgdaSpace{}%
\AgdaBound{r₂}\AgdaSpace{}%
\AgdaOperator{\AgdaPrimitive{*ₙ}}\AgdaSpace{}%
\AgdaSymbol{(}\AgdaBound{k₁}\AgdaSpace{}%
\AgdaOperator{\AgdaPrimitive{*ₙ}}\AgdaSpace{}%
\AgdaBound{j₀}\AgdaSymbol{)}\<%
\\
%
\>[12]\AgdaOperator{\AgdaPrimitive{+ₙ}}\AgdaSpace{}%
\AgdaBound{k₀}\AgdaSpace{}%
\AgdaOperator{\AgdaPrimitive{*ₙ}}\AgdaSpace{}%
\AgdaBound{j₀}\<%
\\
%
\>[12]\AgdaOperator{\AgdaPrimitive{+ₙ}}\AgdaSpace{}%
\AgdaBound{r₁}\AgdaSpace{}%
\AgdaOperator{\AgdaPrimitive{*ₙ}}\AgdaSpace{}%
\AgdaSymbol{(}\AgdaBound{k₀}\AgdaSpace{}%
\AgdaOperator{\AgdaPrimitive{*ₙ}}\AgdaSpace{}%
\AgdaBound{j₁}\AgdaSymbol{)}\<%
\\
%
\>[12]\AgdaOperator{\AgdaPrimitive{+ₙ}}\AgdaSpace{}%
\AgdaBound{r₂}\AgdaSpace{}%
\AgdaOperator{\AgdaPrimitive{*ₙ}}\AgdaSpace{}%
\AgdaSymbol{(}\AgdaBound{r₁}\AgdaSpace{}%
\AgdaOperator{\AgdaPrimitive{*ₙ}}\AgdaSpace{}%
\AgdaSymbol{(}\AgdaBound{j₁}\AgdaSpace{}%
\AgdaOperator{\AgdaPrimitive{*ₙ}}\AgdaSpace{}%
\AgdaBound{k₁}\AgdaSymbol{))}\<%
\\
%
\>[12]\AgdaSymbol{)}\<%
\\
%
\>[6]\AgdaFunction{rearrange-ω-power}\AgdaSpace{}%
\AgdaSymbol{=}\<%
\\
\>[6][@{}l@{\AgdaIndent{0}}]%
\>[8]\AgdaFunction{-ω-cong₂}\<%
\\
\>[8][@{}l@{\AgdaIndent{0}}]%
\>[10]\AgdaInductiveConstructor{refl}\<%
\\
%
\>[10]\AgdaSymbol{(}\AgdaFunction{solve}\<%
\\
\>[10][@{}l@{\AgdaIndent{0}}]%
\>[12]\AgdaNumber{6}\<%
\\
%
\>[12]\AgdaSymbol{(λ}\AgdaSpace{}%
\AgdaBound{r₁ℕ}\AgdaSpace{}%
\AgdaBound{r₂ℕ}\AgdaSpace{}%
\AgdaBound{k₀ℕ}\AgdaSpace{}%
\AgdaBound{k₁ℕ}\AgdaSpace{}%
\AgdaBound{j₀ℕ}\AgdaSpace{}%
\AgdaBound{j₁ℕ}\AgdaSpace{}%
\AgdaSymbol{→}\AgdaSpace{}%
\AgdaSymbol{(}\AgdaBound{r₂ℕ}\AgdaSpace{}%
\AgdaOperator{\AgdaFunction{:*}}\AgdaSpace{}%
\AgdaBound{k₁ℕ}\AgdaSpace{}%
\AgdaOperator{\AgdaFunction{:+}}\AgdaSpace{}%
\AgdaBound{k₀ℕ}\AgdaSymbol{)}\AgdaSpace{}%
\AgdaOperator{\AgdaFunction{:*}}\AgdaSpace{}%
\AgdaSymbol{(}\AgdaBound{r₁ℕ}\AgdaSpace{}%
\AgdaOperator{\AgdaFunction{:*}}\AgdaSpace{}%
\AgdaBound{j₁ℕ}\AgdaSpace{}%
\AgdaOperator{\AgdaFunction{:+}}\AgdaSpace{}%
\AgdaBound{j₀ℕ}\AgdaSymbol{)}\AgdaSpace{}%
\AgdaOperator{\AgdaFunction{:=}}\AgdaSpace{}%
\AgdaBound{r₂ℕ}\AgdaSpace{}%
\AgdaOperator{\AgdaFunction{:*}}\AgdaSpace{}%
\AgdaSymbol{(}\AgdaBound{k₁ℕ}\AgdaSpace{}%
\AgdaOperator{\AgdaFunction{:*}}\AgdaSpace{}%
\AgdaBound{j₀ℕ}\AgdaSymbol{)}\AgdaSpace{}%
\AgdaOperator{\AgdaFunction{:+}}\AgdaSpace{}%
\AgdaBound{k₀ℕ}\AgdaSpace{}%
\AgdaOperator{\AgdaFunction{:*}}\AgdaSpace{}%
\AgdaBound{j₀ℕ}\AgdaSpace{}%
\AgdaOperator{\AgdaFunction{:+}}\AgdaSpace{}%
\AgdaBound{r₁ℕ}\AgdaSpace{}%
\AgdaOperator{\AgdaFunction{:*}}\AgdaSpace{}%
\AgdaSymbol{(}\AgdaBound{k₀ℕ}\AgdaSpace{}%
\AgdaOperator{\AgdaFunction{:*}}\AgdaSpace{}%
\AgdaBound{j₁ℕ}\AgdaSymbol{)}\AgdaSpace{}%
\AgdaOperator{\AgdaFunction{:+}}\AgdaSpace{}%
\AgdaBound{r₂ℕ}\AgdaSpace{}%
\AgdaOperator{\AgdaFunction{:*}}\AgdaSpace{}%
\AgdaSymbol{(}\AgdaBound{r₁ℕ}\AgdaSpace{}%
\AgdaOperator{\AgdaFunction{:*}}\AgdaSpace{}%
\AgdaSymbol{(}\AgdaBound{j₁ℕ}\AgdaSpace{}%
\AgdaOperator{\AgdaFunction{:*}}\AgdaSpace{}%
\AgdaBound{k₁ℕ}\AgdaSymbol{)))}\<%
\\
%
\>[12]\AgdaInductiveConstructor{refl}\<%
\\
%
\>[12]\AgdaBound{r₁}\AgdaSpace{}%
\AgdaBound{r₂}\AgdaSpace{}%
\AgdaBound{k₀}\AgdaSpace{}%
\AgdaBound{k₁}\AgdaSpace{}%
\AgdaBound{j₀}\AgdaSpace{}%
\AgdaBound{j₁}\<%
\\
%
\>[10]\AgdaSymbol{)}\<%
\\
%
\>[6]\AgdaFunction{split-ω}\AgdaSpace{}%
\AgdaSymbol{:}\<%
\\
\>[6][@{}l@{\AgdaIndent{0}}]%
\>[10]\AgdaField{-ω}\<%
\\
\>[10][@{}l@{\AgdaIndent{0}}]%
\>[12]\AgdaSymbol{(}\AgdaBound{r₂}\AgdaSpace{}%
\AgdaOperator{\AgdaPrimitive{*ₙ}}\AgdaSpace{}%
\AgdaBound{r₁}\AgdaSymbol{)}\<%
\\
%
\>[12]\AgdaSymbol{(}\AgdaSpace{}%
\AgdaBound{r₂}\AgdaSpace{}%
\AgdaOperator{\AgdaPrimitive{*ₙ}}\AgdaSpace{}%
\AgdaSymbol{(}\AgdaBound{k₁}\AgdaSpace{}%
\AgdaOperator{\AgdaPrimitive{*ₙ}}\AgdaSpace{}%
\AgdaBound{j₀}\AgdaSymbol{)}\<%
\\
%
\>[12]\AgdaOperator{\AgdaPrimitive{+ₙ}}\AgdaSpace{}%
\AgdaBound{k₀}\AgdaSpace{}%
\AgdaOperator{\AgdaPrimitive{*ₙ}}\AgdaSpace{}%
\AgdaBound{j₀}\<%
\\
%
\>[12]\AgdaOperator{\AgdaPrimitive{+ₙ}}\AgdaSpace{}%
\AgdaBound{r₁}\AgdaSpace{}%
\AgdaOperator{\AgdaPrimitive{*ₙ}}\AgdaSpace{}%
\AgdaSymbol{(}\AgdaBound{k₀}\AgdaSpace{}%
\AgdaOperator{\AgdaPrimitive{*ₙ}}\AgdaSpace{}%
\AgdaBound{j₁}\AgdaSymbol{)}\<%
\\
%
\>[12]\AgdaOperator{\AgdaPrimitive{+ₙ}}\AgdaSpace{}%
\AgdaBound{r₂}\AgdaSpace{}%
\AgdaOperator{\AgdaPrimitive{*ₙ}}\AgdaSpace{}%
\AgdaSymbol{(}\AgdaBound{r₁}\AgdaSpace{}%
\AgdaOperator{\AgdaPrimitive{*ₙ}}\AgdaSpace{}%
\AgdaSymbol{(}\AgdaBound{j₁}\AgdaSpace{}%
\AgdaOperator{\AgdaPrimitive{*ₙ}}\AgdaSpace{}%
\AgdaBound{k₁}\AgdaSymbol{))}\<%
\\
%
\>[12]\AgdaSymbol{)}\<%
\\
%
\>[10]\AgdaOperator{\AgdaDatatype{≡}}\<%
\\
\>[10][@{}l@{\AgdaIndent{0}}]%
\>[12]\AgdaField{-ω}\AgdaSpace{}%
\AgdaSymbol{(}\AgdaBound{r₂}\AgdaSpace{}%
\AgdaOperator{\AgdaPrimitive{*ₙ}}\AgdaSpace{}%
\AgdaBound{r₁}\AgdaSymbol{)}\AgdaSpace{}%
\AgdaSymbol{⦃}\AgdaSpace{}%
\AgdaFunction{m*n≢0}\AgdaSpace{}%
\AgdaBound{r₂}\AgdaSpace{}%
\AgdaBound{r₁}\AgdaSpace{}%
\AgdaSymbol{⦄}\AgdaSpace{}%
\AgdaSymbol{(}\AgdaBound{r₂}\AgdaSpace{}%
\AgdaOperator{\AgdaPrimitive{*ₙ}}\AgdaSpace{}%
\AgdaSymbol{(}\AgdaBound{k₁}\AgdaSpace{}%
\AgdaOperator{\AgdaPrimitive{*ₙ}}\AgdaSpace{}%
\AgdaBound{j₀}\AgdaSymbol{))}\<%
\\
%
\>[10]\AgdaOperator{\AgdaField{*}}\AgdaSpace{}%
\AgdaField{-ω}\AgdaSpace{}%
\AgdaSymbol{(}\AgdaBound{r₂}\AgdaSpace{}%
\AgdaOperator{\AgdaPrimitive{*ₙ}}\AgdaSpace{}%
\AgdaBound{r₁}\AgdaSymbol{)}\AgdaSpace{}%
\AgdaSymbol{⦃}\AgdaSpace{}%
\AgdaFunction{m*n≢0}\AgdaSpace{}%
\AgdaBound{r₂}\AgdaSpace{}%
\AgdaBound{r₁}\AgdaSpace{}%
\AgdaSymbol{⦄}\AgdaSpace{}%
\AgdaSymbol{(}\AgdaBound{k₀}\AgdaSpace{}%
\AgdaOperator{\AgdaPrimitive{*ₙ}}\AgdaSpace{}%
\AgdaBound{j₀}\AgdaSymbol{)}\<%
\\
%
\>[10]\AgdaOperator{\AgdaField{*}}\AgdaSpace{}%
\AgdaField{-ω}\AgdaSpace{}%
\AgdaSymbol{(}\AgdaBound{r₂}\AgdaSpace{}%
\AgdaOperator{\AgdaPrimitive{*ₙ}}\AgdaSpace{}%
\AgdaBound{r₁}\AgdaSymbol{)}\AgdaSpace{}%
\AgdaSymbol{⦃}\AgdaSpace{}%
\AgdaFunction{m*n≢0}\AgdaSpace{}%
\AgdaBound{r₂}\AgdaSpace{}%
\AgdaBound{r₁}\AgdaSpace{}%
\AgdaSymbol{⦄}\AgdaSpace{}%
\AgdaSymbol{(}\AgdaBound{r₁}\AgdaSpace{}%
\AgdaOperator{\AgdaPrimitive{*ₙ}}\AgdaSpace{}%
\AgdaSymbol{(}\AgdaBound{k₀}\AgdaSpace{}%
\AgdaOperator{\AgdaPrimitive{*ₙ}}\AgdaSpace{}%
\AgdaBound{j₁}\AgdaSymbol{))}\<%
\\
%
\>[10]\AgdaOperator{\AgdaField{*}}\AgdaSpace{}%
\AgdaField{-ω}\AgdaSpace{}%
\AgdaSymbol{(}\AgdaBound{r₂}\AgdaSpace{}%
\AgdaOperator{\AgdaPrimitive{*ₙ}}\AgdaSpace{}%
\AgdaBound{r₁}\AgdaSymbol{)}\AgdaSpace{}%
\AgdaSymbol{⦃}\AgdaSpace{}%
\AgdaFunction{m*n≢0}\AgdaSpace{}%
\AgdaBound{r₂}\AgdaSpace{}%
\AgdaBound{r₁}\AgdaSpace{}%
\AgdaSymbol{⦄}\AgdaSpace{}%
\AgdaSymbol{((}\AgdaBound{r₂}\AgdaSpace{}%
\AgdaOperator{\AgdaPrimitive{*ₙ}}\AgdaSpace{}%
\AgdaBound{r₁}\AgdaSymbol{)}\AgdaSpace{}%
\AgdaOperator{\AgdaPrimitive{*ₙ}}\AgdaSpace{}%
\AgdaSymbol{(}\AgdaBound{j₁}\AgdaSpace{}%
\AgdaOperator{\AgdaPrimitive{*ₙ}}\AgdaSpace{}%
\AgdaBound{k₁}\AgdaSymbol{))}\<%
\\
%
\>[6]\AgdaFunction{split-ω}\AgdaSpace{}%
\AgdaSymbol{=}\<%
\\
\>[6][@{}l@{\AgdaIndent{0}}]%
\>[12]\AgdaSymbol{(}\AgdaField{ω-N-k₀+k₁}\AgdaSpace{}%
\AgdaSymbol{\{}\AgdaBound{r₂}\AgdaSpace{}%
\AgdaOperator{\AgdaPrimitive{*ₙ}}\AgdaSpace{}%
\AgdaBound{r₁}\AgdaSymbol{\}}\AgdaSpace{}%
\AgdaSymbol{\{}\AgdaBound{r₂}\AgdaSpace{}%
\AgdaOperator{\AgdaPrimitive{*ₙ}}\AgdaSpace{}%
\AgdaSymbol{(}\AgdaBound{k₁}\AgdaSpace{}%
\AgdaOperator{\AgdaPrimitive{*ₙ}}\AgdaSpace{}%
\AgdaBound{j₀}\AgdaSymbol{)}\AgdaSpace{}%
\AgdaOperator{\AgdaPrimitive{+ₙ}}\AgdaSpace{}%
\AgdaBound{k₀}\AgdaSpace{}%
\AgdaOperator{\AgdaPrimitive{*ₙ}}\AgdaSpace{}%
\AgdaBound{j₀}\AgdaSpace{}%
\AgdaOperator{\AgdaPrimitive{+ₙ}}\AgdaSpace{}%
\AgdaBound{r₁}\AgdaSpace{}%
\AgdaOperator{\AgdaPrimitive{*ₙ}}\AgdaSpace{}%
\AgdaSymbol{(}\AgdaBound{k₀}\AgdaSpace{}%
\AgdaOperator{\AgdaPrimitive{*ₙ}}\AgdaSpace{}%
\AgdaBound{j₁}\AgdaSymbol{)\}}\AgdaSpace{}%
\AgdaSymbol{\{}\AgdaBound{r₂}\AgdaSpace{}%
\AgdaOperator{\AgdaPrimitive{*ₙ}}\AgdaSpace{}%
\AgdaSymbol{(}\AgdaBound{r₁}\AgdaSpace{}%
\AgdaOperator{\AgdaPrimitive{*ₙ}}\AgdaSpace{}%
\AgdaSymbol{(}\AgdaBound{j₁}\AgdaSpace{}%
\AgdaOperator{\AgdaPrimitive{*ₙ}}\AgdaSpace{}%
\AgdaBound{k₁}\AgdaSymbol{))\}}\AgdaSpace{}%
\AgdaSymbol{)}\<%
\\
\>[6][@{}l@{\AgdaIndent{0}}]%
\>[10]\AgdaOperator{\AgdaFunction{⊡}}\AgdaSpace{}%
\AgdaSymbol{(}\AgdaFunction{flip}\AgdaSpace{}%
\AgdaOperator{\AgdaFunction{\$}}\AgdaSpace{}%
\AgdaFunction{cong₂}\AgdaSpace{}%
\AgdaOperator{\AgdaField{\AgdaUnderscore{}*\AgdaUnderscore{}}}\AgdaSymbol{)}\AgdaSpace{}%
\AgdaSymbol{(}\AgdaFunction{-ω-cong₂}\AgdaSpace{}%
\AgdaInductiveConstructor{refl}\AgdaSpace{}%
\AgdaOperator{\AgdaFunction{\$}}\AgdaSpace{}%
\AgdaFunction{sym}\AgdaSpace{}%
\AgdaOperator{\AgdaFunction{\$}}\AgdaSpace{}%
\AgdaFunction{*ₙ-assoc}\AgdaSpace{}%
\AgdaBound{r₂}\AgdaSpace{}%
\AgdaBound{r₁}\AgdaSpace{}%
\AgdaSymbol{(}\AgdaBound{j₁}\AgdaSpace{}%
\AgdaOperator{\AgdaPrimitive{*ₙ}}\AgdaSpace{}%
\AgdaBound{k₁}\AgdaSymbol{))}\<%
\\
%
\>[10]\AgdaSymbol{(}\AgdaSpace{}%
\AgdaSymbol{(}\AgdaField{ω-N-k₀+k₁}\AgdaSpace{}%
\AgdaSymbol{\{}\AgdaBound{r₂}\AgdaSpace{}%
\AgdaOperator{\AgdaPrimitive{*ₙ}}\AgdaSpace{}%
\AgdaBound{r₁}\AgdaSymbol{\}}\AgdaSpace{}%
\AgdaSymbol{\{}\AgdaBound{r₂}\AgdaSpace{}%
\AgdaOperator{\AgdaPrimitive{*ₙ}}\AgdaSpace{}%
\AgdaSymbol{(}\AgdaBound{k₁}\AgdaSpace{}%
\AgdaOperator{\AgdaPrimitive{*ₙ}}\AgdaSpace{}%
\AgdaBound{j₀}\AgdaSymbol{)}\AgdaSpace{}%
\AgdaOperator{\AgdaPrimitive{+ₙ}}\AgdaSpace{}%
\AgdaBound{k₀}\AgdaSpace{}%
\AgdaOperator{\AgdaPrimitive{*ₙ}}\AgdaSpace{}%
\AgdaBound{j₀}%
\>[83]\AgdaSymbol{\}}\AgdaSpace{}%
\AgdaSymbol{\{}\AgdaBound{r₁}\AgdaSpace{}%
\AgdaOperator{\AgdaPrimitive{*ₙ}}\AgdaSpace{}%
\AgdaSymbol{(}\AgdaBound{k₀}\AgdaSpace{}%
\AgdaOperator{\AgdaPrimitive{*ₙ}}\AgdaSpace{}%
\AgdaBound{j₁}\AgdaSymbol{)}%
\>[110]\AgdaSymbol{\}}\AgdaSpace{}%
\AgdaSymbol{)}\<%
\\
%
\>[10]\AgdaOperator{\AgdaFunction{⊡}}%
\>[3537I]\AgdaSymbol{(}\AgdaFunction{flip}\AgdaSpace{}%
\AgdaOperator{\AgdaFunction{\$}}\AgdaSpace{}%
\AgdaFunction{cong₂}\AgdaSpace{}%
\AgdaOperator{\AgdaField{\AgdaUnderscore{}*\AgdaUnderscore{}}}\AgdaSymbol{)}\AgdaSpace{}%
\AgdaInductiveConstructor{refl}\<%
\\
\>[.][@{}l@{}]\<[3537I]%
\>[12]\AgdaSymbol{(}\AgdaField{ω-N-k₀+k₁}\AgdaSpace{}%
\AgdaSymbol{\{}\AgdaBound{r₂}\AgdaSpace{}%
\AgdaOperator{\AgdaPrimitive{*ₙ}}\AgdaSpace{}%
\AgdaBound{r₁}\AgdaSymbol{\}}\AgdaSpace{}%
\AgdaSymbol{\{}\AgdaBound{r₂}\AgdaSpace{}%
\AgdaOperator{\AgdaPrimitive{*ₙ}}\AgdaSpace{}%
\AgdaSymbol{(}\AgdaBound{k₁}\AgdaSpace{}%
\AgdaOperator{\AgdaPrimitive{*ₙ}}\AgdaSpace{}%
\AgdaBound{j₀}\AgdaSymbol{)}%
\>[83]\AgdaSymbol{\}}\AgdaSpace{}%
\AgdaSymbol{\{}\AgdaBound{k₀}\AgdaSpace{}%
\AgdaOperator{\AgdaPrimitive{*ₙ}}\AgdaSpace{}%
\AgdaBound{j₀}%
\>[110]\AgdaSymbol{\}}\AgdaSpace{}%
\AgdaSymbol{)}\<%
\\
%
\>[10]\AgdaSymbol{)}\<%
\end{code}
\subsubsection{Nesting of Sums}


\section{Review of Implementation}
\clearpage
\bibliographystyle{IEEEtran}
\bibliography{Thesis/References}

\end{document}

