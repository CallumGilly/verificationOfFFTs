\section{Conclusion}
This paper provides a formally verified implementation of the Cooley Tukey Fast 
Fourier Transform.
This implementation is built inductively, first defining representations of
tensors and complex numbers.
A trivial implementation of the Discrete Fourier Transform is then provided.
This is then used to define an implementation of the Fast Fourier Transform.

Unlike most verifiable implementations of the FFT, this implementations of the
FFT is radix generic and is defined on an abstract definition of Complex.
Defining the FFT generically allows for any non prime input
to be split optimally, and for the structure of any future parallelism to be 
defined at run time.

Given this general implementation of the FFT, I have then provided a proof that
it is equal to the DFT for all possible inputs.
With a basic implementation of the complex numbers, and a basic compiler,
this allowed for the generation of a runnable, verified, instance of the FFT.

In future research I wish to investigate the generation of optimised code, 
through translation into an intermediate language such as SaC.
The speed and floating point accuracy of such code would ideally be 
comparable to, or an improvement upon, the speed and accuracy of most similar
kernels in FFTW.
Should such comparison show significant improvement over the results of FFTW,
investigation into the generation of a kernel for FFTW would be possible.

Additionally, within the Agda community, this work allows future research 
projects to make use of the FFT interchangeably with the DFT.
This allows for future research into the verification of many signal processing 
algorithms, such as fast convolution or correlation, or for more general 
methods such as fast polynomial multiplication.

