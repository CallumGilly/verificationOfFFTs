\documentclass[12pt, twoside]{article}
\usepackage{graphicx} % Required for inserting images
\usepackage{fancyhdr} % Required to put information in a right and left justified header
\usepackage{enumitem} % Required to make itemize left margin not massive 
\usepackage{hyperref} % Allows linking to the bib in citations
\usepackage{amsmath} % Used to make maths good
\usepackage{amssymb} % USed for maths symbols
\usepackage{minted}
\usepackage{pdflscape}
% Required to set the margin sizes to avoid \vspace{-9em}
\usepackage{geometry}
\geometry{
    a4paper,
    left=35mm,
    right=24mm,
    top=24mm,
    bottom=24mm,
}
\setlength{\headheight}{18.0pt}
\renewcommand{\familydefault}{\rmdefault}
\setlength{\parindent}{0pt}  % Disable paragraph indentation
\pagenumbering{roman}

% Used to stop me submitting the wrong version
%\usepackage{draftwatermark}
%\SetWatermarkText{Draft}
%\SetWatermarkScale{4}
%\SetWatermarkAngle{30}

% https://tex.stackexchange.com/questions/318142/typeseeting-a-multiset-with-double-curly-braces
\newcommand*{\xlbrace}{[\mskip-5mu[}
\newcommand*{\xrbrace}{]\mskip-5mu]}

\usepackage{bm}
% \usepackage{agda}
\usepackage{newunicodechar}



\newunicodechar{₁}{\ensuremath{_1}}
\newunicodechar{₂}{\ensuremath{_2}}
\newunicodechar{₃}{\ensuremath{_3}}
\newunicodechar{₄}{\ensuremath{_4}}
\newunicodechar{₅}{\ensuremath{_5}}
\newunicodechar{ι}{\ensuremath{\iota}}
\newunicodechar{ₗ}{\ensuremath{_l}}
\newunicodechar{ₐ}{\ensuremath{_a}}
\newunicodechar{ₖ}{\ensuremath{_k}}
\newunicodechar{ᵢ}{\ensuremath{_i}}
\newunicodechar{ⱼ}{\ensuremath{_j}}
\newunicodechar{ᵥ}{\ensuremath{_v}}
\newunicodechar{ₕ}{\ensuremath{_h}}

\newunicodechar{ᵣ}{\ensuremath{_r}}
\newunicodechar{ʳ}{\ensuremath{^r}}
\newunicodechar{ˡ}{\ensuremath{^l}}
\newunicodechar{ℕ}{\ensuremath{\mathbb{N}}}
\newunicodechar{∀}{\ensuremath{\forall}}
\newunicodechar{≡}{\ensuremath{\equiv}}
\newunicodechar{≈}{\ensuremath{\approx}}
\newunicodechar{∎}{\ensuremath{\blacksquare}}
\newunicodechar{⊛}{\ensuremath{\circledast}}
\newunicodechar{⊗}{\ensuremath{\otimes}}
\newunicodechar{⊕}{\ensuremath{\oplus}}
\newunicodechar{φ}{\ensuremath{\phi}}
\newunicodechar{ψ}{\ensuremath{\psi}}
\newunicodechar{ε}{\ensuremath{\epsilon}}
\newunicodechar{λ}{\ensuremath{\lambda}}
\newunicodechar{σ}{\ensuremath{\sigma}}
\newunicodechar{′}{{'}}
\newunicodechar{∷}{\ensuremath{\dblcolon}}
\newunicodechar{↔}{\ensuremath{\leftrightarrow}}
\newunicodechar{↦}{\ensuremath{\mapsto}}
\newunicodechar{∘}{\ensuremath{\circ}}
\newunicodechar{⁻}{\ensuremath{^{-}}}
\newunicodechar{∙}{\ensuremath{\boldsymbol{\cdot}}}
\newunicodechar{▹}{\ensuremath{\triangleright}}
\newunicodechar{Σ}{\ensuremath{\Sigma}}
\newunicodechar{∈}{\ensuremath{\in}}
\newunicodechar{⟦}{\ensuremath{\llbracket}}
\newunicodechar{⟧}{\ensuremath{\rrbracket}}
\newunicodechar{⇒}{\ensuremath{\Rightarrow}}
\newunicodechar{⟪}{\ensuremath{\xlbrace}}
\newunicodechar{⟫}{\ensuremath{\xrbrace}}
\newunicodechar{⊠}{\ensuremath{\boxtimes}}
\newunicodechar{⊞}{\ensuremath{\boxplus}}
\newunicodechar{∋}{\ensuremath{\ni}}
\newunicodechar{∶}{\ensuremath{\bm{:}}}
\newunicodechar{↦}{\ensuremath{\mapsto}}
\newunicodechar{ℂ}{\ensuremath{\mathbb{C}}} 

\newunicodechar{ʹ}{\ensuremath{\prime}}


%TC:ignore
\pagestyle{fancy}
\lhead{
    \large
    % Electronics and Computer Science \\
    % Faculty of Engineering and Physical Sciences \\
    % University of Southampton 
    Final Report
    }
\rhead{
    \large 15th August 2025
}

\begin{document}
\thispagestyle{fancy}
\begin{center}
    \vspace*{20mm}
    \Large
    Electronics and Computer Science \\
    Faculty of Engineering and Physical Sciences \\
    University of Southampton 

    
    \vspace*{20mm}
    \Large Callum Gilchrist (\verb|cg3g22@soton.ac.uk|)
    
    \large 15th August 2025
    
    \vspace*{20mm}
    \Huge Formal Verification of Fast Fourier Transforms
    \vspace*{60mm}
    
    % \large Project Brief
    \large \textbf{Supervisor:} Artjoms Šinkarovs (\verb|a.sinkarovs@soton.ac.uk|)
    \large \textbf{Second Examiner:} Vahid Yazdanpanah (\verb|v.yazdanpanah@soton.ac.uk|)

    \vspace*{10mm}
    \normalsize A Project Report submitted for the award of \\
    \large \textbf{BSc Computer Science}
    
    \vspace*{10mm}
    
\end{center}
\clearpage
\addcontentsline{toc}{section}{Abstract}
\section*{Abstract}
Discrete Fourier Transforms (DFTs) are key operations within Digital Signal Processing and other fields, Fast Fourier Transforms (FFTs) allow for the time complexity of computing the DFT to be significantly reduced. 
Implementations of the FFT often comprise large, low-level libraries written with efficiency in mind, making verification of their correctness challenging.
Agda is a dependently typed functional language which implements Martin-Löf type theory allowing proofs to be embedded within code.
As a result of including these proofs, programs written in Agda can contain formal guarantees of their correctness, for the FFT this requires proving that the DFT is equal to the FFT for all cases.


In this project, I have thus far created an Agda definition of the DFT and FFT.
I now plan to implement proof that the two definitions are equal for all possible inputs.
Given such a proof I will then use the FFT definition to generate a low-level library of my own which can perform Fourier Transforms with strict guarantees on its correctness.

\clearpage
%TC:endignore
\include{Thesis/StatementOfOriginality}
\clearpage
% \addcontentsline{toc}{section}{Nomenclature}
% \section*{Nomenclature}
% \begin{table}[h]
%     \centering
%     \begin{tabular}{|c|c|}
%         \hline
%          DFT & Discrete Fourier Transform  \\
%         \hline
%          FFT & Fast Fourier Transform  \\
%         \hline
%     \end{tabular}
%     \caption{Nomenclature}
%     \label{tab:my_label}
% \end{table}
% \clearpage
\addcontentsline{toc}{section}{Contents}
\tableofcontents
\clearpage
% Add space between paragraphs, but only after the TOC
\setlength{\parskip}{0.5em}
\pagenumbering{arabic}
\clearpage
\section{Introduction}

The Discrete Fourier Transform (DFT) is a staple operation within Computer 
Science, Physics, and other fields with many applications.
Fast Fourier Transforms are implementations of the DFT with improved 
performance characteristics.
One such use of the FFT is polynomial multiplication, the time 
complexity of which can be reduced from $\mathcal{O}\left(n^2\right)$ to
$\mathcal{O}\left(n\log n\right)$.

Most current implementations, such as WFFT\cite{Frigo2005}, take the form of 
large libraries written in low-level languages. 
A key component of these libraries is the use of multiple implementations of 
the same algorithm, with each implementation (or kernel) containing 
optimisations suited towards specific input sizes and hardware profiles. 
When the user wants to compute the result of a Fourier Transform, the library 
chooses the optimal kernel based on the input size and the user's hardware.

The large number of kernels makes it very challenging to verify that a given 
FFT library provides the same result as the naïve DFT.
This is because to do so would involve analysing the low-level implementation 
of each kernel, individually, and proving that it gives the same result as the 
naïve DFT for all possible inputs.
An alternate approach is as follows. % This sentence doesn't add much, 
                                     % but I'm not sure that it flows too well 
                                     % without it
Instead of analysing existing code to confirm its correctness, we can create a 
single specification of the FFT such that it can be instantiated to any kernel, 
giving us a usable kernel and formal proof that said kernel computes the 
expected values.

Agda is a dependently typed functional language which allows for formal 
properties of programs written in it to be proven.\cite{Norell2007} 
This paper discusses the use of Agda to create a general case implementation 
of the FFT which is then proven to always compute the same value as the naïve 
DFT.

Such an implementation would allow for future research in Agda to make use of
the FFT in definitions, before substituting it with the DFT when it comes to
generating proofs.
This would be useful for research into algorithms which utilise the FFT, such as
efficient polynomial multiplication.
Such an implementation would also allow for future generation of low-level, 
efficient kernels, with a formally verified basis.

\section{Background}
\subsection{Fourier Transforms}
Described as ``perhaps the most ubiquitous algorithm in use today''\cite{Top10Algos},
Fourier Transforms are mathematical operations which transform functions between 
the time domain and the frequency domain.
Fourier Transforms, and derivatives of, receive their name from the French 
mathematician and physicist Jean-Baptiste-Joseph Fourier who proposed in his 
$1822$\cite{Fourier1822} treatise that any given function can be represented as 
a harmonic series.\cite{Saribulut2013} 
% https://mathoverflow.net/questions/417034/who-introduced-the-discrete-fourier-transform
While bearing Fourier's name, some early forms of the Discrete Fourier 
Transform (DFT), a Fourier Transform which works on evenly spaced samples of a 
function, can be found before Fourier's time.
As discussed by Heideman and Johnson in ``Gauss and the History of the Fast 
Fourier Transform''\cite{Heideman1985}, the earliest known example of this can 
be found in work published by Alexis-Claude Clairaut in $1754$\cite{Clairaut1754}.
Clairaut defined a variation of the DFT which exclusively used what we now refer 
to as the cosine component, thus restricting the input domain to the set of 
even functions\footnote{The term ``even function'' refers to the set of 
functions $f(x)$ such that $f(-x)=f(x)$, that is to say, the set of functions 
which are symmetric over the y-axis.  \cite{Gelfand1990}\cite{Tolstov1962}}.\cite{Heideman1985}
Carl Friedrich Gauss extended Clairaut's definition to make use of both cosine 
and sine components, removing the need for the input domain to be restricted 
to the set of even functions and allowing for the analysis of any periodic 
function.\cite{Gauss1866}\cite{Heideman1985}
This definition was published posthumously in $1866$, however, it is believed 
that it was originally written in $1805$.\cite{Heideman1985} 

We can use the historical definitions discussed above to create our modern 
definition for the DFT as follows.
Given an input sequence $x = 
        (x_0,x_1,\dots,x_{n-1})$,
            where $x_i\in\mathbb{C}
            $,
our transformed sequence $X =
        (X_0,X_1,\dots,X_{n-1})$,
            where $X_i\in\mathbb{C}
            $,
is given as follows.

\begin{align}
    X_j &= \sum_{k=0}^{N-1}x_k\omega_N^{kj}\label{eq:DFT_Definition}
\end{align}
\begin{equation}
    \text{where}~\omega_N~=e^{-\frac{2\pi i}{N}}
    = \cos{\left(\frac{2\pi}{N}\right)}-i\sin{\left(\frac{2\pi}{N}\right)}\label{eq:ComplexRootsOfUnity}
\end{equation}
% Original
%\begin{align}
%    X_j &= \sum_{k=0}^{N-1}\omega_N^{jk}x_k\label{eq:DFT_Definition} \\
%    \text{where}~\omega_N~&=e^{-\frac{2\pi i}{N}}\\
%    &= \cos{\left(\frac{2\pi}{N}\right)}-i\sin{\left(\frac{2\pi}{N}\right)}
%\end{align}

The DFT Eq. \ref{eq:DFT_Definition} has applications in a variety of fields, 
such as digital signal processing\cite{Bellanger2024}.
When implemented naïvely, however, it has poor performance scaling, requiring  
``$\mathcal{O}\left(n^2\right)$ complex operations'' \cite{VanLoan1992}.
Methods to reduce the number of complex operations required when computing the DFT were first investigated by Gauss in his $1805$ treatise such that the ``tediousness of mechanical calculations''\cite{Gauss1866} could be reduced.\cite{Heideman1985}
In part due to his lack of research into the complexity scaling factor of his method, Gauss's research into how computation complexity could be reduced was not widely recognised until $1977$ when H. H. Goldstine highlighted Gauss's research in an article for the Journal of Applied Mathematics and Mechanics.\cite{Heideman1985}\cite{Heinrich1980}
While the DFT continued to be of great use to mathematicians through the 20th century, and with Gauss's work on complexity remaining hidden, some attempts (such as those by Danielson and Lanczos \cite{Danielson1942} and by Good \cite{Good1958}) were made to create Fast Fourier Transform algorithms (FFT algorithms) which could reduce the complexity of computation to $\mathcal{O}\left(n\log n\right)$.
These algorithms, however, where only applicable to a subset of the domain\cite{Good1958}, succeeded only in reducing the constant on $\mathcal{O}\left(n^2\right)$, or did not directly perform the computational complexity\cite{Danielson1942}.

In $1965$ James William Cooley and John Tukey succeeded in discovering an FFT algorithm through the inadvertent reinvention of Gauss's algorithm for fast computation of the DFT; This would henceforth be known as the Cooley-Tukey FFT Algorithm.\cite{Cooley1965}\cite{Heideman1985}
This FFT Algorithm allows for a given DFT to be computed with $\mathcal{O}\left(n\log n\right)$ complex operations through recursive splitting of the input.\cite{Cooley1965}
Although other FFT Algorithms were discovered before and after the Cooley-Tukey FFT, it is commonly considered to be ``the most important FFT''\cite{Frigo2005}.
This is because this improvement in time complexity allowed algorithms with 
time complexity previously bounded by use of the DFT, to reduce this complexity 
to, at the lowest, $\mathcal{O}\left(n^2\right)$.
In the example of polynomial multiplication, this allowed for the computation
to be moved into the frequency domain, reducing the time complexity from 
$\Theta\left(n^2\right)$ to $\Theta\left(n\log n\right)$.\cite{IntroToAlgos}


The Cooley-Tukey FFT can be derived from the DFT Eq. \ref{eq:DFT_Definition} by splitting any non-prime input $n$ into the composite $n=r_1r_2$ and expressing the indices $k$ and $j$ as follows.
\begin{align}\label{eq:IndexManipulation}
    \begin{aligned}
        j&=j_1r_1+j_0 \\
        \text{where }~
        j_0&=(0,1,\dots,r_1-1) \\
        j_1&=(0,1,\dots,r_2-1) 
    \end{aligned}
    \begin{aligned}
        &~&~&~
    \end{aligned}
    \begin{aligned}
        k&=k_1r_2+k_0 \\
        \text{where }~k_0&=(0,1,\dots,r_2-1) \\
        k_1&=(0,1,\dots,r_1-1)
    \end{aligned}
\end{align}
Eq. \ref{eq:DFT_Definition} can then be arranged to take the following form.
\begin{align}
    X_{j_1r_1+j_0}&=\sum^{r_2-1}_{k_0=0}\left[\left(\sum^{r_1-1}_{k_1=0}x_{k_1r_2+k_0}\omega_{r_1}^{k_1j_0}\right)\omega_{r_1r_2}^{k_0j_1}\right]\omega_{r_2}^{k_0j_1}
    \label{eq:FFTDefinitionFromDFT}
\end{align}
When written in this form our recursive step, and thus the core idea of the Cooley-Tukey FFT, be easily observed by noting that the inner sum takes the form of a DFT of length $r_1$.
% \begin{align}
%     j&=j_1r_1+j_0, &k&=k_1r_2+k_0 \\
%     \text{where }~
%      j_0&=(0,1,\dots,r_1-1) 
%     &k_0&=(0,1,\dots,r_2-1) \\
%      j_1&=(0,1,\dots,r_2-1) 
%     &k_1&=(0,1,\dots,r_1-1)
% \end{align}


%\begin{align}
%    n&=n_1
%\end{align}\cite{Cooley1965AnSeries}
%
%We can then represent

%Methods to reduce the number of complex operations required when computing DFTs were first investigated by Gauss in his $1805$ treatise such that the ``tediousness of mechanical calculations''\cite{Gauss1868Nachlass:Tractata} could be reduced.\cite{Heideman1985GaussTransform}
%However, for a variety of reasons, Gauss's research into how computation complexity could be reduced was not widely recognised. \cite{Heideman1985GaussTransform}\cite{The one which worked out Gauss is a cool kid}
%In 1965 James William Cooley and John Tukey inadvertently reinvented Gauss's algorithm for fast computation of the DFT, in what would henceforth be known as the Cooley-Tukey Fast Fourier Transform (FFT) Algorithm.\cite{Cooley1965AnSeries}\cite{Heideman1985GaussTransform}
%This FFT Algorithm allows for a given DFT to be computed with $\mathcal{O}\left(n\log n\right)$ complex operations through recursive splitting on the input vector.\cite{Cooley1965AnSeries}


%never published investigations into the resultant reduction in computational complexity meaning that his methods, which reduced the complexity to $\mathcal{O}\left( N\log N \right)$ would go unnoticed until *AFTER C-T but I ain't mentioned C-T yet...*.



\subsection{Agda}
Agda\footnote{Reference to ``Agda'' throughout this report will always refer to 
version 2 unless explicitly stated otherwise} is a functional programming 
language which implements Martin-Löf Type Theory.\cite{Norell2007}\cite{Martin-Lf1984}
Martin-Löf type theory provides the definition of, and Agda allows for the 
construction of, dependent types.\cite{Norell2007}
These types allow for the definition of invariant properties which are checked
at compile time.\cite{Norell2007}
As well as making a variety of common errors, such as out-of-bound indexing,
unreachable, invariance properties can be used to guarantee functional
properties.\cite{Norell2007}
When evidence that properties hold is non trivial, proofs that the properties
hold must be provided.
This allows for strong guarantees to be formed on any program defined in Agda.
These proofs allow systems to be built which are provably correct allowing for a high confidence in their reliability.\cite{Norell2007}

Agda is not the only such proof assistant, and others exist with Agda's main 
contender being Coq.
Coq considers programs and proofs - or as it refers to them, tactics - 
separately.
This means that ``every concept has to be learned twice''\cite{PLFA}, for its 
program component and tactic component. \cite{Barras1999}
In Agda, however, proofs and programs are considered in the same light, removing
the need for this additional syntax and simplifying the programs within it.

\subsection{Related work}
FFTW\cite{Frigo2005} is a \verb|C| code library which is generally accepted within academia and industry as the fastest method with which the FFT can be correctly computed.\cite{Frigo1999} 
It achieves this title by implementing its own ``special-purpose compiler''\cite{Frigo1999}, \verb|genfft|, this compiler accepts the size of the transform as input and outputs a kernel - a \verb|c| code implementations of some known algorithm (i.e. the Cooley-Tukey FFT \cite{Cooley1965} Eq .\ref{eq:FFTDefinitionFromDFT}) optimised for that sized transform and the current hardware.\cite{Frigo1999} 
Although it is known through rigorous testing and real-world use that FFTW is correct, there is no formal verification of its correctness. %to provide a formal verification of the library would also be a very challenging undertaking requiring analysis of the low-level code which makes up the library.
% An alternate approach to formally verifying its correctens

As FFTW does not come with such formal guarantees 
separate definitions of various FFTs have been created before in proof assistance such as Coq\cite{Barras1999} and Hol\cite{Gordon1993} with various methods and goals. In the paper ``Certifying the Fast Fourier Transform with Coq''\cite{Capretta2001}, Capretta makes use of binary trees to create a definition of the Cooley-Tukey FFT\cite{Cooley1965} for the radix-2 case (when $r_1=2$).
This definition is then proven to be extensionally equal to that of the DFT.
This provides a good definition for the radix-2 case of the FFT, allowing for it to be built on to create future proofs should they require the FFT, however, it does not cover the generalisation on the radix restricting the proof to specific splitting strategies.
% Theres probably allot more waffle I can put in here

In another paper, ``A Methodology for the Formal Verification of FFT Algorithms in 
HOL'',\cite{Akbarpour2004} Akbarpour and Tahar create two definitions of the 
Cooley-Tukey FFT\cite{Cooley1965} in Hol for the radix-2 and radix-4 cases.
With a primary focus on the radix-2 case Akbarpour and Tahar go on to show 
equivalence to the DFT across various levels of abstraction.\cite{Akbarpour2004}
At one stage of this abstraction, Akbarpour and Tahar introduce floating and 
fixed point arithmetic, showing an analysis of the resultant errors.\cite{Akbarpour2004}

Much like Capretta\cite{Capretta2001}, this paper also does not make use of 
a general radix, however, it does highlight how its methodology can be used to 
analyse general radix FFT implementations.
Currently, all previous work to formally verify the Cooley Tukey FFT has used 
fixed radices, while most common implementations utilise mixed radices which 
``are adapted to the hardware''\cite{Frigo2005}.
This show a gap in the existing research, as no verification on these mixed
radix cases is present.

\clearpage
\bibliographystyle{IEEEtran}
\bibliography{Thesis/References}

\end{document}

